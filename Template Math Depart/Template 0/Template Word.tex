\documentclass[10pt,openany,a4paper]{article}
\usepackage{graphicx} 
\usepackage{multirow}
\usepackage{enumitem}
\usepackage{amssymb}
\usepackage{amsmath}
\usepackage{amsthm}
\usepackage{xcolor}
\usepackage{geometry}
	\geometry{
		total = {160mm, 237mm},
		left = 25mm,
		right = 35mm,
		top = 30mm,
		bottom = 30mm,
	}
\usepackage{fancyhdr}
\renewcommand{\headrulewidth}{0pt}
\renewcommand{\arraystretch}{1.1}
\pagestyle{fancy}

\graphicspath{{C:/Users/teoso/OneDrive/Documents/Tugas Kuliah/Template Math Depart/}{D:/Hada Touya/Tugas-Kuliah/Template Math Depart/}}

\newcommand{\R}{\mathbb{R}}
\newcommand{\N}{\mathbb{N}}
\newcommand{\Z}{\mathbb{Z}}
\newcommand{\Q}{\mathbb{Q}}
\newcommand{\jawab}{\textbf{Solusi}:}

\newtheorem*{teorema}{Teorema}
\newtheorem*{definisi}{Definisi}

\begin{document}
\fancyfoot[C]{\raisebox{.5ex}{\rule{0.5cm}{.4pt}}o0o\raisebox{.5ex}{\rule{0.5cm}{.4pt}}}

\begin{tabular}{r c l}
    \includegraphics[width=2cm]{ITS.png}
     & \begin{tabular}{|c|l|l|}
           \hline
           \multirow{3}{*}{{\Huge \textbf{EAS}}}        & \textbf{Matakuliah}    & Geometri Analitik (A,B,C,D)         \\
           \cline{2-3}
                                                        & \textbf{Semester}      & 1                                   \\
           \cline{2-3}
           \multirow{3}{*}{{\large \textbf{GASAL}}}     & \textbf{Kredit SKS}    & 3                                   \\
           \cline{2-3}
           \multirow{3}{*}{{\large \textbf{2023/2024}}} & \textbf{Hari, Tanggal} & Jumat, 15 Desember 2023             \\
           \cline{2-3}
                                                        & \textbf{Waktu}         & \textbf{100 menit}                  \\
           \cline{2-3}
                                                        & \textbf{Dosen}         & \begin{tabular}{l}
                                                                                    Drs. I Gst Ngr Rai Usadha, M.Si.   \\
                                                                                    Dra, Wahyu Fistia Doctorina, M.Si. \\
                                                                                    Drs. Komar Baihaqi, M.Si.          \\
                                                                                    DR. Mont Kistosil Fahim, S.Si, M.Si.
                                                                                \end{tabular} \\
           \hline
       \end{tabular}
     &
    \includegraphics[width=2cm]{M.png}
    \\
\end{tabular}
\begin{center}

\end{center}
\begin{enumerate}
    \item
    \item
    \item
    \item
\end{enumerate}
\begin{center}
    \textbf{== HARAP JUNJUNG TINGGI KEJUJURAN ==}
\end{center}
\end{document}