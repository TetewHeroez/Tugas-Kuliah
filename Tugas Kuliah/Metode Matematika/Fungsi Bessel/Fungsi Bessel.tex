\documentclass[10pt,openany,a4paper]{article}
\usepackage{graphicx} 
\usepackage{multirow}
\usepackage{enumitem}
\usepackage{amssymb}
\usepackage{amsmath}
\usepackage{xcolor}
\usepackage{cancel}
\usepackage{tcolorbox}
\usepackage{geometry}
\usepackage{tikz}
	\geometry{
		total = {160mm, 237mm},
		left = 25mm,
		right = 35mm,
		top = 30mm,
		bottom = 30mm,
	}
\newcommand{\jawab}{\textbf{Jawab}:}

\begin{document}
    \pagenumbering{gobble}
    \begin{tabular}{|lcl|}
     \hline
     Nama&:&Teosofi Hidayah Agung\\
     NRP&:&5002221132\\
     \hline
    \end{tabular}
    \begin{enumerate}
        \setcounter{enumi}{2}
        \item Kembang $\displaystyle f(x)=\begin{cases}
            2x+1\quad&;\quad 0< x\leq 1\\
            0\quad&;\quad -1\leq x<0
        \end{cases}$ pada $a_nP_n(x)$.\\
        \jawab
        \[\boxed{a_n=\frac{2n+1}{2}\int_{-1}^{1}f(x)P_0(x)~dx}\]
        \begin{flalign*}
            a_0&=\frac{1}{2}\int_{-1}^{1}f(x)P_0(x)dx&\\
            &=\frac{1}{2}\int_{-1}^{0}0\,dx+\frac{1}{2}\int_{0}^{1}(2x+1)\,dx&\\
            &=\frac{1}{2}\left[x^2+x\right]_0^1=1&\\
            a_1&=\frac{3}{2}\int_{-1}^{1}f(x)P_1(x)dx&\\
            &=\frac{3}{2}\int_{-1}^{0}0\,dx+\frac{3}{2}\int_{0}^{1}(2x+1)x\,dx&\\
            &=\frac{3}{2}\left[\frac{2}{3}x^3+\frac{1}{2}x^2\right]_0^1=\frac{3}{2}\left[\frac{7}{6}\right]=\frac{7}{4}&\\
            a_2&=\frac{5}{2}\int_{-1}^{1}f(x)P_2(x)dx&\\
            &=\frac{5}{2}\int_{-1}^{0}0\,dx+\frac{5}{2}\int_{0}^{1}(2x+1)\left(\frac{3}{2}x^2-\frac{1}{2}\right)dx&\\
            &=\frac{5}{2}\int_{0}^{1}(6x^3+3x^2-2x-1)\,dx&\\
            &=\frac{5}{2}\left[\frac{3}{2}x^4+x^3-x^2-x\right]_0^1=\frac{5}{2}\left[\frac{1}{2}+\frac{1}{2}\right]=\frac{5}{8}&\\
            a_3&=\frac{7}{2}\int_{-1}^{1}f(x)P_3(x)dx&\\
            &=\frac{7}{2}\int_{-1}^{0}0\,dx+\frac{7}{2}\int_{0}^{1}(2x+1)\left(\frac{5}{2}x^3-\frac{3}{2}x\right)dx&\\
            &=\frac{7}{4}\int_{0}^{1}(10x^4+5x^3-6x^2-3x)\,dx&\\
            &=\frac{7}{4}\left[2x^5+\frac{5}{4}x^4-2x^3-\frac{3}{2}x^2\right]_0^1=\frac{7}{4}\left[2+\frac{5}{4}-2-\frac{3}{2}\right]=-\frac{7}{8}&\\
        \end{flalign*}
        $\displaystyle \therefore\,f(x)=\frac{1}{2}P_0(x)+\frac{7}{4}P_1(x)+\frac{5}{8}P_2(x)-\frac{7}{8}P_3(x)$

        \item Nyatakan $\displaystyle J_{\frac{5}{2}}(x)=\sqrt{\frac{2}{\pi x}}\left(\frac{3-x^2}{x}\sin x-\frac{3}{x}\cos x\right)$.\\
        \jawab
        \begin{flalign*}
            \frac{2n}{x}J_n(x)&=J_{n-1}(x)+J_{n+1}(x)\\
            \frac{3}{x}J_{\frac{3}{2}}(x)&=J_{\frac{1}{2}}(x)+J_{\frac{5}{2}}(x)\\
            J_{\frac{5}{2}}(x)&=\frac{3}{x}J_{\frac{3}{2}}(x)-J_{\frac{1}{2}}(x)\\
            &=\frac{3}{x}\left(\sqrt{\frac{2}{\pi x}}\left(\frac{\sin x}{x}-\cos x\right)\right)-\sqrt{\frac{2}{\pi x}}\sin x\\
            &=\sqrt{\frac{2}{\pi x}}\left(\frac{3-x^2}{x}\sin x-\frac{3}{x}\cos x\right)
        \end{flalign*}
    \end{enumerate}
\end{document}