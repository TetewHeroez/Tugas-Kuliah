\documentclass[10pt]{beamer}
\usepackage{pgf}
\usepackage{colortbl,tabularx,amsfonts,mathrsfs,calligra}
\usepackage{ragged2e}
\usepackage{setspace}
\usepackage{tikz}
\usepackage{filecontents}
\usepackage{amsmath,amssymb}


\bibliographystyle{apalike}

\usetheme{CambridgeUS}
\usecolortheme{orchid}

\usefonttheme{professionalfonts}
\setbeamertemplate{theorems}[numbered]
\setbeamertemplate{bibliography item}{\insertbiblabel}
\setbeamercovered{transparent}

\setbeamercolor{boxbiru}{fg=yellow,bg=black!50!blue!90}
\setbeamercolor{boxpurple}{fg=white,bg=purple}
\setbeamercolor{boxpink}{fg=black,bg=pink}
\setbeamercolor{boxviolet}{fg=white,bg=violet}
\setbeamercolor{boxabu}{fg=black!30!blue!100,bg=gray!40}
\setbeamertemplate{background canvas}[vertical shading][bottom=red!15,top=blue!15]




\usetikzlibrary{shapes.geometric, arrows}

\newtheorem{theorems}{Teorema}
\newtheorem{Lemmas}[theorems]{Lema}
\newtheorem{remark}[theorems]{Catatan}
\newtheorem{proposition}[theorems]{Proposisi}
\newtheorem{algorithm}[theorems]{Algoritma}
\newtheorem{problems}[theorems]{Problem}
\newtheorem{exercise}[theorem]{Latihan}
\newtheorem{contoh}[theorems]{Contoh}
\newtheorem{corollaries}[theorems]{Akibat}
\newtheorem{definisi}{Definisi}
%\newproof{proof}{Proof}
\newenvironment{proofs}{\paragraph{Proof:}}{\hfill$\blacksquare$\newline}



 \pgfdeclareimage[height=.85cm]{logo}{logoITS}
 \logo{\pgfuseimage{logo}}

\date{}
\title{INSTITUT TEKNOLOGI SEPULUH NOPEMBER}
\author{Teori Peluang\part{The Probability Theory}}
\institute{by Team Teaching}



\begin{document}

\begin{frame}[plain]
    %\frametitle{\bf Analisis Real I}
    \transboxout

    \vspace{10ex}

    \begin{beamercolorbox}[wd=\textwidth,rounded=true,shadow=true]{boxbiru}
        \centering
        \resizebox{.8\textwidth}{1em}{%
        \textsf{\textbf{BAB 4: INTEGRAL GARIS DAN APLIKASINYA}}}\\[1.5ex]
    \end{beamercolorbox}

%    \centerline{Bagian I}
%	\centerline{Himpunan dan Fungsi}

    \vspace*{6ex}

    \centerline{\includegraphics[scale=.1]{logoITS}}

    \vspace*{2ex}

    \begin{center}
    \tiny
    \rm
    \color{magenta!50!cyan!100}
    INSTITUT TEKNOLOGI SEPULUH NOPEMBER\\
    Departemen Matematika\\
    %Institut Teknologi Sepuluh Nopember\\
    Indonesia\\
    \end{center}
\end{frame}%--------------------------------------------

\AtBeginSection[]
{
	\begin{frame}
		\frametitle{Daftar Isi}
		\tableofcontents[hidesubsections,currentsection]
	\end{frame}
}
\section{Pendahuluan}

\begin{frame}
    \frametitle{Operator Vektor Delta ($\nabla$)}
    \begin{definisi}
        Operator vektor $\nabla$ atau disebut juga \textbf{del} adalah operator diferensial yang didefinisikan sebagai
        \begin{equation}
            \nabla = \left(\frac{\partial}{\partial x}, \frac{\partial}{\partial y}, \frac{\partial}{\partial z}\right)
        \end{equation}
    \end{definisi}
    \begin{definisi}
        Dalam koordinat silinder, operator $\nabla$ didefinisikan sebagai
        \begin{equation}
            \nabla = \left(\frac{\partial}{\partial r}, \frac{1}{r}\frac{\partial}{\partial \theta}, \frac{\partial}{\partial z}\right)
        \end{equation}
    \end{definisi}
\end{frame}

\begin{frame}
    \frametitle{Operator Vektor Delta ($\nabla$)}
    \begin{definisi}
        Dalam koordinat bola, operator $\nabla$ didefinisikan sebagai
        \begin{equation}
            \nabla = \left(\frac{\partial}{\partial r}, \frac{1}{r}\frac{\partial}{\partial \theta}, \frac{1}{r\sin\theta}\frac{\partial}{\partial \phi}\right)
        \end{equation}
    \end{definisi}
\end{frame}

\begin{frame}
    \frametitle{Gradien}
    \begin{definisi}
        Gradien dari suatu fungsi skalar $f(x,y,z)$ adalah vektor yang didefinisikan sebagai
        \begin{equation}
            \nabla f = \left(\frac{\partial f}{\partial x}, \frac{\partial f}{\partial y}, \frac{\partial f}{\partial z}\right)
        \end{equation}
    \end{definisi}
    Gradien mengubah fungsi skalar menjadi vektor.
\end{frame}

\begin{frame}
    \frametitle{Gradien}
    \begin{example}
        Tentukan gradien dari fungsi $f(x,y,z) = x^2 + y^2 + z^2$.\\
        \textbf{Jawab:}
        \begin{eqnarray*}
            \nabla f &=& \left(\frac{\partial f}{\partial x}, \frac{\partial f}{\partial y}, \frac{\partial f}{\partial z}\right)\\
            &=& \left(2x, 2y, 2z\right)
        \end{eqnarray*}
    \end{example}
\end{frame}

\begin{frame}
    \frametitle{Divergensi}
    \begin{definisi}
        Divergensi dari suatu vektor $F(x,y,z) = (P(x,y,z), Q(x,y,z), R(x,y,z))$ adalah fungsi skalar yang didefinisikan sebagai
        \begin{equation}
            \nabla \cdot F = \frac{\partial P}{\partial x} + \frac{\partial Q}{\partial y} + \frac{\partial R}{\partial z}
        \end{equation}
    \end{definisi}
    Divergensi mengubah vektor menjadi fungsi skalar.
\end{frame}

\begin{frame}
    \frametitle{Divergensi}
    \begin{example}
        Tentukan divergensi dari vektor $F(x,y,z) = (\sin xy, \cos yz, \tan xz)$.\\
        \textbf{Jawab:}
        \begin{eqnarray*}
            \nabla \cdot F &=& \frac{\partial P}{\partial x} + \frac{\partial Q}{\partial y} + \frac{\partial R}{\partial z}\\
            &=& y\cos xy - z \sin yz + x\sec^2 xz
        \end{eqnarray*}
    \end{example}
\end{frame}

\begin{frame}
    \frametitle{Curl}
    \begin{definisi}
        Curl dari suatu vektor $F(x,y,z) = (P(x,y,z), Q(x,y,z), R(x,y,z))$ adalah vektor yang didefinisikan sebagai
        \begin{equation}
            \nabla \times F = \left(\frac{\partial R}{\partial y} - \frac{\partial Q}{\partial z}, \frac{\partial P}{\partial z} - \frac{\partial R}{\partial x}, \frac{\partial Q}{\partial x} - \frac{\partial P}{\partial y}\right)
        \end{equation}
    \end{definisi}
    Curl mengubah vektor menjadi vektor.
\end{frame}

\begin{frame}
    \frametitle{Curl}
    \begin{example}
        Tentukan curl dari vektor $F(x,y,z) = (2y \ln x,4e^{xyz}, z^y )$.\\
        \textbf{Jawab:}
        \begin{eqnarray*}
            \nabla \times F &=& \left(\frac{\partial R}{\partial y} - \frac{\partial Q}{\partial z}, \frac{\partial P}{\partial z} - \frac{\partial R}{\partial x}, \frac{\partial Q}{\partial x} - \frac{\partial P}{\partial y}\right)\\
            &=& (z^y - 0, 0 - 0, 0 - 4e^{xyz})\\
            &=& (z^y, 0, -4e^{xyz})
        \end{eqnarray*}
    \end{example}
\end{frame}

\end{document}