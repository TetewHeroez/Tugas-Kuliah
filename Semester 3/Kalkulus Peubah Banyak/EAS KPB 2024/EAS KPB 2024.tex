\documentclass[11pt,a4paper]{article}
\usepackage{graphicx} 
\usepackage{multirow}
\usepackage{enumitem}
\usepackage{amssymb}
\usepackage{amsmath}
\usepackage{amsthm}
\usepackage{xcolor}
\usepackage{tikz}
\usepackage{pgfplots}
  \pgfplotsset{
        every non boxed x axis/.style={
            xtick align=center,
            tick style={line width=0.5pt, color=black},
            x axis line style={-{Latex[width=1.5mm]},black,line width=0.5pt},
            xlabel style={at={(ticklabel* cs:1.05)}, anchor=mid},
            xlabel=$x$
        },
         every non boxed y axis/.style={
            ytick align=center,
            tick style={line width=0.7pt, color=black},
            y axis line style={-{Latex[width=1.5mm]},black,line width=0.5pt},
            ylabel style={at={(ticklabel* cs:1.08)}, anchor=mid},
            ylabel=$y$
        },
        every non boxed z axis/.style={
            ztick align=center,
            tick style={line width=0.5pt, color=black},
            z axis line style={-{Latex[width=1.5mm]},black,line width=0.5pt},
            zlabel style={at={(ticklabel* cs:1.06)}, anchor=mid},
            zlabel=$z$
        },
        tick label style={
            font=\tiny,
        },
        compat=1.18
    }

  \usepgfplotslibrary{colormaps,patchplots}
  \usepgfplotslibrary{fillbetween}
  \pgfplotsset{colormap={cm}{color(0)=(cyan) color(1)=(cyan!90) color(3)=(cyan!80) color(4)=(cyan!70) color(5)=(cyan!10)}}
  \usetikzlibrary{arrows.meta}
\usepackage{geometry}
	\geometry{
		total = {160mm, 237mm},
		left = 20mm,
		right = 20mm,
		top = 20mm,
		bottom = 20mm,
	}
\usepackage{hyperref}
\hypersetup{
    colorlinks=true,
    linkcolor=blue,
    filecolor=magenta,      
    urlcolor=cyan,
    pdftitle={Overleaf Example},
    pdfpagemode=FullScreen,
    }
\usepackage{fancyhdr}
\renewcommand{\headrulewidth}{0pt}
\renewcommand{\arraystretch}{1.1}
\pagestyle{fancy}

\graphicspath{{C:/Users/teoso/OneDrive/Documents/Tugas Kuliah/Template Math Depart/}{D:/Hada Touya/Tugas-Kuliah/Template Math Depart/}}

\newcommand{\R}{\mathbb{R}}
\newcommand{\N}{\mathbb{N}}
\newcommand{\Z}{\mathbb{Z}}
\newcommand{\Q}{\mathbb{Q}}
\newcommand{\jawab}{\textbf{Solusi}:}

\newtheorem*{teorema}{Teorema}
\newtheorem*{definisi}{Definisi}


\begin{document}
\begin{table}[h!]
  \centering
  \begin{tabular}{|r c|}
    \hline
    \multicolumn{2}{|c|}{\MakeUppercase{\large\bfseries evaluasi tengah semester gasal 2024/2025}}                         \\
     &                                                                                                                     \\
    \begin{tabular}{c}
      \includegraphics[width=2cm]{ITS.png}
      \includegraphics[width=2cm]{M.png}      \\
      {\large\bfseries Departemen Matematika} \\
      {\large\bfseries FSAD}
    \end{tabular}
     & \begin{tabular}{lcll}
         Matakuliah    & : & \multicolumn{2}{l}{Kalkulus Peubah Banyak}                                              \\
         Hari, Tanggal & : & \multicolumn{2}{l}{Selasa, 10 Desember 2024}                                            \\
         Waktu / Sifat & : & \multicolumn{2}{l}{100 menit / \textit{Tertutup}}                                       \\
         Kelas, Dosen  & : & A.                                                & Dra. Nur Asiyah, M.Si.              \\
                       &   & B.                                                & Drs. Suhud Wahyudi, M.Si.           \\
                       &   & C.                                                & Drs. Lukman Hanafi, M.Si.           \\
                       &   & D.                                                & Dr. Didik Khusnul Arif S.Si., M.Si. \\
       \end{tabular}

    \\
    \hline
    \multicolumn{2}{|l|}{\color{red}\MakeUppercase{harap diperhatikan !!!}}                                                \\
    \multicolumn{2}{|l|}{\color{red}Segala jenis pelanggaran (mencontek, kerjasama, dsb) yang dilakukan pada saat ETS/EAS} \\
    \multicolumn{2}{|l|}{\color{red}akan dikenakan sanksi pembatalan matakuliah pada semester yang sedang berjalan.}       \\
    \hline
  \end{tabular}
\end{table}

\begin{enumerate}
  \item Hitung integral berikut ini $\iiint_R dx\,dy\,dz$, dengan $R$ adalah daerah integrasi yang dibatasi oleh bidang-bidang $z=\frac{1}{2}x,\, z=0,\, y=x,\, x+y=2,$ dan $y=0$. Sketsa batas integrasi $R$.
  \item Dapatkan Volume benda yang dibatasi $z=y$, $y=x^2$, dan $x=y^2$ yang berada dalam oktan pertama. Sketsalah batas permukaan benda tersebut.
  \item Gambarkan keping datar homogen yang dibatasi oleh kurva-kurva $x(1-y)=1$, $x(1-y)=2$, $xy=1$, dan $xy=3$. Hitung pula momen inersia terhadap sumbu $y$. (Petunjuk: transformasi ke koordinat baru $(u,v)$)
  \item Dapatkan pusat massa permukaan benda $2z=8-x^2-y^2$, jika densitinya konstan, yang berada dalam silinder $x^2+y^2=3$. Sertai dengan sketsa permukaan benda.
\end{enumerate}
\vspace*{1cm}
\begin{center}
  \textbf{Selamat Mengerjakan Semoga Sukses}
\end{center}
\end{document}