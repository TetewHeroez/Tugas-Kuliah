\documentclass[10pt,openany,a4paper]{article}
\usepackage{graphicx} 
\usepackage{multirow}
\usepackage{enumitem}
\usepackage{amssymb}
\usepackage{amsmath}
\usepackage{amsthm}
\usepackage{xcolor}
\usepackage{cancel}
\usepackage{tikz}
\usepackage{animate}
\usepackage{ragged2e}
\usepackage{geometry}
	\geometry{
		total = {160mm, 237mm},
		left = 25mm,
		right = 35mm,
		top = 30mm,
		bottom = 30mm,
	}
\usepackage{fancyhdr}
\renewcommand{\headrulewidth}{0pt}
\pagestyle{fancy}

\graphicspath{{C:/Users/teoso/OneDrive/Documents/Tugas Kuliah/Template Math Depart/}{D:/Hada Touya/Tugas-Kuliah/Template Math Depart/}}

\newcommand{\R}{\mathbb{R}}
\newcommand{\N}{\mathbb{N}}
\newcommand{\Z}{\mathbb{Z}}
\newcommand{\Q}{\mathbb{Q}}
\newcommand{\jawab}{\textbf{Solusi}:}

\newtheorem*{teorema}{Teorema}
\newtheorem*{definisi}{Definisi}

\begin{document}
\fancyfoot[C]{\raisebox{.5ex}{\rule{0.5cm}{.4pt}}o0o\raisebox{.5ex}{\rule{0.5cm}{.4pt}}}

\begin{tabular}{r c l}
    \includegraphics[width=2cm]{ITS.png}
     & \begin{tabular}{lcll}
           \multicolumn{4}{c}{\begin{tabular}{c}
                                   \MakeUppercase{evaluasi tengah semester gasal 2024/2025} \\
                                   \MakeUppercase{departemen matematika fsad its}           \\
                                   \MakeUppercase{program sarjana}                          \\
                               \end{tabular}}                        \\
           \\
           Matakuliah    & : & \multicolumn{2}{l}{Analisis 1}                                                 \\
           Hari, Tanggal & : & \multicolumn{2}{l}{Kamis, 16 Oktober 2024}                                     \\
           Waktu / Sifat & : & \multicolumn{2}{l}{100 menit / \textit{Closed Book}}                           \\
           Kelas, Dosen  & : & A.                                                   & Dr. Mahmud Yunus, M.Si. \\
                         &   & B.                                                   & Drs. Sadjidon, M.Si.    \\
                         &   & C.                                                   & Dr. Sunarsini, M.Si.    \\
                         &   & D.                                                   & Dr. Rinurwati, M.Si.    \\
       \end{tabular}
     &
    \includegraphics[width=2cm]{M.png}
    \\ \hline
    \multicolumn{3}{|l|}{\color{red}\MakeUppercase{harap diperhatikan !!!}}                                                \\
    \multicolumn{3}{|l|}{\color{red}Segala jenis pelanggaran (mencontek, kerjasama, dsb) yang dilakukan pada saat ETS/EAS} \\
    \multicolumn{3}{|l|}{\color{red}akan dikenakan sanksi pembatalan matakuliah pada semester yang sedang berjalan.}       \\
    \hline
\end{tabular}

\begin{enumerate}
    \justifying
    \item Pandang $A,B\subseteq\R$ dan fungsi $f\,:\,A\,\to\,B$ yang didefinisikan dengan $f(x)\,=\,\dfrac{2x-3}{x+1}$.\\
          Dapatkan \textit{domain} dan \textit{range} dari $f$ agar $f$ bijektif. Buktikan jawaban Anda.\\

    \item Buktikan dengan Induksi Matematika bahwa jumlah $n$ bilangan ganjil yang pertama adalah $n^2$, yaitu $1+3+5+\cdots+(2n-1)=n^2$ untuk setiap $n\in\N$.\\

    \item Buktikan bahwa tidak ada bilangan rasional $q$ sehingga $q^2=3$.\\

    \item Tunjukkan bahwa:
          \begin{enumerate}
              \item Jika $a\in\R$ dan $0\leq a< \varepsilon$ untuk setiap $\varepsilon>0$, maka $a=0$.
              \item Jika $\mathcal{N}_\varepsilon(a)$ adalah persekitaran-$\varepsilon$ dari $a$ dan $x\in\mathcal{N}_\varepsilon(a)$ untuk setiap $\varepsilon>0$, maka $x=a$.\\
          \end{enumerate}

    \item Diberikan barisan bilangan real $X=(x_n)$ dengan $x_n=(2-\frac{1}{n})^2$ untuk semua $n\in\N$.
          \begin{enumerate}
              \item Tunjukkan bahwa $X$ barisan terbatas.
              \item Dapatkan $\inf(X)$ dan $\sup(X)$; dan buktikan jawaban Anda.
              \item Dengan menerapkan definisi konvergensi barisan dan hasil bagian (b), buktikan bahwa $\lim_{n\to\infty}x_n=\sup(X)$.
          \end{enumerate}
\end{enumerate}

\newpage
\fancyfoot[C]{}
\fancyfoot[R]{\color{blue}\textit{Teosofi Hidayah Agung - 5002221132}}
\jawab
\begin{enumerate}
    \item Pertama lakukan oret-oretan untuk mendapatkan \textit{domain} dan \textit{range} (Hal ini bertujuan untuk melakukan suatu klaim atau biasa kita kenal sebagai \textit{conjecture}). Dengan menggunakan konsep kalkulus, didapatkan \textit{domain}-nya adalah $A=\R\backslash\{-1\}$ dan \textit{range}-nya adalah $B=\R\backslash\{2\}$.\\

          Selanjutnya kita perlu buktikan bahwa dengan memilih \textit{domain} dan \textit{range} tersebut, fungsi $f$ menjadi bijektif. Ingat bahwa bijektif adalah gabungan dari sifat Injektif dan sifat Surjektif.\\

          \begin{definisi}
              Sebuah fungsi $f\,:\,A\,\to\,B$ dikatakan injektif jika untuk setiap $x_1,x_2\in A$ dengan $x_1\ne x_2$ maka $f(x_1)\ne f(x_2)$.
          \end{definisi}
          Atau kita bisa menggunakan kontraposisi dari definisi diatas, yaitu
          \begin{definisi}
              Sebuah fungsi $f\,:\,A\,\to\,B$ dikatakan injektif jika untuk setiap $x_1,x_2\in A$ dengan $f(x_1)=f(x_2)$ maka $x_1=x_2$.
          \end{definisi}
          Disini kita perlu mencari apa isi dari $A$ dan $B$. Dengan menggunakan konsep kalkulus, kita bisa mendapatkan $A=\R\backslash\{-1\}$ dan $B=\R\backslash\{2\}$.

          Selanjutnya ambil sebarang $x_1,x_2\in A$. Kemudian misalkan $f(x_1)=f(x_2)$, maka
          \begin{flalign*}
              \dfrac{2x_1-3}{x_1+1} & =\dfrac{2x_2-3}{x_2+1}
          \end{flalign*}
          Sebab $x_1,x_2\ne -1$, maka $x_1+1\ne 0$ dan $x_2+1\ne 0$. Artinya $x_1+1$ dan $x_2+1$ masing-masing punya invers yaitu $\dfrac{1}{x_1+1}$ dan $\dfrac{1}{x_2+1}$. Dengan demikian, kita bisa mengalikan kedua ruas persamaan di atas dengan $(x_1+1)(x_2+1)$ sehingga didapatkan
          \begin{flalign*}
              \dfrac{2x_1-3}{x_1+1}\cdot(x_1+1)(x_2+1) & =\dfrac{2x_2-3}{x_2+1}\cdot(x_1+1)(x_2+1) \\
              2x_1(x_2+1)-3(x_2+1)                     & =2x_2(x_1+1)-3(x_1+1)                     \\
              \cancel{2x_1x_2}+2x_1-3x_2\cancel{-3}    & =\cancel{2x_2x_1}+2x_2-3x_1\cancel{-3}    \\
              2x_1-3x_2                                & =2x_2-3x_1                                \\
              2x_1+3x_1                                & =2x_2+3x_2                                \\
              5x_1                                     & =5x_2                                     \\
              x_1                                      & =x_2
          \end{flalign*}
          didapatkan $x_1=x_2$. Hal ini mengimplikasikan $f$ adalah fungsi injektif.\\

          \begin{definisi}
              Sebuah fungsi $f\,:\,A\,\to\,B$ dikatakan surjektif jika untuk setiap $y\in B$ terdapat $x\in A$ sehingga $f(x)=y$.
          \end{definisi}
          Ambil sebarang $y\in B$. Karena $B$ adalah kodomain dari $f$, maka $y$ dapat dinyatakan sebagai $y=\dfrac{2x-3}{x+1}$ untuk suatu $x\in A$. Dengan demikian, kita bisa mendapatkan $x$ sebagai berikut
          \begin{flalign*}
              y      & =\dfrac{2x-3}{x+1} \\
              y(x+1) & =2x-3              \\
              yx+y   & =2x-3              \\
              yx-2x  & =-3-y              \\
              x(y-2) & =-3-y              \\
              x      & =\dfrac{-3-y}{y-2}
          \end{flalign*}
          Sebab $y\ne 2$, maka $y-2\ne 0$. Artinya selalu ada $x\in A$ sehingga $f(x)=y$. Hal ini mengimplikasikan $f$ adalah fungsi surjektif.\\

          $\therefore$ Dengan demikian, $f$ adalah fungsi bijektif.

    \item Misalkan $P(n)$ adalah pernyataan bahwa $1+3+5+\cdots+(2n-1)=n^2$.
          \begin{itemize}
              \item Untuk $n=1$, maka $P(1)$ adalah $1=1^2$ yang dimana adalah pernyataan yang benar.
              \item Asumsi bahwa $P(k)$ benar untuk suatu $k\in\N$, yaitu $1+3+5+\cdots+(2k-1)=k^2$.
              \item Untuk $n=k+1$, maka $P(k+1)$ adalah
                    \begin{flalign*}
                        \underbrace{1+3+5+\cdots+(2k-1)}_{k^2}+(2(k+1)-1)=k^2+2k+1=(k+1)^2
                    \end{flalign*}
          \end{itemize}
          Dengan demikian, berdasarkan prinsip induksi matematika, kita bisa menyimpulkan bahwa $P(n)$ benar untuk setiap $n\in\N$.

    \item Disini kita dapat melakukan pembuktian seperti yang biasanya dilakukan untuk membuktikan tidak ada bilangan rasional $p$ yang memenuhi $p^2=2$, yaitu dengan menggunakan kontradiksi.\\

          Asumsikan bahwa $q$ adalah bilangan rasional, maka $q$ dapat dinyatakan sebagai $q=\dfrac{a}{b}$ dengan $a,b\in \Z\setminus\{0\}$ dan $\text{fpb}(a,b)=1$. Sehingga didapatkan
          \[q^2=3\iff \left(\dfrac{a}{b}\right)^2=3\iff \dfrac{a^2}{b^2}=3\iff a^2=3b^2\]
          Akibatnya $a^2$ adalah kelipatan dari 3. Sebab $a^2$ adalah bilangan bulat, maka $a$ juga adalah kelipatan dari 3. Dengan demikian, $a$ dapat dinyatakan sebagai $a=3c$ untuk suatu $c\in\Z$. Sehingga didapatkan
          \[a^2=(3c)^2=9c^2=3b^2\iff 3c^2=b^2\iff b^2\]
          Hal ini juga berarti $b^2$ adalah kelipatan dari 3. Dari sini didapatkan hasil $\text{fpb}(a,b)=3$ yang bertentangan dengan asumsi bahwa $\text{fpb}(a,b)=1$.\\

          $\therefore$ Tidak ada bilangan rasional $q$ yang memenuhi $q^2=3$.

    \item
          \begin{enumerate}
              \item Diketahui $a\in\mathbb{R}$, sehingga akan memenuhi sifat-sifat pada bilangan real. Asumsikan $a\neq0$. Maka didapat
                    \begin{flalign*}
                         & 0<a<\varepsilon        & \\
                         & a<\varepsilon\:,\: a>0
                    \end{flalign*}
                    Karena berlaku untuk setiap $\varepsilon>0$, dapat dipilih $\varepsilon_0=\frac{1}{2}a$ sehingga
                    \begin{flalign*}
                         & a<\varepsilon_0\:,\: a>0                           & \\
                         & a<\frac{1}{2}a\:,\:a>0\quad \textrm{(Kontradiksi)}
                    \end{flalign*}
                    Akibatnya asumsi $a\neq0$ salah. Jadi haruslah $a=0$.

              \item Sebelumnya kita harus mengetahui terlebih dahulu definisi dari persekitaran-$\varepsilon$ dari $a$.
                    \begin{definisi}
                        Sebuah himpunan $\mathcal{N}_\varepsilon(a)$ disebut sebagai persekitaran-$\varepsilon$ dari $a$ jika terdapat $\varepsilon>0$ sehingga $\mathcal{N}_\varepsilon(a)=\{x\in\mathbb{R}\,:\,|x-a|<\varepsilon\}$.
                    \end{definisi}
                    Perhatikan bahwa soal mengiginkan untuk setiap $\varepsilon>0$, maka untuk $x\in\mathcal{N}_\varepsilon(a)$ haruslah $|x-a|<\varepsilon$ memenuhi untuk setiap $\varepsilon>0$. Sehingga dari informasi pada soal (a), didapatkan $|x-a|=0$ yang berarti $x=a$.
          \end{enumerate}

    \item
          \begin{enumerate}
              \item Barisan dikatakan terbatas jika dia terbatas di atas dan terbatas di bawah. Sekarang perhatikan bahwa untuk setiap $n\in\N$, berlaku
                    \[\left(2-\frac{1}{n}\right)^2>0\]
                    Sehingga didapatkan bahwa $(x_n)$ terbatas di bawah oleh 0.\\
                    Selanjutnya, perhatikan bahwa untuk setiap $n\in\N$, berlaku
                    \begin{flalign*}
                        \frac{1}{n}                  & >0 \\
                        -\frac{1}{n}                 & <0 \\
                        0<2-\frac{1}{n}              & <2 \\
                        \left(2-\frac{1}{n}\right)^2 & <4
                    \end{flalign*}
                    Sehingga didapatkan bahwa $(x_n)$ terbatas di atas oleh 4.\\

                    $\therefore$ Barisan $X$ adalah barisan terbatas.

              \item Untuk mencari $\inf(X)$ dan $\sup(X)$, kita perlu menggunakan konsep kalkulus kembali seperti menggunakan limit dsb. Dari hasil analisa, dapat kita klaim bahwa $\inf(X)=1$ dan $\sup(X)=4$.\\

                    Sekarang kita akan buktikan klaim tersebut.
                    \begin{itemize}
                        \item Pertama kita akan buktikan bahwa $\inf(X)=1$.
                              \begin{teorema}
                                  Misalkan $H$ adalah himpunan tak kosong dari $\R$. Sebuah bilangan $a\in\R$ dikatakan sebagai infimum dari $H$ jika
                                  \begin{enumerate}
                                      \item $a\leq x$ untuk setiap $x\in H$.
                                      \item Untuk setiap $\varepsilon>0$, terdapat $x\in H$ sehingga $a\leq x<a+\varepsilon$.\\
                                  \end{enumerate}
                              \end{teorema}

                              Menggunakan teorema di atas, diperloleh
                              \begin{enumerate}
                                  \item Untuk setiap $n\geq 1,n\in\N$, berlaku
                                        \[\frac{1}{n}\leq 1\iff 2-\frac{1}{n}\geq 1\iff \left(2-\frac{1}{n}\right)^2\geq 1\]
                                        Sehingga $1$ adalah batas bawah dari barisan $X$.
                                  \item Untuk setiap $\varepsilon>0$, dapat dengan mudah kita pilih $x=1\in X$ sehingga $1\leq x<1+\varepsilon$.\\
                              \end{enumerate}

                              $\therefore$ Terbukti $\inf(X)=1$.\\

                        \item Kedua kita akan buktikan bahwa $\sup(X)=4$.
                              \begin{teorema}
                                  Misalkan $H$ adalah himpunan tak kosong dari $\R$. Sebuah bilangan $b\in\R$ dikatakan sebagai supremum dari $H$ jika
                                  \begin{enumerate}
                                      \item $b\geq x$ untuk setiap $x\in H$.
                                      \item Untuk setiap $\varepsilon>0$, terdapat $x\in H$ sehingga $b-\varepsilon<x\leq b$.\\
                                  \end{enumerate}
                              \end{teorema}
                              Menggunakan teorema di atas, diperloleh
                              \begin{enumerate}
                                  \item Dari hasil (a), jelas bahwa $4$ adalah batas atas dari barisan $X$.
                                  \item Dengan sifat Archimedes, didapatkan untuk setiap $\varepsilon>0$ selalu terdapat $n_\varepsilon\in\N$ sehingga $\frac{1}{n_\varepsilon}<\varepsilon$. Informasi tersebut dapat digunakan sebagai berikut
                                        \[\left(2-\frac{1}{n_\varepsilon}\right)^2=4-\frac{4}{n_\varepsilon}+\frac{1}{n^2_\varepsilon}>4-\frac{4}{n_\varepsilon}\]
                                        Pilih $n_\varepsilon=\dfrac{4}{\varepsilon}$, sehingga
                                        \[4-\frac{4}{n_\varepsilon}>4-\varepsilon\]
                                        Hal ini menunjukkan terdapat $x=\left(2-\frac{1}{n_\varepsilon}\right)^2\in X$ sehingga $4-\varepsilon<x\leq 4$.\\
                              \end{enumerate}

                              $\therefore$ Terbukti $\sup(X)=4$.
                    \end{itemize}
              \item Ingat tentang definisi konvergensi barisan
                    \begin{definisi}
                        Sebuah barisan $(x_n)$ dikatakan konvergen ke $a\in\R$ jika untuk setiap $\varepsilon>0$, terdapat $N_\varepsilon\in\N$ sehingga untuk setiap $n\geq N_\varepsilon$ berlaku $|x_n-a|<\varepsilon$ .
                    \end{definisi}
                    Dari hasil sebelumnya, kita sudah mengetahui bahwa $\sup(X)=4$. Sehingga kita bisa klaim bahwa $\lim_{n\to\infty}x_n=4$.\\

                    Sekarang akan kita buktikan klaim tersebut. Perhatikan bahwa
                    \begin{flalign*}
                        |x_n-4| & =\left|\left(2-\frac{1}{n}\right)^2-4\right|=\left|\frac{1}{n^2}-\frac{4}{n}\right|\leq\frac{1}{n^2}+\frac{4}{n}<\frac{4}{n}
                    \end{flalign*}
                    Pilih $N_\varepsilon=\dfrac{4}{\varepsilon}$, sehingga untuk setiap $n\geq N_\varepsilon$ berlaku
                    \[|x_n-4|<\varepsilon\]
                    Sehingga berdasarkan definisi konvergensi barisan, didapatkan $\lim_{n\to\infty}x_n=4$.
          \end{enumerate}
\end{enumerate}

\tikz[remember picture,overlay]
\node at ([xshift=-4cm,yshift=0cm]current page.east)
{\animategraphics[autoplay,loop,width=0.3\textwidth]{30}{Kita Ikuyo Doodle/Kita Ikuyo Doodle-}{0}{85}};

\end{document}