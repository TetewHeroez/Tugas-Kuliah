\documentclass[10pt,openany,a4paper]{article}
\usepackage{graphicx} 
\usepackage{multirow}
\usepackage{enumitem}
\usepackage{amssymb}
\usepackage{amsmath}
\usepackage{xcolor}
\usepackage{geometry}
	\geometry{
		total = {160mm, 237mm},
		left = 25mm,
		right = 35mm,
		top = 30mm,
		bottom = 30mm,
	}
\usepackage{fancyhdr}
\renewcommand{\headrulewidth}{0pt}
\pagestyle{fancy}

\begin{document}
\fancyfoot[C]{\raisebox{.5ex}{\rule{0.5cm}{.4pt}}o0o\raisebox{.5ex}{\rule{0.5cm}{.4pt}}}

\begin{tabular}{r c l}
    \includegraphics[width=2cm]{ITS.png}
     &\begin{tabular}{lcll}
        \multicolumn{4}{c}{\begin{tabular}{c}
          \MakeUppercase{evaluasi tengah semester gasal 2023/2024}\\
          \MakeUppercase{departemen matematika fsad its}\\
          \MakeUppercase{program sarjana}\\
     \end{tabular}}\\
     \\
          Matakuliah&:&\multicolumn{2}{l}{Analisis 1}\\
          Hari, Tanggal&:&\multicolumn{2}{l}{Kamis, 19 Oktober 2023}\\
          Waktu / Sifat&:&\multicolumn{2}{l}{100 menit / \textit{Closed Book}}\\
          Kelas, Dosen&:&A.&Dr. Sunarsini, M.Si.\\
          &&B,F&Dr. mont. Kistosil Fahim, M.Si.\\
          &&C.&Dr. Mahmud Yunus, M.Si.\\
          &&D,G&Drs. Sadjidon, M.Si.\\
          &&E.&Dr. Rinurwati, M.Si.\\
     \end{tabular}
     & 
     \includegraphics[width=2cm]{M.png}
     \\ \hline
\multicolumn{3}{|l|}{\MakeUppercase{harap diperhatikan !!!}}\\
\multicolumn{3}{|l|}{Segala jenis pelanggaran (mencontek, kerjasama, dsb) yang dilakukan pada saat ETS/EAS}\\
\multicolumn{3}{|l|}{akan dikenakan sanksi pembatalan matakuliah pada semester yang sedang berjalan.}\\
\hline
\end{tabular}

\begin{enumerate}
    \item Diberikan bilangan real $a$. Tunjukkan bahwa :
    \begin{enumerate}
        \item Jika untuk setiap $\varepsilon>0$ berlaku $0\leq a<\varepsilon$ maka $a=0$.
        \item jika $a\neq0$ maka $a^2>0$.
    \end{enumerate}
    \vspace{1cm}
    
    \item Misal $S$ himpunan bagian tak-kosong dari $\mathbb{R}$. Jika $S$ terbatas diatas, tunjukkan bahwa
    \[\textrm{sup}(S)=-\textrm{inf}\{-s:s\in S\}.\]\\

    \item Diberikan himpunan $B\subset\mathbb{R}$ yang tidak kosong dan terbatas dibawah. Jika $v\in\mathbb{R}$ memenuhi sifat untuk setiap $n\in\mathbb{N}$ berlaku:
    \begin{enumerate}[label=(\roman*)]
        \item $v+\frac{1}{n}$ bukan batas bawah dari $B$;
        \item $v-\frac{1}{n}$ batas bawah dari $B$,
    \end{enumerate}
    tunjukkan bahwa inf$(B)=v$.\\~\\

    \item Dengan menggunakan definisi konvergensi barisan, tunjukkan bahwa lim$(\frac{1}{3^n})=0$.\\~\\

    \item Misal $(x_n)$ suatu barisan konvergen, dan diketahui $(y_n)$ suatu barisan dengan sifat bahwa untuk setiap $\varepsilon>0$ terdapat $M$ sehingga $|x_n-y_n|<\varepsilon$ untuk semua $n\geq M$. Apakah hal tersebut berakibat bahwa $(y_n)$ barisan yang konvergen? Berikan penjelasan untuk jawaban Anda.
\end{enumerate}

\newpage
\begin{tabular}{|lcl|}
     \hline
     Nama&:&Teosofi Hidayah Agung\\
     NRP&:&5002221132\\
     \hline
\end{tabular}
\begin{enumerate}
    \item Diketahui $a\in\mathbb{R}$, sehingga akan memenuhi sifat-sifat pada bilangan real.
    \begin{enumerate}
        \item Asumsikan $a\neq0$. Maka didapat
        \begin{flalign*}
            &0<a<\varepsilon&\\
            &a<\varepsilon\:,\: a>0
        \end{flalign*}
        Karena berlaku untuk setiap $\varepsilon>0$, dapat dipilih $\varepsilon_0=\frac{1}{2}a$ sehingga
        \begin{flalign*}
            &a<\varepsilon_0\:,\: a>0&\\
            &a<\frac{1}{2}a\:,\:a>0\quad \textrm{(Kontradiksi)}
        \end{flalign*}
        Akibatnya asumsi $a\neq0$ salah. Jadi haruslah $a=0$.
        \item Akan dibagi menjadi 2 kasus:
        \begin{enumerate}[label=(\roman*)]
            \item untuk $a>0$, maka $a\in\mathbb{P}$ dan jelas bahwa $a\cdot a=a^2\in\mathbb{P}$ (sifat keterurutan)
            \item untuk $a<0$, maka $-a\in\mathbb{P}$ sehingga $(-a)\cdot(-a)\in\mathbb{P}$. Secara Pembuktian sifat aljabar di $\mathbb{R}$, didapatkan $(-a)\cdot(-a)=a^2\in\mathbb{P}$.
        \end{enumerate}
        Jadi untuk $a\neq0$ berlaku $a^2>0$.
    \end{enumerate}
    \vspace{1cm}
    
    \item $S$ terbatas diatas berakibat $S$ mempunyai supremum, yaitu sup$(S)$. Dengan sifat supremum maka $s\leq\textrm{sup}(S)$, untuk setiap $s\in S$. Akibatnya dapat diperoleh $-\textrm{sup}(S)\leq -s,\:\forall s\in S$\dotfill(1)
    \\~\\
    Perhatikan bahwa dari (1) dapat disimpulkan $-\textrm{sup}(S)$ merupakan batas bawah dari $\{-s:s\in S\}$ yang berakibat himpunan tersebut memiliki infimum, yaitu inf$\{-s:s\in S\}$. Dengan memanfaatkan sifat infimum didapatkan $-\textrm{sup}(S)\leq\textrm{inf}\{-s:s\in S\}$ atau dapat ditulis sebagai $\textrm{sup}(S)\geq-\textrm{inf}\{-s:s\in S\}$\dotfill(2)
    \\~\\
    Ingat kembali bahwa $\textrm{inf}\{-s:s\in S\}\leq-s,\forall s\in S$ atau $s\leq-\textrm{inf}\{-s:s\in S\},\forall s\in S$. Didapatkan fakta bahwa inf$\{-s:s\in S\}$ merupakan batas atas dari $S$. Akibatnya diperoleh $\textrm{sup}(S)\leq-\textrm{inf}\{-s:s\in S\}$\dotfill(3)
    \\~\\
    $\therefore$ Dari (2) dan (3) dapat disimpulkan $\textrm{sup}(S)=-\textrm{inf}\{-s:s\in S\}$
    \vspace{1cm}
    
    \item Dengan menggunakan sifat infimum, dari (i) diperoleh inf$(B)< v+\frac{1}{n}$ dan dari (ii) diperoleh $v-\frac{1}{n}\leq \textrm{inf}(B)$. Sehingga didapatkan pertidaksamaan
    \begin{flalign*}
        v-\frac{1}{n}\leq \textrm{inf}(B)< v+\frac{1}{n}\\
        -\frac{1}{n}\leq \textrm{inf}(B)-v<\frac{1}{n}\\
        0\leq|\textrm{inf}(B)-v|<\frac{1}{n}
    \end{flalign*}
    Dengan sifat archimedes, didapatkan fakta bahwa $\frac{1}{n_\varepsilon}<\varepsilon,\:\forall\varepsilon>0$. Dengan demikian didapatkan
    \begin{flalign*}
        0\leq|\textrm{inf}(B)-v|<\frac{1}{n_\varepsilon}<\varepsilon\\
        0\leq|\textrm{inf}(B)-v|<\varepsilon
    \end{flalign*}
    Dengan menggunakan teorema pada soal 1.(a) dapat disimpulkan bahwa $\textrm{inf}(B)-v=0$ atau $\textrm{inf}(B)=v$. \textbf{(Terbukti)}
    \vspace{1cm}

    \item Menggunakan definisi lim$(\frac{1}{3^n})=0$ didapatkan untuk setiap $\varepsilon>0$, terdapat bilangan asli $K(\varepsilon)$ sehingga $\forall n\geq K(\varepsilon)$ berlaku $|\frac{1}{3^n}-0|<\varepsilon\Rightarrow|\frac{1}{3^n}|<\varepsilon\Rightarrow\frac{1}{3^n}<\varepsilon$.\\~\\
    Perhatikan bahwa dengan induksi matematika, dapat dibuktikan bahwa $3^n>n,\:\forall n\in\mathbb{N}$ atau ekuivalen dengan $\frac{1}{3^n}<\frac{1}{n}$. Dilanjutkan dengan sifat archimedes didapatkan $\frac{1}{3^{n_\varepsilon}}<\frac{1}{n_\varepsilon}<\varepsilon\Rightarrow\frac{1}{3^{n_\varepsilon}}<\varepsilon$.\\~\\
    Dengan demikian dapat disimpulkan bahwa lim$(\frac{1}{3^n})=0$ terbukti.
    \vspace{1cm}

    \item Perhatikan bahwa $0\leq|x_n-y_n|<\varepsilon$ untuk setiap $\varepsilon>0$. Dengan menggunakan teorema yang sudah dibuktikan pada soal 1.(a) didapatkan $x_n-y_n=0$ atau $x_n=y_n$. Hal ini berakibat barisan $(x_n)$ dan barisan $(y_n)$ adalah barisan yang sama, sehingga karena $(x_n)$ barisan konvergen maka $(y_n)$ haruslah barisan yang konvergen juga.
\end{enumerate}
\end{document}
