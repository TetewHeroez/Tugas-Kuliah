\documentclass[14pt,aspectratio=169]{beamer}
\usetheme{Madrid}

\usepackage{graphicx}
\usepackage{tikz}
\usepackage{enumitem}
\usepackage{amssymb}
\usepackage{amsmath}
\usepackage{xcolor}
\usepackage{color, colortbl}

\newcommand*{\defeq}{\stackrel{\text{def}}{=}}
\newcommand{\inlineitem}{%
\leavevmode\usebeamertemplate{itemize item}
}

\title{Tugas Video Aljabar 1}
\author{Teosofi Hidayah Agung\\ 5002221132}
\date{December 2023}

\begin{document}
    \maketitle
    \begin{frame}
        \begin{block}{ETS Aljabar 1 2023 no.1}
            Diberikan $a\in\mathbb{R}$ dan $n=\frac{\sqrt{3-|a-1|}+\sqrt{|a-1|-3}}{a+2}+\frac{1+2a}{a-3}$. Tentukan digit terakhir dari nilai $n^{2023}$. 
        \end{block}
        \begin{alertblock}{Ingat}<2->
            \inlineitem $\sqrt{f(x)}$ terdefinisi, jika $f(x)\geq0$\\
            \inlineitem $\frac{f(x)}{g(x)}$ terdefinisi untuk $g(x)\neq0$
        \end{alertblock}
    \end{frame}
    \begin{frame}
        \onslide<2->{Untuk $\sqrt{3-|a-1|}$, maka
        \[3-|a-1|\geq 0\Longrightarrow|a-1|\leq3.............(1)\]}
        
        \onslide<3->{Untuk $\sqrt{|a-1|-3}$, maka
        \[|a-1|-3\geq 0\Longrightarrow|a-1|\geq3.............(2)\]}

        \onslide<4->{Dari (1) dan (2) didapatkan $|a-1|=3$. Sehingga nilai $a$ yang memenuhi adalah $a=-2$ atau $a=4$.}
    \end{frame}
    \begin{frame}
        \onslide<2->{\begin{exampleblock}{Perhatikan}
            $a=-2$ tidak memenuhi sebab membuat penyebut menjadi $0$. Sehingga $a=4$ adalah solusi satu-satunya.
        \end{exampleblock}}
        \onslide<3->{Nilai $n=\frac{\sqrt{3-|4-1|}+\sqrt{|4-1|-3}}{4+2}+\frac{1+2(4)}{4-3}=9$}
        \onslide<4->{\begin{block}{Grup modulo}
            Perhatikan bahwa untuk menentukan digit terakhir suatu bilangan, dapat digunakan konsep grup $(\mathbb{Z}_{10},\cdot)$. Dimana untuk $[9]_{10}$ berorde $2$ pada grup tersebut.
        \end{block}}
        \onslide<5->{$([9]_{10})^{2023}=([9]_{10})^{2022}\cdot[9]_{10}=([9]_{10}^2)^{1011}\cdot[9]_{10}=([1]_{10})^{1011}\cdot[9]_{10}=[9]_{10}$.\\~\\}
        \onslide<6->{$\therefore$ Digit terakhir $9^{2023}$ adalah $9$.}
    \end{frame}
    \begin{frame}
        \begin{block}{Kuis 2 Aljabar 1 Kelas B 2023 no.4}
            Diberikan $f:G\to G'$ homomorpisma grup. Tunjukkan bahwa ker$(f)$ subgrup normal dari $G$.
        \end{block}
        \onslide<2->{\begin{exampleblock}{Diketahui}
            \inlineitem $f$ memenuhi $f(ab)=f(a)f(b),\forall a,b\in G$.\\
            \inlineitem ker$(f)=\{x\in G\:|\:f(x)=e_{G'}\}$
        \end{exampleblock}}
        \onslide<3->{Pertama-tama akan ditunjukkan bahwa ker$(f)<G$. Ambil sembarang $a,b\in$ ker$(f)$, maka}
        \onslide<4->{\[f(ab^{-1})=f(a)f(b^{-1})=f(a)f(b)^{-1}=e_{G'}e_{G'}^{-1}=e_{G'}\]}
        \onslide<5->{Jadi $ab^{-1}\in$ ker$(f)$ yang secara definisi adalah ker$(f)<G$.}
    \end{frame}
    \begin{frame}
        \onslide<2->{Selanjutnya akan ditunjukkan bahwa ker$(f)\triangleleft G$. Ambil sembarang $g\in G$ dan $x\in$ ker$(f)$, maka}
        \onslide<3->{\[f(gxg^{-1})=f(g)f(x)f(g^{-1})=f(g)e_{G'}f(g)^{-1}=f(g)f(g)^{-1}=e_{G'}\]}
        \onslide<4->{Jadi $gxg^{-1}\in$ ker$(f)$ yang secara definisi adalah ker$(f)\triangleleft G$.\\}
        \onslide<5->{$\therefore$ Terbukti bahwa ker$(f)$ subgrup normal dari $G$}
    \end{frame}
\end{document}