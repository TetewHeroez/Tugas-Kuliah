\documentclass{article}
\usepackage{graphicx} 
\usepackage{enumitem}
\usepackage{amssymb}
\usepackage{amsmath}
\usepackage{xcolor}
\usepackage{color, colortbl}
\newcommand*{\defeq}{\stackrel{\text{def}}{=}}
\title{Tugas Aljabar I}
\author{Teosofi Hidayah Agung\\
5002221132}
\date{}
\begin{document}
\maketitle
\pagenumbering{gobble}
\setlength{\belowdisplayskip}{-4.5mm}
\setlength{\abovedisplayskip}{1.5mm}
\allowdisplaybreaks

\begin{enumerate}
    %nomor 1
    \item Buatlah contoh grup.\\~\\
    \textbf{Jawab}:\\
    Himpunan bilangan rasional $\mathbb{Q}\setminus\{0\}$ terhadap operasi perkalian $(\boldsymbol{\cdot})$.
    \begin{enumerate}[label=(\arabic*)]
        \item Sifat \textbf{asosiatif}.\\
        jelas karena $\mathbb{Q}\setminus\{0\}\subset\mathbb{R}$ yang dimana bilangan real $\mathbb{R}$ sudah terbukti sifat asosiatifnya.
        \item Eksistensi \textbf{identitas}\\
        Terdapat identitas $e\in\mathbb{Q}\setminus\{0\}$ yaitu $1\in\mathbb{Q}\setminus\{0\}$, sedemikian sehingga untuk sembarang $a\in\mathbb{Q}\setminus\{0\}$ berlaku $1\boldsymbol{\cdot}a=a\boldsymbol{\cdot}1=a$ 
        \item Eksistensi \textbf{invers}\\
        Untuk setiap $a\in\mathbb{Q}\setminus\{0\}$ terdapat $a^{-1}\in\mathbb{Q}\setminus\{0\}$ yaitu $\frac{1}{a}\in\mathbb{Q}\setminus\{0\}$ yang saling invers, sehingga $a\boldsymbol{\cdot}\frac{1}{a}=\frac{1}{a}\boldsymbol{\cdot}a=e=1$
    \end{enumerate}
    $\therefore$ Himpunan bilangan rasional $\mathbb{Q}\setminus\{0\}$ termasuk grup terhadap operasi perkalian $(\boldsymbol{\cdot})$.
    
    %nomor 2
    \item Buatlah contoh bukan grup.\\~\\
    \textbf{Jawab}:\\
    Didefinisikan $G=\{a,b\in\mathbb{R}|a\:\heartsuit\:b=ab+a+b\}$.\\
    Mempunyai identitas yaitu $0\in G$ sebab\\
    $a\:\heartsuit\:0=0\:\heartsuit\:a=a(0)+a+0=(0)a+0+a=a$.\\
    Sekarang perhatikan untuk $-1\in G$. Asumsikan $-1\in G$ mempunyai invers yaitu $p\in G$, Sehingga nilai $b$:
    \begin{flalign*}
        -1\:\heartsuit\:p\Longleftrightarrow(-1)p+(-1)+p&=0&\\
        -p-1+p&=0&\\
        p-p-1+p&=p+0&\\
        0-1+p&=p&\\
        -1+p-p&=p-p&\\
        -1&=0\quad\textrm{\textbf{(Kontradiksi)}}&\\
    \end{flalign*}
    Karena pembuktian mendapati kontradiksi, maka kesimpulannya adalah bahwa asumsi awalnya salah. Oleh karena itu, haruslah $-1\in G$ tidak mempunyai invers.

    \vspace{0.1mm}
    $\therefore$ $G$ bukan termasuk grup
    %nomor 3
    \item Tunjukkan bahwa bila $G$ adalah grup komutatif, maka untuk semua $a, b \in G$ dan untuk semua bilangan bulat $n$, $(ab)^n = a^nb^n$.\\~\\
    \textbf{Jawab}:\\
    \[(ab)^n\defeq\underbrace{(ab)\boldsymbol{\cdot} (ab)\boldsymbol{\cdot}(ab)\boldsymbol{\cdot}...\boldsymbol{\cdot} (ab)}_n\]
    Perhatikan untuk $n=0$, diperoleh $a^0=b^0=(ab)^0=a^0b^0=e\in G$. Selanjutnya akan digunakan induksi matematika untuk membuktikan setiap $n\in \mathbb{N}$ memenuhi.\\
    Untuk setiap $a,b\in G$, misalkan $P(k):(ab)^n=a^nb^n$.Untuk $P(1)$ terbukti benar karena $(ab)^1=ab=a^1b^1$. Selanjutnya anggap benar untuk $P(k):(ab)^k=a^kb^k$ dan akan dibuktikan benar pula untuk $P(k+1).$  
    \begin{flalign*}
        (ab)^{k+1} &= (ab)^k(ab)\quad\textrm{\textbf{(Definisi)}}&\\
        &= (a^kb^k)(ab)\quad\textrm{\textbf{(Asumsi $P(k)$ benar)}}&\\
        &= (a^kb^k)(ba)\quad\textrm{\textbf{(Komutatif)}}&\\
        &= a^k(b^k(ba))\quad\textrm{\textbf{(Asosiatif)}}&\\
        &= a^k((b^kb)a)\quad\textrm{\textbf{(Asosiatif)}}&\\
        &= a^k(b^{k+1}a)\quad\textrm{\textbf{(Definisi)}}&\\
        &= a^k(ab^{k+1})\quad\textrm{\textbf{(Komutatif)}}&\\
        &= (a^ka)b^{k+1}\quad\textrm{\textbf{(Asosiatif)}}&\\
        &= a^{k+1}b^{k+1}\quad\textrm{\textbf{(Definisi)}}&\\
    \end{flalign*}
    Oleh karena itu, $P(n)$ terbukti benar untuk setiap $n\in \mathbb{N}$.
    
    \vspace{0.1mm}
    Sekarang akan dibuktikan $P(n)$ benar untuk $n$ bilangan bulat negatif. Selanjutnya ambil sembarang $a,b\in G$ yang berakibat $(ab)^{-k}$ merupakan invers dari $(ab)^{k}$, Sehingga notasi $(-k)$ dapat mempresentasikan bilangan bulat negatif untuk $k\in\mathbb{N}$.
    \begin{flalign*}
        (ab)^{-k} &= b^{-k}a^{-k}\quad\textrm{\textbf{(Invers $a^kb^k$)}}&\\
        &= a^{-k}b^{-k}\quad\textrm{\textbf{(Komutatif)}}&\\
    \end{flalign*}
    Juga $P(n)$ terbukti benar untuk setiap $n$ bilangan bulat negatif.
    
    \vspace{0.1mm}
    $\therefore (ab)^n = a^nb^n$ terbukti benar $\forall n \in\mathbb{Z}$
    
    %nomor 4
    \item Dapatkan invers masing-masing elemen dari $\mathbb{U}(10)$ dan $\mathbb{U}(15)$.\\
    \textit{Catatan}: $\mathbb{U}(n)=\{[u]_n\in\mathbb{Z}\:|\:\textrm{fpb}(u,n)=1\}\subset\mathbb{Z}_n$\\~\\
    \textbf{Jawab}:\\
    Identitas masing-masing adalah $[1]_{10}\in\mathbb{U}(10)$ dan $[1]_{15}\in\mathbb{U}(15)$.\\
    
    \vspace{0.1mm}
    $\mathbb{U}(10)=\{[1]_{10},[3]_{10},[7]_{10},[9]_{10}\}$
    \newcolumntype{l}{>{\columncolor{lime}}c}
    \begin{center}
    \begin{tabular}{|l|| c c c c|} 
        \hline
        \rowcolor{cyan}
         \color{purple}$\times$ & $[1]_{10}$ & $[3]_{10}$ & $[7]_{10}$ & $[9]_{10}$ \\
         \hline\hline
         $[1]_{10}$ & $[1]_{10}$ & $[3]_{10}$ & $[7]_{10}$ & $[9]_{10}$ \\
         $[3]_{10}$ & $[3]_{10}$ & $[9]_{10}$ & $[1]_{10}$ & $[7]_{10}$ \\
         $[7]_{10}$ & $[7]_{10}$ & $[1]_{10}$ & $[9]_{10}$ & $[3]_{10}$ \\
         $[9]_{10}$ & $[9]_{10}$ & $[7]_{10}$ & $[3]_{10}$ & $[1]_{10}$ \\
    \end{tabular}
    \end{center}
    \begin{itemize}
        \item $[1]_{10}^{-1}=[1]_{10}$
        \item $[3]_{10}^{-1}=[7]_{10}$
        \item $[7]_{10}^{-1}=[3]_{10}$
        \item $[9]_{10}^{-1}=[9]_{10}$
    \end{itemize}
    
    \vspace{0.1mm}
    $\mathbb{U}(15)=\{[1]_{15},[2]_{15},[4]_{15},[7]_{15},[8]_{15},[11]_{15},[13]_{15},[14]_{15}\}$
    \newcolumntype{l}{>{\columncolor{lime}}c}
    \begin{center}
    \begin{tabular}{|l|| c c c c c c c c|} 
        \hline
        \rowcolor{cyan}
         \color{purple}$\times$ & $[1]_{15}$ & $[2]_{15}$ & $[4]_{15}$ & $[7]_{15}$ & $[8]_{15}$ & $[11]_{15}$ & $[13]_{15}$ & $[14]_{15}$\\
         \hline\hline
         $[1]_{15}$ & $[1]_{15}$ & $[2]_{15}$ & $[4]_{15}$ & $[7]_{15}$ & $[8]_{15}$ & $[11]_{15}$ & $[13]_{15}$ & $[14]_{15}$\\
         $[2]_{15}$ & $[2]_{15}$ & $[4]_{15}$ & $[8]_{15}$ & $[14]_{15}$ & $[1]_{15}$ & $[7]_{15}$ & $[11]_{15}$ & $[13]_{15}$\\
         $[4]_{15}$ & $[4]_{15}$ & $[8]_{15}$ & $[1]_{15}$ & $[13]_{15}$ & $[2]_{15}$ & $[14]_{15}$ & $[7]_{15}$ & $[11]_{15}$\\
         $[7]_{15}$ & $[7]_{15}$ & $[14]_{15}$ & $[13]_{15}$ & $[4]_{15}$ & $[11]_{15}$ & $[2]_{15}$ & $[1]_{15}$ & $[8]_{15}$\\
         $[8]_{15}$ & $[8]_{15}$ & $[1]_{15}$ & $[2]_{15}$ & $[11]_{15}$ & $[4]_{15}$ & $[13]_{15}$ & $[14]_{15}$ & $[7]_{15}$\\
         $[11]_{15}$ & $[11]_{15}$ & $[7]_{15}$ & $[14]_{15}$ & $[2]_{15}$ & $[13]_{15}$ & $[1]_{15}$ & $[8]_{15}$ & $[4]_{15}$\\
         $[13]_{15}$ & $[13]_{15}$ & $[11]_{15}$ & $[7]_{15}$ & $[1]_{15}$ & $[14]_{15}$ & $[8]_{15}$ & $[4]_{15}$ & $[2]_{15}$\\
         $[14]_{15}$ & $[14]_{15}$ & $[13]_{15}$ & $[11]_{15}$ & $[8]_{15}$ & $[7]_{15}$ & $[4]_{15}$ & $[2]_{15}$ & $[1]_{15}$\\
         \hline
    \end{tabular}
    \end{center}
    \begin{itemize}
        \item $[1]_{15}^{-1}=[1]_{15}$
        \item $[2]_{15}^{-1}=[8]_{15}$
        \item $[4]_{15}^{-1}=[4]_{15}$
        \item $[7]_{15}^{-1}=[13]_{15}$
        \item $[8]_{15}^{-1}=[2]_{15}$
        \item $[11]_{15}^{-1}=[11]_{15}$
        \item $[13]_{15}^{-1}=[7]_{15}$
        \item $[14]_{15}^{-1}=[14]_{15}$
    \end{itemize}
    
    %nomor 5
    \item Apakah $H_{2\times 2}(\mathbb{R})=\left\{A=\begin{bmatrix}a&b\\c&d\end{bmatrix}\Big|\textrm{ det}(A)=2 \textrm{ dan }a,b,c,d\in\mathbb{R} \right\}$ merupakan grup?\\~\\
    \textbf{Jawab}:\\
    $H_{2\times 2}(\mathbb{R})$ bukan merupakan grup karena elemen identitas $I=\begin{bmatrix}1&0\\0&1\end{bmatrix}\notin H_{2\times 2}(\mathbb{R})$ dikarenakan det$(I)=1\neq2$, sehingga $H_{2\times 2}(\mathbb{R})$ tidak mempunyai elemen identitas.
    
    \vspace{0.1mm}
    $\therefore H_{2\times 2}(\mathbb{R})$ bukan merupakan grup.
\end{enumerate}
\end{document}
