\documentclass[10pt,openany,a4paper]{article}
\usepackage{graphicx} 
\usepackage{multirow}
\usepackage{enumitem}
\usepackage{amssymb}
\usepackage{amsmath}
\usepackage{xcolor}
\usepackage{physics}
\usepackage{soul}
\usepackage{hyperref}
\usepackage{multicol}
\usepackage{geometry}
	\geometry{
		total = {160mm, 237mm},
		left = 25mm,
		right = 35mm,
		top = 30mm,
		bottom = 30mm,
	}
\usepackage{fancyhdr}
\renewcommand{\headrulewidth}{0pt}
\pagestyle{fancy}

\graphicspath{{C:/Users/teoso/OneDrive/Documents/Tugas Kuliah/Template Math Depart/}}

\newcommand{\R}{\mathbb{R}}
\newcommand{\N}{\mathbb{N}}
\newcommand{\Z}{\mathbb{Z}}
\newcommand{\Q}{\mathbb{Q}}
\newcommand{\jawab}{\textbf{Solusi}:}

\begin{document}
\fancyfoot[C]{\raisebox{.5ex}{\rule{0.5cm}{.4pt}}o0o\raisebox{.5ex}{\rule{0.5cm}{.4pt}}}

\begin{tabular}{r c l}
    \includegraphics[width=2cm]{ITS.png}
     &\begin{tabular}{lcll}
        \multicolumn{4}{c}{\begin{tabular}{c}
          \MakeUppercase{evaluasi tengah semester gasal 2023/2024}\\
          \MakeUppercase{departemen matematika fsad its}\\
          \MakeUppercase{program sarjana}\\
     \end{tabular}}\\
     \\
          Matakuliah&:&\multicolumn{2}{l}{Aljabar 1}\\
          Hari, Tanggal&:&\multicolumn{2}{l}{Rabu, 18 Oktober 2023}\\
          Waktu / Sifat&:&\multicolumn{2}{l}{100 menit / \textit{Closed Book}}\\
          Kelas, Dosen&:&A.&Prof. Dr. Drs. Subiono, MS\\
          &&B.&Dian Winda S., M.Si\\
          &&C.&Soleha, M.Si\\
     \end{tabular}
     & 
     \includegraphics[width=2cm]{M.png}
     \\ \hline
\multicolumn{3}{|l|}{\MakeUppercase{harap diperhatikan !!!}}\\
\multicolumn{3}{|l|}{Segala jenis pelanggaran (mencontek, kerjasama, dsb) yang dilakukan pada saat ETS/EAS}\\
\multicolumn{3}{|l|}{akan dikenakan sanksi pembatalan matakuliah pada semester yang sedang berjalan.}\\
\hline
\end{tabular}

\begin{enumerate}
    \item Diberikan $a\in\R$ dan $\displaystyle n=\frac{\sqrt{3-|a-1|}+\sqrt{|a-1|-3}}{a+2}+\frac{1+2a}{a-3}$. Tentukan digit terakhir dari nilai $n^{2023}$.
    \item Diberikan grup siklik $\langle a\rangle$ dan $\langle b\rangle$ masing-masing mempunyai orde $8$ dan $20$. Tentukan semua generator dari masing-masing grup siklik tersebut.
    \item Tunjukkan bahwa bila $G$ adalah suatu grup yang belum tentu komutatif dan $a,b\in G$, maka $|ab|=|ba|$ dan $|aba^{-1}|=|b|$.
    \item Diberikan $S_{2\times 2}(\Z_3)=\left\{A=\left.\begin{pmatrix}a&b\\0&c\end{pmatrix}\,\right|\, \det(A)\ne 0 \text{ dan } a,b,c\in\Z_3\right\},\,(S_{2\times 2}(\Z_3),\cdot)$ grup dan $H=\left\{\left.\begin{pmatrix}1&d\\0&1\end{pmatrix}\,\right|\, d\in\Z_3\right\}$ subgrup dari $S_{2\times 2}(\Z_3)$. 
    \begin{enumerate}
        \item Tentukan banyaknya koset kanan/koset kiri dari $H$ dalam grup $S_{2\times 2}(\Z_3)$ yang berbeda.
        \item Tentukan semua koset kanan dari $H$ dalam grup $S_{2\times 2}(\Z_3)$ yang berbeda.
        \item Tentukan semua koset kiri dari $H$ dalam grup $S_{2\times 2}(\Z_3)$ yang berbeda.
        \item Dari hasil (b) dan (c), apakah $H$ subgrup normal dari $S_{2\times 2}(\Z_3)$? Jelaskan.
    \end{enumerate}
\end{enumerate}

\newpage
\jawab
\begin{enumerate}
    \item Ingat akan 2 hal berikut yang pernah dipelajari di Kalkulus:
    \begin{itemize}
        \item $\sqrt{f(x)}$ terdefinisi, jika $f(x)\geq0$.
        \item $\displaystyle\frac{f(x)}{g(x)}$ terdefinisi untuk $g(x)\neq0$.
    \end{itemize}
    Sehingga untuk $\sqrt{3-|a-1|}$ dan $\sqrt{|a-1|-3}$, didaptkankan dua pertidaksamaan
    \begin{flalign}
        3-|a-1|\geq 0&\Longrightarrow|a-1|\leq3\label{eq1}\\
        |a-1|-3\geq 0&\Longrightarrow|a-1|\geq3\label{eq2}
    \end{flalign}
    Dari \eqref{eq1} dan \eqref{eq2} diperoleh $|a-1|=3$ Sehingga nilai $a$ yang memenuhi adalah $a=-2$ atau $a=4$. Dapat di cek bahwa $a=-2$ tidak memenuhi sebab membuat penyebut menjadi $0$. Sehingga $a=4$ adalah solusi satu-satunya.
    \[n=\frac{\sqrt{3-|4-1|}+\sqrt{|4-1|-3}}{4+2}+\frac{1+2(4)}{4-3}=9\]
    Perhatikan bahwa untuk menentukan digit terakhir suatu bilangan, dapat digunakan konsep \st{grup}\footnote{Mungkin saja konsep seperti ini kurang tepat, namun setidaknya pendekatan inilah yang dapat kita hubungkan melalui materi aljabar yang sudah kita pelajari} $(\mathbb{Z}_{10},\cdot)$. Dimana untuk $[9]_{10}$ berorde $2$ pada grup tersebut.
    \[([9]_{10})^{2023}=([9]_{10})^{2022}\cdot[9]_{10}=([9]_{10}^2)^{1011}\cdot[9]_{10}=([1]_{10})^{1011}\cdot[9]_{10}=[9]_{10}\]
    $\therefore$ Digit terakhir $9^{2023}$ adalah $9$.
    \item Ingat kembali bahwa grup siklik adalah grup yang dibangun oleh hanya satu elemen. Cara termudah untuk menentukan generator dari grup siklik adalah dengan mencari elemen ordenya sama dengan orde grupnya\footnote{Bisa didapatkan dengan mencari pangkat elemen yang relatif prima dengan orde grup}. Maka generator masing-masing grup siklik adalah
    \begin{itemize}
        \item Himpunan generator dari $\langle a\rangle$ adalah $\{a^1,a^3,a^5,a^7\}$.
        \item Himpunan generator dari $\langle b\rangle$ adalah $\{b^1,b^3,b^7,b^9,b^{11},b^{13},b^{17},b^{19}\}$.
    \end{itemize}
    \item Misalkan $a,b\in G$ dengan $G$ grup. 
    \begin{itemize}
        \item Agar jawaban kita mendapat nilai sempurna, perlu dituliskan pembuktian yang lengkap.\\
        Sekarang andaikan $|ab|=n$ dan $|ba|=m$, maka
        \begin{flalign*}
            (ab)^n&=e&\\
            \underbrace{(ab)(ab)\cdots (ab)}_{n}&=e&\\
            a\underbrace{(ba)\cdots (ba)}_{n-1}b&=e&\\
            \underbrace{(ba)\cdots (ba)}_{n-1}b&=a^{-1}&\\
            \underbrace{(ba)\cdots (ba)}_{n-1}&=a^{-1}b^{-1}&\\
            \underbrace{(ba)\cdots (ba)}_{n-1}&=(ba)^{-1}&\\
            \underbrace{(ba)\cdots (ba)}_{n}&=e&\\
            (ba)^n&=e&
        \end{flalign*}
        Dari hasil di atas didapatkan $n=k_1m$ dengan $k_1=1,2,3,\ldots$.\\

        Selanjutnya dapat kita tinjau 
        \begin{flalign*}
            (ba)^m&=e&\\
            \underbrace{(ba)(ba)\cdots (ba)}_{m}&=e&\\
            b\underbrace{(ab)\cdots (ab)}_{m-1}a&=e&\\
            \underbrace{(ab)\cdots (ab)}_{m-1}a&=b^{-1}&\\
            \underbrace{(ab)\cdots (ab)}_{m-1}&=a^{-1}b^{-1}&\\
            \underbrace{(ab)\cdots (ab)}_{m-1}&=(ab)^{-1}&\\
            \underbrace{(ab)\cdots (ab)}_{m}&=e&\\
            (ab)^m&=e&
        \end{flalign*}
        Dari hasil di atas didapatkan $m=k_2n$ dengan $k_2=1,2,3,\ldots$.\\

        Alhasil kita dapatkan $n=k_1m\longrightarrow n=k_1(k_2n)\longrightarrow k_1=k_2=1$. Sehingga didapatkan kesimpulan $|ab|=|ba|$.
        \item Dengan cara yang sama, seperti yang dilakukan pada poin pertama.\\
        Andaikan $|aba^{-1}|=n$, maka
        \begin{flalign*}
            (aba^{-1})^n&=e&\\
            \underbrace{(aba^{-1})(aba^{-1})\cdots (aba^{-1})}_{n}&=e&\\
            ab(a^{-1}a)b(a^{-1}a)\cdots (a^{-1}a)ba^{-1}&=e&\\
            a\underbrace{bb\cdots b}_na^{-1}&=e&\\
            a^{-1}a\underbrace{bb\cdots b}_na^{-1}a&=a^{-1}a&\\
            b^n&=e&
        \end{flalign*}
        Dari hasil di atas didapatkan $n=k_1|b|$.\\

        Kemudian andaikan $|b|=m$, maka dengan cara yang sama didapatkan $m=k_2|aba^{-1}|$. Disini nantinya berakibat $k_1=k_2=1$.
    \end{itemize}
    $\therefore \,|aba^{-1}|=|b|$.

    \item Perhatikan bahwa $|\Z_3|=3$ dan agar $\det(A)\ne 0$, maka $a,c\ne 0$. Dengan menggunakan kaidah perkalian akan ada $2\cdot 2\cdot 3=12$ elemen dalam $S_{2\times 2}(\Z_3)$.\footnote{Lebih jelasnya bisa dilihat di web Prof. Bi \href{https://sites.google.com/view/aljabar-i/grup/pembahasan-soal-soal?authuser=0}{disini}}
    \begin{eqnarray*}
        S_{2\times 2}(\mathbb{Z}_3)&=&\left\{\left(\begin{array}{rr}
        1 & 0 \\
        0 & 1
        \end{array}\right), \left(\begin{array}{rr}
        2 & 0 \\
        0 & 1
        \end{array}\right), \left(\begin{array}{rr}
        1 & 1 \\
        0 & 1
        \end{array}\right), \left(\begin{array}{rr}
        1 & 0 \\
        0 & 2
        \end{array}\right), \left(\begin{array}{rr}
        2 & 1 \\
        0 & 1
        \end{array}\right), \left(\begin{array}{rr}
        2 & 0 \\
        0 & 2
        \end{array}\right)\right.,\\
        && \left. \left(\begin{array}{rr}
        1 & 2 \\
        0 & 1
        \end{array}\right), \left(\begin{array}{rr}
        1 & 1 \\
        0 & 2
        \end{array}\right), \left(\begin{array}{rr}
        2 & 2 \\
        0 & 1
        \end{array}\right), \left(\begin{array}{rr}
        2 & 1 \\
        0 & 2
        \end{array}\right), \left(\begin{array}{rr}
        1 & 2 \\
        0 & 2
        \end{array}\right), \left(\begin{array}{rr}
        2 & 2 \\
        0 & 2
        \end{array}\right)\right\}
        \end{eqnarray*}
    \[H = \left\{\left(\begin{array}{rr}
        1 & 0 \\
        0 & 1
        \end{array}\right), \left(\begin{array}{rr}
        1 & 1 \\
        0 & 1
        \end{array}\right),  \left(\begin{array}{rr}
        1 & 2 \\
        0 & 1
        \end{array}\right) \right\}.\]
    \begin{enumerate}
        \item Menggunakan teorema Lagrange, banyak kosetnya adalah 
        \[|S_{2\times 2}(\Z_3):H|=\frac{|S_{2\times 2}(\Z_3)|}{|H|}=\frac{12}{3}=4\]
        \item Koset kanan dari $H$ dalam $S_{2\times 2}(\Z_3)$ adalah
        \begin{itemize}
            \item Untuk $g_1=I$, maka $Hg_1=H$.
            \item Untuk $g_2=\left(\begin{array}{rr}
                2 & 0 \\
                0 & 1
                \end{array}\right)$, maka $Hg_2 =\left\{\left(\begin{array}{rr}
                    2 & 0 \\
                    0 & 1
                    \end{array}\right), \left(\begin{array}{rr}
                    2 & 1 \\
                    0 & 1
                    \end{array}\right), \left(\begin{array}{rr}
                    2 & 2 \\
                    0 & 1
                    \end{array}\right)\right\}$.
            \item Untuk $g_3=\left(\begin{array}{rr}
                1 & 0 \\
                0 & 2
                \end{array}\right)$, maka $Hg_3 = \left\{\left(\begin{array}{rr}
                    1 & 0 \\
                    0 & 2
                    \end{array}\right), \left(\begin{array}{rr}
                    1 & 2 \\
                    0 & 2
                    \end{array}\right), \left(\begin{array}{rr}
                    1 & 1 \\
                    0 & 2
                    \end{array}\right)\right\}$.
            \item Untuk $g_4=\left(\begin{array}{rr}
                2 & 0 \\
                0 & 2
                \end{array}\right)$, maka $Hg_4 =\left\{\left(\begin{array}{rr}
                    2 & 0 \\
                    0 & 2
                    \end{array}\right), \left(\begin{array}{rr}
                    2 & 2 \\
                    0 & 2
                    \end{array}\right), \left(\begin{array}{rr}
                    2 & 1 \\
                    0 & 2
                    \end{array}\right)\right\}$
        \end{itemize}
        \item Koset kiri dari $H$ dalam $S_{2\times 2}(\Z_3)$ adalah
        \begin{itemize}
            \item Untuk $g_1=I$, maka $g_1H=H$.
            \item Untuk $g_2=\left(\begin{array}{rr}
                2 & 0 \\
                0 & 1
                \end{array}\right)$, maka $g_2H=\left\{\left(\begin{array}{rr}
                2 & 0 \\
                0 & 1
                \end{array}\right), \left(\begin{array}{rr}
                2 & 2 \\
                0 & 1
                \end{array}\right), \left(\begin{array}{rr}
                2 & 1 \\
                0 & 1
                \end{array}\right)\right\}.$.
            \item Untuk $g_3=\left(\begin{array}{rr}
                1 & 0 \\
                0 & 2
                \end{array}\right)$, maka $g_3H = \left\{\left(\begin{array}{rr}
                    1 & 0 \\
                    0 & 2
                    \end{array}\right), \left(\begin{array}{rr}
                    1 & 1 \\
                    0 & 2
                    \end{array}\right), \left(\begin{array}{rr}
                    1 & 2 \\
                    0 & 2
                    \end{array}\right)\right\}.$
            \item Untuk $g_4=\left(\begin{array}{rr}
                2 & 0 \\
                0 & 2
                \end{array}\right)$, maka $g_4H = \left\{\left(\begin{array}{rr}
                    2 & 0 \\
                    0 & 2
                    \end{array}\right), \left(\begin{array}{rr}
                    2 & 1 \\
                    0 & 2
                    \end{array}\right), \left(\begin{array}{rr}
                    2 & 2 \\
                    0 & 2
                    \end{array}\right)\right\}.$
        \end{itemize}
        \item Dapat dilihat bahwa koset kiri dan kanan dari $H$ dalam $S_{2\times 2}(\Z_3)$ sama. Sehingga $H$ merupakan subgrup normal dari $S_{2\times 2}(\Z_3)$.
    \end{enumerate}

    \begin{flushright}
        -\textit{Teosofi Hidayah Agung}
    \end{flushright}
\end{enumerate}
\end{document}