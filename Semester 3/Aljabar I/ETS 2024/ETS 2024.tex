\documentclass[10pt,openany,a4paper]{article}
\usepackage{graphicx} 
\usepackage{multirow}
\usepackage{enumitem}
\usepackage{amssymb}
\usepackage{amsmath}
\usepackage{xcolor}
\usepackage{physics}
\usepackage{soul}
\usepackage{hyperref}
\usepackage{multicol}
\usepackage{geometry}
	\geometry{
		total = {160mm, 237mm},
		left = 25mm,
		right = 35mm,
		top = 30mm,
		bottom = 30mm,
	}
\usepackage{fancyhdr}
\usepackage{animate}
\renewcommand{\headrulewidth}{0pt}
\pagestyle{fancy}

\graphicspath{{C:/Users/teoso/OneDrive/Documents/Tugas Kuliah/Template Math Depart/}}

\newcommand{\R}{\mathbb{R}}
\newcommand{\N}{\mathbb{N}}
\newcommand{\Z}{\mathbb{Z}}
\newcommand{\Q}{\mathbb{Q}}
\newcommand{\jawab}{\textbf{Solusi}:}

\begin{document}
\fancyfoot[C]{\raisebox{.5ex}{\rule{0.5cm}{.4pt}}o0o\raisebox{.5ex}{\rule{0.5cm}{.4pt}}}

\begin{tabular}{r c l}
    \includegraphics[width=2cm]{ITS.png}
     &\begin{tabular}{lcll}
        \multicolumn{4}{c}{\begin{tabular}{c}
          \MakeUppercase{evaluasi tengah semester gasal 2024/2025}\\
          \MakeUppercase{departemen matematika fsad its}\\
          \MakeUppercase{program sarjana}\\
     \end{tabular}}\\
     \\
          Matakuliah&:&\multicolumn{2}{l}{Aljabar 1}\\
          Hari, Tanggal&:&\multicolumn{2}{l}{Senin, 14 Oktober 2024}\\
          Waktu / Sifat&:&\multicolumn{2}{l}{100 menit / \textit{Closed Book}}\\
          Kelas, Dosen&:&A.&Prof. Dr. Drs. Subiono, MS\\
          &&B.&Soleha, M.Si\\
          &&C.&Drs. Komar Baihaqi, M.Si\\
          &&D.&Dr. Dian Winda S., M.Si\\
     \end{tabular}
     & 
     \includegraphics[width=2cm]{M.png}
     \\ \hline
\multicolumn{3}{|l|}{\MakeUppercase{harap diperhatikan !!!}}\\
\multicolumn{3}{|l|}{Segala bentuk kecurangan dalam ujian akan berakibat kegagalan kuliah dalam 1 semester}\\
\hline
\end{tabular}

\begin{enumerate}
    \item Tentukan semua subgrup dari grup $(\Z_{15}, +)$. Jelaskan.
    \item Diberikan grup $G = \{e, a, b, \}$ dimana $e\ne a\ne b$, $e$ elemen identitas dari $G$ dan $ab=ba=e$.
    \begin{enumerate}
        \item Buatlah tabel operasi grup $G$.
        \item Tunjukkan bahwa $G$ grup siklik.
    \end{enumerate}
    \item Diberikan $GL(2,\R)=\left\{A=\left.\begin{pmatrix} a & b \\ c & d \end{pmatrix} \right| \det(A)\ne 0 \text{ dan } a,b,c,d\in\R\right\}$ dan diberikan himpunan $H=\left\{\left.\begin{pmatrix} r & 0 \\ 0 & r \end{pmatrix} \right| r\in\R, r\ne 0\right\}$.
    \begin{enumerate}
        \item Tunjukkan bahwa $H$ adalah subgrup dari $GL(2,\R)$.
        \item Apakah $H$ subgrup normal dari $GL(2,\R)$? Jelaskan.
    \end{enumerate}
    \item Diberikan $(\Z_{18},+)$ grup, $H=\langle [2]_{18}\rangle$ dan $K=\langle [6]_{18}\rangle$ subgrup normal dari $\Z_{18}$.
    \begin{enumerate}
        \item Tentukan semua anggota grup faktor $\Z_{18}/K$.
        \item Tentukan semua anggota grup faktor $H/K$.
    \end{enumerate}
\end{enumerate}

\newpage
Kita Ikuyo!!!
\begin{figure}[ht]
    \centering
    \animategraphics[autoplay,loop,width=0.5\textwidth]{30}{gif/Kita Ikuyo Doodle-}{0}{85}
    \caption{Doodle}
\end{figure}
\end{document}