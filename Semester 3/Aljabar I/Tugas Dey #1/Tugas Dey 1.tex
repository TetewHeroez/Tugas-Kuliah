\documentclass{article}
\usepackage{graphicx} 
\usepackage{enumitem}
\usepackage{amssymb}
\usepackage{amsmath}

\begin{document}

\pagenumbering{gobble}
Definisi Relasi Ekivalensi:\\

Suatu relasi $\sim$ pada himpunan $A$ disebut relasi ekivalensi jika memenuhi tiga syarat berikut:
\begin{itemize}
  \item $a\sim a$ (refleksif)
  \item $a\sim b \implies b\sim a$ (simetris)
  \item $a\sim b \land b\sim c \implies a\sim c$ (transitif)
\end{itemize}
\begin{enumerate}
  \item Dalam $\mathbb{R}$, relasi $a \sim b$ bila dan hanya bila $|a| = |b|$.\\
        \textbf{Jawab:}\\
        Definisikan sebuah relasi
        \begin{align*}
          R:=\{(a,-a) \text{ dengan } a\geq 0\}
        \end{align*}
        Selanjutnya perhatikan bahwa $\{a,-a\}\times\{a,-a\} = \{(a,a),(-a,-a),(a,-a),(-a,a)\}$\\
        Jelas bahwa pasangan terurut ini memenuhi syarat relasi ekivalensi.\\
        (Buktikan lebih lanjut)\\
        Jadi kelas ekivalensinya adalah $[a]:=\{a,-a\}$.
  \item[4.] Dalam $\mathbb{R}$, relasi $a \sim b$ bila dan hanya bila $|a-b| \leq 1$.\\
        \textbf{Jawab:}\\
        Misalkan $a=0$, $b=1$ dan $c=2$.\\
        Maka $|a-b| = |0-1| = 1 \leq 1$ sehingga $a\sim b$.\\
        Dan $|b-c| = |1-2| = 1 \leq 1$ sehingga $b\sim c$.\\
        Namun $|a-c| = |0-2| = 2 \not\leq 1$ sehingga $a \not\sim c$.\\
        Jadi relasi ini tidak memenuhi syarat transitif, sehingga bukan relasi ekivalensi.
\end{enumerate}

\end{document}