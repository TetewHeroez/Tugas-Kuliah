\documentclass[10pt,openany,a4paper]{article}
\usepackage{graphicx} 
\usepackage{multirow}
\usepackage{enumitem}
\usepackage{amssymb}
\usepackage{amsmath}
\usepackage{xcolor}
\usepackage{physics}
\usepackage{soul}
\usepackage{hyperref}
\usepackage{multicol}
\usepackage{geometry}
	\geometry{
		total = {160mm, 237mm},
		left = 25mm,
		right = 35mm,
		top = 30mm,
		bottom = 30mm,
	}
\usepackage{fancyhdr}
\renewcommand{\headrulewidth}{0pt}
\pagestyle{fancy}

\graphicspath{{C:/Users/teoso/OneDrive/Documents/Tugas Kuliah/Template Math Depart/}}

\newcommand{\R}{\mathbb{R}}
\newcommand{\N}{\mathbb{N}}
\newcommand{\Z}{\mathbb{Z}}
\newcommand{\Q}{\mathbb{Q}}
\newcommand{\U}{\mathbb{U}}
\newcommand{\jawab}{\textbf{Solusi}:}

\begin{document}
\fancyfoot[C]{\raisebox{.5ex}{\rule{0.5cm}{.4pt}}o0o\raisebox{.5ex}{\rule{0.5cm}{.4pt}}}

\begin{tabular}{r c l}
    \includegraphics[width=2cm]{ITS.png}
     &\begin{tabular}{lcll}
        \multicolumn{4}{c}{\begin{tabular}{c}
          \MakeUppercase{evaluasi akhir semester gasal 2024/2025}\\
          \MakeUppercase{departemen matematika fsad its}\\
          \MakeUppercase{program sarjana}\\
     \end{tabular}}\\
     \\
          Matakuliah&:&\multicolumn{2}{l}{Aljabar 1}\\
          Hari, Tanggal&:&\multicolumn{2}{l}{Jum'at, 13 Desember 2024}\\
          Waktu / Sifat&:&\multicolumn{2}{l}{100 menit / TERTUTUP}\\
          Kelas, Dosen&:&\multicolumn{2}{l}{Prof. Dr. Drs. Subiono, M.S,}\\
          &&\multicolumn{2}{l}{Drs. Komar Baihaqi, M.Si,}\\
          &&\multicolumn{2}{l}{Dr. Dian Winda S., M.Si,}\\
          &&\multicolumn{2}{l}{Soleha, M.Si,}\\
     \end{tabular}
     & 
     \includegraphics[width=2cm]{M.png}
     \\ \hline
\multicolumn{3}{|l|}{\MakeUppercase{harap diperhatikan !!!}}\\
\multicolumn{3}{|l|}{Segala jenis pelanggaran (mencontek, kerjasama, dsb) yang dilakukan pada saat ETS/EAS}\\
\multicolumn{3}{|l|}{akan dikenakan sanksi pembatalan matakuliah pada semester yang sedang berjalan.}\\
\hline
\end{tabular}\\

\indent\textbf{Kerjakan soal-soal di bawah ini!}

\begin{enumerate}
    \item Misalkan $\theta : G \to H$ homomorfisma grup. Tunjukkan bahwa $\theta(G)$ abelian jika dan hanya jika $xyx^{-1}y^{-1} \in \ker(\theta), \quad \forall \, x, y \in G.$

    \item Diberikan grup $\mathbb{Z}_8 \times \U(8)$. Didefinisikan $([a]_8, [b]_8) * ([c]_8, [d]_8) = ([a + c]_8, [b \cdot d]_8)$, Untuk setiap $([a]_8, [b]_8), ([c]_8, [d]_8) \in \mathbb{Z}_8 \times \U(8)$:
    \begin{enumerate}
        \item Tentukan $([3]_8, [3]_8) * ([5]_8, [5]_8)$.
        \item Tentukan order dari $([3]_8, [3]_8)$.
        \item Tentukan invers dari $([5]_8, [5]_8)$.
    \end{enumerate}

    \item Diberikan $(\mathbb{Z}_2, +), (S_3, \circ)$, dan $\mathbb{Z}_2 \times S_3$ masing-masing merupakan grup. Untuk setiap 
    $([a]_2, b),$ $([c]_2, d) \in \mathbb{Z}_2 \times S_3$ didefinisikan $([a]_2, b) * ([c]_2, d) = ([a + c]_2, b \circ d)$. Jika $\phi : \mathbb{Z}_2 \times S_3 \to \mathbb{Z}_2$ didefinisikan $\phi(([x]_2, y)) = [x]_2$ untuk setiap $([x]_2, y) \in \mathbb{Z}_2 \times S_3$, maka:
    \begin{enumerate}
        \item Tunjukkan $\phi$ homomorfisma grup.
        \item Tentukan $\ker(\phi)$ dan $\operatorname{Im}(\phi)$.
        \item Tunjukkan $\mathbb{Z}_2 \times S_3 / (\{0\} \times S_3) \cong \mathbb{Z}_2$.
    \end{enumerate}

    \item Diberikan $G = \{1, -1, i, -i\}$.
    \begin{enumerate}
        \item Tentukan generator dari $G$.
        \item Tentukan $\operatorname{Inn}(G)$.
        \item Tentukan $\operatorname{Aut}(G)$.
    \end{enumerate}
\end{enumerate}

\end{document}