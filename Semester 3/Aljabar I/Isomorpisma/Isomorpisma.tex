\documentclass{article}
\usepackage{graphicx}
\usepackage{tikz}
\usepackage{enumitem}
\usepackage{amssymb}
\usepackage{amsmath}
\usepackage{xcolor}
\usepackage{color, colortbl}
\newcommand*{\defeq}{\stackrel{\text{def}}{=}}

\begin{document}
\pagenumbering{gobble}
\setlength{\belowdisplayskip}{-4.5mm}
\setlength{\abovedisplayskip}{1.5mm}
\allowdisplaybreaks
\setlength\parindent{0pt}

    \begin{tabular}{|lcl|}
     \hline
     Nama&:&Teosofi Hidayah Agung\\
     NRP&:&5002221132\\
     \hline
    \end{tabular}

    \begin{enumerate}
        \item Buatlah grup permutasi yang isomorphisma dengan grup $\mathbb{Z}_8$.
        \textbf{Jawab}:
        \begin{flalign*}
            \phi_0&=\begin{pmatrix}0&1&2&3&4&5&6&7\\0&1&2&3&4&5&6&7\end{pmatrix}=\begin{pmatrix}1\end{pmatrix}&\\
            \phi_1&=\begin{pmatrix}0&1&2&3&4&5&6&7\\1&2&3&4&5&6&7&0\end{pmatrix}=\begin{pmatrix}0&1&2&3&4&5&6&7\end{pmatrix}&\\
            \phi_2&=\begin{pmatrix}0&1&2&3&4&5&6&7\\2&3&4&5&6&7&0&1\end{pmatrix}=\begin{pmatrix}0&2&4&6\end{pmatrix}\begin{pmatrix}1&3&5&7\end{pmatrix}&\\
            \phi_3&=\begin{pmatrix}0&1&2&3&4&5&6&7\\3&4&5&6&7&0&1&2\end{pmatrix}=\begin{pmatrix}0&3&6&1&4&7&2&5\end{pmatrix}&\\
            \phi_4&=\begin{pmatrix}0&1&2&3&4&5&6&7\\4&5&6&7&0&1&2&3\end{pmatrix}=\begin{pmatrix}0&4\end{pmatrix}\begin{pmatrix}1&5\end{pmatrix}\begin{pmatrix}2&6\end{pmatrix}\begin{pmatrix}3&7\end{pmatrix}&\\
            \phi_5&=\begin{pmatrix}0&1&2&3&4&5&6&7\\5&6&7&0&1&2&3&4\end{pmatrix}=\begin{pmatrix}0&5&2&7&4&1&6&3\end{pmatrix}&\\
            \phi_6&=\begin{pmatrix}0&1&2&3&4&5&6&7\\6&7&0&1&2&3&4&5\end{pmatrix}=\begin{pmatrix}0&6&4&2\end{pmatrix}\begin{pmatrix}1&7&5&3\end{pmatrix}&\\
            \phi_7&=\begin{pmatrix}0&1&2&3&4&5&6&7\\7&0&1&2&3&4&5&6\end{pmatrix}=\begin{pmatrix}7&6&5&4&3&2&1&0\end{pmatrix}&\\
            \therefore \overline{\mathbb{Z}_8}&=\{\phi_0,\phi_1,\phi_2,\phi_3,\phi_4,\phi_5,\phi_6,\phi_7\}&\\
        \end{flalign*}
        
        \item Buatlah grup permutasi yang isomorphisma dengan grup $\mathbb{U}(8)$.\\
        \textbf{Jawab}:
        \begin{flalign*}
            \phi_1&=\begin{pmatrix}1&3&5&7\\1&3&5&7\end{pmatrix}=\begin{pmatrix}1\end{pmatrix}&\\
            \phi_3&=\begin{pmatrix}1&3&5&7\\3&1&7&5\end{pmatrix}=\begin{pmatrix}1&3\end{pmatrix}\begin{pmatrix}5&7\end{pmatrix}&\\
            \phi_5&=\begin{pmatrix}1&3&5&7\\5&7&1&3\end{pmatrix}=\begin{pmatrix}1&5\end{pmatrix}\begin{pmatrix}3&7\end{pmatrix}&\\
            \phi_7&=\begin{pmatrix}1&3&5&7\\7&5&3&1\end{pmatrix}=\begin{pmatrix}1&7\end{pmatrix}\begin{pmatrix}3&5\end{pmatrix}&\\
            \therefore \overline{\mathbb{U}(8)}&=\{\phi_1,\phi_3,\phi_5,\phi_7\}&\\
        \end{flalign*}
        
        \item Buatlah grup permutasi yang isomorphisma dengan grup $G={A,B,C,D}$,dimana
        \[A=\begin{bmatrix}1&0\\0&1\end{bmatrix},
        B=\begin{bmatrix}1&0\\0&-1\end{bmatrix},
        C=\begin{bmatrix}-1&0\\0&1\end{bmatrix},
        D=\begin{bmatrix}-1&0\\0&-1\end{bmatrix}\]\\~\\
        Catatan: Buatlah domain untuk permutasinya $A,B,C,D$.\\
        \textbf{Jawab}:
        \begin{flalign*}
            \phi_A&=\begin{pmatrix}
                \begin{bmatrix}1&0\\0&1\end{bmatrix}&\begin{bmatrix}1&0\\0&-1\end{bmatrix}&\begin{bmatrix}-1&0\\0&1\end{bmatrix}&\begin{bmatrix}-1&0\\0&-1\end{bmatrix}\\\\
                \begin{bmatrix}1&0\\0&1\end{bmatrix}&\begin{bmatrix}1&0\\0&-1\end{bmatrix}&\begin{bmatrix}-1&0\\0&1\end{bmatrix}&\begin{bmatrix}-1&0\\0&-1\end{bmatrix}
            \end{pmatrix}=\begin{pmatrix}A\end{pmatrix}&\\
            \\
            \phi_B&=\begin{pmatrix}
                \begin{bmatrix}1&0\\0&1\end{bmatrix}&\begin{bmatrix}1&0\\0&-1\end{bmatrix}&\begin{bmatrix}-1&0\\0&1\end{bmatrix}&\begin{bmatrix}-1&0\\0&-1\end{bmatrix}\\\\
                \begin{bmatrix}1&0\\0&-1\end{bmatrix}&\begin{bmatrix}1&0\\0&1\end{bmatrix}&\begin{bmatrix}-1&0\\0&-1\end{bmatrix}&\begin{bmatrix}-1&0\\0&1\end{bmatrix}
            \end{pmatrix}=\begin{pmatrix}A&B\end{pmatrix}\begin{pmatrix}C&D\end{pmatrix}&\\
            \\
            \phi_C&=\begin{pmatrix}
                \begin{bmatrix}1&0\\0&1\end{bmatrix}&\begin{bmatrix}1&0\\0&-1\end{bmatrix}&\begin{bmatrix}-1&0\\0&1\end{bmatrix}&\begin{bmatrix}-1&0\\0&-1\end{bmatrix}\\\\
                \begin{bmatrix}-1&0\\0&1\end{bmatrix}&\begin{bmatrix}-1&0\\0&-1\end{bmatrix}&\begin{bmatrix}1&0\\0&1\end{bmatrix}&\begin{bmatrix}1&0\\0&-1\end{bmatrix}
            \end{pmatrix}=\begin{pmatrix}A&C\end{pmatrix}\begin{pmatrix}B&D\end{pmatrix}&\\
            \\
            \phi_D&=\begin{pmatrix}
                \begin{bmatrix}1&0\\0&1\end{bmatrix}&\begin{bmatrix}1&0\\0&-1\end{bmatrix}&\begin{bmatrix}-1&0\\0&1\end{bmatrix}&\begin{bmatrix}-1&0\\0&-1\end{bmatrix}\\\\
                \begin{bmatrix}-1&0\\0&-1\end{bmatrix}&\begin{bmatrix}-1&0\\0&1\end{bmatrix}&\begin{bmatrix}1&0\\0&-1\end{bmatrix}&\begin{bmatrix}1&0\\0&1\end{bmatrix}
            \end{pmatrix}=\begin{pmatrix}A&D\end{pmatrix}\begin{pmatrix}B&C\end{pmatrix}&\\
            \therefore \overline{G}&=\{\phi_A,\phi_B,\phi_C,\phi_D\}&\\
        \end{flalign*}
        
        \item Misalkan $\phi:\mathbb{Z}_{12}\rightarrow\mathbb{Z}_{3}$ adalah suatu homomorpisma dengan
        \[\textrm{ker}(\phi)=\left\{[0]_{12},[3]_{12},[6]_{12},[9]_{12}\right\}\]\\
        dan $\phi([4]_{12})=[2]_{3}$. Dapatkan semua $x\in\mathbb{Z}_{12}$ yang memenuhi $\phi(x)=[1]_{3}$, dan tunjukkan bahwa himpunan $\{x\in\mathbb{Z}_{12}\:|\:\phi(x)=[1]_3\}$ suatu koset dari ker$(\phi)$ dalam $\mathbb{Z}_{12}$.\\
        \textbf{Jawab}:\\
        Diketahui bahwa $\phi([4]_{12})=[2]_{3}$ dan ker$(\phi)=\{[0]_{12},[3]_{12},[6]_{12},[9]_{12}\}$. Dengan sifat homomorpisma grup didapatkan
        \begin{flalign*}
            \bullet&\:\phi([7]_{12})=\phi([4]_{12}+[3]_{12})=\phi([4]_{12})+\phi([3]_{12})=[2]_{3}+[0]_{3}=[2]_{3}&\\
            \bullet&\:\phi([10]_{12})=\phi([4]_{12}+[6]_{12})=\phi([4]_{12})+\phi([6]_{12})=[2]_{3}+[0]_{3}=[2]_{3}&\\
            \bullet&\:\phi([1]_{12})=\phi([4]_{12}+[9]_{12})=\phi([4]_{12})+\phi([9]_{12})=[2]_{3}+[0]_{3}=[2]_{3}&\\
        \end{flalign*}\\
        Sehingga kita bisa dapatkan himpunan $\{x\in\mathbb{Z}_{12}\:|\:\phi(x)=[1]_3\}$
        \begin{flalign*}
            \bullet&\:\phi([2]_{12})=\phi([1]_{12}+[1]_{12})=\phi([1]_{12})+\phi([1]_{12})=[2]_{3}+[2]_{3}=[1]_{3}&\\
            \bullet&\:\phi([5]_{12})=\phi([4]_{12}+[1]_{12})=\phi([4]_{12})+\phi([1]_{12})=[2]_{3}+[2]_{3}=[1]_{3}&\\
            \bullet&\:\phi([8]_{12})=\phi([7]_{12}+[1]_{12})=\phi([7]_{12})+\phi([1]_{12})=[2]_{3}+[2]_{3}=[1]_{3}&\\
            \bullet&\:\phi([11]_{12})=\phi([10]_{12}+[1]_{12})=\phi([10]_{12})+\phi([1]_{12})=[2]_{3}+[2]_{3}=[1]_{3}&\\
        \end{flalign*}\\
        Perhatikan bahwa hal diatas tidak lain adalah salah satu koset dari ker$(\phi)$ dalam $\mathbb{Z}_{12}$, yaitu ker$(\phi)_{[2]_{12}}$. \textbf{(Terbukti)}\\
        
        \item Diberikan suatu grup $G,a\in G$ dan $\phi:\mathbb{Z}\rightarrow G$ adalah homomorpisma yang diberikan oleh $\phi(n)=a^n,\forall n \in\mathbb{Z}$. Uraikan semua ker$(\phi)$ yang mungkin.\\
        \textbf{Jawab}:
        \begin{flalign*}
            \textrm{ker}(\phi)&=\{n\in\mathbb{Z}\:|\:\phi(n)=e_G\}&\\
            &=\{n\in\mathbb{Z}\:|\:a^n=e_G\}&\\
            &=\{n\in\mathbb{Z}\:|\:(a^m)^k=e_G \textrm{ dengan } |a|=m\}&\\
            &=\{k|a|\:|\:k\in\mathbb{Z}\textrm{ dan }a\in G\}&\\
        \end{flalign*}
        
        \item Diberikan $f:\mathbb{U}(10)\rightarrow\mathbb{Z}_4$ homomorphisma grup dimana dimana $f([3]_{10})=[3]_4$.
        \begin{enumerate}
            \item Tentukan  $f([x]_{10})$ untuk setiap $[x]_{10}\in\mathbb{U}(10)$.\\
            \textbf{Jawab}:
            \begin{flalign*}
                f([1]_{10})&=[0]_4\quad\textit{Sifat Homgrup}&\\
                f([3]_{10})&=[3]_4&\\
                f([7]_{10})&=f([3\cdot3\cdot3]_{10})=[3]_4+[3]_4+[3]_4=[1]_4&\\
                f([9]_{10})&=f([3\cdot3]_{10})=[3]_4+[3]_4=[2]_4&\\
            \end{flalign*}
            
            \item Tentukan ker$(f)$.\\
            \textbf{Jawab}:\\
            Dapat dilihat pada nomor 6a bahwa hanya ada satu anggota ker$(f)$ yaitu $\{[1]_{10}\}$.\\
            
            \item Apakah $f$ isomorphisma?\\
            \textbf{Jawab}:\\
            $f$ akan isomorphisma jika dan hanya jika $f$ homomorphisma yang pemetaannya bijektif.\\~\\
            Adit. bahwa $f$ injektif dan surjektif.\\
            $f$ bersifat injektif karena
            \begin{flalign*}
                \bullet\:&[0]_4=[0]_4\Longrightarrow[1]_{10}=[1]_{10}&\\
                \bullet\:&[1]_4=[1]_4\Longrightarrow[7]_{10}=[7]_{10}&\\
                \bullet\:&[2]_4=[2]_4\Longrightarrow[9]_{10}=[9]_{10}&\\
                \bullet\:&[3]_4=[1]_4\Longrightarrow[3]_{10}=[3]_{10}&\\
            \end{flalign*}
            $f$ bersifat surjektif karena
            \begin{flalign*}
                \bullet\:&[0]_4=f([1]_{10})&\\
                \bullet\:&[1]_4=f([7]_{10})&\\
                \bullet\:&[2]_4=f([9]_{10})&\\
                \bullet\:&[3]_4=f([3]_{10})&\\
            \end{flalign*}
            $\therefore$ $f$ merupakan isomorphisma.
        \end{enumerate}
    \end{enumerate}
    

\end{document}