\documentclass[10pt,openany,a4paper]{article}
\usepackage{graphicx} 
\usepackage{multirow}
\usepackage{enumitem}
\usepackage{amssymb}
\usepackage{amsmath}
\usepackage{amsthm}
\usepackage{xcolor}
\usepackage{fancyhdr}
\usepackage{geometry}
	\geometry{
		total = {160mm, 237mm},
		left = 25mm,
		right = 25mm,
		top = 35mm,
        bottom = 30mm,
        headheight=2cm
	}
  \usepackage{hyperref}
\hypersetup{
    colorlinks=true,
    linkcolor=blue,
    filecolor=magenta,      
    urlcolor=cyan,
    pdftitle={Overleaf Example},
    pdfpagemode=FullScreen,
    }
\renewcommand{\headrulewidth}{0pt}

\graphicspath{{C:/Users/teoso/OneDrive/Documents/Tugas Kuliah/Template Math Depart/}}

\newcommand{\R}{\mathbb{R}}
\newcommand{\N}{\mathbb{N}}
\newcommand{\Z}{\mathbb{Z}}
\newcommand{\Q}{\mathbb{Q}}
\newcommand{\jawab}{\textbf{Solusi}:}

\newtheorem*{teorema}{Teorema}
\newtheorem*{definisi}{Definisi}

\pagestyle{fancy}
\fancyhf{}
\fancyhead[L]{\includegraphics[width=2cm]{ITS.png}}
\fancyhead[C]{\textbf{\MakeUppercase{Evaluasi Akhir Semester Gasal 2024/2025}\\~\\
                \MakeUppercase{Departemen Matematika}\\ 
                \MakeUppercase{Fakultas Sains dan Analitika Data}\\
                \MakeUppercase{Institut Teknologi Sepuluh Nopember}\\
                \MakeUppercase{Program Sarjana}}}
\fancyhead[R]{\includegraphics[width=2cm]{M.png}}
\begin{document}
\hrule
\hrule
\begin{table}[h!]
  \centering
  \underline{\MakeUppercase{Evaluasi Akhir Semester Gasal 2024/2025}}
  \begin{tabular}{l l l}
    Mata kuliah  & : & Persamaan Diferensial Biasa           \\
    Hari/tanggal & : & Rabu, 11 Desember 2024                \\
    Semester     & : & III                                   \\
    Waktu        & : & 100 Menit (09.00 - 10.40 WIB)         \\
    Sifat        & : & Opened Note (1 lembar folio)          \\
    Penguji      & : & Drs. I Gusti Ngurah Rai Usadha, M.Si. \\
                 &   & Dra. Nur Asiyah, M.Si.                \\
                 &   & Dr. Tahiyatul Asfihani, S.Si, M.Si.   \\
                 &   & Amirul Hakam, S.Si., M.Si.
  \end{tabular}
\end{table}
\hrule
\hrule
\fbox{%
  \begin{minipage}{0.95\textwidth}
    \textbf{HARAP DIPERHATIKAN !!!}

    Segala jenis pelanggaran (\textit{mencontek, kerjasama, dsb}) yang dilakukan saat EAS akan dikenakan sanksi pembatalan semua mata kuliah yang sedang berjalan.
  \end{minipage}
}
\begin{enumerate}
  \item Dapatkan penyelesaian masalah nilai awal berikut ini dengan menggunakan transformasi Laplace:
        \[
          y'' + 2y' + y = 4e^{-t}, \quad y(0) = 2, \quad y'(0) = -2
        \]

  \item Diberikan Persamaan Diferensial Biasa:
        $
          y''' - 5y'' + 8y' - 4y = 0, \quad \text{dimana } \; y' = \frac{dy}{dt}.
        $
        \begin{enumerate}
          \item Nyatakan PD Biasa tersebut dalam bentuk sistem PD
                \(\dot{X} = A X\), dan dapatkan nilai Eigen dari matriks \(A\).
          \item Berdasarkan poin (a) tersebut di atas, dapatkan penyelesaian sistem
                \(\dot{X} = A X\) dan matriks Fundamentelnya.
        \end{enumerate}

  \item Diberikan sistem persamaan diferensial biasa linier non-homogen (SPDBL-NH) sebagai berikut:
        \[
          x' =
          \begin{bmatrix}
            -2 & 1  \\
            1  & -2
          \end{bmatrix}x
          +
          \begin{bmatrix}
            2e^{-t} \\
            3t
          \end{bmatrix}
        \]
        \begin{enumerate}
          \item Hitung nilai dan vektor eigennya serta penyelesaian homogen-nya.
          \item Dapatkan penyelesaian partikularnya.
          \item Tentukan penyelesaian umum dari SPDBL-NH ini.
        \end{enumerate}

  \item Diberikan sistem autonomous:
        \[
          \begin{cases}
            \dfrac{dx}{dt} = -x + 2y, \\[6pt]
            \dfrac{dy}{dt} = 2x - y
          \end{cases}
        \]
        \begin{enumerate}
          \item Tentukan titik setimbang dan jenisnya.\\
          \item Tentukan kestabilan sistem tersebut.\\
          \item Gambarkan portrait phasanya.
        \end{enumerate}
\end{enumerate}

\end{document}
