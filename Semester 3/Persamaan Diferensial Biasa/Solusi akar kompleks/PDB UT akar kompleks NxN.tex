\documentclass[a4paper]{article}
\usepackage{amsmath}
\usepackage{amssymb}
\usepackage{amsthm}
\usepackage{graphicx}
\usepackage{hyperref}
  \hypersetup{
    colorlinks=true,
    linkcolor=black,
  }
\usepackage{enumitem}
\usepackage{geometry}
\usepackage{enumitem}
\usepackage{fancyhdr}
\usepackage{bm}

\newtheoremstyle{definisi}
{}{}% 
{\normalfont}{}% 
{\bfseries}{}% 
{\newline}% 
{\underline{\thmname{#1}\thmnumber{ #2}\thmnote{ #3}.}}

\theoremstyle{definisi}
\newtheorem{contoh}{Contoh}[subsection]
\newtheorem{definisi}{Definisi}[subsection]
\newtheorem{teorema}{Teorema}[subsection]
\newtheorem{algoritma}{Algoritma}[subsection]
\newtheorem{catatan}{Catatan}[subsection]

\newcommand{\bfxi}{\boldsymbol{\xi}}
\newcommand{\solusi}{\textbf{\underline{Penyelesaian:}}\\}

\numberwithin{equation}{section}
\setcounter{section}{12}
\setcounter{subsection}{2}

\newcommand{\R}{\mathbb{R}}

\begin{document}
  \begin{algoritma}\label{algoritma}
  Setelah mengetahui solusi polinomial karakteristiknya terdapat akar kompleks, maka langkah selanjutnya adalah mencari solusi umum dari PD linear homogen dimensi-$n$. Berikut adalah langkah-langkahnya:
  \begin{enumerate}[label=Langkah \arabic*: ,leftmargin=*]
    \item Misalkan $\bfxi_{m+1}$ dan $\bfxi_{m+2}$ masing-masing adalah vektor eigen yang saling konjugat serta berkoresponden dengan pasangan konjugat nilai eigen $r_{m+1} = \alpha + i\beta$ dan $r_{m+2} = \alpha - i\beta$.
    \item Ambil salah satu dari kedua vektor eigen kemudian pisahkan bagian real dan imajiner. Misalkan $\bfxi_{m+1} = \mathbf{u} + i\mathbf{v}$.
    \item Selanjutnya kita bisa dapatkan dua solusi real yaitu
    \begin{equation}\label{eq:solusi_vektor_eigen}
      \begin{split}
        \mathbf{x}_{m+1}(t) &= e^{\alpha t}(\mathbf{u}\cos(\beta t) - \mathbf{v}\sin(\beta t))\\
        \mathbf{x}_{m+2}(t) &= e^{\alpha t}(\mathbf{u}\sin(\beta t) + \mathbf{v}\cos(\beta t))
      \end{split}
    \end{equation}
    Dimana sudah merupakan solusi dari pasangan nilai eigen $r_{m+1}$ dan $r_{m+2}$.
    \item Jika terdapat vektor eigen kompleks lainnya, maka ulangi langkah 1 sampai 3 untuk mendapatkan semua solusi nilai eigen kompleks. Jika tidak, maka solusi umum dari sistem PD linear homogen dimensi-$n$ adalah
    \begin{equation}
      \mathbf{x}(t) = c_1\mathbf{x}_1(t) + c_2\mathbf{x}_2(t) + \cdots + c_n\mathbf{x}_n(t)
    \end{equation}
    \item Jika terdapat syarat awal $x_0$, maka subtitusi syarat awal tersebut ke dalam solusi umum yang telah didapatkan.
    \begin{equation}\label{eq:solusi_umum_syarat_awal}
      \mathbf{x}(x_0) = c_1\mathbf{x}_1(x_0) + c_2\mathbf{x}_2(x_0) + \cdots + c_n\mathbf{x}_n(x_0)
    \end{equation}
    \item Selesaikan persmaan \eqref{eq:solusi_umum_syarat_awal} untuk mencari konstanta $c_1, c_2, \ldots, c_n$. 
  \end{enumerate}
  \end{algoritma}

  \begin{catatan}
    Jikalau kita memilih vektor eigen $\bfxi_{m+2}=\mathbf{u}-i\mathbf{v}$ yang berkoresponden dengan nilai eigen $r_{m+2} = \alpha - i\beta$, maka solusi umumnya menjadi
    \begin{equation}\label{eq:solusi_vektor_eigen_konjugat}
      \begin{split}
        \mathbf{x}_{m+1}(t) &= e^{\alpha t}(\mathbf{u}\cos(-\beta t) - (-\mathbf{v})\sin(-\beta t))\\
        &= e^{\alpha t}(\mathbf{u}\cos(\beta t) - \mathbf{v}\sin(\beta t))\\
        \mathbf{x}_{m+2}(t) &= e^{\alpha t}(\mathbf{u}\sin(-\beta t) + (-\mathbf{v})\cos(-\beta t))\\
        &= -e^{\alpha t}(\mathbf{u}\sin(\beta t) + \mathbf{v}\cos(\beta t))
      \end{split}
    \end{equation}
    Dapat dilihat bahwa solusi \eqref{eq:solusi_vektor_eigen} dan \eqref{eq:solusi_vektor_eigen_konjugat} hanya berbeda tanda pada $\mathbf{x}_{m+2}(t)$. Namun karena solusi umum mencakup sebuah sembarang konstanta $c_2$, maka perbedaan tanda tersebut tidak mempengaruhi solusi umumnya.
  \end{catatan}
  
  \begin{contoh}
    Tentukan solusi umum dari 
    \begin{equation}\label{eq:contoh_pd_trench_2}
      \mathbf{y'} = \begin{pmatrix}
        1 & -1 & -2\\
        1 & 3 & 2\\
        1 & -1 & 2
      \end{pmatrix}\mathbf{y}.
    \end{equation}
    \solusi
    Polinomial karakteristik dari matriks \eqref{eq:contoh_pd_trench_2} adalah
    \begin{equation}
      \begin{vmatrix}
        1-\lambda & -1 & -2\\
        1 & 3-\lambda & 2\\
        1 & -1 & 2-\lambda
      \end{vmatrix} = -(\lambda-2)((\lambda-2)^2+4).
    \end{equation}
    Nilai eigen dari $A$ adalah $\lambda_1 = 2, \lambda_2 = 2+2i$ dan $\lambda_3 = 2-2i$. Matriks \textit{augmented} dari $(A-2I)\mathbf{x} = \mathbf{0}$ adalah
    \begin{equation*}
      \begin{pmatrix}
        -1 & -1 & -2 &:& 0\\
        1 & 1 & 2 &:& 0\\
        1 & -1 & 0 &:& 0
      \end{pmatrix} \sim \begin{pmatrix}
        1 & 0 & 1 &:& 0\\
        0 & 1 & 1 &:& 0\\
        0 & 0 & 0 &:& 0
      \end{pmatrix}.
    \end{equation*}
    Diperoleh $x_1 = -x_3$ dan $x_2 = -x_3$. Ambil $x_3 = 1$ sehingga
    \[\mathbf{x}_1 = \begin{pmatrix}-1\\-1\\1\end{pmatrix}.\]
    Jadi
    \[\mathbf{y}_1 = \begin{pmatrix}-1\\-1\\1\end{pmatrix}e^{2t}\]
    adalah solusi dari \eqref{eq:contoh_pd_trench_2}. 
    
    Matriks \textit{augmented} dari $(A-(2+2i)I)\mathbf{x} = \mathbf{0}$ adalah
    \begin{equation*}
      \begin{pmatrix}
        -1-2i & -1 & -2 &:& 0\\
        1 & 1-2i & 2 &:& 0\\
        1 & -1 & -2i &:& 0
      \end{pmatrix} \sim \begin{pmatrix}
        1 & 0 & -i &:& 0\\
        0 & 1 & i &:& 0\\
        0 & 0 & 0 &:& 0
      \end{pmatrix}.
    \end{equation*}
    Sehingga $x_1 = ix_3$ dan $x_2 = -ix_3$. Ambil $x_3 = 1$ maka
    \[\mathbf{x}_2 = \begin{pmatrix}i\\-i\\1\end{pmatrix}.\]
    Dengan Algoritma \ref{algoritma} dapat diperoleh
    \begin{enumerate}[label=Langkah \arabic*: ,leftmargin=*]
      \item Perhatikan bahwa $\alpha=2$ dan $\beta=2$. 
      \item Pilih vektor eigen $\mathbf{x}_2 = \begin{pmatrix}i\\-i\\1\end{pmatrix}$, maka $\mathbf{u}= \begin{pmatrix}0\\0\\1\end{pmatrix}$ dan $\mathbf{v} = \begin{pmatrix}1\\-1\\0\end{pmatrix}$.
      \item Dua solusi lainnya adalah
      \begin{align*}
        \mathbf{y}_2(t) &= e^{2t}\left(\begin{pmatrix}0\\0\\1\end{pmatrix}\cos(2t) - \begin{pmatrix}1\\-1\\0\end{pmatrix}\sin(2t)\right)=\begin{pmatrix}-\sin(2t)\\\sin(2t)\\\cos(2t)\end{pmatrix}
      \end{align*}
      dan
      \begin{align*}
        \mathbf{y}_3(t) &= e^{2t}\left(\begin{pmatrix}0\\0\\1\end{pmatrix}\sin(2t) + \begin{pmatrix}1\\-1\\0\end{pmatrix}\cos(2t)\right)=\begin{pmatrix}\cos(2t)\\-\cos(2t)\\\sin(2t)\end{pmatrix}
      \end{align*}
      \item Solusi umumnya adalah
      \[\mathbf{y}=c_1\begin{pmatrix}-1\\-1\\1\end{pmatrix}e^{2t} + c_2e^{2t}\begin{pmatrix}-\sin(2t)\\\sin(2t)\\\cos(2t)\end{pmatrix} + c_3e^{2t}\begin{pmatrix}\cos(2t)\\-\cos(2t)\\\sin(2t)\end{pmatrix}.\]
    \end{enumerate}
  \end{contoh}

  \begin{contoh}
    Carilah solusi dari 
    \begin{equation}\label{eq:contoh_pd_trench_1}
      \mathbf{y'} = \begin{pmatrix}
        -5 & 5 & 4\\
        -8 & 7 & 6\\
        1 & 0 & 0
      \end{pmatrix}\mathbf{y}
    \end{equation}
    Dengan syarat awal $\mathbf{y}(0) = \begin{pmatrix}0&3&1\end{pmatrix}^T$.\\
    \solusi
    Polinomial karakteristiknya dari matriks $A$ pada \eqref{eq:contoh_pd_trench_1} adalah
    \begin{equation}
      \begin{vmatrix}
        -5-\lambda & 5 & 4\\
        -8 & 7-\lambda & 6\\
        1 & 0 & -\lambda
      \end{vmatrix}=-(\lambda-2)(\lambda^2+1).
    \end{equation}
    Sehingga didapatkan nilai eigen dari $A$ adalah $\lambda_1 = 2, \lambda_2 = i$ dan $ \lambda_3 = -i$. Selanjutnya matriks \textit{augmented} dari $(A-\lambda 2I)\mathbf{x} = \mathbf{0}$ adalah
    \begin{equation*}
      \begin{pmatrix}
        -7 & 5 & 4 &:& 0\\
        -8 & 5 & 6 &:& 0\\
        1 & 0 & -2 &:& 0
      \end{pmatrix} \sim \begin{pmatrix}
        1 & 0 & -2 &:& 0\\
        0 & 1 & -2 & :&0\\
        0 & 0 & 0 & :&0
      \end{pmatrix}.
    \end{equation*}
    diperoleh $x_1 = x_2 = 2x_3$. Ambil saja $x_3 = 1$ akibatnya 
    \[\mathbf{x}_1 = \begin{pmatrix}2\\2\\1\end{pmatrix}.\]
    Jadi
    \[\mathbf{y}_1=\begin{pmatrix}2\\2\\1\end{pmatrix}e^{2t}\]
    adalah solusi dari \eqref{eq:contoh_pd_trench_1}. 
    Kemudian, matriks \textit{augmented} dari $(A-iI)\mathbf{x} = \mathbf{0}$ adalah
    \begin{equation*}
      \begin{pmatrix}
        -5-i & 5 & 4 &:& 0\\
        -8 & 7-i & 6 &:& 0\\
        1 & 0 & -i &:& 0
      \end{pmatrix} \sim \begin{pmatrix}
        1 & 0 & -i &:& 0\\
        0 & 1 & 1-i & :&0\\
        0 & 0 & 0 & :&0
      \end{pmatrix}.
    \end{equation*}
    didapatkan $x_1 = ix_3$ dan $x_2 = (1-i)x_3$. Dengan mengambil $x_3 = 1$ diperoleh
    \[\mathbf{x}_2 = \begin{pmatrix}i\\-1+i\\1\end{pmatrix}.\]
    Lakukan yang telah dicontohkan pada Algoritma \ref{algoritma} sebagai berikut:
    \begin{enumerate}[label=Langkah \arabic*: ,leftmargin=*]
      \item Karena $\lambda_2=0+i$ dan $\lambda_3=0-i$ saling konjugat begitu juga dengan vektor eigen nya yaitu
      \[\mathbf{x_2}= \begin{pmatrix}i\\-1+i\\1\end{pmatrix} \quad \text{dan} \quad \mathbf{x_3}= \begin{pmatrix}-i\\-1-i\\1\end{pmatrix}.\]
      \item Pilih salah satu vektor eigen, misalnya $\mathbf{x_2}$ sehingga 
      \[\mathbf{x_2} = \mathbf{u}+i\mathbf{v}= \begin{pmatrix}0\\-1\\1\end{pmatrix} + i\begin{pmatrix}1\\1\\0\end{pmatrix}.\]
      didapat $\mathbf{u} = \begin{pmatrix}0\\-1\\1\end{pmatrix}$ dan $\mathbf{v} = \begin{pmatrix}1\\1\\0\end{pmatrix}$.
      \item Solusi kedua dan ketiga masing-masing adalah
      \begin{align*}
        \mathbf{y}_2(t) &= e^{0t}\left(\begin{pmatrix}0\\-1\\1\end{pmatrix}\cos(t) - \begin{pmatrix}1\\1\\0\end{pmatrix}\sin(t)\right)=\begin{pmatrix}-\sin(t)\\-\cos(t)-\sin(t)\\\cos(t)\end{pmatrix}
      \end{align*}
      dan
      \begin{align*}
        \mathbf{y}_3(t) &= e^{0t}\left(\begin{pmatrix}0\\-1\\1\end{pmatrix}\sin(t) + \begin{pmatrix}1\\1\\0\end{pmatrix}\cos(t)\right)=\begin{pmatrix}\cos(t)\\-\sin(t)+\cos(t)\\\sin(t)\end{pmatrix}
      \end{align*}
      \item solusi umumnya adalah
      \[\mathbf{y}=c_1\begin{pmatrix}2\\2\\1\end{pmatrix}e^{2t} + c_2\begin{pmatrix}-\sin(t)\\-\cos(t)-\sin(t)\\\cos(t)\end{pmatrix} + c_3\begin{pmatrix}\cos(t)\\-\sin(t)+\cos(t)\\\sin(t)\end{pmatrix}.\]
      \item Subtitusi syarat awal $\mathbf{y}(0)$ sehingga didapatkan
      \begin{align*}
        \begin{pmatrix}0\\3\\1\end{pmatrix} &= c_1\begin{pmatrix}2\\2\\1\end{pmatrix} + c_2\begin{pmatrix}0\\-1\\1\end{pmatrix} + c_3\begin{pmatrix}1\\1\\0\end{pmatrix}= \begin{pmatrix}2c_1+c_3\\2c_1-c_2+c_3\\c_1+c_2\end{pmatrix}.
      \end{align*}
      \item Selesaikan persamaan menggunakan sembarang metode yang telah anda diketahui. Pada contoh ini akan digunakan metode eliminasi Gauss-Jordan. 
      \begin{align*}
        \begin{pmatrix}
          2 & 0 & 1 &:& 0\\
          2 & -1 & 1 &:& 3\\
          1 & 1 & 0 &:& 1
        \end{pmatrix}
        &\sim \begin{pmatrix}
          1 & 0 & 0 &:& \frac{4}{3}\\
          0 & 1 & 0 &:& -\frac{1}{3}\\
          0 & 0 & 1 &:& -\frac{8}{3}
        \end{pmatrix}.
      \end{align*}
      Didapat $c_1 = \frac{4}{3}, c_2 = -\frac{1}{3}$ dan $c_3 = -\frac{8}{3}$. Jadi solusi dari \eqref{eq:contoh_pd_trench_1} adalah
      \[\mathbf{y} = \frac{4e^{2t}}{3}\begin{pmatrix}2\\2\\1\end{pmatrix} - \frac{1}{3}\begin{pmatrix}-\sin(t)\\-\cos(t)-\sin(t)\\\cos(t)\end{pmatrix} - \frac{8}{3}\begin{pmatrix}\cos(t)\\-\sin(t)+\cos(t)\\\sin(t)\end{pmatrix}.\]
    \end{enumerate}
  \end{contoh}

  \begin{contoh}
    Carilah solusi PD linear orde tiga berikut
    \begin{equation}\label{eq:contoh_pd_3}
      y''' - 4y'' + 6y' - 4y = 0.
    \end{equation}
    \solusi
    Persamaan \eqref{eq:contoh_pd_3} dapat diselesaikan menggunakan sistem PD linear homogen dimensi-$3$. Misalkan
    \begin{align*}
      y_1 &= y \implies y_1' = y' = y_2\\
      y_2 &= y' \implies y_2' = y'' = y_3\\
      y_3 &= y''\implies y_3' = y''' = 4y'' - 6y' + 4y.
    \end{align*}
    Dengan demikian didapatkan sistem sebagai berikut
    \begin{equation}\label{eq:contoh_pd_3_sistem}
      \begin{pmatrix}
        y_1'\\
        y_2'\\
        y_3'
      \end{pmatrix} = \begin{pmatrix}
        0 & 1 & 0\\
        0 & 0 & 1\\
        4 & -6 & 4
      \end{pmatrix}\begin{pmatrix}
        y_1\\
        y_2\\
        y_3
      \end{pmatrix}.
    \end{equation}
    Dari matriks \eqref{eq:contoh_pd_3_sistem} didapatkan nilai eigen $\lambda_1 = 2, \lambda_2 = 1+i$ dan $\lambda_3 = 1-i$. Untuk vektor eigen yang berkoresponden adalah
    \[\bfxi_1 = \begin{pmatrix}1\\2\\4\end{pmatrix} \quad \bfxi_1 = \begin{pmatrix}i\\1+i\\2\end{pmatrix} \quad \bfxi_3 = \begin{pmatrix}-i\\1-i\\2\end{pmatrix}.\]
    \begin{enumerate}[label=Langkah \arabic*: ,leftmargin=*]
      \item Terlihat bahwa $\bfxi_2$ dan $\bfxi_3$ adalah pasangan konjugat.
      \item Ambil $\bfxi_2 = \begin{pmatrix}i\\1+i\\2\end{pmatrix}$, didapat $\mathbf{u}= \begin{pmatrix}0\\1\\2\end{pmatrix}$ dan $\mathbf{v} = \begin{pmatrix}1\\1\\0\end{pmatrix}$.
      \item Solusi untuk nilai eigen $\lambda_2$ dan $\lambda_3$ adalah
      \begin{align*}
        \mathbf{y}_2(x) &= e^{x}\left(\begin{pmatrix}0\\1\\2\end{pmatrix}\cos(x) - \begin{pmatrix}1\\1\\0\end{pmatrix}\sin(x)\right)=\begin{pmatrix}-\sin(x)\\\cos(x)-\sin(x)\\2\cos(x)\end{pmatrix}e^x\\
        \mathbf{y}_3(x) &= e^{x}\left(\begin{pmatrix}0\\1\\2\end{pmatrix}\sin(x) + \begin{pmatrix}1\\1\\0\end{pmatrix}\cos(x)\right)=\begin{pmatrix}\cos(x)\\\sin(x)+\cos(x)\\2\sin(x)\end{pmatrix}e^x
      \end{align*}
      \item Solusi umum dari \eqref{eq:contoh_pd_3} adalah
      \[\mathbf{y}(x) = c_1e^{2x}\begin{pmatrix}1\\2\\4\end{pmatrix} + c_2e^x\begin{pmatrix}-\sin(x)\\\cos(x)-\sin(x)\\2\cos(x)\end{pmatrix} + c_3e^x\begin{pmatrix}\cos(x)\\\sin(x)+\cos(x)\\2\sin(x)\end{pmatrix}.\]
    \end{enumerate}
    Ingat bahwa $\mathbf{y}(x)=\begin{pmatrix}y(x)\\y'(x)\\y''(x)\end{pmatrix}$ artinya solusi PD orde tiga ada pada baris pertama solusi umum. Jadi solusi dari \eqref{eq:contoh_pd_3} adalah
    \[y(x) = c_1e^{2x} + c_2e^x\cos(x) + c_3e^x\sin(x).\] 
  \end{contoh}

  \begin{contoh}
    Diberikan sistem PD linear homogen dimensi-$4$ berikut
    \begin{equation}\label{eq:contoh_pd_4}
      \mathbf{x'}=\begin{pmatrix}
        0 & -1 & 0 & 0\\
        1 & 0 & 0 & 0\\
        0 & 0 & 0 & 1\\
        0 & 0 & -1 & 0 
      \end{pmatrix}\mathbf{x}
    \end{equation}
    Tentukan solusi umum dari sistem PD tersebut.\\
    \solusi
    Nilai eigen dari matriks \eqref{eq:contoh_pd_4} adalah
    \[\lambda_{1,2} = i, \quad \lambda_{3,4} = -i\]
    dan vektor eigen yang berkoresponden dengan nilai eigen tersebut adalah
    \[\bfxi_1 = \begin{pmatrix}i\\1\\0\\0\end{pmatrix} \quad \bfxi_2 = \begin{pmatrix}0\\0\\i\\1\end{pmatrix} \quad \bfxi_3 = \begin{pmatrix}-i\\1\\0\\0\end{pmatrix} \quad \bfxi^{(4)} = \begin{pmatrix}0\\0\\-i\\1\end{pmatrix}\]
    \begin{enumerate}[label=Langkah \arabic*: ,leftmargin=*]
      \item Pilih vektor eigen $\bfxi_1$ dan $\bfxi_3$ sebagai vektor eigen yang saling konjugat.
      \item Ambil $\bfxi_1 = \begin{pmatrix}i\\1\\0\\0\end{pmatrix}$, maka $\mathbf{u} = \begin{pmatrix}0\\1\\0\\0\end{pmatrix}$ dan $\mathbf{v} = \begin{pmatrix}1\\0\\0\\0\end{pmatrix}$.
      \item Solusi untuk nilai eigen $\lambda_1$ dan $\lambda_3$ adalah
      \begin{equation}\label{eq:contoh_pd_4_solusi_1}
        \begin{split}
          \mathbf{x}_1(t) &= \left(\begin{pmatrix}0\\1\\0\\0\end{pmatrix}\cos(t) - \begin{pmatrix}1\\0\\0\\0\end{pmatrix}\sin(t)\right) = \begin{pmatrix}-\sin(t)\\ \cos(t)\\0\\0\end{pmatrix}\\
          \mathbf{x}_2(t) &= \left(\begin{pmatrix}0\\1\\0\\0\end{pmatrix}\sin(t) + \begin{pmatrix}1\\0\\0\\0\end{pmatrix}\cos(t)\right) = \begin{pmatrix}\cos(t)\\ \sin(t)\\0\\0\end{pmatrix}
        \end{split}
      \end{equation}
      \item Karena terdapat vektor kompleks eigen lain, yaitu $\bfxi_2$ dan $\bfxi_4$.
      \item Ambil $\bfxi_2 = \begin{pmatrix}0\\0\\i\\1\end{pmatrix}$, maka $\mathbf{u} = \begin{pmatrix}0\\0\\1\\0\end{pmatrix}$ dan $\mathbf{v} = \begin{pmatrix}0\\0\\0\\1\end{pmatrix}$.
      \item Solusi untuk nilai eigen $\lambda_2$ dan $\lambda_4$ adalah
      \begin{equation}\label{eq:contoh_pd_4_solusi_2}
        \begin{split}
          \mathbf{x}_3(t) &= \left(\begin{pmatrix}0\\0\\1\\0\end{pmatrix}\cos(t) - \begin{pmatrix}0\\0\\0\\1\end{pmatrix}\sin(t)\right) = \begin{pmatrix}0\\0\\-\sin(t)\\ \cos(t)\end{pmatrix}\\
          \mathbf{x}_4(t) &= \left(\begin{pmatrix}0\\0\\1\\0\end{pmatrix}\sin(t) + \begin{pmatrix}0\\0\\0\\1\end{pmatrix}\cos(t)\right) = \begin{pmatrix}0\\0\\ \cos(t)\\ \sin(t)\end{pmatrix}
        \end{split}
      \end{equation}
      \item Solusi umum dari \eqref{eq:contoh_pd_4} adalah 
      \[\mathbf{x}(t) = c_1\begin{pmatrix}-\sin(t)\\ \cos(t)\\0\\0\end{pmatrix} + c_2\begin{pmatrix}\cos(t)\\ \sin(t)\\0\\0\end{pmatrix} + c_3\begin{pmatrix}0\\0\\-\sin(t)\\ \cos(t)\end{pmatrix} + c_4\begin{pmatrix}0\\0\\ \cos(t)\\ \sin(t)\end{pmatrix}.\]
    \end{enumerate}
  \end{contoh}

  \newpage
  \noindent Untuk memperdalam pemahaman Anda mengenai pengertian sistem PD linear homogen dimensi-$n$ dan penyelesaiannya, kerjakanlah latihan berikut!
  \begin{enumerate}
    \item Diberikan sistem PD linear berikut
    \begin{align*}
      \mathbf{x'} = \begin{pmatrix}
        1 & 2 & -3\\
        0 & -3 & 2\\
        0 & k & 5
      \end{pmatrix}\mathbf{x}
    \end{align*}
    Tentukan nilai $k$ agar sistem PD tersebut memiliki nilai eigen kompleks yang bagian imajinernya tak nol.
    \item Carilah solusi homogen dari PD linear berikut
    \begin{align*}
      \mathbf{x'} = \begin{pmatrix}
        1 & 0 & 0\\
        2 & 1 & -2\\
        3 & 2 & 1
      \end{pmatrix}\mathbf{x}
    \end{align*}

    \item Diberikan sebuah sistem PD linear homogen berikut
    \begin{equation*}
        \frac{dx}{dt}=-3x+2z,\quad
        \frac{dy}{dt}=x-y,\quad
        \frac{dz}{dt}=-2x-y
    \end{equation*}
    Tentukan solusi umum dari sistem PD tersebut.
    
    \item Carilah solusi umum sistem PD dimensi-$4$ berikut
    \[\mathbf{y'}=\begin{pmatrix}
      1 & -2 & 1 & -3\\
      0 & 5 & 0 & 0\\
      0 & 2 & 0 & -1\\
      1 & 0 & -2 & 0
    \end{pmatrix}\mathbf{y}\]
    
    \item Diberikan suatu sistem tiga pegas dengan dua buah massa yang dirumuskan dalam PD berikut
    \begin{equation*}
      \begin{split}
        m_1\frac{d^2x_1}{dt^2}&=-(k_1+k_2)x_1+k_2x_2\\
        m_2\frac{d^2x_2}{dt^2}&=k_2x_1-(k_2+k_3)x_2\\
      \end{split}
    \end{equation*}
    Jika $m_1=m_2=2,$ dan $k_1=k_2=k_3=6$. Tentukan solusi umum dari sistem PD tersebut.
  \end{enumerate}
  \newpage
  \underline{JAWABAN LATIHAN 12.2}
  \begin{enumerate}
    \item Polinomial karakteristik matriks $A$ adalah 
    \begin{equation*}
      \begin{vmatrix}
        1-\lambda & 2 & -3\\
        0 & -3-\lambda & 2\\
        0 & k & 5-\lambda
      \end{vmatrix} = (1-\lambda)(\lambda^2-2\lambda-15-2k). 
    \end{equation*}
    Sehingga agar terdapat nilai eigen kompleks yang bagian imajinernya tak nol, maka haruslah diskriminan dari polinomial $\lambda^2-2\lambda-15-2k$ negatif. 
    \[\Delta = b^2-4ac = (-2)^2-4(1)(-15-2k)=64+8k < 0 \implies k < -8.\]
    Jadi nilai $k$ yang memenuhi adalah $\{x\in\R\,|\,x<-8\}$.

    \item Nilai eigen dari matriks adalah
    \[\lambda_1 = 1, \quad \lambda_2 = 1 -2i, \quad \lambda_3 = 1 + 2i.\]
    dan vektor eigen yang berkoresponden dengan nilai eigen tersebut adalah
    \[\bfxi_1 = \begin{pmatrix}2\\-3\\2\end{pmatrix} \quad \bfxi_2 = \begin{pmatrix}0\\-i\\1\end{pmatrix} \quad \bfxi_3 = \begin{pmatrix}0\\i\\1\end{pmatrix}.\]
    Selanjutnya didapatkan masing-masing solusi $\mathbf{x}_1, \mathbf{x}_2$ dan $\mathbf{x}_3$ sebagai berikut
    \begin{align*}
      \mathbf{x}_1(t) &= \begin{pmatrix}2\\-3\\2\end{pmatrix}e^t\\
      \mathbf{x}_2(t) &= e^t \left(\begin{pmatrix}0\\0\\1\end{pmatrix}\cos(2t) - \begin{pmatrix}0\\1\\0\end{pmatrix}\sin(2t)\right)=\begin{pmatrix}0\\-\sin(2t)\\ \cos(2t)\end{pmatrix}e^t\\
      \mathbf{x}_3(t) &= e^t \left(\begin{pmatrix}0\\0\\1\end{pmatrix}\sin(2t) + \begin{pmatrix}0\\1\\0\end{pmatrix}\cos(2t)\right)=\begin{pmatrix}0\\ \cos(2t)\\ \sin(2t)\end{pmatrix}e^t
    \end{align*}
    Jadi solusi umum dari sistem PD tersebut adalah
    \[\mathbf{x}(t) = c_1\begin{pmatrix}2\\-3\\2\end{pmatrix}e^t + c_2\begin{pmatrix}0\\-\sin(2t)\\ \cos(2t)\end{pmatrix}e^t + c_3\begin{pmatrix}0\\ \cos(2t)\\ \sin(2t)\end{pmatrix}e^t.\]
    
    \item Misalkan $\mathbf{y} = \begin{pmatrix}x\\y\\z\end{pmatrix}$, maka sistem PD tersebut dapat dituliskan sebagai
    \begin{equation*}
      \mathbf{y'}=\begin{pmatrix}
        -3 & 0 & 2\\
        1 & -1 & 0\\
        -2 & -1 & 0
      \end{pmatrix}\mathbf{y}.
    \end{equation*}
    Selanjutnya nilai eigen dari matriks tersebut adalah
    \[\lambda_1 = -2, \quad \lambda_2 = -1+\sqrt{2}i, \quad \lambda_3 = -1-\sqrt{2}i\]
    dengan masing-masing vektor eigen
    \[\bfxi_1 = \begin{pmatrix}2\\-2\\1\end{pmatrix} \quad \bfxi_2 = \begin{pmatrix}2-\sqrt{2}i\\-1-\sqrt{2}i\\3\end{pmatrix} \quad \bfxi_3 = \begin{pmatrix}2+\sqrt{2}i\\-1+\sqrt{2}i\\3\end{pmatrix}.\]
    Dari vektor eigen $\bfxi_2$ dan $\bfxi_3$ dapat diperoleh solusi kompleks
    \begin{align*}
      \mathbf{y}_2(t) &= e^{-t}\left(\begin{pmatrix}2\\-1\\3\end{pmatrix}\cos(\sqrt{2}t) - \begin{pmatrix}-\sqrt{2}\\-\sqrt{2}\\0\end{pmatrix}\sin(\sqrt{2}t)\right)=e^{-t}\begin{pmatrix}2\cos(\sqrt{2}t)+\sqrt{2}\sin(\sqrt{2}t)\\-\cos(\sqrt{2}t)+\sqrt{2}\sin(\sqrt{2}t)\\3\cos(\sqrt{2}t)\end{pmatrix}\\
      \mathbf{y}_3(t) &= e^{-t}\left(\begin{pmatrix}2\\-1\\3\end{pmatrix}\sin(\sqrt{2}t) + \begin{pmatrix}-\sqrt{2}\\-\sqrt{2}\\0\end{pmatrix}\cos(\sqrt{2}t)\right)=e^{-t}\begin{pmatrix}2\sin(\sqrt{2}t)-\sqrt{2}\cos(\sqrt{2}t)\\-\sin(\sqrt{2}t)-\sqrt{2}\cos(\sqrt{2}t)\\3\sin(\sqrt{2}t)\end{pmatrix}
    \end{align*} 
    Sehingga solusi umum dari sistem PD tersebut adalah
    \[\mathbf{y}(t) = c_1\begin{pmatrix}2\\-2\\1\end{pmatrix}e^{-2t} + c_2\begin{pmatrix}2\cos(\sqrt{2}t)+\sqrt{2}\sin(\sqrt{2}t)\\-\cos(\sqrt{2}t)+\sqrt{2}\sin(\sqrt{2}t)\\3\cos(\sqrt{2}t)\end{pmatrix}e^{-t} + c_3\begin{pmatrix}2\sin(\sqrt{2}t)-\sqrt{2}\cos(\sqrt{2}t)\\-\sin(\sqrt{2}t)-\sqrt{2}\cos(\sqrt{2}t)\\3\sin(\sqrt{2}t)\end{pmatrix}e^{-t}.\]

    \item Nilai eigen dari matriks adalah
    \[\lambda_1 = 1, \quad \lambda_2 = 5, \quad \lambda_3 = 1 + \sqrt{2}i \quad \lambda_4 = 1 - \sqrt{2}i\]
    dan vektor eigen yang berkoresponden dengan nilai eigen tersebut adalah
    \[\bfxi_1 = \begin{pmatrix}1\\0\\1\\1\end{pmatrix} \quad \bfxi_2 = \begin{pmatrix}2\\-9\\-4\\2\end{pmatrix} \quad \bfxi_3 = \begin{pmatrix}1-5\sqrt{2}i\\0\\\-1-\sqrt{2}i\\3\end{pmatrix} \quad \bfxi^{(4)} = \begin{pmatrix}1+5\sqrt{2}i\\0\\\-1+\sqrt{2}i\\3\end{pmatrix}.\]
    Dengan mudah didapatkan solusi untuk $\mathbf{y}_1$ dan $\mathbf{y}_2$ adalah
    \begin{align*}
      \mathbf{y}_1(t) = \begin{pmatrix}1\\0\\1\\1\end{pmatrix}e^t,\quad
      \mathbf{y}_2(t) = \begin{pmatrix}2\\-9\\-4\\2\end{pmatrix}e^{5t}
    \end{align*}
    dan untuk $\mathbf{y}_3$ dan $\mathbf{y}_4$ karena bilangan kompleks maka solusinya adalah
    \begin{align*}
      \mathbf{y}_3(t) &= e^t\left(\begin{pmatrix}1\\0\\1\\3\end{pmatrix}\cos(\sqrt{2}t) - \begin{pmatrix}-5\sqrt{2}\\0\\-\sqrt{2}\\0\end{pmatrix}\sin(\sqrt{2}t)\right)= e^t\begin{pmatrix}\cos(\sqrt{2}t)+5\sqrt{2}\sin(\sqrt{2}t)\\0\\\cos(\sqrt{2}t)+\sqrt{2}\sin(\sqrt{2}t)\\3\cos(\sqrt{2}t)\end{pmatrix}\\
      \mathbf{y}_4(t) &= e^t\left(\begin{pmatrix}1\\0\\1\\3\end{pmatrix}\sin(\sqrt{2}t) + \begin{pmatrix}-5\sqrt{2}\\0\\-\sqrt{2}\\0\end{pmatrix}\cos(\sqrt{2}t)\right)= e^t\begin{pmatrix}\sin(\sqrt{2}t)-5\sqrt{2}\cos(\sqrt{2}t)\\0\\\sin(\sqrt{2}t)-\sqrt{2}\cos(\sqrt{2}t)\\3\sin(\sqrt{2}t)\end{pmatrix}
    \end{align*}
    Jadi solusi umum dari sistem PD tersebut adalah
    \[\mathbf{y}(t) = c_1\begin{pmatrix}1\\0\\1\\1\end{pmatrix}e^t + c_2\begin{pmatrix}2\\-9\\-4\\2\end{pmatrix}e^{5t} + c_3\begin{pmatrix}\cos(\sqrt{2}t)+5\sqrt{2}\sin(\sqrt{2}t)\\0\\\cos(\sqrt{2}t)+\sqrt{2}\sin(\sqrt{2}t)\\3\cos(\sqrt{2}t)\end{pmatrix}e^t + c_4\begin{pmatrix}\sin(\sqrt{2}t)-5\sqrt{2}\cos(\sqrt{2}t)\\0\\\sin(\sqrt{2}t)-\sqrt{2}\cos(\sqrt{2}t)\\3\sin(\sqrt{2}t)\end{pmatrix}e^t.\]
    
    \item Subtitusi nilai variabel yang diketahui sehingga diperoleh sistem PD
    \begin{align*}
        \frac{d^2x_1}{dt^2}&=-6x_1+3x_2\\
        \frac{d^2x_2}{dt^2}&=3x_1-6x_2
    \end{align*}
    Dapat kita misalkan $y_1=x_1,\,y_2=x_2,\,y_3=x_1',\,y_4=x_2'$, sehingga diperoleh informasi
    \begin{align*}
        y_1'&=x_1'=y_3,\\
        y_2'&=x_2'=y_4,\\
        y_3'&=x_1''=-6x_1+3x_2=-6y_1+3y_2,\\
        y_4'&=x_2''=3x_1-6x_2=3y_1-6y_2.
    \end{align*}
    atau jika dituliskan dalam bentuk matriks adalah sebagai berikut
    \begin{equation*}
        \begin{pmatrix}
            y_1'\\
            y_2'\\
            y_3'\\
            y_4'
        \end{pmatrix}=\begin{pmatrix}
            0 & 0 & 1 & 0\\
            0 & 0 & 0 & 1\\
            -6 & 3 & 0 & 0\\
            3 & -6 & 0 & 0
        \end{pmatrix}\begin{pmatrix}
            y_1\\
            y_2\\
            y_3\\
            y_4
        \end{pmatrix}.
    \end{equation*}
    sehingga untuk mencari solusinya dapat dilakukan dengan cara yang sama seperti contoh-contoh sebelumnya. Nilai eigen dari matriks tersebut adalah
    \[\lambda_1 = 3i, \quad \lambda_2 = -3i, \quad \lambda_3 = \sqrt{3}i, \quad \lambda_4 = -\sqrt{3}i\]
    dan vektor eigen yang berkoresponden dengan nilai eigen tersebut adalah
    \[\bfxi_1 = \begin{pmatrix}i\\-i\\-3\\3\end{pmatrix} \quad \bfxi_2 = \begin{pmatrix}-i\\i\\-3\\3\end{pmatrix} \quad \bfxi_3 = \begin{pmatrix}-\sqrt{3}i\\-\sqrt{3}i\\3\\3\end{pmatrix} \quad \bfxi^{(4)} = \begin{pmatrix}\sqrt{3}i\\\sqrt{3}i\\3\\3\end{pmatrix}.\]
    Untuk masing-masing pasangan vektor eigen dapat dikelompokkan yaitu $\bfxi_1$ dan $\bfxi_2$ serta $\bfxi_3$ dan $\bfxi^{(4)}$. Solusi untuk $\bfxi_1$ dan $\bfxi_2$ adalah
    \begin{align*}
      \mathbf{y}_1(t) &= e^{0}\left(\begin{pmatrix}0\\0\\-3\\3\end{pmatrix}\cos(3t) - \begin{pmatrix}1\\-1\\0\\0\end{pmatrix}\sin(3t)\right)=\begin{pmatrix}-\sin(3t)\\\sin(3t)\\-3\cos(3t)\\3\cos(3t)\end{pmatrix}\\
      \mathbf{y}_2(t) &= e^{0}\left(\begin{pmatrix}0\\0\\-3\\3\end{pmatrix}\sin(3t) + \begin{pmatrix}1\\-1\\0\\0\end{pmatrix}\cos(3t)\right)=\begin{pmatrix}\cos(3t)\\-\cos(3t)\\-3\sin(3t)\\3\sin(3t)\end{pmatrix}
    \end{align*} 
    dan untuk $\bfxi_3$ dan $\bfxi^{(4)}$ adalah
    \begin{align*}
      \mathbf{y}_3(t) &= e^{0}\left(\begin{pmatrix}0\\0\\3\\3\end{pmatrix}\cos(\sqrt{3}t) - \begin{pmatrix}-\sqrt{3}\\-\sqrt{3}\\0\\0\end{pmatrix}\sin(\sqrt{3}t)\right)=\begin{pmatrix}\sqrt{3}\sin(\sqrt{3}t)\\\sqrt{3}\sin(\sqrt{3}t)\\3\cos(\sqrt{3}t)\\3\cos(\sqrt{3}t)\end{pmatrix}\\
      \mathbf{y}_4(t) &= e^{0}\left(\begin{pmatrix}0\\0\\3\\3\end{pmatrix}\sin(\sqrt{3}t) + \begin{pmatrix}-\sqrt{3}\\-\sqrt{3}\\0\\0\end{pmatrix}\cos(\sqrt{3}t)\right)=\begin{pmatrix}-\sqrt{3}\cos(\sqrt{3}t)\\-\sqrt{3}\cos(\sqrt{3}t)\\3\sin(\sqrt{3}t)\\3\sin(\sqrt{3}t)\end{pmatrix}
    \end{align*}
    Sehingga solusi umum dari sistem PD tersebut adalah
    \[\mathbf{y}(t) = c_1\begin{pmatrix}-\sin(3t)\\\sin(3t)\\-3\cos(3t)\\3\cos(3t)\end{pmatrix} + c_2\begin{pmatrix}\cos(3t)\\-\cos(3t)\\-3\sin(3t)\\3\sin(3t)\end{pmatrix} + c_3\begin{pmatrix}\sqrt{3}\sin(\sqrt{3}t)\\\sqrt{3}\sin(\sqrt{3}t)\\3\cos(\sqrt{3}t)\\3\cos(\sqrt{3}t)\end{pmatrix} + c_4\begin{pmatrix}-\sqrt{3}\cos(\sqrt{3}t)\\-\sqrt{3}\cos(\sqrt{3}t)\\3\sin(\sqrt{3}t)\\3\sin(\sqrt{3}t)\end{pmatrix}.\]
    Karena yang diminta adalah solusi untuk $x_1$ dan $x_2$, maka solusi masing-masingnya adalah
    \begin{align*}
      x_1(t) &= -c_1\sin(3t) + c_2\cos(3t) + c_3\sqrt{3}\sin(\sqrt{3}t) - c_4\sqrt{3}\cos(\sqrt{3}t)\\
      x_2(t) &= c_1\sin(3t) - c_2\cos(3t) + c_3\sqrt{3}\sin(\sqrt{3}t)- c_4\sqrt{3}\cos(\sqrt{3}t)
    \end{align*}
    dengan $c_1, c_2, c_3, c_4$ sebarang konstanta.
  \end{enumerate}
  \newpage
  \noindent\textbf{TES FORMATIF}\\
  Pilihlah satu jawaban yang paling tepat!
  \begin{enumerate}
    \item Diberikan sistem PD linear homogen dimensi-$3$ berikut
    \[\mathbf{x'} = \begin{pmatrix}
      6 & 0 & -3\\
      -3 & 3 & 3\\
      1 & -2 & 6
    \end{pmatrix}\mathbf{x}.\]
    Banyaknya nilai eigen yang bagian imajinernya tak nol adalah
    \begin{enumerate}
      \item $0$
      \item $1$
      \item $2$
      \item $3$
    \end{enumerate}

    \item 
  \end{enumerate}
\end{document}