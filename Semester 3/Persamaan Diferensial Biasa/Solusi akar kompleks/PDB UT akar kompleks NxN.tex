\documentclass[a4paper]{article}
\usepackage{amsmath}
\usepackage{amssymb}
\usepackage{amsthm}
\usepackage{graphicx}
\usepackage{hyperref}
  \hypersetup{
    colorlinks=true,
    linkcolor=black,
  }
\usepackage{enumitem}
\usepackage{geometry}
\usepackage{enumitem}
\usepackage{fancyhdr}
\usepackage{bm}

\newtheoremstyle{definisi}
{}{}% 
{\normalfont}{}% 
{\bfseries}{}% 
{\newline}% 
{\underline{\thmname{#1}\thmnumber{ #2}\thmnote{ #3}.}}

\theoremstyle{definisi}
\newtheorem{contoh}{Contoh}[section]
\newtheorem{definisi}{Definisi}[section]
\newtheorem{teorema}{Teorema}[section]
\newtheorem{algoritma}{Algoritma}[section]
\newtheorem{catatan}{Catatan}[section]
\newtheorem{akibat}{Akibat}[section]

\newcommand{\bfxi}{\boldsymbol{\xi}}
\newcommand{\penyelesaian}{\textbf{\underline{Penyelesaian:}}\\}

\numberwithin{equation}{section}
\setcounter{equation}{3}
\setcounter{catatan}{2}
\setcounter{algoritma}{1}
\setcounter{teorema}{1}
\setcounter{definisi}{0}
\setcounter{contoh}{6}
\setcounter{section}{12}

\newcommand{\R}{\mathbb{R}}
\newcommand{\C}{\mathbb{C}}

\begin{document}
  Pada Kegiatan Belajar 2 kali ini akan dibahas mengenai penyelesaian sistem PD linear homogen dimensi-$n$ yang memiliki nilai eigen kompleks. Sebagai pengingat, bilangan kompleks biasanya ditulis dalam bentuk $z=x+iy$ dengan $x,y\in\R\text{ dan }i=\sqrt{-1}$. 

  \begin{catatan}
    Pada umumnya bilangan real juga merupakan bilangan kompleks ($\R\subseteq\C$). Namun pada Kegiatan Belajar ini, bilangan kompleks yang dimaksud adalah bilangan yang memiliki bagian imajiner tak nol ($z=x+iy,\,y\ne 0$).
  \end{catatan}

  Selanjutnya akan diberikan sebuah definisi sebagai pendahuluan terkait nilai eigen dan vektor eigen untuk bilangan kompleks.

  \begin{definisi}\label{definisi_vektor_kompleks}
    Vektor $\mathbf{u}$ dan $\mathbf{v}$ masing-masing disebut bagian real dan bagian imajiner dari sebuah vektor $\bfxi$ jika $\bfxi$ dapat ditulis sebagai $\mathbf{u} + i\mathbf{v}$. Disisi lain, $\overline{\bfxi} = \mathbf{u} - i\mathbf{v}$ disebut konjugat dari $\bfxi$.
  \end{definisi}
  \begin{akibat}\label{akibat_vektor_kompleks}
    Jika $\bfxi=\mathbf{u}+i\mathbf{v}$ adalah vektor eigen yang berkoresponden dengan nilai eigen kompleks $\lambda=\alpha+i\beta$, maka $\overline{\bfxi}=\mathbf{u}-i\mathbf{v}$ adalah vektor eigen yang berkoresponden dengan nilai eigen konjugatnya yaitu $\overline{\lambda}=\alpha-i\beta$.
  \end{akibat}

  Definisi \ref{definisi_vektor_kompleks} dan Akibat \ref{akibat_vektor_kompleks} akan digunakan untuk memisalkan suatu vektor eigen yang berkoresponden dengan nilai eigen kompleks. Tinjau sistem (12.1), kemudian misalkan $\lambda=\alpha+i\beta$ dan $\bfxi=\mathbf{u}+i\mathbf{v}$ masing-masing adalah nilai eigen dan vektor eigen dari $A$, maka salah satu penyelesaian umumnya adalah
  \begin{equation}\label{eq:umum_eigen_kompleks}
    \mathbf{x}(t) = e^{\lambda t}\bfxi = e^{(\alpha+i\beta)t}(\mathbf{u}+i\mathbf{v})
  \end{equation}
  Jika kita mengingat kembali tentang persamaan Euler, maka bentuk $e^{(\alpha+i\beta)t}$ dapat ditulis sebagai
  \begin{equation}\label{eq:euler_eigen_kompleks}
    e^{(\alpha+i\beta)t} =e^{\alpha t}e^{i\beta t}=e^{\alpha t}(\cos\beta t + i\sin\beta t).
  \end{equation}
  Dari \eqref{eq:umum_eigen_kompleks} dan \eqref{eq:euler_eigen_kompleks} kita dapatkan sebuah penyelesaian kompleks
  \begin{equation}\label{eq:penyelesaian_eigen_kompleks}
    \mathbf{x}(t) = e^{\alpha t}(\cos\beta t + i\sin\beta t)(\mathbf{u}+i\mathbf{v}).
  \end{equation}
  Namun pada umumnya kita lebih memfokuskan pada penyelesaian yang bersifat real, sehingga dari \eqref{eq:penyelesaian_eigen_kompleks} perlu kita tinjau ulang untuk mencari semua penyelesaian yang bersifat real. Berikut adalah teorema yang akan membantu kita dalam mencari penyelesaian dari sistem (12.1) yang memiliki nilai eigen kompleks

  \begin{teorema}\label{thm:penyelesaian_nilai_eigen_kompleks}
    Misalkan $A$ matriks berukuran $n\times n$ dengan entri-entrinya real. Andaikan $\lambda = \alpha + i\beta\,(\beta \ne 0)$ adalah nilai eigen kompleks dari $A$ dan $\bfxi=\mathbf{u}+i\mathbf{v}$ adalah vektor eigen yang bersesuaian, dimana $\mathbf{u}$ dan $\mathbf{v}$ adalah komponen real yang masing-masing tak nol. Maka 
    \[\mathbf{y_1} = e^{\alpha t}(\mathbf{u}\cos\beta t - \mathbf{v}\sin\beta t)\quad\text{dan}\quad \mathbf{y_2} = e^{\alpha t}(\mathbf{u}\sin\beta t + \mathbf{v}\cos\beta t)\]
    yang merupakan bagian real dan imajiner dari
    \begin{equation}
      e^{\alpha t}(\cos\beta t + i\sin\beta t)(\mathbf{u} + i\mathbf{v}) 
    \end{equation}
    merupakan penyelesaian yang bebas linear dari $\mathbf{y'}=A\mathbf{y}$.
  \end{teorema}

  \begin{catatan}
    Teorema \ref{thm:penyelesaian_nilai_eigen_kompleks} menunjukan bahwa penyelesaian $\mathbf{y_1}$ dan $\mathbf{y_2}$ sudah merupakan satu set penyelesaian yang bebas linear dari nilai eigen kompleks dan konjugatnya.
  \end{catatan}

  Sebelumnya pada Modul 11 Kegiatan Belajar 2, telah dijelaskan langkah-langkah penyelesaiannya untuk sistem PD linear homogen dimensi-$2$. Pada Modul 12 Kegiatan Belajar 2 ini akan dibahas secara lanjut langkah-langkah yang sama namun dengan kasus yang lebih umum yaitu dimensi-$n$. 

  \begin{algoritma}\label{algoritma}
  Setelah mengetahui penyelesaian polinomial karakteristik dari sistem homogen (12.1) terdapat akar kompleks, maka langkah selanjutnya adalah mencari penyelesaian umum dari PD linear homogen dimensi-$n$. Berikut adalah langkah-langkahnya:
  \begin{enumerate}[label=Langkah \arabic*: ,leftmargin=*]
    \item Pisahkan himpunan nilai eigen menjadi dua kelompok, yaitu nilai eigen real dan kompleks. Misalkan terdapat $r$ nilai eigen real dan $s$ nilai eigen kompleks, maka $\{\lambda_1, \lambda_2, \ldots, \lambda_r\}$ dan $\{\mu_1, \mu_2, \ldots, \mu_s\}$ adalah himpunan nilai eigen real dan kompleks dari matriks $A$.
    \item Carilah solusi dari nilai-nilai eigen real dengan menggunakan metode yang telah dijelaskan pada KB 1. Sehingga didapatkan penyelesaian umum dari nilai eigen real adalah
    \begin{equation}\label{eq:penyelesaian_nilai_eigen_real}
      \mathbf{x}_\mathcal{R}=c_1\mathbf{x_1}+c_2\mathbf{x_2}+\cdots+c_r\mathbf{x_r}
    \end{equation}
    \item Selanjutnya untuk nilai eigen kompleks, dapat kita ambil sepasang $\mu = \alpha + i\beta$ dan $\overline{\mu} = \alpha - i\beta$. Kemudian cukup cari vektor eigen $\bfxi$ yang berkoresponden dengan nilai eigen $\mu$ menggunakan $(A-\mu I)\bfxi = \mathbf{0}$
    \item Dapatkan bagian real dan imajinernya dengan menuliskannya sebagai $\bfxi = \mathbf{u} + i\mathbf{v}$.
    \item Selanjutnya kita bisa dapatkan dua penyelesaian real untuk nilai eigen kompleks yaitu
    \begin{equation}\label{eq:penyelesaian_vektor_eigen}
      \begin{split}
        \mathbf{z}_1(t) &= e^{\alpha t}(\mathbf{u}\cos(\beta t) - \mathbf{v}\sin(\beta t))\\
        \mathbf{z}_2(t) &= e^{\alpha t}(\mathbf{u}\sin(\beta t) + \mathbf{v}\cos(\beta t))
      \end{split}
    \end{equation}
    Dimana sudah merupakan penyelesaian dari pasangan nilai eigen $r$ dan $\overline{r}$ sesuai dengan Teorema \ref{thm:penyelesaian_nilai_eigen_kompleks}.
    \item Jika terdapat vektor eigen kompleks lainnya, maka ulangi langkah 3 sampai 5 untuk mendapatkan semua penyelesaian nilai eigen kompleks lain. Jika tidak ada lagi, maka penyelesaian umumnya adalah kombinasi linear dari \eqref{eq:penyelesaian_nilai_eigen_real} dan penyelesaian nilai eigen kompleks.
    \begin{equation}
      \mathbf{x}(t) = \mathbf{x}_\mathcal{R}+ c_1\mathbf{z}_1(t) + c_2\mathbf{z}_2(t) + \cdots + c_s\mathbf{z}_s(t).
    \end{equation}
  \end{enumerate}
  \end{algoritma}

  \begin{catatan}
    Pada langkah 3, jikalau kita menggunakan vektor eigen $\overline{\bfxi}=\mathbf{u}-i\mathbf{v}$ yang berkoresponden dengan nilai eigen $\overline{\mu} = \alpha - i\beta$, maka penyelesaian umumnya menjadi
    \begin{equation}\label{eq:penyelesaian_vektor_eigen_konjugat}
      \begin{split}
        \mathbf{z}_1(t) &= e^{\alpha t}(\mathbf{u}\cos(-\beta t) - (-\mathbf{v})\sin(-\beta t))\\
        &= e^{\alpha t}(\mathbf{u}\cos(\beta t) - \mathbf{v}\sin(\beta t))\\
        \mathbf{z}_2(t) &= e^{\alpha t}(\mathbf{u}\sin(-\beta t) + (-\mathbf{v})\cos(-\beta t))\\
        &= -e^{\alpha t}(\mathbf{u}\sin(\beta t) + \mathbf{v}\cos(\beta t))
      \end{split}
    \end{equation}
    Dapat dilihat bahwa penyelesaian \eqref{eq:penyelesaian_vektor_eigen} dan \eqref{eq:penyelesaian_vektor_eigen_konjugat} hanya berbeda tanda pada $\mathbf{z}_2(t)$. Namun karena penyelesaian umum mencakup sebuah sembarang konstanta $c_2$, maka perbedaan tanda tersebut tidak mempengaruhi penyelesaian umumnya.
  \end{catatan}

  \begin{catatan}
    Jika terdapat syarat awal $\mathbf{x}(x_0)$, maka subtitusi syarat awal tersebut ke dalam penyelesaian umum yang telah didapatkan.
    \begin{equation}\label{eq:penyelesaian_umum_syarat_awal}
      \mathbf{x}(x_0) = c_1\mathbf{x}_1(x_0) + c_2\mathbf{x}_2(x_0) + \cdots + c_n\mathbf{x}_n(x_0)
    \end{equation}
    Kemudian selesaikan persmaan \eqref{eq:penyelesaian_umum_syarat_awal} untuk mencari konstanta $c_1, c_2, \ldots, c_n$. 
  \end{catatan}
  
  \begin{contoh}
    Tentukan penyelesaian umum dari 
    \begin{equation}\label{eq:contoh_pd_trench_2}
      \mathbf{y'} = \begin{pmatrix}
        1 & -1 & -2\\
        1 & 3 & 2\\
        1 & -1 & 2
      \end{pmatrix}\mathbf{y}.
    \end{equation}
    \penyelesaian
    Polinomial karakteristik dari matriks \eqref{eq:contoh_pd_trench_2} adalah
    \begin{equation}
      \begin{vmatrix}
        1-\lambda & -1 & -2\\
        1 & 3-\lambda & 2\\
        1 & -1 & 2-\lambda
      \end{vmatrix} = -(\lambda-2)((\lambda-2)^2+4).
    \end{equation}
    Nilai eigen dari $A$ adalah $\lambda_1 = 2, \, \lambda_2 = 2+2i$ dan $\lambda_3 = 2-2i$. Karena terdapat nilai eigen kompleks, maka akan digunakan Algoritma \ref{algoritma} untuk mencari penyelesaiannya
    \begin{enumerate}[label=Langkah \arabic*: ,leftmargin=*]
      \item $\lambda_1 = 2$ adalah nilai eigen real dan $\{\lambda_2, \lambda_3\}$ adalah himpunan nilai eigen kompleks.
      \item Untuk menentukan penyelesaian dari nilai eigen real, kita cari matriks \textit{augmented} dari $(A-2I)\mathbf{x} = \mathbf{0}$.
      \begin{equation*}
        \begin{pmatrix}
          -1 & -1 & -2 &:& 0\\
          1 & 1 & 2 &:& 0\\
          1 & -1 & 0 &:& 0
        \end{pmatrix} \sim \begin{pmatrix}
          1 & 0 & 1 &:& 0\\
          0 & 1 & 1 &:& 0\\
          0 & 0 & 0 &:& 0
        \end{pmatrix}.
      \end{equation*}
      Diperoleh $x_1 = -x_3$ dan $x_2 = -x_3$. Ambil $x_3 = 1$ sehingga
      \[\mathbf{x}_1 = \begin{pmatrix}-1\\-1\\1\end{pmatrix}.\]
      Jadi
      \[\mathbf{y}_1 = \begin{pmatrix}-1\\-1\\1\end{pmatrix}e^{2t}\]
      adalah penyelesaian untuk nilai eigen real dari \eqref{eq:contoh_pd_trench_2}. 

      \item Selanjutnya tersisa sepasang nilai eigen kompleks $\lambda_2$ dan $\lambda_3$. Matriks \textit{augmented} dari $(A-\lambda_2I)\mathbf{x} = \mathbf{0}$ adalah
      \begin{equation*}
        \begin{pmatrix}
          -1-2i & -1 & -2 &:& 0\\
          1 & 1-2i & 2 &:& 0\\
          1 & -1 & -2i &:& 0
        \end{pmatrix} \sim \begin{pmatrix}
          1 & 0 & -i &:& 0\\
          0 & 1 & i &:& 0\\
          0 & 0 & 0 &:& 0
        \end{pmatrix}.
      \end{equation*}
      Sehingga $x_1 = ix_3$ dan $x_2 = -ix_3$. Ambil $x_3 = 1$ maka
      \[\mathbf{x}_2 = \begin{pmatrix}i\\-i\\1\end{pmatrix}.\]
      \item Perhatikan bahwa $\mathbf{x}_2 = \begin{pmatrix}i\\-i\\1\end{pmatrix}=\begin{pmatrix}0\\0\\1\end{pmatrix}+i\begin{pmatrix}1\\-1\\0\end{pmatrix}$, maka $\mathbf{u}= \begin{pmatrix}0\\0\\1\end{pmatrix}$ dan $\mathbf{v} = \begin{pmatrix}1\\-1\\0\end{pmatrix}$.
      \item Dua penyelesaian untuk nilai eigen kompleks adalah
      \begin{align*}
        \mathbf{y}_2(t) &= e^{2t}\left(\begin{pmatrix}0\\0\\1\end{pmatrix}\cos(2t) - \begin{pmatrix}1\\-1\\0\end{pmatrix}\sin(2t)\right)=\begin{pmatrix}-\sin(2t)\\\sin(2t)\\\cos(2t)\end{pmatrix}
      \end{align*}
      dan
      \begin{align*}
        \mathbf{y}_3(t) &= e^{2t}\left(\begin{pmatrix}0\\0\\1\end{pmatrix}\sin(2t) + \begin{pmatrix}1\\-1\\0\end{pmatrix}\cos(2t)\right)=\begin{pmatrix}\cos(2t)\\-\cos(2t)\\\sin(2t)\end{pmatrix}
      \end{align*}
      \item Sehingga penyelesaian umum untuk \eqref{eq:contoh_pd_trench_2} adalah
      \[\mathbf{y}=c_1\begin{pmatrix}-1\\-1\\1\end{pmatrix}e^{2t} + c_2e^{2t}\begin{pmatrix}-\sin(2t)\\\sin(2t)\\\cos(2t)\end{pmatrix} + c_3e^{2t}\begin{pmatrix}\cos(2t)\\-\cos(2t)\\\sin(2t)\end{pmatrix}.\]
    \end{enumerate}
  \end{contoh}

  \begin{contoh}
    Carilah penyelesaian dari 
    \begin{equation}\label{eq:contoh_pd_trench_1}
      \mathbf{y'} = \begin{pmatrix}
        -5 & 5 & 4\\
        -8 & 7 & 6\\
        1 & 0 & 0
      \end{pmatrix}\mathbf{y}
    \end{equation}
    Dengan syarat awal $\mathbf{y}(0) = \begin{pmatrix}0&3&1\end{pmatrix}^T$.\\
    \penyelesaian
    Polinomial karakteristiknya dari matriks $A$ pada \eqref{eq:contoh_pd_trench_1} adalah
    \begin{equation}
      \begin{vmatrix}
        -5-\lambda & 5 & 4\\
        -8 & 7-\lambda & 6\\
        1 & 0 & -\lambda
      \end{vmatrix}=-(\lambda-2)(\lambda^2+1).
    \end{equation}
    Sehingga didapatkan nilai eigen dari $A$ adalah $\lambda_1 = 2, \lambda_2 = i$ dan $ \lambda_3 = -i$. Gunakan Algoritma \ref{algoritma} untuk mencari penyelesaiannya
    \begin{enumerate}[label=Langkah \arabic*: ,leftmargin=*]
      \item $\lambda_1$ adalah nilai eigen real dan sisanya adalah nilai eigen kompleks.
      \item Selanjutnya matriks \textit{augmented} dari $(A-\lambda_1I)\mathbf{x} = \mathbf{0}$ adalah
      \begin{equation*}
        \begin{pmatrix}
          -7 & 5 & 4 &:& 0\\
          -8 & 5 & 6 &:& 0\\
          1 & 0 & -2 &:& 0
        \end{pmatrix} \sim \begin{pmatrix}
          1 & 0 & -2 &:& 0\\
          0 & 1 & -2 & :&0\\
          0 & 0 & 0 & :&0
        \end{pmatrix}.
      \end{equation*}
      diperoleh $x_1 = x_2 = 2x_3$. Ambil saja $x_3 = 1$ akibatnya 
      \[\mathbf{x}_1 = \begin{pmatrix}2\\2\\1\end{pmatrix}.\]
      Jadi
      \[\mathbf{y}_1=\begin{pmatrix}2\\2\\1\end{pmatrix}e^{2t}\]
      adalah penyelesaian real dari \eqref{eq:contoh_pd_trench_1}. 
      \item Kemudian untuk nilai eigen kompleks, matriks \textit{augmented} dari $(A-iI)\mathbf{x} = \mathbf{0}$ adalah
      \begin{equation*}
        \begin{pmatrix}
          -5-i & 5 & 4 &:& 0\\
          -8 & 7-i & 6 &:& 0\\
          1 & 0 & -i &:& 0
        \end{pmatrix} \sim \begin{pmatrix}
          1 & 0 & -i &:& 0\\
          0 & 1 & 1-i & :&0\\
          0 & 0 & 0 & :&0
        \end{pmatrix}.
      \end{equation*}
      didapatkan $x_1 = ix_3$ dan $x_2 = (1-i)x_3$. Dengan mengambil $x_3 = 1$ diperoleh
      \[\mathbf{x_2} = \begin{pmatrix}i\\1-i\\1\end{pmatrix}.\]
      \item Dapat dilihat bahwa $\mathbf{u}= \begin{pmatrix}0\\1\\1\end{pmatrix}$ dan $\mathbf{v} = \begin{pmatrix}1\\-1\\0\end{pmatrix}$.
      \item Penyelesaian kedua dan ketiga masing-masing adalah
      \begin{align*}
        \mathbf{y}_2(t) &= e^{0t}\left(\begin{pmatrix}0\\-1\\1\end{pmatrix}\cos(t) - \begin{pmatrix}1\\1\\0\end{pmatrix}\sin(t)\right)=\begin{pmatrix}-\sin(t)\\-\cos(t)-\sin(t)\\\cos(t)\end{pmatrix}
      \end{align*}
      dan
      \begin{align*}
        \mathbf{y}_3(t) &= e^{0t}\left(\begin{pmatrix}0\\-1\\1\end{pmatrix}\sin(t) + \begin{pmatrix}1\\1\\0\end{pmatrix}\cos(t)\right)=\begin{pmatrix}\cos(t)\\-\sin(t)+\cos(t)\\\sin(t)\end{pmatrix}
      \end{align*}
      \item Didapatkan penyelesaian umumnya adalah
      \[\mathbf{y}=c_1\begin{pmatrix}2\\2\\1\end{pmatrix}e^{2t} + c_2\begin{pmatrix}-\sin(t)\\-\cos(t)-\sin(t)\\\cos(t)\end{pmatrix} + c_3\begin{pmatrix}\cos(t)\\-\sin(t)+\cos(t)\\\sin(t)\end{pmatrix}.\]
    \end{enumerate}
    Langkah terakhir adalah subtitusi syarat awal $\mathbf{y}(0)$ sehingga didapatkan
    \begin{align*}
      \begin{pmatrix}0\\3\\1\end{pmatrix} &= c_1\begin{pmatrix}2\\2\\1\end{pmatrix} + c_2\begin{pmatrix}0\\-1\\1\end{pmatrix} + c_3\begin{pmatrix}1\\1\\0\end{pmatrix}= \begin{pmatrix}2c_1+c_3\\2c_1-c_2+c_3\\c_1+c_2\end{pmatrix}
    \end{align*}
    kemudian selesaikan persamaan menggunakan sembarang metode yang telah anda diketahui. Pada contoh ini akan digunakan metode eliminasi Gauss-Jordan. 
    \begin{align*}
      \begin{pmatrix}
        2 & 0 & 1 &:& 0\\
        2 & -1 & 1 &:& 3\\
        1 & 1 & 0 &:& 1
      \end{pmatrix}
      &\sim \begin{pmatrix}
        1 & 0 & 0 &:& \frac{4}{3}\\
        0 & 1 & 0 &:& -\frac{1}{3}\\
        0 & 0 & 1 &:& -\frac{8}{3}
      \end{pmatrix}.
    \end{align*}
    Didapat $c_1 = \frac{4}{3}, c_2 = -\frac{1}{3}$ dan $c_3 = -\frac{8}{3}$. Jadi penyelesaian dari \eqref{eq:contoh_pd_trench_1} adalah
    \[\mathbf{y} = \frac{4e^{2t}}{3}\begin{pmatrix}2\\2\\1\end{pmatrix} - \frac{1}{3}\begin{pmatrix}-\sin(t)\\-\cos(t)-\sin(t)\\\cos(t)\end{pmatrix} - \frac{8}{3}\begin{pmatrix}\cos(t)\\-\sin(t)+\cos(t)\\\sin(t)\end{pmatrix}.\]
  \end{contoh}

  \begin{contoh}
    Carilah penyelesaian PD linear orde tiga berikut
    \begin{equation}\label{eq:contoh_pd_3}
      y''' - 4y'' + 6y' - 4y = 0.
    \end{equation}
    \penyelesaian
    Persamaan \eqref{eq:contoh_pd_3} dapat diselesaikan menggunakan sistem PD linear homogen dimensi-$3$. Misalkan
    \begin{align*}
      y_1 &= y \implies y_1' = y' = y_2\\
      y_2 &= y' \implies y_2' = y'' = y_3\\
      y_3 &= y''\implies y_3' = y''' = 4y'' - 6y' + 4y.
    \end{align*}
    Dengan demikian didapatkan sistem sebagai berikut
    \begin{equation}\label{eq:contoh_pd_3_sistem}
      \begin{pmatrix}
        y_1'\\
        y_2'\\
        y_3'
      \end{pmatrix} = \begin{pmatrix}
        0 & 1 & 0\\
        0 & 0 & 1\\
        4 & -6 & 4
      \end{pmatrix}\begin{pmatrix}
        y_1\\
        y_2\\
        y_3
      \end{pmatrix}.
    \end{equation}
    Dari matriks \eqref{eq:contoh_pd_3_sistem} didapatkan nilai eigen $\lambda_1 = 2, \lambda_2 = 1+i$ dan $\lambda_3 = 1-i$. Sama seperti contoh sebelumnya, gunakan Algoritma \ref{algoritma} 
    \begin{enumerate}[label=Langkah \arabic*: ,leftmargin=*]
      \item $\lambda_1 = 2$ adalah nilai eigen real dan $\{\lambda_2, \lambda_3\}$ adalah nilai eigen kompleks.
      \item Untuk nilai eigen real, dapat dicek bahwa 
      \[\mathbf{y}_1(x) = e^{2x}\begin{pmatrix}1\\2\\4\end{pmatrix}\]
      adalah penyelesaian dari nilai eigen real.
      \item Tinjau untuk $\lambda_2 = 1+i$ memiliki eigen vektor
      \[\bfxi_2 = \begin{pmatrix}i\\1+i\\2\end{pmatrix}=\begin{pmatrix}0\\1\\2\end{pmatrix}+i\begin{pmatrix}1\\1\\0\end{pmatrix}.\]
      \item Didapat $\mathbf{u}= \begin{pmatrix}0\\1\\2\end{pmatrix}$ dan $\mathbf{v} = \begin{pmatrix}1\\1\\0\end{pmatrix}$.
      \item Penyelesaian untuk nilai eigen $\lambda_2$ dan $\lambda_3$ adalah
      \begin{align*}
        \mathbf{y}_2(x) &= e^{x}\left(\begin{pmatrix}0\\1\\2\end{pmatrix}\cos(x) - \begin{pmatrix}1\\1\\0\end{pmatrix}\sin(x)\right)=\begin{pmatrix}-\sin(x)\\\cos(x)-\sin(x)\\2\cos(x)\end{pmatrix}e^x\\
        \mathbf{y}_3(x) &= e^{x}\left(\begin{pmatrix}0\\1\\2\end{pmatrix}\sin(x) + \begin{pmatrix}1\\1\\0\end{pmatrix}\cos(x)\right)=\begin{pmatrix}\cos(x)\\\sin(x)+\cos(x)\\2\sin(x)\end{pmatrix}e^x
      \end{align*}
      \item Penyelesaian umum dari \eqref{eq:contoh_pd_3} adalah
      \[\mathbf{y}(x) = c_1e^{2x}\begin{pmatrix}1\\2\\4\end{pmatrix} + c_2e^x\begin{pmatrix}-\sin(x)\\\cos(x)-\sin(x)\\2\cos(x)\end{pmatrix} + c_3e^x\begin{pmatrix}\cos(x)\\\sin(x)+\cos(x)\\2\sin(x)\end{pmatrix}.\]
    \end{enumerate}
    Ingat bahwa $\mathbf{y}(x)=\begin{pmatrix}y(x)\\y'(x)\\y''(x)\end{pmatrix}$ artinya penyelesaian PD orde tiga ada pada baris pertama penyelesaian umum. Jadi penyelesaian dari \eqref{eq:contoh_pd_3} adalah
    \[y(x) = c_1e^{2x} + c_2e^x\cos(x) + c_3e^x\sin(x).\] 
  \end{contoh}

  \begin{contoh}
    Diberikan sistem PD linear homogen dimensi-$4$ berikut
    \begin{equation}\label{eq:contoh_pd_4}
      \mathbf{x'}=\begin{pmatrix}
        0 & -1 & 0 & 0\\
        1 & 0 & 0 & 0\\
        0 & 0 & 0 & 2\\
        0 & 0 & -2 & 0 
      \end{pmatrix}\mathbf{x}
    \end{equation}
    Tentukan penyelesaian umum dari sistem PD tersebut.\\
    \penyelesaian
    Nilai eigen dari matriks \eqref{eq:contoh_pd_4} adalah
    \[\lambda_1 = i, \quad \lambda_2 = -i,\quad \lambda_2 = 2i, \quad \lambda_2 = -2i\]
    Ternyata terdapat dua pasang nilai eigen kompleks yang saling konjugat. Gunakan Algoritma \ref{algoritma} untuk mencari penyelesaiannya
    \begin{enumerate}[label=Langkah \arabic*: ,leftmargin=*]
      \item Perhatikan bahwa semua nilai eigen $\lambda_1,...,\lambda_4$ adalah nilai eigen kompleks.
      \item Karena tidak ada nilai eigen real, maka langkah ini dilewati.
      \item Ambil sepasang nilai eigen kompleks pertama, yaitu $\lambda_1$ dan $\lambda_2$. Selanjutnya dengan menyelesaikan $(A-iI)\bfxi = \mathbf{0}$, diperoleh
      $\bfxi_1 = \begin{pmatrix}i\\1\\0\\0\end{pmatrix}$
      \item Selanjutnya definisikan $\mathbf{u} = \begin{pmatrix}0\\1\\0\\0\end{pmatrix}$ dan $\mathbf{v} = \begin{pmatrix}1\\0\\0\\0\end{pmatrix}$ sebagai bagian real dan imajiner dari $\bfxi_1$.
      \item Penyelesaian untuk nilai eigen $\lambda_1$ dan $\lambda_2$ adalah
      \begin{equation}\label{eq:contoh_pd_4_penyelesaian_1}
        \begin{split}
          \mathbf{x}_1(t) &= \left(\begin{pmatrix}0\\1\\0\\0\end{pmatrix}\cos(t) - \begin{pmatrix}1\\0\\0\\0\end{pmatrix}\sin(t)\right) = \begin{pmatrix}-\sin(t)\\ \cos(t)\\0\\0\end{pmatrix}\\
          \mathbf{x}_2(t) &= \left(\begin{pmatrix}0\\1\\0\\0\end{pmatrix}\sin(t) + \begin{pmatrix}1\\0\\0\\0\end{pmatrix}\cos(t)\right) = \begin{pmatrix}\cos(t)\\ \sin(t)\\0\\0\end{pmatrix}
        \end{split}
      \end{equation}
      \item Karena terdapat nilai eigen lain, yaitu $\lambda_3$ dan $\lambda_4$. Maka perlu untuk mencari vektor eigen $\bfxi_3$ dengan menyelesaikan $(A-2iI)\bfxi = \mathbf{0}$, sehingga diperoleh $\bfxi_3 = \begin{pmatrix}0\\0\\-i\\1\end{pmatrix}$.
      \item Diperoleh $\mathbf{u} = \begin{pmatrix}0\\0\\0\\1\end{pmatrix}$ dan $\mathbf{v} = \begin{pmatrix}0\\0\\-1\\0\end{pmatrix}$.
      \item Penyelesaian untuk nilai eigen $\lambda_3$ dan $\lambda_4$ adalah
      \begin{equation}\label{eq:contoh_pd_4_penyelesaian_2}
        \begin{split}
          \mathbf{x}_3(t) &= \left(\begin{pmatrix}0\\0\\0\\1\end{pmatrix}\cos(2t) - \begin{pmatrix}0\\0\\-1\\0\end{pmatrix}\sin(2t)\right) = \begin{pmatrix}0\\0\\\sin(2t)\\ \cos(2t)\end{pmatrix}\\
          \mathbf{x}_4(t) &= \left(\begin{pmatrix}0\\0\\0\\1\end{pmatrix}\sin(2t) + \begin{pmatrix}0\\0\\-1\\0\end{pmatrix}\cos(2t)\right) = \begin{pmatrix}0\\0\\-\cos(2t)\\ \sin(2t)\end{pmatrix}
        \end{split}
      \end{equation}
      \item Penyelesaian umum dari \eqref{eq:contoh_pd_4} adalah kombinasi linear dari \eqref{eq:contoh_pd_4_penyelesaian_1} dan \eqref{eq:contoh_pd_4_penyelesaian_2}, yaitu
      \[\mathbf{x}(t) = c_1\begin{pmatrix}-\sin(t)\\ \cos(t)\\0\\0\end{pmatrix} + c_2\begin{pmatrix}\cos(t)\\ \sin(t)\\0\\0\end{pmatrix} + c_3\begin{pmatrix}0\\0\\\sin(2t)\\ \cos(2t)\end{pmatrix} + c_4\begin{pmatrix}0\\0\\ -\cos(2t)\\ \sin(2t)\end{pmatrix}.\]
    \end{enumerate}
  \end{contoh}

  \newpage
  \noindent Untuk memperdalam pemahaman Anda mengenai pengertian sistem PD linear homogen dimensi-$n$ dan penyelesaiannya, kerjakanlah latihan berikut!
  \begin{enumerate}
    \item Diberikan sistem PD linear berikut
    \begin{align*}
      \mathbf{x'} = \begin{pmatrix}
        1 & 2 & -3\\
        0 & -3 & 2\\
        0 & k & 5
      \end{pmatrix}\mathbf{x}
    \end{align*}
    Tentukan nilai $k$ agar sistem PD tersebut memiliki nilai eigen kompleks yang bagian imajinernya tak nol.
    \item Carilah penyelesaian homogen dari PD linear berikut
    \begin{align*}
      \mathbf{x'} = \begin{pmatrix}
        1 & 0 & 0\\
        2 & 1 & -2\\
        3 & 2 & 1
      \end{pmatrix}\mathbf{x}
    \end{align*}

    \item Diberikan sebuah sistem PD linear homogen berikut
    \begin{equation*}
        \frac{dx}{dt}=-3x+2z,\quad
        \frac{dy}{dt}=x-y,\quad
        \frac{dz}{dt}=-2x-y
    \end{equation*}
    Tentukan penyelesaian umum dari sistem PD tersebut.
    
    \item Carilah penyelesaian umum sistem PD dimensi-$4$ berikut
    \[\mathbf{y'}=\begin{pmatrix}
      1 & -2 & 1 & -3\\
      0 & 5 & 0 & 0\\
      0 & 2 & 0 & -1\\
      1 & 0 & -2 & 0
    \end{pmatrix}\mathbf{y}\]
    
    \item Diberikan suatu sistem tiga pegas dengan dua buah massa yang dirumuskan dalam PD berikut
    \begin{equation*}
      \begin{split}
        m_1\frac{d^2x_1}{dt^2}&=-(k_1+k_2)x_1+k_2x_2\\
        m_2\frac{d^2x_2}{dt^2}&=k_2x_1-(k_2+k_3)x_2\\
      \end{split}
    \end{equation*}
    Jika $m_1=m_2=2,$ dan $k_1=k_2=k_3=6$. Tentukan penyelesaian umum dari sistem PD tersebut.
  \end{enumerate}
  \newpage
  \underline{JAWABAN LATIHAN 12.2}
  \begin{enumerate}
    \item Polinomial karakteristik matriks $A$ adalah 
    \begin{equation*}
      \begin{vmatrix}
        1-\lambda & 2 & -3\\
        0 & -3-\lambda & 2\\
        0 & k & 5-\lambda
      \end{vmatrix} = (1-\lambda)(\lambda^2-2\lambda-15-2k). 
    \end{equation*}
    Sehingga agar terdapat nilai eigen kompleks yang bagian imajinernya tak nol, maka haruslah diskriminan dari polinomial $\lambda^2-2\lambda-15-2k$ negatif. 
    \[\Delta = b^2-4ac = (-2)^2-4(1)(-15-2k)=64+8k < 0 \implies k < -8.\]
    Jadi nilai $k$ yang memenuhi adalah $\{x\in\R\,|\,x<-8\}$.

    \item Nilai eigen dari matriks adalah
    \[\lambda_1 = 1, \quad \lambda_2 = 1 -2i, \quad \lambda_3 = 1 + 2i.\]
    dan vektor eigen yang berkoresponden dengan nilai eigen tersebut adalah
    \[\bfxi_1 = \begin{pmatrix}2\\-3\\2\end{pmatrix} \quad \bfxi_2 = \begin{pmatrix}0\\-i\\1\end{pmatrix} \quad \bfxi_3 = \begin{pmatrix}0\\i\\1\end{pmatrix}.\]
    Selanjutnya didapatkan masing-masing penyelesaian $\mathbf{x}_1, \mathbf{x}_2$ dan $\mathbf{x}_3$ sebagai berikut
    \begin{align*}
      \mathbf{x}_1(t) &= \begin{pmatrix}2\\-3\\2\end{pmatrix}e^t\\
      \mathbf{x}_2(t) &= e^t \left(\begin{pmatrix}0\\0\\1\end{pmatrix}\cos(2t) - \begin{pmatrix}0\\1\\0\end{pmatrix}\sin(2t)\right)=\begin{pmatrix}0\\-\sin(2t)\\ \cos(2t)\end{pmatrix}e^t\\
      \mathbf{x}_3(t) &= e^t \left(\begin{pmatrix}0\\0\\1\end{pmatrix}\sin(2t) + \begin{pmatrix}0\\1\\0\end{pmatrix}\cos(2t)\right)=\begin{pmatrix}0\\ \cos(2t)\\ \sin(2t)\end{pmatrix}e^t
    \end{align*}
    Jadi penyelesaian umum dari sistem PD tersebut adalah
    \[\mathbf{x}(t) = c_1\begin{pmatrix}2\\-3\\2\end{pmatrix}e^t + c_2\begin{pmatrix}0\\-\sin(2t)\\ \cos(2t)\end{pmatrix}e^t + c_3\begin{pmatrix}0\\ \cos(2t)\\ \sin(2t)\end{pmatrix}e^t.\]
    
    \item Misalkan $\mathbf{y} = \begin{pmatrix}x\\y\\z\end{pmatrix}$, maka sistem PD tersebut dapat dituliskan sebagai
    \begin{equation*}
      \mathbf{y'}=\begin{pmatrix}
        -3 & 0 & 2\\
        1 & -1 & 0\\
        -2 & -1 & 0
      \end{pmatrix}\mathbf{y}.
    \end{equation*}
    Selanjutnya nilai eigen dari matriks tersebut adalah
    \[\lambda_1 = -2, \quad \lambda_2 = -1+\sqrt{2}i, \quad \lambda_3 = -1-\sqrt{2}i\]
    dengan masing-masing vektor eigen
    \[\bfxi_1 = \begin{pmatrix}2\\-2\\1\end{pmatrix} \quad \bfxi_2 = \begin{pmatrix}2-\sqrt{2}i\\-1-\sqrt{2}i\\3\end{pmatrix} \quad \bfxi_3 = \begin{pmatrix}2+\sqrt{2}i\\-1+\sqrt{2}i\\3\end{pmatrix}.\]
    Dari vektor eigen $\bfxi_2$ dan $\bfxi_3$ dapat diperoleh penyelesaian kompleks
    \begin{align*}
      \mathbf{y}_2(t) &= e^{-t}\left(\begin{pmatrix}2\\-1\\3\end{pmatrix}\cos(\sqrt{2}t) - \begin{pmatrix}-\sqrt{2}\\-\sqrt{2}\\0\end{pmatrix}\sin(\sqrt{2}t)\right)=e^{-t}\begin{pmatrix}2\cos(\sqrt{2}t)+\sqrt{2}\sin(\sqrt{2}t)\\-\cos(\sqrt{2}t)+\sqrt{2}\sin(\sqrt{2}t)\\3\cos(\sqrt{2}t)\end{pmatrix}\\
      \mathbf{y}_3(t) &= e^{-t}\left(\begin{pmatrix}2\\-1\\3\end{pmatrix}\sin(\sqrt{2}t) + \begin{pmatrix}-\sqrt{2}\\-\sqrt{2}\\0\end{pmatrix}\cos(\sqrt{2}t)\right)=e^{-t}\begin{pmatrix}2\sin(\sqrt{2}t)-\sqrt{2}\cos(\sqrt{2}t)\\-\sin(\sqrt{2}t)-\sqrt{2}\cos(\sqrt{2}t)\\3\sin(\sqrt{2}t)\end{pmatrix}
    \end{align*} 
    Sehingga penyelesaian umum dari sistem PD tersebut adalah
    \[\mathbf{y}(t) = c_1\begin{pmatrix}2\\-2\\1\end{pmatrix}e^{-2t} + c_2\begin{pmatrix}2\cos(\sqrt{2}t)+\sqrt{2}\sin(\sqrt{2}t)\\-\cos(\sqrt{2}t)+\sqrt{2}\sin(\sqrt{2}t)\\3\cos(\sqrt{2}t)\end{pmatrix}e^{-t} + c_3\begin{pmatrix}2\sin(\sqrt{2}t)-\sqrt{2}\cos(\sqrt{2}t)\\-\sin(\sqrt{2}t)-\sqrt{2}\cos(\sqrt{2}t)\\3\sin(\sqrt{2}t)\end{pmatrix}e^{-t}.\]

    \item Nilai eigen dari matriks adalah
    \[\lambda_1 = 1, \quad \lambda_2 = 5, \quad \lambda_3 = 1 + \sqrt{2}i \quad \lambda_4 = 1 - \sqrt{2}i\]
    dan vektor eigen yang berkoresponden dengan nilai eigen tersebut adalah
    \[\bfxi_1 = \begin{pmatrix}1\\0\\1\\1\end{pmatrix} \quad \bfxi_2 = \begin{pmatrix}2\\-9\\-4\\2\end{pmatrix} \quad \bfxi_3 = \begin{pmatrix}1-5\sqrt{2}i\\0\\\-1-\sqrt{2}i\\3\end{pmatrix} \quad \bfxi^{(4)} = \begin{pmatrix}1+5\sqrt{2}i\\0\\\-1+\sqrt{2}i\\3\end{pmatrix}.\]
    Dengan mudah didapatkan penyelesaian untuk $\mathbf{y}_1$ dan $\mathbf{y}_2$ adalah
    \begin{align*}
      \mathbf{y}_1(t) = \begin{pmatrix}1\\0\\1\\1\end{pmatrix}e^t,\quad
      \mathbf{y}_2(t) = \begin{pmatrix}2\\-9\\-4\\2\end{pmatrix}e^{5t}
    \end{align*}
    dan untuk $\mathbf{y}_3$ dan $\mathbf{y}_4$ karena bilangan kompleks maka penyelesaiannya adalah
    \begin{align*}
      \mathbf{y}_3(t) &= e^t\left(\begin{pmatrix}1\\0\\1\\3\end{pmatrix}\cos(\sqrt{2}t) - \begin{pmatrix}-5\sqrt{2}\\0\\-\sqrt{2}\\0\end{pmatrix}\sin(\sqrt{2}t)\right)= e^t\begin{pmatrix}\cos(\sqrt{2}t)+5\sqrt{2}\sin(\sqrt{2}t)\\0\\\cos(\sqrt{2}t)+\sqrt{2}\sin(\sqrt{2}t)\\3\cos(\sqrt{2}t)\end{pmatrix}\\
      \mathbf{y}_4(t) &= e^t\left(\begin{pmatrix}1\\0\\1\\3\end{pmatrix}\sin(\sqrt{2}t) + \begin{pmatrix}-5\sqrt{2}\\0\\-\sqrt{2}\\0\end{pmatrix}\cos(\sqrt{2}t)\right)= e^t\begin{pmatrix}\sin(\sqrt{2}t)-5\sqrt{2}\cos(\sqrt{2}t)\\0\\\sin(\sqrt{2}t)-\sqrt{2}\cos(\sqrt{2}t)\\3\sin(\sqrt{2}t)\end{pmatrix}
    \end{align*}
    Jadi penyelesaian umum dari sistem PD tersebut adalah
    \[\mathbf{y}(t) = c_1\begin{pmatrix}1\\0\\1\\1\end{pmatrix}e^t + c_2\begin{pmatrix}2\\-9\\-4\\2\end{pmatrix}e^{5t} + c_3\begin{pmatrix}\cos(\sqrt{2}t)+5\sqrt{2}\sin(\sqrt{2}t)\\0\\\cos(\sqrt{2}t)+\sqrt{2}\sin(\sqrt{2}t)\\3\cos(\sqrt{2}t)\end{pmatrix}e^t + c_4\begin{pmatrix}\sin(\sqrt{2}t)-5\sqrt{2}\cos(\sqrt{2}t)\\0\\\sin(\sqrt{2}t)-\sqrt{2}\cos(\sqrt{2}t)\\3\sin(\sqrt{2}t)\end{pmatrix}e^t.\]
    
    \item Subtitusi nilai variabel yang diketahui sehingga diperoleh sistem PD
    \begin{align*}
        \frac{d^2x_1}{dt^2}&=-6x_1+3x_2\\
        \frac{d^2x_2}{dt^2}&=3x_1-6x_2
    \end{align*}
    Dapat kita misalkan $y_1=x_1,\,y_2=x_2,\,y_3=x_1',\,y_4=x_2'$, sehingga diperoleh informasi
    \begin{align*}
        y_1'&=x_1'=y_3,\\
        y_2'&=x_2'=y_4,\\
        y_3'&=x_1''=-6x_1+3x_2=-6y_1+3y_2,\\
        y_4'&=x_2''=3x_1-6x_2=3y_1-6y_2.
    \end{align*}
    atau jika dituliskan dalam bentuk matriks adalah sebagai berikut
    \begin{equation*}
        \begin{pmatrix}
            y_1'\\
            y_2'\\
            y_3'\\
            y_4'
        \end{pmatrix}=\begin{pmatrix}
            0 & 0 & 1 & 0\\
            0 & 0 & 0 & 1\\
            -6 & 3 & 0 & 0\\
            3 & -6 & 0 & 0
        \end{pmatrix}\begin{pmatrix}
            y_1\\
            y_2\\
            y_3\\
            y_4
        \end{pmatrix}.
    \end{equation*}
    sehingga untuk mencari penyelesaiannya dapat dilakukan dengan cara yang sama seperti contoh-contoh sebelumnya. Nilai eigen dari matriks tersebut adalah
    \[\lambda_1 = 3i, \quad \lambda_2 = -3i, \quad \lambda_3 = \sqrt{3}i, \quad \lambda_4 = -\sqrt{3}i\]
    dan vektor eigen yang berkoresponden dengan nilai eigen tersebut adalah
    \[\bfxi_1 = \begin{pmatrix}i\\-i\\-3\\3\end{pmatrix} \quad \bfxi_2 = \begin{pmatrix}-i\\i\\-3\\3\end{pmatrix} \quad \bfxi_3 = \begin{pmatrix}-\sqrt{3}i\\-\sqrt{3}i\\3\\3\end{pmatrix} \quad \bfxi_4 = \begin{pmatrix}\sqrt{3}i\\\sqrt{3}i\\3\\3\end{pmatrix}.\]
    Untuk masing-masing pasangan vektor eigen dapat dikelompokkan yaitu $\bfxi_1$ dan $\bfxi_2$ serta $\bfxi_3$ dan $\bfxi_4$. Penyelesaian untuk $\bfxi_1$ dan $\bfxi_2$ adalah
    \begin{align*}
      \mathbf{y}_1(t) &= e^{0}\left(\begin{pmatrix}0\\0\\-3\\3\end{pmatrix}\cos(3t) - \begin{pmatrix}1\\-1\\0\\0\end{pmatrix}\sin(3t)\right)=\begin{pmatrix}-\sin(3t)\\\sin(3t)\\-3\cos(3t)\\3\cos(3t)\end{pmatrix}\\
      \mathbf{y}_2(t) &= e^{0}\left(\begin{pmatrix}0\\0\\-3\\3\end{pmatrix}\sin(3t) + \begin{pmatrix}1\\-1\\0\\0\end{pmatrix}\cos(3t)\right)=\begin{pmatrix}\cos(3t)\\-\cos(3t)\\-3\sin(3t)\\3\sin(3t)\end{pmatrix}
    \end{align*} 
    dan untuk $\bfxi_3$ dan $\bfxi_4$ adalah
    \begin{align*}
      \mathbf{y}_3(t) &= e^{0}\left(\begin{pmatrix}0\\0\\3\\3\end{pmatrix}\cos(\sqrt{3}t) - \begin{pmatrix}-\sqrt{3}\\-\sqrt{3}\\0\\0\end{pmatrix}\sin(\sqrt{3}t)\right)=\begin{pmatrix}\sqrt{3}\sin(\sqrt{3}t)\\\sqrt{3}\sin(\sqrt{3}t)\\3\cos(\sqrt{3}t)\\3\cos(\sqrt{3}t)\end{pmatrix}\\
      \mathbf{y}_4(t) &= e^{0}\left(\begin{pmatrix}0\\0\\3\\3\end{pmatrix}\sin(\sqrt{3}t) + \begin{pmatrix}-\sqrt{3}\\-\sqrt{3}\\0\\0\end{pmatrix}\cos(\sqrt{3}t)\right)=\begin{pmatrix}-\sqrt{3}\cos(\sqrt{3}t)\\-\sqrt{3}\cos(\sqrt{3}t)\\3\sin(\sqrt{3}t)\\3\sin(\sqrt{3}t)\end{pmatrix}
    \end{align*}
    Sehingga penyelesaian umum dari sistem PD tersebut adalah
    \[\mathbf{y}(t) = c_1\begin{pmatrix}-\sin(3t)\\\sin(3t)\\-3\cos(3t)\\3\cos(3t)\end{pmatrix} + c_2\begin{pmatrix}\cos(3t)\\-\cos(3t)\\-3\sin(3t)\\3\sin(3t)\end{pmatrix} + c_3\begin{pmatrix}\sqrt{3}\sin(\sqrt{3}t)\\\sqrt{3}\sin(\sqrt{3}t)\\3\cos(\sqrt{3}t)\\3\cos(\sqrt{3}t)\end{pmatrix} + c_4\begin{pmatrix}-\sqrt{3}\cos(\sqrt{3}t)\\-\sqrt{3}\cos(\sqrt{3}t)\\3\sin(\sqrt{3}t)\\3\sin(\sqrt{3}t)\end{pmatrix}.\]
    Karena yang diminta adalah penyelesaian untuk $x_1$ dan $x_2$, maka penyelesaian masing-masingnya adalah
    \begin{align*}
      x_1(t) &= -c_1\sin(3t) + c_2\cos(3t) + c_3\sqrt{3}\sin(\sqrt{3}t) - c_4\sqrt{3}\cos(\sqrt{3}t)\\
      x_2(t) &= c_1\sin(3t) - c_2\cos(3t) + c_3\sqrt{3}\sin(\sqrt{3}t)- c_4\sqrt{3}\cos(\sqrt{3}t)
    \end{align*}
    dengan $c_1, c_2, c_3, c_4$ sebarang konstanta.
  \end{enumerate}
  \newpage
  \textbf{RANGKUMAN}
  \begin{itemize}
    \item Solusi untuk setiap pasangan nilai eigen kompleks $\lambda$ dan $\overline{\lambda}$ adalah bagian real ($\mathbf{u}$) dan imajiner ($\mathbf{v}$) dari vektor eigennya ($\bfxi=\mathbf{u}+i\mathbf{v}$).
    \item Penyelesaian umum dari sistem PD linear homogen (12.1) yang memiliki nilai eigen kompleks adalah
    \[\mathbf{x}(t) = \mathbf{x}_\mathcal{R}+ c_1\mathbf{z}_1(t) + c_2\mathbf{z}_2(t) + \cdots + c_s\mathbf{z}_s(t).\]
    Dengan $\mathbf{x_\mathcal{R}}$ adalah penyelesaian untuk nilai eigen real dan $\mathbf{z}_1,\mathbf{z}_2,\dots,\mathbf{z}_n$ adalah penyelesaian untuk nilai eigen kompleks.
  \end{itemize}
  \newpage
  \noindent\textbf{TES FORMATIF}\\
  Pilihlah satu jawaban yang paling tepat!
  \begin{enumerate}
    \item Diberikan sistem PD linear homogen dimensi-$3$ berikut
    \[\mathbf{x'} = \begin{pmatrix}
      6 & 0 & -3\\
      -3 & 3 & 3\\
      1 & -2 & 6
    \end{pmatrix}\mathbf{x}.\]
    Banyaknya nilai eigen yang bagian imajinernya tak nol adalah\dots
    
    \begin{enumerate}[label=\Alph*.]
      \item $0$
      \item $1$
      \item $2$
      \item $3$
    \end{enumerate}

    \item Sistem PD berikut
    \begin{align*}
      \mathbf{y'} = \begin{pmatrix}
        1&2&-2\\
        0&2&-1\\
        1&0&0
      \end{pmatrix}\mathbf{y}
    \end{align*}
    memiliki salah satu penyelesaian yaitu\dots
    \begin{enumerate}[label=\Alph*.]
      \item $e^t\begin{pmatrix}1+2\cos t\\1+\cos t\\\cos t+\sin t\end{pmatrix}$
      \item $e^t\begin{pmatrix}2+2\cos t\\2+\cos t\\2+\cos t+\sin t\end{pmatrix}$
      \item $e^t\begin{pmatrix}\sin t+\cos t\\\cos t\\\sin t\end{pmatrix}$
      \item $e^t\begin{pmatrix}2\sin t\\1+\cos t\\\cos t+\sin t\end{pmatrix}$
    \end{enumerate}

    \item Diberikan sistem PD linear homogen sebagai berikut
    \begin{align*}
      \frac{dx}{dt}&=3x-4y-2z\\
      \frac{dy}{dt}&=-5x+7y-8z\\
      \frac{dz}{dt}&=-10x+13y-8z
    \end{align*}
    Penyelesaian umum dari sistem PD tersebut adalah\dots
    \begin{enumerate}[label=\Alph*.]
      \item $\begin{pmatrix}
        x\\y\\z
      \end{pmatrix}=c_1e^{-2t}\begin{pmatrix}
        1\\1\\\frac{1}{2}
      \end{pmatrix}+c_2e^{2t}\begin{pmatrix}
        \cos 3t+\sin 3t\\\sin 3t\\-\cos 3t
      \end{pmatrix}+c_3e^{2t}\begin{pmatrix}
        \cos 3t-\sin 3t\\\cos 3t\\\sin 3t
      \end{pmatrix}$
      \item $\begin{pmatrix}
        x\\y\\z
      \end{pmatrix}=c_1e^{-2t}\begin{pmatrix}
        2\\2\\1
      \end{pmatrix}+c_2e^{t}\begin{pmatrix}
        \cos 3t+\sin 3t\\\sin 3t\\-\cos 3t
      \end{pmatrix}+c_3e^{t}\begin{pmatrix}
        \cos 3t+\sin 3t\\\cos 3t\\\sin 3t
      \end{pmatrix}$
      \item $\begin{pmatrix}
        x\\y\\z
      \end{pmatrix}=c_1e^{-2t}\begin{pmatrix}
        2\\2\\1
      \end{pmatrix}+c_2e^{2t}\begin{pmatrix}
        \cos t+\sin t\\\sin t\\-\cos t
      \end{pmatrix}+c_3e^{2t}\begin{pmatrix}
        \cos t-\sin t\\\cos t\\\sin t
      \end{pmatrix}$
      \item $\begin{pmatrix}
        x\\y\\z
      \end{pmatrix}=c_1e^{-2t}\begin{pmatrix}
        1\\1\\\frac{1}{2}
      \end{pmatrix}+c_2e^{2t}\begin{pmatrix}
        \cos 3t\\\sin 3t\\-\cos 3t
      \end{pmatrix}+c_3e^{2t}\begin{pmatrix}
        \cos 3t\\\cos 3t\\\sin 3t
      \end{pmatrix}$
    \end{enumerate}
    
    \item Diberikan PD linear orde 3 sebagai berikut
    \begin{align*}
      y'''-9y''+25y'-25y=0
    \end{align*}
    dengan $y(0)=2, y'(0)=0, y''(0)=0$. Penyelesaian dari PD tersebut adalah\dots
    \begin{enumerate}[label=\Alph*.]
      \item $y=e^{5x}+2e^{2x}\sin x+e^{2x}\cos x$
      \item $y=e^{5x}+4e^{2x}\cos x-3e^{2x}\sin x$
      \item $y=e^{5x}-7e^{2x}\sin x+e^{2x}\cos x$
      \item $y=e^{5x}+e^{x}\cos x-7e^{x}\sin x$
    \end{enumerate}
    
    \item Perhatikan sistem PD dengan nilai awal berikut
    \begin{align*}
      \mathbf{x'}=\begin{pmatrix}
        1&1&2\\
        1&0&-1\\
        -1&-2&-1
      \end{pmatrix}\mathbf{x},\quad \mathbf{x}(0)=\begin{pmatrix}
        -1\\2\\2
      \end{pmatrix}.
    \end{align*}
    Penyelesaian sistem PD diatas adalah\dots
    \begin{enumerate}[label=\Alph*.]
      \item $\mathbf{x}(t)=e^{t}\begin{pmatrix}\sin t - \cos t\\2\cos t -\sin t\\2\cos t + \sin t\end{pmatrix}$
      \item $\mathbf{x}(t)=e^{-2t}\begin{pmatrix}-1\\1\\1\end{pmatrix}$
      \item $\mathbf{x}(t)=e^{-2t}\begin{pmatrix}-1\\1\\1\end{pmatrix}+e^t\begin{pmatrix}\sin t\\\sin t+\cos t\\\cos t-\sin t\end{pmatrix}$
      \item $\mathbf{x}(t)=2e^{-2t}\begin{pmatrix}-1\\1\\1\end{pmatrix}+e^t\begin{pmatrix}\cos t\\\sin t\\-\sin t\end{pmatrix}$
    \end{enumerate}

    \item Diberikan PD linear orde 3 sebagai berikut
    \begin{equation*}
      x'''-7x''+18x'-12x=0
    \end{equation*}
    Penyelesaian umum untuk $x''$ adalah\dots
    \begin{enumerate}[label=\Alph*.]
      \item $x''(t)=c_1+c_2e^{3t}\cos \sqrt{3}t+c_3e^{3t}\sin \sqrt{3}t$
      \item $x''(t)=c_1e^t+c_2e^{3t}\cos 3t+c_3e^{3t}\sin 3t$
      \item $x''(t)=c_1e^{2t}+c_2e^{t}\cos \sqrt{3}t+c_3e^{t}\sin \sqrt{3}t$
      \item $x''(t)=c_1e^t+c_2e^{3t}\cos \sqrt{3}t+c_3e^{3t}\sin \sqrt{3}t$
    \end{enumerate}

    \item Dibawah ini yang \textbf{bukan} merupakan penyelesaian dari sistem PD linear homogen dimensi-$4$ yang mempunyai nilai eigen kompleks konjugat adalah\dots
    \begin{enumerate}[label=\Alph*.]
      \item $\begin{pmatrix}
        2\cos t\\\cos t\\0\\0
      \end{pmatrix}e^{2t}$
      \item $\begin{pmatrix}
        \frac{1}{2}\cos t\\0\\2\cos t\\\sin t
      \end{pmatrix}e^t$
      \item $\begin{pmatrix}
        2\cos t+\sin t\\2\sin t\\2\cos t\\2\sin t
      \end{pmatrix}e^{-3t}$
      \item $\begin{pmatrix}
        2\cos t\\2\sin t\\-2\cos t\\-2\sin t
      \end{pmatrix}e^t$
    \end{enumerate}
    
    \item Salah satu penyelesaian untuk
    \begin{equation*}
      \mathbf{y'}=\begin{pmatrix}
        5&2&-1&1\\
        -3&2&2&0\\
        1&3&2&2\\
        0&0&0&2
      \end{pmatrix}\mathbf{y}
    \end{equation*}
    \begin{enumerate}[label=\Alph*.]
      \item $\mathbf{y}=\begin{pmatrix}
        1\\-1\\2\\0
      \end{pmatrix}e^t+\begin{pmatrix}
        2\\-4\\3\\5
      \end{pmatrix}e^{2t}+e^{4t}\begin{pmatrix}
        \sin t\\0\\\cos t\\0
      \end{pmatrix}$
      \item $\mathbf{y}=\begin{pmatrix}
        1\\-1\\2\\0
      \end{pmatrix}e^t+e^{4t}\begin{pmatrix}
        2\sin t\\2\cos t\\2\sin t+2\cos t\\0
      \end{pmatrix}$
      \item $\mathbf{y}=\begin{pmatrix}
        1\\-1\\2\\0
      \end{pmatrix}e^t+\begin{pmatrix}
        2\\-4\\3\\5
      \end{pmatrix}e^{2t}+e^{t}\begin{pmatrix}
        2\sin t\\2\cos t\\2\sin t+2\cos t\\0
      \end{pmatrix}$
      \item $\mathbf{y}=\begin{pmatrix}
        1\\-1\\2\\0
      \end{pmatrix}e^t+\begin{pmatrix}
        2\\-4\\3\\1
      \end{pmatrix}e^{2t}$
    \end{enumerate}
    \item $\mathbf{y}=\begin{pmatrix}
      3\\5\\-4\\5
    \end{pmatrix}\sin(4t)+\begin{pmatrix}
      1\\5\\7\\0
    \end{pmatrix}\cos(4t)$ merupakan penyelesaian dari sistem $\mathbf{y'}=A\mathbf{y}$ dengan matriks $A=$...
    \begin{enumerate}[label=\Alph*.]
      \item $\begin{pmatrix}
        0&1&1&-1\\
        -1&0&3&-1\\
        -1&-3&0&-2\\
        1&1&2&0
      \end{pmatrix}$
      \item $\begin{pmatrix}
        0&1&1&-1\\
        -1&4&3&-1\\
        -1&-3&1&-2\\
        1&1&2&0
      \end{pmatrix}$
      \item $\begin{pmatrix}
        0&1&1&1\\
        1&0&3&1\\
        1&3&0&2\\
        1&1&2&0
      \end{pmatrix}$
      \item $\begin{pmatrix}
        1&-2&1&-1\\
        -1&2&3&-1\\
        -1&2&0&-2\\
        1&-2&2&0
      \end{pmatrix}$
    \end{enumerate}
    \item Diberikan PD orde 4 sebagai berikut
    \begin{equation*}
      y^{(4)}+7y''+10y=0
    \end{equation*}
    Dengan kondisi awal $y(0)=1,\, y'(0)=1,\, y''(0)=1,\, y'''(0)=3$. Penyelesaian untuk $y'''$ adalah\dots
    \begin{enumerate}[label=\Alph*.]
      \item $\frac{5\sqrt{5}}{3}\sin\sqrt{5}t-5\cos\sqrt{5}t-\frac{8\sqrt{2}}{3}\sin\sqrt{2}t-4\cos\sqrt{2}t$
      \item $\frac{5}{2}\sin\sqrt{5}t-2\cos\sqrt{5}t-\frac{8\sqrt{2}}{3}\sin\sqrt{2}t+2\cos\sqrt{2}t$
      \item $\frac{\sqrt{5}}{3}\sin\sqrt{5}t-4\cos\sqrt{5}t+8\sin\sqrt{2}t-4\cos\sqrt{2}t$
      \item $5\sin\sqrt{5}t-3\cos\sqrt{5}t-8\sqrt{2}\sin\sqrt{2}t-\cos\sqrt{2}t$
    \end{enumerate}
  \end{enumerate}
  KUNCI JAWABAN:
  \begin{enumerate}
    \item C
    \item B
    \item A
    \item C
    \item D
    \item D
    \item A
    \item B
    \item A
    \item A
  \end{enumerate}
\end{document}