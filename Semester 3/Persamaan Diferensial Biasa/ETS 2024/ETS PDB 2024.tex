\documentclass[10pt,openany,a4paper]{article}
\usepackage{graphicx} 
\usepackage{multirow}
\usepackage{enumitem}
\usepackage{amssymb}
\usepackage{amsmath}
\usepackage{amsthm}
\usepackage{xcolor}
\usepackage{fancyhdr}
\usepackage{geometry}
	\geometry{
		total = {160mm, 237mm},
		left = 30mm,
		right = 35mm,
		top = 35mm,
        bottom = 30mm,
        headheight=2cm
	}
\renewcommand{\headrulewidth}{0pt}

\graphicspath{{C:/Users/teoso/OneDrive/Documents/Tugas Kuliah/Template Math Depart/}}

\newcommand{\R}{\mathbb{R}}
\newcommand{\N}{\mathbb{N}}
\newcommand{\Z}{\mathbb{Z}}
\newcommand{\Q}{\mathbb{Q}}
\newcommand{\jawab}{\textbf{Solusi}:}

\newtheorem*{teorema}{Teorema}
\newtheorem*{definisi}{Definisi}

\pagestyle{fancy}
\fancyhf{}
\fancyhead[L]{\includegraphics[width=1.2cm]{ITS.png}}
\fancyhead[C]{\textbf{\MakeUppercase{Evaluasi Tengah Semester}\\ 
                \MakeUppercase{Semester Ganjil 2024/2025}\\ 
                \MakeUppercase{Departemen Matematika - FSAD ITS}\\ 
                \MakeUppercase{Program Sarjana}}}
\fancyhead[R]{\includegraphics[width=1.2cm]{M.png}}
\begin{document}
\begin{table}[h!]
  \begin{tabular}{l l l}
    Mata kuliah / kode & : & Persamaan Diferensial Biasa (SM234305) \\
    Hari / tanggal     & : & Selasa, 15 Oktober 2024                \\
    Sifat / waktu      & : & Tutup buku - 100 menit                 \\
    Dosen              & : & Drs. I Gusti Ngurah Rai Usadha, M.Si.  \\
                       &   & Dra. Nur Asiyah, M.Si.                 \\
                       &   & Dr. Tahiyatul Asfihani, S.Si, M.Si.    \\
                       &   & Amirul Hakam, S.Si., M.Si.
  \end{tabular}
\end{table}
\begin{center}
  \textbf{Aturan Pengerjaan:}
\end{center}
\begin{itemize}
  \item Dilarang bekerja sama dalam bentuk apa pun. Segala jenis pelanggaran (mencontek, kerja sama, dll) yang dilakukan saat ETS akan dikenakan sanksi pembatalan mata kuliah pada semester yang sedang berjalan.
  \item Dilarang membuka HP, kalkulator, dan sejenisnya. Bobot Nilai Setiap Soal Sama; Kerjakan yang lebih mudah dahulu menurut Anda.\\
\end{itemize}
\begin{enumerate}
  \item Diberikan persamaan diferensial sebagai berikut:
        $
          \dfrac{dy}{dx} = ky - ay^2
        $
        dengan $k, a$ adalah konstanta.

        \begin{enumerate}
          \item Identifikasi jenis persamaan diferensial tersebut.
          \item Dapatkan penyelesaian umum PD tersebut.
          \item Jelaskan perilaku penyelesaian jika $x\to \infty$.
        \end{enumerate}

  \item Populasi tikus pada suatu tempat mengalami pertumbuhan sebesar 60\% per bulan. Di tempat itu juga terdapat kucing yang ada pada daerah tersebut, kucing-kucing tersebut memakan tikus 3 ekor tiap bulan.

        \begin{enumerate}
          \item Dapatkan model matematika dari perubahan tikus tiap bulan $(dp/dt)$.
          \item Dapatkan persamaan populasi $(p(t))$, jika diketahui populasi awal tikus atau $p(0)$ sama dengan 100.
        \end{enumerate}

  \item Dapatkan penyelesaian umum persamaan diferensial berikut:
        \[
          y'' + 9y = 9 \sec^2(3x), \quad 0 \le x \le \pi/6
        \]
        Petunjuk:
        $
          \dfrac{d}{dx}(\ln|\tan x+\sec x|) = \sec x
        $

  \item Diberikan PD linear tingkat dua dengan koefisien tidak konstan:
        \[
          x^2y'' - 3xy' + 4y = x^2-2x
        \]

        \begin{enumerate}
          \item Ubahlah PD tersebut menjadi PD dengan koefisien konstan.
          \item Tentukan penyelesaian umum PD tersebut.
        \end{enumerate}
\end{enumerate}
\end{document}