\documentclass[10pt,openany,a4paper]{article}
\usepackage{graphicx} 
\usepackage{multirow}
\usepackage{enumitem}
\usepackage{amssymb}
\usepackage{amsmath}
\usepackage{amsthm}
\usepackage{xcolor}
\usepackage{fancyhdr}
\usepackage{geometry}
	\geometry{
		total = {160mm, 237mm},
		left = 30mm,
		right = 35mm,
		top = 35mm,
        bottom = 30mm,
        headheight=2cm
	}
  \usepackage{hyperref}
\hypersetup{
    colorlinks=true,
    linkcolor=blue,
    filecolor=magenta,      
    urlcolor=cyan,
    pdftitle={Overleaf Example},
    pdfpagemode=FullScreen,
    }
\renewcommand{\headrulewidth}{0pt}

\graphicspath{{C:/Users/teoso/OneDrive/Documents/Tugas Kuliah/Template Math Depart/}}

\newcommand{\R}{\mathbb{R}}
\newcommand{\N}{\mathbb{N}}
\newcommand{\Z}{\mathbb{Z}}
\newcommand{\Q}{\mathbb{Q}}
\newcommand{\jawab}{\textbf{Solusi}:}

\newtheorem*{teorema}{Teorema}
\newtheorem*{definisi}{Definisi}

\pagestyle{fancy}
\fancyhf{}
\fancyhead[L]{\includegraphics[width=1.2cm]{ITS.png}}
\fancyhead[C]{\textbf{\MakeUppercase{Evaluasi Tengah Semester}\\ 
                \MakeUppercase{Semester Ganjil 2024/2025}\\ 
                \MakeUppercase{Departemen Matematika - FSAD ITS}\\ 
                \MakeUppercase{Program Sarjana}}}
\fancyhead[R]{\includegraphics[width=1.2cm]{M.png}}
\begin{document}
\begin{table}[h!]
  \begin{tabular}{l l l}
    Mata kuliah / kode & : & Persamaan Diferensial Biasa (SM234305) \\
    Hari / tanggal     & : & Selasa, 15 Oktober 2024                \\
    Sifat / waktu      & : & Tutup buku - 100 menit                 \\
    Dosen              & : & Drs. I Gusti Ngurah Rai Usadha, M.Si.  \\
                       &   & Dra. Nur Asiyah, M.Si.                 \\
                       &   & Dr. Tahiyatul Asfihani, S.Si, M.Si.    \\
                       &   & Amirul Hakam, S.Si., M.Si.
  \end{tabular}
\end{table}
\begin{center}
  \textbf{Aturan Pengerjaan:}
\end{center}
\begin{itemize}
  \item Dilarang bekerja sama dalam bentuk apa pun. Segala jenis pelanggaran (mencontek, kerja sama, dll) yang dilakukan saat ETS akan dikenakan sanksi pembatalan mata kuliah pada semester yang sedang berjalan.
  \item Dilarang membuka HP, kalkulator, dan sejenisnya. Bobot Nilai Setiap Soal Sama; Kerjakan yang lebih mudah dahulu menurut Anda.\\
\end{itemize}
\begin{enumerate}
  \item Diberikan persamaan diferensial sebagai berikut:
        $
          \dfrac{dy}{dx} = ky - ay^2
        $
        dengan $k, a$ adalah konstanta.

        \begin{enumerate}
          \item Identifikasi jenis persamaan diferensial tersebut.
          \item Dapatkan penyelesaian umum PD tersebut.
          \item Jelaskan perilaku penyelesaian jika $x\to \infty$.
        \end{enumerate}

  \item Populasi tikus pada suatu tempat mengalami pertumbuhan sebesar 60\% per bulan. Di tempat itu juga terdapat kucing yang ada pada daerah tersebut, kucing-kucing tersebut memakan tikus 3 ekor tiap bulan.

        \begin{enumerate}
          \item Dapatkan model matematika dari perubahan tikus tiap bulan $(dp/dt)$.
          \item Dapatkan persamaan populasi $(p(t))$, jika diketahui populasi awal tikus atau $p(0)$ sama dengan 100.
        \end{enumerate}

  \item Dapatkan penyelesaian umum persamaan diferensial berikut:
        \[
          y'' + 9y = 9 \sec^2(3x), \quad 0 \le x \le \pi/6
        \]
        Petunjuk:
        $
          \dfrac{d}{dx}(\ln|\tan x+\sec x|) = \sec x
        $

  \item Diberikan PD linear tingkat dua dengan koefisien tidak konstan:
        \[
          x^2y'' - 3xy' + 4y = x^2-2x
        \]

        \begin{enumerate}
          \item Ubahlah PD tersebut menjadi PD dengan koefisien konstan.
          \item Tentukan penyelesaian umum PD tersebut.
        \end{enumerate}
\end{enumerate}
\newpage
\fancyhead{}
\renewcommand{\headrulewidth}{1pt}
\fancyhead[L]{\textit{Solution By: \hyperlink{https://github.com/TetewHeroez}{Tetew}}}
\fancyfoot{}
% \fancyfoot[R]{\animategraphics[autoplay,loop,width=0.1\textwidth]{15}{Kuru Kuru Herta/kuru kuru-}{0}{5}}
\begin{center}
  \textbf{SOLUSI}
\end{center}
\begin{enumerate}
  \item \begin{enumerate}
          \item Persamaan Diferensial Bernoulli dengan $n=2$. Sekedar informasi, PD Bernoulli memiliki bentuk umum:
                \[
                  \dfrac{dy}{dx} + P(x)y = Q(x)y^n
                \]
          \item Misalkan $v = y^{1-2} = y^{-1}$, maka
                \[
                  \dfrac{dv}{dx} = -y^{-2}\dfrac{dy}{dx} = -\dfrac{1}{y^2}(ky - ay^2) = -kv + a
                \]
                sehingga diperoleh persamaan diferensial linear orde satu:
                \[
                  \dfrac{dv}{dx} + kv = a
                \]
                Faktor integrasi dari PD tersebut adalah $e^{\int k dx} = e^{kx}$, sehingga
                \[
                  e^{kx}v = \int ae^{kx} dx = \dfrac{a}{k}e^{kx} + C
                \]
                atau
                \[
                  v = \dfrac{a}{k} + Ce^{-kx}
                \]
                Kembali ke variabel $y$, diperoleh penyelesaian umumnya:
                \[
                  y = \dfrac{1}{\dfrac{a}{k} + Ce^{-kx}} = \dfrac{k e^{kx}}{a e^{kx} + Ck}
                \]
          \item Akan dibagi menjadi dua kasus, yaitu $k>0$ dan $k<0$.
                \begin{itemize}
                  \item Jika $k>0$, maka $\displaystyle\lim_{x\to\infty} e^{kx} = \infty$. Sehingga
                        \[
                          \lim_{x\to\infty} y = \lim_{x\to\infty} \dfrac{k e^{kx}}{a e^{kx} + Ck} = \lim_{x\to\infty} \dfrac{k}{a + Ck e^{-kx}} = \dfrac{k}{a+0} = \dfrac{k}{a}
                        \]
                  \item Jika $k<0$, maka $\displaystyle\lim_{x\to\infty} e^{kx} = 0$. Sehingga
                        \[
                          \lim_{x\to\infty} y = \lim_{x\to\infty} \dfrac{k e^{kx}}{a e^{kx} + Ck} = 0
                        \]
                \end{itemize}
        \end{enumerate}
  \item \begin{enumerate}
          \item Misalkan $p$ adalah populasi tikus pada waktu $t$ (dalam bulan), artinya $dp/dt$ adalah perubahan populasi tikus tiap bulan. Diketahui bahwa pertumbuhan populasi tikus adalah $3/5p$ (60\% per bulan) dan kucing memakan tikus sebanyak 3 ekor tiap bulan, sehingga bisa dituliskan sebagai berikut:
                \[
                  \dfrac{dp}{dt} = \frac{3}{5}p - 3
                \]
          \item Dapat kita tuliskan ulang persamaan diferensial tersebut menjadi
                \[
                  \dfrac{dp}{dt} - \frac{3}{5}p = -3
                \]
                Faktor integrasi dari PD tersebut adalah $e^{\int -3/5 dt} = e^{-3t/5}$, sehingga
                \[
                  e^{-3t/5}p = \int -3e^{-3t/5} dt = 5e^{-3t/5} + C
                \]
                atau
                \[
                  p = 5 + Ce^{3t/5}
                \]
                Dengan kondisi awal $p(0) = 100$, diperoleh
                \[
                  100 = 5 + C \implies C = 95
                \]
                sehingga persamaan populasi tikus menjadi
                \[
                  p(t) = 5 + 95e^{3t/5}
                \]
        \end{enumerate}
  \item Pertama kita cari penyelesaian homogen dari persamaan diferensial tersebut, yaitu
        \[
          y'' + 9y = 0
        \]
        dengan karakteristik persamaan $r^2 + 9 = 0$, sehingga diperoleh $r = \pm 3i$. Dengan demikian, penyelesaian homogen dari PD tersebut adalah
        \[
          y_h = C_1 \cos 3x + C_2 \sin 3x
        \]
        Selanjutnya, kita cari penyelesaian partikular dari PD tersebut. Dari petunjuk yang diberikan, kita gunakan metode variasi parameter. Misalkan
        \[
          y_p = L_1(x) \cos 3x + L_2(x) \sin 3x
        \]
        dengan syarat
        \begin{align*}
          L_1'(x) \cos 3x + L_2'(x) \sin 3x    & = 0            \\
          -3L_1'(x) \sin 3x + 3L_2'(x) \cos 3x & = 9 \sec^2(3x)
        \end{align*}
        persamaan di atas dapat dituliskan dalam bentuk matriks sebagai berikut:
        \[
          \begin{bmatrix}
            \cos 3x   & \sin 3x  \\
            -3\sin 3x & 3\cos 3x
          \end{bmatrix}
          \begin{bmatrix}
            L_1'(x) \\
            L_2'(x)
          \end{bmatrix}
          =
          \begin{bmatrix}
            0 \\
            9 \sec^2(3x)
          \end{bmatrix}
        \]
        Dengan menggunakan aturan Cramer, diperoleh
        \begin{align*}
          L_1'(x) & = \dfrac{
            \begin{vmatrix}
              0           & \sin 3x  \\
              9\sec^2(3x) & 3\cos 3x
            \end{vmatrix}
          }{
            \begin{vmatrix}
              \cos 3x   & \sin 3x  \\
              -3\sin 3x & 3\cos 3x
            \end{vmatrix}
          } = \dfrac{-9\sin 3x \sec^2(3x)}{3} = -3\tan 3x \sec 3x \\
          L_2'(x) & = \dfrac{
            \begin{vmatrix}
              \cos 3x   & 0           \\
              -3\sin 3x & 9\sec^2(3x)
            \end{vmatrix}
          }{
            \begin{vmatrix}
              \cos 3x   & \sin 3x  \\
              -3\sin 3x & 3\cos 3x
            \end{vmatrix}
          } = \dfrac{9\cos 3x \sec^2(3x)}{3} = 3\sec 3x
        \end{align*}
        Selanjutnya, kita integralkan $L_1'(x)$ dan $L_2'(x)$:
        \begin{align*}
          L_1(x) & = \int -3\tan 3x \sec 3x dx = -\int \dfrac{d}{dx}(\sec 3x) dx = -\sec 3x \\
          L_2(x) & = \int 3\sec 3x dx = \ln|\tan 3x + \sec 3x|
        \end{align*}
        sehingga penyelesaian partikular dari PD tersebut adalah
        \[
          y_p = -\sec 3x \cos 3x + \ln|\tan 3x + \sec 3x| \sin 3x = -1 + \ln|\tan 3x + \sec 3x| \sin 3x
        \]
        Dengan demikian, penyelesaian umum dari PD tersebut adalah
        \[
          y = C_1 \cos 3x + C_2 \sin 3x + \ln|\tan 3x + \sec 3x| \sin 3x -1
        \]

  \item \begin{enumerate}
          \item PD diatas merupakan PD Cauchy-Euler. Misalkan $x = e^t$, sehingga $t = \ln x$. Dengan demikian, diperoleh
                \[
                  \dfrac{dy}{dx} = \dfrac{dy}{dt} \cdot \dfrac{dt}{dx} = \dfrac{1}{x} \dfrac{dy}{dt}
                \]
                dan
                \[
                  \dfrac{d^2y}{dx^2} = \dfrac{d}{dx}\left(\dfrac{1}{x} \dfrac{dy}{dt}\right) = -\dfrac{1}{x^2} \dfrac{dy}{dt} + \dfrac{1}{x} \cdot \dfrac{1}{x} \cdot \dfrac{d^2y}{dt^2} = -\dfrac{1}{x^2} \dfrac{dy}{dt} + \dfrac{1}{x^2} \dfrac{d^2y}{dt^2}
                \]
                Substitusi ke dalam PD yang diberikan menghasilkan
                \[
                  x^2\left(-\dfrac{1}{x^2} \dfrac{dy}{dt} + \dfrac{1}{x^2} \dfrac{d^2y}{dt^2}\right) - 3x\left(\dfrac{1}{x} \dfrac{dy}{dt}\right) + 4y = x^2 - 2x
                \]
                atau
                \[
                  -\dfrac{dy}{dt} + \dfrac{d^2y}{dt^2} - 3\dfrac{dy}{dt} + 4y = e^{2t} - 2e^t
                \]
                sehingga diperoleh PD dengan koefisien konstan:
                \[
                  y'' - 4y' + 4y = e^{2t} - 2e^t
                \]
          \item Pertama kita cari penyelesaian homogen dari persamaan diferensial tersebut, yaitu
                \[
                  y'' - 4y' + 4y = 0
                \]
                dengan karakteristik persamaan $(r-2)^2 = 0$, sehingga diperoleh $r = 2$ (kembar). Dengan demikian, penyelesaian homogen dari PD tersebut adalah
                \[
                  y_h = (C_1 + C_2 t)e^{2t}
                \]
                Selanjutnya, kita cari penyelesaian partikular dari PD tersebut. Misalkan
                \[
                  y_p = Ae^{2t} + Be^t
                \]
                sehingga
                \[
                  y_p' = 2Ae^{2t} + Be^t
                \]
                dan
                \[
                  y_p'' = 4Ae^{2t} + Be^t
                \]
                Substitusi ke dalam PD menghasilkan
                \[
                  (4Ae^{2t} + Be^t) - 4(2Ae^{2t} + Be^t) + 4(Ae^{2t} + Be^t) = e^{2t} - 2e^t
                \]
                atau
                \[
                  (4A - 8A + 4A)e^{2t} + (B - 4B + 4B)e^t = e^{2t} - 2e^t
                \]
                sehingga diperoleh
                \[
                  0e^{2t} + Be^t = e^{2t} - 2e^t
                \]
                Dari persamaan di atas, diperoleh $B = -2$ dan tidak ada nilai untuk $A$ karena koefisien di depan $e^{2t}$ adalah 0. Oleh karena itu, kita dapat menyimpulkan bahwa
                \[
                  y_p = -2e^t
                \]
                Dengan demikian, penyelesaian umum dari PD tersebut adalah
                \[
                  y = (C_1 + C_2 t)e^{2t} - 2e^t
                \]
                Kembali ke variabel $x$, diperoleh
                \[
                  y = (C_1 + C_2 \ln x)x^2 - 2x
                \]
        \end{enumerate}
\end{enumerate}
\end{document}