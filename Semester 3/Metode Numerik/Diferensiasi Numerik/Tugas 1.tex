\documentclass[a4paper,12pt]{article}
\usepackage{amsmath}
\usepackage{amsfonts}
\usepackage{amssymb}
\usepackage{graphicx}
\usepackage{hyperref}
\usepackage{color}
\usepackage[table]{xcolor}
%\usepackage[T1]{fontenc}
\usepackage{etoolbox}
\usepackage{multicol}
\usepackage{multirow}
\usepackage{fancyhdr}
\usepackage{array}
\usepackage{animate}
\usepackage{amsthm}
\usepackage{caption}

\graphicspath{{C:/Users/teoso/OneDrive/Documents/Tugas Kuliah/Template Math Depart/}}

\newcommand{\jawab}{\textbf{Solusi}:}

\begin{document}
\fancyhead[L]{\textit{Teosofi Hidayah Agung}}
\fancyhead[R]{\textit{5002221132}}
\pagestyle{fancy}

Cari nilai turunan sebuah fungsi dengan forward-difference formula dan backward-difference formula serta nilai erornya.
\vspace*{-0.5cm}
\begin{multicols}{2}
    \[
    \begin{array}{|c|c|c|}
    \hline
    x & f(x) & f'(x) \\
    \hline
    -0.3 & 1.9507 & \dots\\
    -0.2 & 2.0421 & \dots\\
    -0.1 & 2.0601 & \dots\\
    \hline
    \end{array}
    \]

    \[
    \begin{array}{|c|c|c|}
    \hline
    x & f(x) & f'(x) \\
    \hline
    0.5 & 0.4794 & \dots\\
    0.6 & 0.5646 & \dots\\
    0.7 & 0.6442 & \dots\\
    \hline
    \end{array}
    \]
\end{multicols}
\noindent\jawab\\
Dari tabel, diketahui bahwa $h=0.1$ dan rumus turunan menggunakan forward-difference formula dan backward-difference formula masing-masing adalah sebagai berikut:
\begin{align}
    f'(x) &\approx \frac{f(x+h)-f(x)}{h} \label{eq:forward}\\
    f'(x) &\approx \frac{f(x)-f(x-h)}{h} \label{eq:backward}
\end{align}
\begin{itemize}
    \item Untuk $x = -0.3$, kita hanya dapat menggunakan forward-difference formula karena tidak ada nilai $f(x)$ yang lebih kecil dari $x$.
    \begin{equation*}
        f'(-0.3) \approx \frac{f(-0.3+0.1)-f(-0.3)}{0.1} = \frac{f(-2.0)-f(-0.3)}{0.1} = 0.914
    \end{equation*}
    \item Untuk $x = -0.2$, kita dapat menggunakan forward-difference formula dan backward-difference formula keduanya.
    \begin{align*}
        f'(-0.2) &\approx \frac{f(-0.2+0.1)-f(-0.2)}{0.1} = \frac{f(-0.1)-f(-0.2)}{0.1} = 0.18\\
        f'(-0.2) &\approx \frac{f(-0.2)-f(-0.2-0.1)}{0.1} = \frac{f(-0.2)-f(-0.3)}{0.1} = 0.914
    \end{align*}
    \item Untuk $x = -0.1$, kita hanya dapat menggunakan backward-difference formula karena tidak ada nilai $f(x)$ yang lebih besar dari $x$.
    \begin{equation*}
        f'(-0.1) \approx \frac{f(-0.1)-f(-0.1-0.1)}{0.1} = \frac{f(-0.1)-f(-0.2)}{0.1} = 0.18
    \end{equation*}
    \item Untuk $x = 0.5$, kita hanya dapat menggunakan forward-difference formula karena tidak ada nilai $f(x)$ yang lebih kecil dari $x$.
    \begin{equation*}
        f'(0.5) \approx \frac{f(0.5+0.1)-f(0.5)}{0.1} = \frac{f(0.6)-f(0.5)}{0.1} = 0.852
    \end{equation*}
    \item Untuk $x = 0.6$, kita dapat menggunakan forward-difference formula dan backward-difference formula keduanya.
    \begin{align*}
        f'(0.6) &\approx \frac{f(0.6+0.1)-f(0.6)}{0.1} = \frac{f(0.7)-f(0.6)}{0.1} = 0.796\\
        f'(0.6) &\approx \frac{f(0.6)-f(0.6-0.1)}{0.1} = \frac{f(0.6)-f(0.5)}{0.1} = 0.852
    \end{align*}
    \item Untuk $x = 0.7$, kita hanya dapat menggunakan backward-difference formula karena tidak ada nilai $f(x)$ yang lebih besar dari $x$.
    \begin{equation*}
        f'(0.7) \approx \frac{f(0.7)-f(0.7-0.1)}{0.1} = \frac{f(0.7)-f(0.6)}{0.1} = 0.796
    \end{equation*}
\end{itemize}
Selanjutnya untuk eror disini tidak dapat dihitung karena tidak ada nilai eksak dari turunan fungsi yang diberikan.
\end{document}