\documentclass[10pt,openany,a4paper]{article}
\usepackage{graphicx} 
\usepackage{multirow}
\usepackage{enumitem}
\usepackage{amssymb}
\usepackage{amsmath}
\usepackage{amsthm}
\usepackage{xcolor}
\usepackage{geometry}
	\geometry{
		total = {160mm, 237mm},
		left = 20mm,
		right = 20mm,
		top = 30mm,
		bottom = 30mm,
	}
\usepackage{fancyhdr}
\renewcommand{\headrulewidth}{0pt}
\renewcommand{\arraystretch}{1.1}
\pagestyle{fancy}

\graphicspath{{C:/Users/teoso/OneDrive/Documents/Tugas Kuliah/Template Math Depart/}}

\newcommand{\R}{\mathbb{R}}
\newcommand{\N}{\mathbb{N}}
\newcommand{\Z}{\mathbb{Z}}
\newcommand{\Q}{\mathbb{Q}}
\newcommand{\jawab}{\textbf{Solusi}:}

\newtheorem*{teorema}{Teorema}
\newtheorem*{definisi}{Definisi}

\begin{document}
\pagenumbering{gobble}
\begin{tabular}{r c l}
    \includegraphics[width=2cm]{ITS.png}
     &\begin{tabular}{|c|l|l|}
        \hline
        \multirow{3}{*}{{\Huge\textbf{EAS}}}&\textbf{Matakuliah}&Geometri Analitik (A,B,C,D)\\
        \cline{2-3}
        &\textbf{Semester}&1\\
        \cline{2-3}
        \multirow{3}{*}{{\large\textbf{GASAL}}}&\textbf{Kredit SKS}&3\\
        \cline{2-3}
        \multirow{3}{*}{{\large\textbf{2023/2024}}}&\textbf{Hari, Tanggal}&Jumat, 15 Desember 2023\\
        \cline{2-3}
        &\textbf{Waktu}&\textbf{100 menit}\\
        \cline{2-3}
        &\textbf{Dosen}&\begin{tabular}{l}
            Drs. I Gst Ngr Rai Usadha, M.Si.\\
            Dra, Wahyu Fistia Doctorina, M.Si.\\
            Drs. Komar Baihaqi, M.Si.\\
            DR. Mont Kistosil Fahim, S.Si, M.Si.
        \end{tabular}\\
        \hline
     \end{tabular}
     & 
     \includegraphics[width=2cm]{M.png}
     \\ 
\end{tabular}
\begin{center}
    
\end{center}
\begin{enumerate}
    \item Misal garis singgung dari hiperbola \(2x^2 - ky^2 = l\) di titik \((4, 2)\) dan melalui titik \((1, -2)\). Dapatkan nilai dari \(k\) dan \(l\).
    
    \item Diberikan irisan kerucut dengan persamaan derajat dua: \(5x^2 - 4xy + 2y^2 = 30\).
    \begin{enumerate}
        \item Lakukan suatu transformasi sehingga persamaan di atas menjadi persamaan dasar dari irisan kerucut pada sumbu koordinat baru. Sebutkan jenis transformasinya. Mengapa?
        \item Gambarkan kurvanya.
    \end{enumerate}
    
    \item Diketahui kurva permukaan pada \(\mathbb{R}^3\) dengan persamaan \(9x^2 + 9y^2 + 9z^2 - 18x - 36y - 72z + 72 = 0\).
    \begin{enumerate}
        \item Tentukan jenis kurva permukaan tersebut, titik pusat dan jari-jarinya, serta sketlah kurvanya.
        \item Tentukan persamaan pada trace \(xz\) (pada bidang \(y=0\)), serta sketlah kurvanya.
    \end{enumerate}
    
    \item 
    \begin{enumerate}
        \item Tunjukkan bahwa 3 titik \(A(-2,1,1)\), \(B(0,2,3)\) dan \(C(1,0,-1)\) berada pada satu bidang dalam \(\mathbb{R}^3\).
        \item Tentukan persamaan bidang yang melalui 3 titik tersebut.
    \end{enumerate}
\end{enumerate}
\begin{center}
    
\end{center}
\begin{center}
    
\end{center}
\begin{center}
    
\end{center}
\begin{center}
    
\end{center}
\begin{center}
    \textbf{== HARAP JUNJUNG TINGGI KEJUJURAN ==}
\end{center}
\newpage
\jawab
\begin{enumerate}
    \item Karena titik $(4,2)$ berada pada hiperbola $2x^2 - ky^2 = l$, maka didapatkan persamaan
    \begin{equation}
         4k + l = 32 \label{eq1}
    \end{equation}
    \item 
\end{enumerate}
\end{document}