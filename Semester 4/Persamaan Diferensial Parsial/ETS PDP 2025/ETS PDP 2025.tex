\documentclass{article}
\usepackage{amsmath,amssymb,amsfonts,amsthm}
\usepackage{multicol}
\usepackage{multirow}
\usepackage{mathtools}
\usepackage{soul}
\usepackage{hyperref}
\hypersetup{
    colorlinks=true,
    linkcolor=blue,
    filecolor=magenta,      
    urlcolor=cyan,
    pdftitle={Overleaf Example},
    pdfpagemode=FullScreen,
    }
\usepackage{color}
\usepackage[table]{xcolor}
\usepackage[T1]{fontenc}
\usepackage{etoolbox}
\usepackage{multicol}
\usepackage{multirow}
\usepackage{fancyhdr}
\usepackage{graphicx}
\usepackage{tcolorbox}
\usepackage{array}
\usepackage{amsthm}
\usepackage{titlesec}
\usepackage{tikz, tkz-euclide}
  \usetikzlibrary{arrows.meta,calc}
  \newcommand\rightAngle[4]{
  \pgfmathanglebetweenpoints{\pgfpointanchor{#2}{center}}{\pgfpointanchor{#1}{center}}
  \coordinate (tmpRA) at ($(#2)+(\pgfmathresult+45:#4)$);
  \draw[red!60!black,thick] ($(#2)!(tmpRA)!(#1)$) -- (tmpRA) -- ($(#2)!(tmpRA)!(#3)$);
}
\renewcommand{\baselinestretch}{1.2}

\titleformat*{\section}{\large\bfseries}
\titleformat*{\subsection}{\normalsize\bfseries}

\newtheorem{theorem}{Theorem}
\newtheorem*{teorema}{Teorema}
\newtheorem*{definisi}{Definisi}
\theoremstyle{definition}
\newtheorem*{bukti}{Bukti}

\newcommand{\Arg}{\text{Arg}}
\newcommand{\R}{\mathbb{R}}
\newcommand{\C}{\mathbb{C}}
\newcommand{\N}{\mathbb{N}}
\newcommand{\Z}{\mathbb{Z}}

\newtcolorbox{solution}[1][]{
    colback=blue!5!white, 
    colframe=blue!75!black,
    fonttitle=\bfseries, 
    colbacktitle=blue!85!black,
    title=Solusi,
    #1
}

\begin{document}
\fancyhead[L]{\textit{Teosofi Hidayah Agung}}
\fancyhead[R]{\textit{5002221132}}
\pagestyle{fancy}
\begin{enumerate}
  \item Diberikan PDP \( u_x - 2u = x \)
  \begin{itemize}
      \item[(a)] Dapatkan penyelesaian umum PDP tersebut.
      \item[(b)] Jika diberikan syarat batas: \( u(0,y) = y \). Dapatkan penyelesaian partikular PDP tersebut.
      \item[(c)] Jika diberikan syarat batas: \( u(x,0) = \frac{3}{4} - \frac{x}{2} \). Tunjukkan PDP tersebut tidak mempunyai penyelesaian.
  \end{itemize}

  \item Dapatkan penyelesaian PDP berikut dengan menggunakan metode Lagrange:
  \[
  y \frac{\partial u}{\partial x} + x \frac{\partial u}{\partial y} = xy
  \]

  \item Dapatkan penyelesaian PDP berikut dengan menggunakan metode karakteristik:
  \[
  u_x + u_y = -\frac{1}{2}u
  \]
  dengan kondisi awal \( u(x,0) = 1 \).

  \item Dapatkan penyelesaian PDP berikut dengan menggunakan metode pemisahan variabel:
  \[
  u_x + 2u_y = 0 \quad \text{jika} \quad u(0,y) = 3e^{-2y} - 4e^y
  \]
\end{enumerate}
\centering\textbf{\underline{SOLUSI}}
\begin{enumerate}
  \item \begin{enumerate}
    \item Kalikan seluruh persamaan dengan faktor integrasi \( e^{-2x} \):
    \[
    e^{-2x}u_x - 2e^{-2x}u = xe^{-2x}
    \quad \Rightarrow \quad
    \frac{d}{dx}(e^{-2x}u) = xe^{-2x}
    \]
    
    Integralkan kedua ruas:
    \[
    e^{-2x}u = \int xe^{-2x} \, dx
    \]
    Gunakan integrasi parsial:
    \[
    \int xe^{-2x} \, dx = -\frac{1}{2}xe^{-2x} - \frac{1}{4}e^{-2x} + f(y)
    \]
    
    Sehingga:
    \[
    e^{-2x}u = -\frac{1}{2}xe^{-2x} - \frac{1}{4}e^{-2x} + f(y)
    \]
    Kalikan dengan \( e^{2x} \):
    \[
    u(x,y) = f(y)e^{2x} - \frac{1}{2}x - \frac{1}{4}
    \]
    \item Substitusi \(x = 0\):
    \[
    u(0,y) = f(y) - \frac{1}{4}
    \]
    Sehingga:
    \[
    f(y) = y + \frac{1}{4}
    \]
    Penyelesaian partikular:
    \[
    u(x,y) = \left( y + \frac{1}{4} \right) e^{2x} - \frac{1}{2}x - \frac{1}{4}
    \]
    \item Substitusi \(y = 0\):
    \begin{align*}
      u(x,0) &= f(0)e^{2x} - \frac{1}{2}x - \frac{1}{4}\\
      \frac{3}{4} - \frac{x}{2} &= f(0)e^{2x} - \frac{1}{2}x - \frac{1}{4}\\
      f(0) = e^{2x}
    \end{align*}
    Hal ini kontradiksi karena fungsi $f$ bergantung pada $y$ saja. Sehingga tidak ada penyelesaian untuk syarat batas ini.
  \end{enumerate}
  \item Karakteristik:
  \[
  \frac{dx}{y} = \frac{dy}{x} = \frac{du}{xy}
  \]
  
  Dari \(\frac{dx}{y} = \frac{dy}{x}\), diperoleh:
  \[
  x\,dx = y\,dy
  \quad \Rightarrow \quad
  \frac{1}{2}x^2 = \frac{1}{2}y^2 + C
  \quad \Rightarrow \quad
  x^2 - y^2 = k
  \]
  
  Dari \(\frac{dx}{y} = \frac{du}{xy}\), didapatkan:
  \[
  du = x\,dx
  \quad \Rightarrow \quad
  u = \frac{1}{2}x^2 + f(x^2 - y^2)
  \]
  
  Jadi solusi umum:
  \[
  u(x,y) = \frac{1}{2}x^2 + f(x^2 - y^2)
  \]
  \item Karakteristik:
  \[
  \frac{dx}{1} = \frac{dy}{1} = \frac{du}{-\frac{1}{2}u}
  \]
  
  Dari \(\frac{dx}{1} = \frac{dy}{1}\):
  \[
  x - y = k
  \]
  
  Dari \(\frac{du}{-\frac{1}{2}u}\):
  \[
  \frac{du}{u} = -\frac{1}{2}ds
  \quad \Rightarrow \quad
  \ln u = -\frac{1}{2}s + C
  \quad \Rightarrow \quad
  u = A(k)e^{-\frac{1}{2}s}
  \]
  
  Sehingga:
  \[
  u(x,y) = f(x-y)e^{-\frac{1}{2}x}
  \]
  
  Gunakan syarat \(u(x,0) = 1\):
  \[
  f(x)e^{-\frac{1}{2}x} = 1
  \quad \Rightarrow \quad
  f(x) = e^{\frac{1}{2}x}
  \]
  
  Maka:
  \[
  u(x,y) = e^{-\frac{1}{2}y}
  \]
  \item 
  
  Misalkan solusi berbentuk:
  \[
  U(x,y) = X(x)Y(y)
  \]
  
  Maka persamaan menjadi:
  \[
  X'Y + 2XY' = 0 \Rightarrow \frac{X'}{X} = -2\frac{Y'}{Y}
  \]
  
  Pisahkan variabel:
  \[
  \frac{X'}{2X} = -\frac{Y'}{Y} = \lambda
  \]
  
  Maka diperoleh dua persamaan diferensial biasa (PDB):
  
  \begin{enumerate}
      \item $\displaystyle \frac{X'}{X} = 2\lambda \Rightarrow X(x) = C_1 e^{2\lambda x}$
      \item $\displaystyle \frac{Y'}{Y} = -\lambda \Rightarrow Y(y) = C_2 e^{-\lambda y}$
  \end{enumerate}
  
  Sehingga solusi umum:
  \[
  U(x,y) = C_1C_2 e^{2\lambda x} e^{-\lambda y} = A e^{\lambda(2x - y)}
  \]
  
  Dengan $A = C_1C_2$, maka superposisi solusi umum:
  \[
  U(x,y) = A_1 e^{\lambda_1(2x - y)} + A_2 e^{\lambda_2(2x - y)}
  \]
  
  Gunakan syarat batas \( u(0,y) = 3e^{-2y} - 4e^y \). Maka:
  \[
  U(0,y) = A_1 e^{-\lambda_1 y} + A_2 e^{-\lambda_2 y} = 3e^{-2y} - 4e^{y}
  \]
  
  Dapat dilihat bahwa:
  \[
  \lambda_1 = 2, \quad A_1 = 3; \qquad \lambda_2 = -1, \quad A_2 = -4
  \]
  
  Maka solusi partikular:
  \[
  U(x,y) = 3e^{2(2x - y)} - 4e^{-1(2x - y)} = 3e^{4x - 2y} - 4e^{-2x + y}
  \]

\end{enumerate}
\end{document}