\documentclass[a4paper,extrafontsizes, 9pt]{memoir}

\usepackage{amsmath, amssymb, amsfonts, amsthm}
\usepackage{multicol}
\usepackage{multirow}
\usepackage{lipsum}
\usepackage{mathtools}
\usepackage{pgfplots}
\usepackage{soul}
\usepackage{hyperref}
\usepackage{enumitem}
\usepackage{fancyhdr}

\pagestyle{empty}

\setlrmarginsandblock{1cm}{1cm}{*}
\setulmarginsandblock{1cm}{1cm}{*}
\checkandfixthelayout

\DeclareMathOperator{\Cov}{Cov}
\DeclareMathOperator{\Var}{Var}

\let\bf\textbf{}

\setlength{\parindent}{0pt}
\renewcommand{\arraystretch}{1.5}

\newcommand{\textbetweenrules}[2][.4pt]{%
  \par\vspace{\topsep}
  \noindent\makebox[\textwidth]{%
    \sbox0{#2}%
    \dimen0=.5\dimexpr\ht0+#1\relax
    \dimen2=-.5\dimexpr\ht0-#1\relax
    \leaders\hrule height \dimen0 depth \dimen2\hfill
    \quad #2\quad
    \leaders\hrule height \dimen0 depth \dimen2\hfill
  }\par\nopagebreak\vspace{\topsep}
}

\begin{document}
\footnotesize \textbf{{Cheatsheet -- Persamaan Diferensial Parsial}} \hfill \textit{Teosofi Hidayah Agung/5002221132}
	\begin{multicols}{3}			
        \subsection*{\small Klasifikasi PDP}
            Bentuk umum dari PDP linier order 2 dengan dua variabel adalah
\begin{equation*}
Au_{xx} + Bu_{xy} + Cu_{yy} + Du_x + Eu_y + Fu = G
\end{equation*}
dengan $A, B, C, \ldots, G$ dan $u$ adalah fungsi $(x, y)$ dan dapat diturunkan 2 kali pada domain $\Omega$.

Klasifikasi PDP linier order 2 dengan dua variabel:
\begin{enumerate}
    \item PDP Hiperbolik \hfill $B^2 - 4AC > 0$
    \item PDP Parabolik \hfill $B^2 - 4AC = 0$
    \item PDP Eliptik \hfill $B^2 - 4AC < 0$
\end{enumerate}
\begin{align*}
A^* &= A \xi_x^2 + B \xi_x \xi_y + C \xi_y^2 \\
B^* &= 2A \xi_x \eta_x + B \left( \xi_x \eta_y + \xi_y \eta_x \right) + 2C \xi_y \eta_y \\
C^* &= A \eta_x^2 + B \eta_x \eta_y + C \eta_y^2 \\
D^* &= A \xi_{xx} + B \xi_{xy} + C \xi_{yy} + D \xi_x + E \xi_y \\
E^* &= A \eta_{xx} + B \eta_{xy} + C \eta_{yy} + D \eta_x + E \eta_y \\
F^* &= F \\
G^* &= G
\end{align*}
Persamaan (3.15) adalah bentuk kanonik dari PDP linier order 2 dengan dua variabel.
\begin{align*}
    A^*w_{\xi\xi} + B^*w_{\xi\eta} + C^*w_{\eta\eta} + D^*w_\xi + E^*w_\eta + F^*w = G^*
\end{align*}
            \subsection*{\small Hiperbolik}
                Bentuk kanonik PDP hiperbolik adalah
\[
w_{\xi \eta} = H^*_1(\xi, \eta, w, w_\xi, w_\eta).
\]

Langkah-langkah mengubah PDP hiperbolik ke bentuk kanoniknya:

\begin{enumerate}
    \item Tentukan tipe PDP dengan mengecek
    \[
    B^2 - 4AC > 0 \rightarrow \text{PDP hiperbolik}
    \]

    \item Dapatkan persamaan karakteristiknya
    \begin{align*}
        \frac{dy}{dx} &= \frac{B + \sqrt{B^2 - 4AC}}{2A} \\\rightarrow& \text{solusinya } \phi_1(x,y) = c_1 \\
        \frac{dy}{dx} &= \frac{B - \sqrt{B^2 - 4AC}}{2A} \\\rightarrow& \text{solusinya } \phi_2(x,y) = c_2
    \end{align*}
    sehingga \( \xi(x,y) = \phi_1(x,y) \) dan \( \eta(x,y) = \phi_2(x,y) \)

    \item Dapatkan turunan \( \xi, \eta \) terhadap \( x, y \) sehingga didapat
    \[
    \xi_x, \xi_y, \xi_{xx}, \xi_{xy}, \xi_{yy}, \eta_x, \eta_y, \eta_{xx}, \eta_{xy}, \eta_{yy}
    \]

    \item Substitusi hasil no (3) ke persamaan (3.15) untuk mendapatkan
    \[
    A^*, B^*, C^*, D^*, E^*, F^*, \text{ dan } G^*
    \]

    \item Selesaikan hasil no (4) sehingga didapat hasil
    $
    w(\xi, \eta)
    $

    \item Transformasi hasil no (5) ke bentuk
    $
    u(x, y)
    $
\end{enumerate}
\subsection*{\small Parabolik}
    Bentuk kanonik PDP parabolik adalah
\[
w_{\eta\eta} = H_2^*(\xi, \eta, w, w_\xi, w_\eta)
\]
atau
\[
w_{\xi\xi} = H_3^*(\xi, \eta, w, w_\xi, w_\eta).
\]

Langkah-langkah mengubah PDP parabolik ke bentuk kanoniknya:

\begin{enumerate}
    \item Tentukan tipe PDP dengan mengecek
    \[
    B^2 - 4AC = 0 \rightarrow \text{PDP parabolik}
    \]

    \item Dapatkan persamaan karakteristiknya
    \[
    \frac{dy}{dx} = \frac{B}{2A} \rightarrow \text{solusinya } \phi_1(x,y) = c_1
    \]
    sehingga \( \xi(x,y) = \phi_1(x,y) \) dan \( \eta(x,y) \) adalah sebarang fungsi \( (x,y) \) dengan catatan
    \[
    J = 
    \begin{vmatrix}
        \xi_x & \xi_y \\
        \eta_x & \eta_y
    \end{vmatrix} \ne 0
    \]

    \item Dapatkan turunan \( \xi, \eta \) terhadap \( x, y \) sehingga didapat
    \[
    \xi_x, \xi_y, \xi_{xx}, \xi_{xy}, \xi_{yy}, \eta_x, \eta_y, \eta_{xx}, \eta_{xy}, \eta_{yy}
    \]

    \item Substitusi hasil no (3) ke persamaan (3.15) untuk mendapatkan
    \[
    A^*, B^*, C^*, D^*, E^*, F^*, \text{ dan } G^*
    \]

    \item Selesaikan hasil no (4) sehingga didapat hasil
    $
    w(\xi, \eta)
    $

    \item Transformasi hasil no (5) ke bentuk
    $
    u(x, y)
    $
\end{enumerate}
\subsection*{\small Eliptik}
Langkah-langkah mengubah PDP eliptik ke bentuk kanoniknya:

\begin{enumerate}
    \item Tentukan tipe PDP dengan mengecek
    \[
    B^2 - 4AC < 0 \rightarrow \text{PDP eliptik}
    \]

    \item Dapatkan persamaan karakteristiknya
    \begin{align*}
        \frac{dy}{dx} &= \frac{B - i\sqrt{4AC - B^2}}{2A} \\ &\rightarrow \text{solusinya } \alpha(x,y) = c_1 \\
        \frac{dy}{dx} &= \frac{B + i\sqrt{4AC - B^2}}{2A} \\ &\rightarrow \text{solusinya } \beta(x,y) = c_2
    \end{align*}
    sehingga diperoleh 
    \[
    \xi(x,y) = \frac{\alpha + \beta}{2} \quad \text{dan} \quad \eta(x,y) = \frac{\alpha - \beta}{2i}
    \]

    \item Dapatkan turunan \( \xi, \eta \) terhadap \( x, y \) sehingga didapat
    \[
    \xi_x, \xi_y, \xi_{xx}, \xi_{xy}, \xi_{yy}, \eta_x, \eta_y, \eta_{xx}, \eta_{xy}, \eta_{yy}
    \]

    \item Substitusi hasil no (3) ke persamaan (3.15) untuk mendapatkan
    \[
    A^*, B^*, C^*, D^*, E^*, F^*, \text{ dan } G^*
    \]

    \item Selesaikan hasil no (4) sehingga didapat hasil
    \[
    w(\xi, \eta)
    \]

    \item Transformasi hasil no (5) ke bentuk
    \[
    u(x, y)
    \]
\end{enumerate}
\section*{\small Metode Pemisahan Variabel}
Bentuk umum PDP order 2 dengan 2 variabel homogen adalah
\[
\resizebox{\linewidth}{!}{$
A^* u_{x^*x^*} + B^* u_{x^*y^*} + C^* u_{y^*y^*} + D^* u_{x^*} + E^* u_{y^*} + F^* u = 0
$}
\]

dan bentuk kanoniknya adalah
\[
A u_{xx} + B u_{xy} + C u_{yy} + D u_x + E u_y + F u = 0
\]

diasumsikan solusi dengan pemisah variabel yaitu
\[
u(x,y) = X(x) Y(y) \ne 0
\]

maka akan diperoleh
\[
\begin{aligned}
u_x &= X' Y \\
u_y &= X Y' \\
u_{xx} &= X'' Y \\
u_{xy} &= X' Y' \\
u_{yy} &= X Y''
\end{aligned}
\]

    \section*{\small Getaran Dawai}
           Solusi permasalahan getaran dawai 
\[
u_{tt} - c^2 u_{xx} = 0, \quad 0 < x < \ell, \ t > 0
\]
dengan kondisi awal 
\[
\begin{aligned}
u(x,0) &= f(x), &\quad 0 \leq x \leq \ell \\
u_t(x,0) &= g(x), &\quad 0 \leq x \leq \ell \\
\end{aligned}
\]
dan batas
\[
\begin{aligned}
u(0,t) &= 0, &\quad t \geq 0 \\
u(\ell,t) &= 0, &\quad t \geq 0
\end{aligned}
\]
adalah
\[
\resizebox{\linewidth}{!}{$
u(x,t) = \sum_{n=1}^{\infty} \sin\left( \frac{n\pi}{\ell} x \right)
\left[ a_n \cos\left( \frac{cn\pi}{\ell} t \right) + b_n \sin\left( \frac{cn\pi}{\ell} t \right) \right]
$}
\]
dengan
\[
a_n = \frac{2}{\ell} \int_0^{\ell} f(x) \sin\left( \frac{n\pi}{\ell} x \right) dx,
\]
\[
b_n = \frac{2}{n\pi c} \int_0^{\ell} g(x) \sin\left( \frac{n\pi}{\ell} x \right) dx.
\]

    \section*{\small Konduksi Panas}
Solusi permasalahan konduksi panas \( u_t = ku_{xx}, \; 0 < x < \ell, \; t > 0 \) dengan kondisi awal dan batas
\[
\begin{aligned}
u(x,0) &= f(x), \quad &&0 \leq x \leq \ell \\
u(0,t) &= 0, \quad &&t \geq 0 \\
u(\ell,t) &= 0, \quad &&t \geq 0
\end{aligned}
\]

adalah

\[
u(x,t) = \sum_{n=1}^{\infty} a_n \sin\left( \frac{n\pi}{\ell} x \right) e^{-\left( \frac{n\pi}{\ell} \right)^2 kt}
\]

dengan

\[
a_n = \frac{2}{\ell} \int_0^{\ell} f(x) \sin\left( \frac{n\pi}{\ell} x \right) dx.
\]
\subsection*{\small Metode Ekspansi Fungsi Eigen}
Metode ini dapat digunakan jika permasalahan nonhomogennya berupa \textbf{PDP nonhomogen dengan kondisi batas homogen}.

Langkah-langkah metode eigen function expansion.

\begin{enumerate}
    \item Dapatkan eigen function dari masalah homogen, misal $\phi_n(x)$ = eigen function, maka solusinya adalah
    \begin{equation}
        u(x,t) = \sum_{n=0}^{\infty} C_n(t)\phi_n(x).
    \end{equation}

    \item Bagian nonhomogen dari PDP ($G(x,t)$) diekspansi dari eigen function
    \begin{equation}
        G(x,t) = \sum_{n=0}^{\infty} h_n(t)\phi_n(x).
    \end{equation}

    \item Substitusi ke PDP nonhomogen diperoleh PDB $C_n(t)$.
    
    \item Cari solusi PDB $C_n(t)$.

    \item Masukkan kondisi awal dan diperoleh $C_n(0)$ dan $C_n'(0)$.
\end{enumerate}
\subsection*{\small PD Non-hom Kondisi \& Batas Non-hom}
Misalkan persamaan diferensial parsial nonhomogen:
\[
u_{tt} = c^2 u_{xx} + h(x,t)
\]
dengan kondisi awal:
\[
u(x,0) = f(x), \quad u_t(x,0) = g(x)
\]
dan kondisi batas nonhomogen:
\[
u(0,t) = p(t), \quad u(\ell,t) = q(t)
\]

Langkah-langkah penyelesaian:

\begin{enumerate}
    \item Pemisahan solusi:
    \[
    u(x,t) = v(x,t) + U(x,t)
    \]
    dengan $v(x,t)$ memenuhi kondisi batas homogen dan $U(x,t)$ mengatasi kondisi batas nonhomogen.
    
    \item Tentukan fungsi $U(x,t)$ agar:
    \[
    U(0,t) = p(t), \quad U(\ell,t) = q(t)
    \]
    Ambil:
    \[
    U(x,t) = p(t) + \frac{x}{\ell} \left(q(t) - p(t)\right)
    \]
    sehingga kondisi batas homogen diperoleh untuk $v(x,t)$:
    \[
    v(0,t) = 0, \quad v(\ell,t) = 0
    \]
    
    \item Substitusi ke persamaan utama:
    \[
    v_{tt} - c^2 v_{xx} = h(x,t) - U_{tt}(x,t)
    \]
    atau:
    \[
    v_{tt} - c^2 v_{xx} = H(x,t)
    \]
    
    \item Tentukan kondisi awal untuk $v$:
    \begin{align*}
        v(x,0) &= f(x) - U(x,0) \\
        v_t(x,0) &= g(x) - U_t(x,0)
    \end{align*}
    
    \item Selesaikan PDP untuk $v(x,t)$ dengan kondisi batas homogen dan sumber baru $H(x,t)$ menggunakan metode \textit{eigen function expansion}.
    
    \item Gabungkan solusi:
    \[
    u(x,t) = v(x,t) + U(x,t)
    \]
\end{enumerate}
\section*{\small Solusi PDP Linier Tingkat Dua dengan Koefisien Konstan}

Bentuk umum PDP linier tingkat dua adalah:
\[
a\,u_{xx} + b\,u_{xy} + c\,u_{yy} + d\,u_x + e\,u_y + f\,u = g
\]
dengan \( a, b, c, d, e, f, g \) fungsi-fungsi dari \( x \) dan \( y \), atau konstan.

Untuk kasus koefisien konstan, bentuknya menjadi:
\[
a\,u_{xx} + b\,u_{xy} + c\,u_{yy} = g(x, y)
\]
dengan \( a, b, c \) konstan.

Definisikan operator:
\begin{align*}
    D &= \frac{\partial}{\partial x}, \quad \mathcal{D} = \frac{\partial}{\partial y},\\
    D^2 &= \frac{\partial^2}{\partial x^2}, \quad D\mathcal{D} = \frac{\partial^2}{\partial x \partial y},\\
    \mathcal{D}^2 &= \frac{\partial^2}{\partial y^2}
\end{align*}

Sehingga operator diferensial:
\[
F(D, \mathcal{D}) = a D^2 + b D \mathcal{D} + c \mathcal{D}^2
\]
dan bentuk PDP menjadi:
\[
F(D, \mathcal{D}) u = g(x, y)
\]

\textbf{Penyelesaian umum} dari PDP ini adalah:
\[
u = u_c + u_p
\]
dengan:
\begin{itemize}
    \item \( u_c \): \textit{penyelesaian komplementer}, yaitu solusi dari persamaan homogen \( F(D, \mathcal{D}) u = 0 \)
    \item \( u_p \): \textit{penyelesaian partikular}, yaitu solusi dari persamaan nonhomogen \( F(D, \mathcal{D}) u = g(x, y) \)
\end{itemize}

Untuk menyelesaikan bagian homogen:
\[
a u_{xx} + b u_{xy} + c u_{yy} = 0
\]
gunakan substitusi \( u = e^{y + mx} \), maka:
\begin{align*}
    u_x &= m e^{y + mx}, \\
    u_y &= e^{y + mx}, \\
    u_{xx} &= m^2 e^{y + mx}, \\
    u_{xy} &= m e^{y + mx}, \\
    u_{yy} &= e^{y + mx}
\end{align*}

Substitusi ke PDE menghasilkan persamaan karakteristik:
\[
a m^2 + b m + c = 0
\]
Dengan solusi \( m_1, m_2 \), bentuk umum solusi \( u_c \) bergantung pada sifat akarnya:

\begin{enumerate}
    \item Jika \( m_1 \ne m_2 \) (real dan berbeda):
    \[
    u_c = f(y + m_1 x) + g(y + m_2 x)
    \]
    
    \item Jika \( m_1 = m_2 = k \) (real dan kembar):
    \[
    u_c = f(y + kx) + x\,g(y + kx)
    \]
    atau
    \[
    u_c = f(y + kx) + y\,g(y + kx)
    \]
    
    \item Jika \( m_1, m_2 = p \pm qi \) (kompleks konjugat):
    \[
    u_c = f(y + (p + qi)x) + g(y + (p - qi)x)
    \]
\end{enumerate}

Untuk penyelesaian partikularnya:
\begin{enumerate}
    \item Jika \( F(D, \mathcal{D})u = e^{ax + by} \), maka:
    \begin{align*}
        u_p &= \frac{1}{F(D, \mathcal{D})} e^{ax + by} \\
            &= \frac{e^{ax + by}}{F(a,b)} \quad \text{dengan } F(a,b) \ne 0
    \end{align*}
    
    \item Jika \( F(D^2, D\mathcal{D}, \mathcal{D}^2) u = \sin(ax + by) \), maka:
    \begin{align*}
        u_p &= \frac{1}{F(D^2, D\mathcal{D}, \mathcal{D}^2)} \sin(ax + by) \\
            &= \frac{\sin(ax + by)}{F(-a^2, -ab, -b^2)}  \\
            & \text{dengan } F(-a^2, -ab, -b^2) \ne 0
    \end{align*}
    
    \item Jika \( F(D^2, D\mathcal{D}, \mathcal{D}^2) u = \cos(ax + by) \), maka:
    \begin{align*}
        u_p &= \frac{1}{F(D^2, D\mathcal{D}, \mathcal{D}^2)} \cos(ax + by) \\
            &= \frac{\cos(ax + by)}{F(-a^2, -ab, -b^2)}  \\
            & \text{dengan } F(-a^2, -ab, -b^2) \ne 0
    \end{align*}
    
    \item Jika \( F(D, \mathcal{D}) u = x^p y^q \) dengan \( p, q \) bilangan bulat positif, maka:
    \begin{align*}
        u_p &= \frac{1}{F(D, \mathcal{D})} x^p y^q
    \end{align*}
    \begin{itemize}
        \item jika \( q < p \), deretkan dalam \( \frac{\mathcal{D}}{D} \) sampai \( \left( \frac{\mathcal{D}}{D} \right)^q \)
        \item jika \( p < q \), deretkan dalam \( \frac{D}{\mathcal{D}} \) sampai \( \left( \frac{D}{\mathcal{D}} \right)^p \)
    \end{itemize}

    \item $\dfrac{1}{D - a\mathcal{D}} f(x,y) = \int f(x, c - ax) \, dx$ dengan $c$ diganti $y + ax$ setelah integrasi

\end{enumerate}
	\end{multicols}
\end{document}