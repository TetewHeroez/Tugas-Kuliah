\documentclass[a4paper,extrafontsizes, 9pt]{memoir}

\usepackage{amsmath, amssymb, amsfonts, amsthm}
\usepackage{multicol}
\usepackage{multirow}
\usepackage{lipsum}
\usepackage{mathtools}
\usepackage{pgfplots}
\usepackage{soul}
\usepackage{hyperref}
\usepackage{enumitem}
\usepackage{fancyhdr}

\pagestyle{empty}

\setlrmarginsandblock{1cm}{1cm}{*}
\setulmarginsandblock{1cm}{1cm}{*}
\checkandfixthelayout

\DeclareMathOperator{\Cov}{Cov}
\DeclareMathOperator{\Var}{Var}

\let\bf\textbf{}

\setlength{\parindent}{0pt}
\renewcommand{\arraystretch}{1.5}

\newcommand{\textbetweenrules}[2][.4pt]{%
  \par\vspace{\topsep}
  \noindent\makebox[\textwidth]{%
    \sbox0{#2}%
    \dimen0=.5\dimexpr\ht0+#1\relax
    \dimen2=-.5\dimexpr\ht0-#1\relax
    \leaders\hrule height \dimen0 depth \dimen2\hfill
    \quad #2\quad
    \leaders\hrule height \dimen0 depth \dimen2\hfill
  }\par\nopagebreak\vspace{\topsep}
}

\begin{document}
\footnotesize \textbf{{Cheatsheet -- Persamaan Diferensial Parsial}} \hfill \textit{Teosofi Hidayah Agung/5002221132}
	\begin{multicols}{3}			
        \subsection*{\small Klasifikasi PDP}
            Bentuk umum dari PDP linier order 2 dengan dua variabel adalah
\begin{equation*}
Au_{xx} + Bu_{xy} + Cu_{yy} + Du_x + Eu_y + Fu = G
\end{equation*}
dengan $A, B, C, \ldots, G$ dan $u$ adalah fungsi $(x, y)$ dan dapat diturunkan 2 kali pada domain $\Omega$.

Klasifikasi PDP linier order 2 dengan dua variabel:
\begin{enumerate}
    \item PDP Hiperbolik \hfill $B^2 - 4AC > 0$
    \item PDP Parabolik \hfill $B^2 - 4AC = 0$
    \item PDP Eliptik \hfill $B^2 - 4AC < 0$
\end{enumerate}
\begin{align*}
A^* &= A \xi_x^2 + B \xi_x \xi_y + C \xi_y^2 \\
B^* &= 2A \xi_x \eta_x + B \left( \xi_x \eta_y + \xi_y \eta_x \right) + 2C \xi_y \eta_y \\
C^* &= A \eta_x^2 + B \eta_x \eta_y + C \eta_y^2 \\
D^* &= A \xi_{xx} + B \xi_{xy} + C \xi_{yy} + D \xi_x + E \xi_y \\
E^* &= A \eta_{xx} + B \eta_{xy} + C \eta_{yy} + D \eta_x + E \eta_y \\
F^* &= F \\
G^* &= G
\end{align*}
            \subsection*{\small Hiperbolik}
                Bentuk kanonik PDP hiperbolik adalah
\[
w_{\xi \eta} = H^*_1(\xi, \eta, w, w_\xi, w_\eta).
\]

Langkah-langkah mengubah PDP hiperbolik ke bentuk kanoniknya:

\begin{enumerate}
    \item Tentukan tipe PDP dengan mengecek
    \[
    B^2 - 4AC > 0 \rightarrow \text{PDP hiperbolik}
    \]

    \item Dapatkan persamaan karakteristiknya
    \begin{align*}
        \frac{dy}{dx} &= \frac{B + \sqrt{B^2 - 4AC}}{2A} \\\rightarrow& \text{solusinya } \phi_1(x,y) = c_1 \\
        \frac{dy}{dx} &= \frac{B - \sqrt{B^2 - 4AC}}{2A} \\\rightarrow& \text{solusinya } \phi_2(x,y) = c_2
    \end{align*}
    sehingga \( \xi(x,y) = \phi_1(x,y) \) dan \( \eta(x,y) = \phi_2(x,y) \)

    \item Dapatkan turunan \( \xi, \eta \) terhadap \( x, y \) sehingga didapat
    \[
    \xi_x, \xi_y, \xi_{xx}, \xi_{xy}, \xi_{yy}, \eta_x, \eta_y, \eta_{xx}, \eta_{xy}, \eta_{yy}
    \]

    \item Substitusi hasil no (3) ke persamaan (3.15) untuk mendapatkan
    \[
    A^*, B^*, C^*, D^*, E^*, F^*, \text{ dan } G^*
    \]

    \item Selesaikan hasil no (4) sehingga didapat hasil
    $
    w(\xi, \eta)
    $

    \item Transformasi hasil no (5) ke bentuk
    $
    u(x, y)
    $
\end{enumerate}
\subsection*{\small Parabolik}
    Bentuk kanonik PDP parabolik adalah
\[
w_{\eta\eta} = H_2^*(\xi, \eta, w, w_\xi, w_\eta)
\]
atau
\[
w_{\xi\xi} = H_3^*(\xi, \eta, w, w_\xi, w_\eta).
\]

Langkah-langkah mengubah PDP parabolik ke bentuk kanoniknya:

\begin{enumerate}
    \item Tentukan tipe PDP dengan mengecek
    \[
    B^2 - 4AC = 0 \rightarrow \text{PDP parabolik}
    \]

    \item Dapatkan persamaan karakteristiknya
    \[
    \frac{dy}{dx} = \frac{B}{2A} \rightarrow \text{solusinya } \phi_1(x,y) = c_1
    \]
    sehingga \( \xi(x,y) = \phi_1(x,y) \) dan \( \eta(x,y) \) adalah sebarang fungsi \( (x,y) \) dengan catatan
    \[
    J = 
    \begin{vmatrix}
        \xi_x & \xi_y \\
        \eta_x & \eta_y
    \end{vmatrix} \ne 0
    \]

    \item Dapatkan turunan \( \xi, \eta \) terhadap \( x, y \) sehingga didapat
    \[
    \xi_x, \xi_y, \xi_{xx}, \xi_{xy}, \xi_{yy}, \eta_x, \eta_y, \eta_{xx}, \eta_{xy}, \eta_{yy}
    \]

    \item Substitusi hasil no (3) ke persamaan (3.15) untuk mendapatkan
    \[
    A^*, B^*, C^*, D^*, E^*, F^*, \text{ dan } G^*
    \]

    \item Selesaikan hasil no (4) sehingga didapat hasil
    $
    w(\xi, \eta)
    $

    \item Transformasi hasil no (5) ke bentuk
    $
    u(x, y)
    $
\end{enumerate}
\subsection*{\small Eliptik}
Langkah-langkah mengubah PDP eliptik ke bentuk kanoniknya:

\begin{enumerate}
    \item Tentukan tipe PDP dengan mengecek
    \[
    B^2 - 4AC < 0 \rightarrow \text{PDP eliptik}
    \]

    \item Dapatkan persamaan karakteristiknya
    \begin{align*}
        \frac{dy}{dx} &= \frac{B - i\sqrt{4AC - B^2}}{2A} \\ &\rightarrow \text{solusinya } \alpha(x,y) = c_1 \\
        \frac{dy}{dx} &= \frac{B + i\sqrt{4AC - B^2}}{2A} \\ &\rightarrow \text{solusinya } \beta(x,y) = c_2
    \end{align*}
    sehingga diperoleh 
    \[
    \xi(x,y) = \frac{\alpha + \beta}{2} \quad \text{dan} \quad \eta(x,y) = \frac{\alpha - \beta}{2i}
    \]

    \item Dapatkan turunan \( \xi, \eta \) terhadap \( x, y \) sehingga didapat
    \[
    \xi_x, \xi_y, \xi_{xx}, \xi_{xy}, \xi_{yy}, \eta_x, \eta_y, \eta_{xx}, \eta_{xy}, \eta_{yy}
    \]

    \item Substitusi hasil no (3) ke persamaan (3.15) untuk mendapatkan
    \[
    A^*, B^*, C^*, D^*, E^*, F^*, \text{ dan } G^*
    \]

    \item Selesaikan hasil no (4) sehingga didapat hasil
    \[
    w(\xi, \eta)
    \]

    \item Transformasi hasil no (5) ke bentuk
    \[
    u(x, y)
    \]
\end{enumerate}
\section*{\small Metode Pemisahan Variabel}
Bentuk umum PDP order 2 dengan 2 variabel homogen adalah
\[
\resizebox{\linewidth}{!}{$
A^* u_{x^*x^*} + B^* u_{x^*y^*} + C^* u_{y^*y^*} + D^* u_{x^*} + E^* u_{y^*} + F^* u = 0
$}
\]

dan bentuk kanoniknya adalah
\[
A u_{xx} + B u_{xy} + C u_{yy} + D u_x + E u_y + F u = 0
\]

diasumsikan solusi dengan pemisah variabel yaitu
\[
u(x,y) = X(x) Y(y) \ne 0
\]

maka akan diperoleh
\[
\begin{aligned}
u_x &= X' Y \\
u_y &= X Y' \\
u_{xx} &= X'' Y \\
u_{xy} &= X' Y' \\
u_{yy} &= X Y''
\end{aligned}
\]

    \section*{\small Getaran Dawai}
           Solusi permasalahan getaran dawai 
\[
u_{tt} - c^2 u_{xx} = 0, \quad 0 < x < \ell, \ t > 0
\]
dengan kondisi awal dan batas
\[
\begin{aligned}
u(x,0) &= f(x), &\quad 0 \leq x \leq \ell \\
u_t(x,0) &= g(x), &\quad 0 \leq x \leq \ell \\
u(0,t) &= 0, &\quad t \geq 0 \\
u(\ell,t) &= 0, &\quad t \geq 0
\end{aligned}
\]
adalah
\[
\resizebox{\linewidth}{!}{$
u(x,t) = \sum_{n=1}^{\infty} \sin\left( \frac{n\pi}{\ell} x \right)
\left[ a_n \cos\left( \frac{cn\pi}{\ell} t \right) + b_n \sin\left( \frac{cn\pi}{\ell} t \right) \right]
$}
\]
dengan
\[
a_n = \frac{2}{\ell} \int_0^{\ell} f(x) \sin\left( \frac{n\pi}{\ell} x \right) dx,
\]
\[
b_n = \frac{2}{n\pi c} \int_0^{\ell} g(x) \sin\left( \frac{n\pi}{\ell} x \right) dx.
\]

    \section*{\small Konduksi Panas}
Solusi permasalahan konduksi panas \( u_t = ku_{xx}, \; 0 < x < \ell, \; t > 0 \) dengan kondisi awal dan batas
\[
\begin{aligned}
u(x,0) &= f(x), \quad &&0 \leq x \leq \ell \\
u(0,t) &= 0, \quad &&t \geq 0 \\
u(\ell,t) &= 0, \quad &&t \geq 0
\end{aligned}
\]

adalah

\[
u(x,t) = \sum_{n=1}^{\infty} a_n \sin\left( \frac{n\pi}{\ell} x \right) e^{-\left( \frac{n\pi}{\ell} \right)^2 kt}
\]

dengan

\[
a_n = \frac{2}{\ell} \int_0^{\ell} f(x) \sin\left( \frac{n\pi}{\ell} x \right) dx.
\]
	\end{multicols}
\end{document}