\documentclass[a4paper]{article}
\usepackage{amsmath,amssymb,amsfonts,amsthm}
\usepackage{multicol}
\usepackage{multirow}
\usepackage{mathtools}
\usepackage{soul}
\usepackage{hyperref}
\hypersetup{
    colorlinks=true,
    linkcolor=blue,
    filecolor=magenta,      
    urlcolor=cyan,
    pdftitle={Overleaf Example},
    pdfpagemode=FullScreen,
    }
\usepackage{color}
\usepackage[table]{xcolor}
\usepackage[T1]{fontenc}
\usepackage{etoolbox}
\usepackage{multicol}
\usepackage{multirow}
\usepackage{fancyhdr}
\usepackage{graphicx}
\usepackage{array}
\usepackage{amsthm}
\usepackage{titlesec}
\usepackage{tikz}
\usetikzlibrary{arrows.meta,calc}
\renewcommand{\baselinestretch}{1.2}

\titleformat*{\section}{\large\bfseries}
\titleformat*{\subsection}{\normalsize\bfseries}

\graphicspath{{C:/Users/teoso/OneDrive/Documents/Tugas Kuliah/Template Math Depart/}}

\newtheorem{theorem}{Theorem}
\newtheorem*{teorema}{Teorema}
\newtheorem*{definisi}{Definisi}
\theoremstyle{definition}
\newtheorem*{bukti}{Bukti}

\newcommand{\Arg}{\text{Arg}}

\begin{document}
\fancyhead[L]{\textit{Teosofi Hidayah Agung}}
\fancyhead[R]{\textit{5002221132}}
\pagestyle{fancy}
\begin{enumerate}
  \item \[u_{xx} + x^2u_{yy} = 0,\, x\ne 0\]
  dan bentuk kanonikalnya adalah
  \begin{equation}
    \omega_{\xi\xi} + \omega_{\eta\eta} = -\frac{1}{2\eta}\omega_{\eta}\label{eq:1}
  \end{equation}
  Bentuk separable 
  \begin{equation}
    \omega(\xi,\eta) = X(\xi)Y(\eta)\label{eq:2}
  \end{equation}
  Subtitusi \eqref{eq:2} ke \eqref{eq:1} diperoleh
  \begin{equation}
    X''(\xi)Y(\eta) + X(\xi)Y''(\eta) = -\frac{1}{2\eta}X(\xi)Y'(\eta)\label{eq:3}
  \end{equation}
  Selanjutnya kita bagi \eqref{eq:3} dengan $X(\xi)Y(\eta)$:
  \begin{equation}
    \frac{X''(\xi)}{X(\xi)} + \frac{Y''(\eta)}{Y(\eta)} = -\frac{1}{2\eta}\frac{Y'(\eta)}{Y(\eta)}\label{eq:4}
  \end{equation}
  dengan pemisahan variabel
  \begin{equation}
    \frac{X''(\xi)}{X(\xi)} = -\lambda,\, \frac{Y''(\eta)}{Y(\eta)} + \frac{1}{2\eta}\frac{Y'(\eta)}{Y(\eta)} = \lambda\label{eq:5}
  \end{equation}
  \begin{itemize}
    \item $X''(\xi) + \lambda X(\xi) = 0$\\
    $X(\xi) = A\cos(\sqrt{\lambda}\xi) + B\sin(\sqrt{\lambda}\xi)$
    \item $Y''(\eta) + \frac{1}{2\eta}Y'(\eta) + \lambda Y(\eta) = 0 = Y_\eta(\eta)$
  \end{itemize}
  \begin{equation}
    \omega(\xi,\eta) = \sum_{n=1}^{\infty}\left[A\cos(\sqrt{\lambda}\xi) + B\sin(\sqrt{\lambda}\xi)\right]Y_\eta(\eta) \label{eq:6}
  \end{equation}
  \begin{equation}
      \eta = \frac{y+\frac{1}{2}x^2}{2},\quad u = \frac{y-\frac{1}{2}x^2}{2}\label{eq:7}
  \end{equation}
  Subtitusi \eqref{eq:7} ke \eqref{eq:6} sehingga solusinya
  \begin{equation*}
      \omega(\xi,\eta) = \sum_{n=1}^{\infty}\left[A\cos\left(\sqrt{\lambda}\left(\frac{y+\frac{1}{2}x^2}{2}\right)\right) + B\sin\left(\sqrt{\lambda}\left(\frac{y+\frac{1}{2}x^2}{2}\right)\right)\right]Y_\eta\left(\frac{y-\frac{1}{2}x^2}{2}\right)
  \end{equation*}
  \item Bentuk kanonik : $\omega_{\xi\xi} + \omega_{\eta\eta} =  \dfrac{2}{3} \omega_{\eta}-\dfrac{2\sqrt{3}}{3} \omega_{\eta}$\\
  Asumsikan solusinya berbentuk
  \begin{equation}
      \omega(\xi,\eta) = e^{\alpha\xi+\beta\eta}\label{eq:8}
  \end{equation}
  Turunkan parsial \eqref{eq:8} terhadap $\xi$ dan $\eta$:
  \begin{equation*}
      \omega_\xi = \alpha e^{\alpha\xi+\beta\eta},\quad
      \omega_{\xi\xi} = \alpha^2 e^{\alpha\xi+\beta\eta},\quad
      \omega_{\eta} = \beta e^{\alpha\xi+\beta\eta},\quad
      \omega_{\eta\eta} = \beta^2 e^{\alpha\xi+\beta\eta},\quad
  \end{equation*}
  Subtitusi ke dalam persamaan \eqref{eq:8}:
  \begin{align*}
      &\alpha^2 e^{\alpha\xi+\beta\eta} + \beta^2 e^{\alpha\xi+\beta\eta} = \frac{2}{3}\alpha e^{\alpha\xi+\beta\eta}-\frac{2\sqrt{3}}{3}\beta e^{\alpha\xi+\beta\eta}\\
      &\implies \alpha^2 + \beta^2 = \frac{2}{3}\alpha-\frac{2\sqrt{3}}{3}\beta
  \end{align*}
  Misalkan $\alpha =  1$ maka $\beta$ diperoleh dari
  \begin{align*}
      1 + \beta^2 &= \frac{2}{3}-\frac{2\sqrt{3}}{3}\beta\\
      \beta^2 + \frac{2\sqrt{3}}{3}\beta + \frac{1}{3} &= 0\\
      3\beta^2 + 2\sqrt{3}\beta + 1 &= 0\\
      \beta &= \frac{-2\sqrt{3}\pm\sqrt{(2\sqrt{3})^2-4.3.1}}{2\cdot 3}\\
      &= \frac{-2\sqrt{3}\pm\sqrt{12-12}}{6}\\
      &= -\frac{\sqrt{3}}{3}\\
  \end{align*}
  Artinya
  \begin{align*}
      \omega(\xi,\eta) &= e^{\xi-\frac{\sqrt{3}}{3}\eta}\\
      \omega(\xi,\eta) &= f(\xi-\frac{\sqrt{3}}{3}\eta)
  \end{align*}
  \begin{itemize}
    \item $\xi = y-\dfrac{1}{2}x$
    \item $\eta = \dfrac{\sqrt{3}}{3}x$
  \end{itemize}
  Oleh karena itu diperoleh
  \begin{align*}
    u(x,y) &= f(\eta-\frac{\sqrt{3}}{3}\xi)\\
    &= f\left(y-\frac{1}{2}x-\frac{\sqrt{3}}{3}\cdot \dfrac{\sqrt{3}}{3}x\right)\\
    &= f\left(y-\frac{1}{2}x-\frac{1}{2}x\right)\\
    u(x,y)&= f\left(y-x\right)
  \end{align*}
\end{enumerate}
\end{document}