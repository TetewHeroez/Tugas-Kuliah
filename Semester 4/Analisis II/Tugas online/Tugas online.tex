\documentclass{article}
\usepackage{graphicx} 
\usepackage{multirow}
\usepackage{enumitem}
\usepackage{amssymb}
\usepackage{amsmath}
\usepackage{xcolor}
\usepackage{cancel}
\usepackage{geometry}
    \geometry{
        paperwidth=17.5cm, 
        paperheight=24cm
        }
\newcommand{\jawab}{\textbf{Jawab}:}
\newcommand{\R}{\mathbb{R}}
\newcommand{\N}{\mathbb{N}}
\begin{document}
    \pagenumbering{gobble}
    \begin{tabular}{|lcl|}
     \hline
     Nama&:&Teosofi Hidayah Agung\\
     NRP&:&5002221132\\
     \hline
    \end{tabular}
    \begin{enumerate}
        \item Tunjukkan bahwa barisan $x^n/(1+x^n)$ tidak konvergen seragam pada $[0,2]$ 
        dengan menunjukkan bahwa limit fungsinya tidak kontinu pada $[0,2]$\\
        \jawab\\
        Perhatikan bahwa $\lim(x^n/(1+x^n))$ menuju ke fungsi sepotong-sepotong $g$ yaitu
        \begin{flalign*}
            g=\begin{cases}
                0,&\quad x=0\\
                \frac{1}{2},&\quad x\in(0,1]\\
                1,&\quad x\in(1,2]\\
            \end{cases}
        \end{flalign*}
        Dapat dilihat bahwa $g$ tidak kontinu untuk semua $x\in[0,2]$, sehingga menggunakan 
        \textbf{Teorema 8.2.2} dapat disimpulkan bahwa barisan diatas tidak konvergen seragam. 
        $\blacksquare$
        \item[4.] Misalkan $(f_n)$ barisan fungsi yang kontinu pada interval $I$ juga 
        konvergen seragam pada $I$ ke fungsi $f$. Jika $(x_n)\subseteq I$ konvergen ke 
        $x_0\in I$, tunjukkan bahwa $\lim(f_n(x_n))=f(x_0)$.\\
        \jawab\\
        Karena $f_n$ konvergen seragam pada $I$, maka $\lim(f_n(x))=f(x)$ untuk setiap $x\in I$.
        Sebab $(x_n)\subseteq I$, maka untuk setiap $n$ berlaku $f_n(x_n)$ akan konvergen ke 
        $f(x_n)$. Kemudian barisan $(x_n)$ juga konvergen ke $x_0\in I$ atau $\lim(x_n)=x_0$. 
        Sehingga dari informasi diatas dapat disimpulkan
        \begin{flalign*}
            \lim(f_n(x_n))=f(\lim(x_n))=f(x_0)\quad\blacksquare
        \end{flalign*}  
        \item[7.] Misalkan barisan $(f_n)$ konvergen seragam ke $f$ pada himpunan $A$, 
        dan andaikan setiap $f_n$ terbatas pada $A$. (Artinya, untuk setiap $n$ terdapat 
        konstanta $M_n$ sedemikian sehingga $|f_n(x)|\leq M_n$ untuk setiap $x\in A$.) 
        Tunjukkan bahwa fungsi $f$ terbatas di $A$.\\
        \jawab\\
        Karena barisan $f_n$ selalu terbatas pada $M_n$, maka barisan $M_n$ juga terbatas pada suatu 
        $M$. Dengan menggunakan fakta tersebut maka didapatkan
        \begin{flalign*}
           \lim|f_n(x)|&\leq\lim(M_n)
        \end{flalign*}
        Fakta bahwa $f_n$ konvergen seragam berakibat $\lim|f_n(x)|=|\lim(f_n(x))|$.
        \begin{flalign*}
            |\lim(f_n(x))|&\leq M\quad x\in A\\
            |f(x)|&\leq M,\quad x\in A
        \end{flalign*}
        Hai diatas menunjukkan bahwa $f$ konvergen terbatas di $A$. $\blacksquare$
    \end{enumerate}
\end{document}