\documentclass{article}
\usepackage{graphicx} 
\usepackage{multirow}
\usepackage{enumitem}
\usepackage{amssymb}
\usepackage{amsmath}
\usepackage{xcolor}
\usepackage{cancel}
\usepackage{geometry}
    \geometry{
        paperwidth=17.5cm, 
        paperheight=24cm
        }

\begin{document}
    \pagenumbering{gobble}
    \begin{tabular}{|lcl|}
     \hline
     Nama&:&Teosofi Hidayah Agung\\
     NRP&:&5002221132\\
     \hline
    \end{tabular}
    \begin{enumerate}
        %1
        \item Given sets of matrices as follows
        \[M=\left\{\left.\begin{bmatrix}
            a_{11}&a_{12}&\dots&a_{1n}\\
            a_{21}&a_{22}&\dots&a_{2n}\\
            \vdots&\vdots&\ddots&\vdots\\
            a_{n1}&a_{n2}&\dots&a_{nn}
        \end{bmatrix}\right|a_{ij}=0,\,\textrm{if }i>j\right\}.\]
        Determine whether $M$ with usual matrix addition and multiplication operations forms a ring!\\
        \textbf{Proof}:\\
        By definition, $M$ is an upper triangular matrix and can be rewritten as follows
        \[\begin{bmatrix}
            a_{11}&a_{12}&\dots&a_{1n}\\
            0&a_{22}&\dots&a_{2n}\\
            \vdots&\vdots&\ddots&\vdots\\
            0&0&\dots&a_{nn}
        \end{bmatrix}\]
        We known $M\subseteq \mathbb{R}^{n\times n}$ and $\mathbb{R}^{n\times n}$ is ring. Then we can check $M$ is 
        subring of $\mathbb{R}^{n\times n}$ iff $\forall A,B\in M$ satisfying 
        \begin{enumerate}[label=(\arabic*)]
            \item $A-B\in M$
            \item $AB\in M$
        \end{enumerate}
        Lets check:
        \begin{flalign*}
            A-B&=\begin{bmatrix}
                a_{11}&a_{12}&\dots&a_{1n}\\
                0&a_{22}&\dots&a_{2n}\\
                \vdots&\vdots&\ddots&\vdots\\
                0&0&\dots&a_{nn}
            \end{bmatrix}-
            \begin{bmatrix}
                b_{11}&b_{12}&\dots&b_{1n}\\
                0&b_{22}&\dots&b_{2n}\\
                \vdots&\vdots&\ddots&\vdots\\
                0&0&\dots&b_{nn}
            \end{bmatrix}
            \\
            &=\begin{bmatrix}
                a_{11}-b_{11}&a_{12}-b_{12}&\dots&a_{1n}-b_{1n}\\
                0&a_{22}-b_{22}&\dots&a_{2n}-b_{2n}\\
                \vdots&\vdots&\ddots&\vdots\\
                0&0&\dots&a_{nn}-b_{nn}
            \end{bmatrix}\in M
        \end{flalign*}
        and
        \begin{flalign*}
            AB&=\begin{bmatrix}
                a_{11}&a_{12}&\dots&a_{1n}\\
                0&a_{22}&\dots&a_{2n}\\
                \vdots&\vdots&\ddots&\vdots\\
                0&0&\dots&a_{nn}
            \end{bmatrix}
            \begin{bmatrix}
                b_{11}&b_{12}&\dots&b_{1n}\\
                0&b_{22}&\dots&b_{2n}\\
                \vdots&\vdots&\ddots&\vdots\\
                0&0&\dots&b_{nn}
            \end{bmatrix}
            \\
            &=\begin{bmatrix}
                a_{11}b_{11}&a_{11}b_{12}+a_{12}b_{22}&\dots&\sum_{k=1}^{n}a_{1k}b_{kn}\\
                0&a_{22}b_{22}&\dots&\sum_{k=2}^{n}a_{2k}b_{kn}\\
                \vdots&\vdots&\ddots&\vdots\\
                0&0&\dots&a_{nn}b_{nn}
            \end{bmatrix}\in M
        \end{flalign*}
        Because (1) and (2) satisfied, so we can conclude that $M$ is subring of $\mathbb{R}^{n\times n}$.\\
        $\therefore$ $M$ is ring $\blacksquare$\\

        %2
        \item Is the set of three-dimension vectors i.e., $\mathbb{R}^3$ with the addition operation
        \[\begin{bmatrix}u_1\\u_2\\u_3\end{bmatrix}+\begin{bmatrix}v_1\\v_2\\v_3\end{bmatrix}:=\begin{bmatrix}u_1+v_1\\u_2+v_2\\u_3+v_3\end{bmatrix}\]
        and multiplication operation
        \[\begin{bmatrix}u_1\\u_2\\u_3\end{bmatrix}\times\begin{bmatrix}v_1\\v_2\\v_3\end{bmatrix}:=\begin{bmatrix}u_2v_3-u_3v_2\\-(u_1v_3-u_3v_1)\\u_1v_2-u_2v_1\end{bmatrix}\]
        forms a ring? Explain your reasoning!\\
        \textbf{Proof}:\\
        No, because is not satisfy the associative law for multiplication. In this case take $\vec{u}=[1,2,3]^T,\vec{v}=[1,1,1]^T,\vec{w}=[1,2,1]^T$,
        then we can check
        \begin{flalign*}
            (\vec{u}\times\vec{v})\times\vec{w}&=\left(\begin{bmatrix}1\\2\\3\end{bmatrix}\times\begin{bmatrix}1\\1\\1\end{bmatrix}\right)
            \times\begin{bmatrix}1\\2\\1\end{bmatrix}=\begin{bmatrix}2-3\\-1+3\\1-2\end{bmatrix}\times\begin{bmatrix}1\\2\\1\end{bmatrix}=\begin{bmatrix}-1\\2\\-1\end{bmatrix}\times\begin{bmatrix}1\\2\\1\end{bmatrix}&\\
            &=\begin{bmatrix}2+2\\1-1\\-2-2\end{bmatrix}=\begin{bmatrix}4\\0\\-4\end{bmatrix}
        \end{flalign*}
        \begin{flalign*}
            \vec{u}\times(\vec{v}\times\vec{w})&=\begin{bmatrix}1\\2\\3\end{bmatrix}\times\left(\begin{bmatrix}1\\1\\1\end{bmatrix}
            \times\begin{bmatrix}1\\2\\1\end{bmatrix}\right)=\begin{bmatrix}1\\2\\3\end{bmatrix}\times\begin{bmatrix}1-2\\-1+1\\2-1\end{bmatrix}
            =\begin{bmatrix}1\\2\\3\end{bmatrix}\times\begin{bmatrix}-1\\0\\1\end{bmatrix}&\\
            &=\begin{bmatrix}2-0\\-(1+3)\\0+2\end{bmatrix}=\begin{bmatrix}2\\-4\\2\end{bmatrix}
        \end{flalign*}
        Thus $(\vec{u}\times\vec{v})\times\vec{w}\neq\vec{u}\times(\vec{v}\times\vec{w})$ or the multiplication isn't 
        associative.\\~\\
        $\therefore$ The set is not ring under that operation.\\

        \fbox{\begin{minipage}{12 cm}
            As known from Elementary Linear Algebra course (ELA). the multiplication operation from the question is 
            "cross product". The definition of multiplication can be rewritten as follows
            \[\begin{bmatrix}u_1\\u_2\\u_3\end{bmatrix}\times\begin{bmatrix}v_1\\v_2\\v_3\end{bmatrix}:=
            \begin{vmatrix}
                i&j&k\\
                u_1&u_2&u_3\\
                v_1&v_2&v_3
            \end{vmatrix}
            =\begin{bmatrix}M_{11}\\M_{12}\\M_{13}\end{bmatrix}\]
            and $M_{ij}$ is \textit{minor} of the entry in the $i^{\textrm{th}}$ row and $j^{\textrm{th}}$ column (also called the $(i,j)$ minor).\\
            Then the one of property \textit{cross product} is
            \begin{align*}
                (\vec{u}\times\vec{v})\times\vec{w}=(u\cdot w)v-(v\cdot w)u\\
                \vec{u}\times(\vec{v}\times\vec{w})=(u\cdot w)v-(u\cdot v)w
            \end{align*}
            and from that property we can conclude $(\vec{u}\times\vec{v})\times\vec{w}\neq\vec{u}\times(\vec{v}\times\vec{w})$
            \end{minipage}}

        %3
        \item Is the set of three-dimension vectors i.e., $\mathbb{R}^3$ with the addition operation
        \[\begin{bmatrix}u_1\\u_2\\u_3\end{bmatrix}+\begin{bmatrix}v_1\\v_2\\v_3\end{bmatrix}:=\begin{bmatrix}u_1+v_1\\u_2+v_2\\u_3+v_3\end{bmatrix}\]
        and multiplication operation
        \[\begin{bmatrix}u_1\\u_2\\u_3\end{bmatrix}\times\begin{bmatrix}v_1\\v_2\\v_3\end{bmatrix}:=\begin{bmatrix}u_1v_1\\u_2v_2\\u_3v_3\end{bmatrix}\]
        forms a ring? Explain your reasoning!\\
        \textbf{Proof}:\\
        The set and operation similarly to \textbf{Contoh 6.1.6}. As we known 
        \[\mathbb{R}^3=\mathbb{R}\times\mathbb{R}\times\mathbb{R}\]
        or $\mathbb{R}^3$ is ordered pairs (External Direct Product).\\
        Because $\mathbb{R}$ is ring under addition and multiplication. Thus conclude $\mathbb{R}^3$ also ring $\blacksquare$
    \end{enumerate}
\end{document}