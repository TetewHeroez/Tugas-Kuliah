\documentclass{article}
\usepackage{graphicx} 
\usepackage{multirow}
\usepackage{enumitem}
\usepackage{amssymb}
\usepackage{amsmath}
\usepackage{amsthm}
\usepackage{xcolor}
\usepackage{cancel}
\usepackage{geometry}
    \geometry{
        paperwidth=17.5cm, 
        paperheight=24cm
        }
\newtheorem*{teorema}{Teorema}

\usepackage{scalerel}
\setcounter{MaxMatrixCols}{20}

\newcommand{\longdiv}{\smash{\mkern-0.43mu\vstretch{1.31}{\hstretch{.7}{)}}\mkern-5.2mu\vstretch{1.31}{\hstretch{.7}{)}}}}

\newcommand{\R}{\mathbb{R}}
\newcommand{\N}{\mathbb{N}}
\newcommand{\Z}{\mathbb{Z}}
\newcommand{\Q}{\mathbb{Q}}
\newcommand{\jawab}{\textbf{Solusi}:}

\begin{document}
\begin{center}
    \textbf{QUIS 2}
\end{center}
\begin{enumerate}
    \item Diberikan $f(x),g(x)\in \Z_7[x]$ dimana $f(x)=x^4+x^3+x^2+x$ dan $g(x)=x^3+1$.
    \begin{enumerate}
        \item Tentukan $\gcd(f(x),g(x))$.\\
        \jawab\\
        Dengan menggunakan algoritma Euclid
        \begin{itemize}
            \item Cari hasil dan sisa bagi $f(x)$ dengan $g(x)$
            \[
        \arraycolsep=1pt
        \renewcommand\arraystretch{1.2}
        \begin{array}{*1r @{\hskip\arraycolsep}c@{\hskip\arraycolsep} *{11}r}
            & &x&+1&\\
        \cline{2-6}
        x^3+1 & \longdiv & x^4 & + x^3 &+ x^2 & +x  \\
            &  & x^4 &+ x \\
        \cline{2-5}
            &  & &x^3 &+ x^2  \\
            &  & &x^3 &+1\\
        \cline{4-6}
            &  & & &x^2 &- 1\\
        \end{array}\]
        Hasil bagi adalah $x+1$ dan sisa bagi adalah $r_1(x)=x^2+6$.
        \begin{equation}\label{1}
            f(x)=(x+1)g(x)+(x^2+6)
        \end{equation}
            \item Kemudian hasil dan sisa bagi $g(x)$ dengan $r_1(x)$
            \[
        \arraycolsep=1pt
        \renewcommand\arraystretch{1.2}
        \begin{array}{*1r @{\hskip\arraycolsep}c@{\hskip\arraycolsep} *{11}r}
            & &x&\\
        \cline{2-4}
        x^2+6 & \longdiv & x^3 & + 1  \\
            &  & x^3 &+ 6x \\
        \cline{2-5}
            &  & &-6x & +1  \\
        \end{array}\]
        Hasil bagi adalah $x$ dan sisa bagi adalah $r_2(x)=x+1$.
        \begin{equation}\label{2}
            g(x)=(x)r_1(x)+(x+1)
        \end{equation}
        \item Selanjutnya hasil dan sisa bagi $r_1(x)$ dengan $r_2(x)$
        \[
        \arraycolsep=1pt
        \renewcommand\arraystretch{1.2}
        \begin{array}{*1r @{\hskip\arraycolsep}c@{\hskip\arraycolsep} *{11}r}
            & &x&-1\\
        \cline{2-5}
        x+1 & \longdiv & x^2 & + 6  \\
            &  & x^2 &+ x \\
        \cline{2-5}
            &  & &-x & +6  \\
            &  & &-x & -1  \\
        \cline{3-5}
            &  & & &7 &=0 \\
        \end{array}\]
        Hasil bagi adalah $x+6$ dan sisa bagi adalah $0$.
        \begin{equation}\label{3}
            r_1(x)=(x+6)r_2(x)+0
        \end{equation}
        Ketika sisa bagi adalah $0$ maka proses algoritma Euclid selesai.
        \end{itemize}
        Hal diatas menyatakan bahwa $\gcd(r_1(x),r_2(x))=r_2(x)=x+1$.\\
        Jadi $\gcd(f(x),g(x))=\gcd(g(x),r_1(x))=\gcd(r_1(x),r_2(x))=\boxed{x+1}$.\\

        \item Nyatakan $\gcd(f(x),g(x))$ sebagai kombinasi linear dari $f(x)$ dan $g(x)$.\\
        \jawab\\
        Dari persamaan \eqref{1} dapat diubah
        \begin{equation*}
            f(x)-(x+1)g(x)=r_1(x)
        \end{equation*}
        subtistusi persamaan diatas ke persamaan \eqref{2}
        \begin{flalign*}
            g(x)=x\left[f(x)-(x+1)g(x)\right]+(x+1)\\
            g(x)-x\left[f(x)-(x+1)g(x)\right]=x+1\\
            g(x)-xf(x)+x(x+1)g(x)=x+1\\
            (-x)f(x)+(x^2+x+1)g(x)=x+1\\
            \boxed{(6x)f(x)+(x^2+x+1)g(x)=x+1}
        \end{flalign*}

    \end{enumerate}
    \item Tunjukkan bahwa $f(x)=7x^3+9x^2+4x+11$ tak tereduksi di $\Z[x]$.\\
    Catatan: Terapkan Teorema berikut:
    \begin{teorema}
        Misalkan $f (x) \in \Z[x]$ dengan $\deg( f (x)) \leq 1$. Untuk suatu bilangan prima $p$,
        polinomial $\mathcal{F}(x) \in \Z_p[x]$ diperoleh dari $f (x) \in \Z[x]$ dengan melakukan semua koefisien
        menjadi modulo $p$. Bila $\deg( f (x)) = \deg(\mathcal{F}(x))$ dan $\mathcal{F}(x)$ tak-tereduksi di $\Z_p[x]$, maka $f (x)$
        tak-tereduksi di $\Z[x]$.
    \end{teorema}
    \jawab\\
    Kalimat "Untuk suatu bilangan prima $p$" berarti cukup pilih satu nilai $p$. Disini kita pilih $p=2$.\\
    {\color{red}CATAT*:Jika nilai $p$ nantinya membuat polinomial $\mathcal{F}(x)$ tereduksi, maka kita harus memilih nilai $p$ yang lain. Dikarenakan Teorema tersebut berbunyi \textbf{"Jika Maka"} yang nantinya kita tak dapat menarik kesimpulan jika $\mathcal{F}(x)$ tereduksi}\\

    Kita dapatkan $\mathcal{F}(x)=x^3+x^2+1$ dengan melakukan semua koefisien dari $f(x)$ menjadi modulo $2$. Jelas bahwa $\deg(f(x))=\deg(\mathcal{F}(x))=3$, kemudian kita cek apakah $\mathcal{F}(x)$ tereduksi di $\Z_2[x]$ atau tidak.\\

    Dalam kasus ini, disebabkan $\mathcal{F}(x)$ polinom derajat 3, maka kemungkinan tereduksinya adalah menjadi polinom derajat 1 dan 2. Dapat ditulis
    \[\mathcal{F}(x)=(x-a)(x^2+bx+c)\]
    Disini kita cukup mencari nilai akarnya yaitu $a$ yang memenuhi $\mathcal{F}(a)=0$.
    Subtistusi semua $x\in\Z_2$ ke $\mathcal{F}(x)$ 
    \begin{flalign*}
        x=0&\implies\mathcal{F}(0)=0^3+0^2+1=1\\
        x=1&\implies\mathcal{F}(1)=1^3+1^2+1=3=1
    \end{flalign*}
    Dari hasil diatas, kita tidak dapat menemukan nilai $a$ yang memenuhi $\mathcal{F}(a)=0$. Sehingga $\mathcal{F}(x)$ tidak tereduksi di $\Z_2[x]$.\\

    Dengan demikian, berdasarkan Teorema diatas, $f(x)$ tidak tereduksi di $\Z[x]$.\\
    \item \begin{enumerate}
        \item Tunjukkan bahwa $I=\langle x^3+x+1 \rangle$ bukan merupakan ideal maksimal dari $\Z_3[x]$.\\
        \jawab
        \begin{teorema}[Teorema 8.6.2]
            Misalkan $\mathbb{F}$ adalah suatu lapangan. Suatu ideal nontrivial $I = \langle p(x)\rangle$ adalah
            suatu ideal maksimal dalam $\mathbb{F}[x]$ jika dan hanya jika $p(x)$ tak-tereduksi atas $\mathbb{F}$.
        \end{teorema}
        Dengan cara yang sama seperti soal nomor 2, derajat dari $x^3+x+1$ adalah 3. Kita cek apakah $x^3+x+1$ tereduksi di $\Z_3[x]$.
        \begin{flalign*}
            x=0&\implies 0^3+0+1=1\\
            x=1&\implies 1^3+1+1=3=0\\
            x=2&\implies 2^3+2+1=11=2
        \end{flalign*}
        Ternyata terdapat akar dari $x^3+x+1$ yaitu $x=1$. Sehingga $x^3+x+1$ tereduksi di $\Z_3[x]$.\\

        $\therefore$ Berdasarkan Teorema diatas, $I=\langle x^3+x+1 \rangle$ bukan merupakan ideal maksimal dari $\Z_3[x]$.

        \item Tentukan ideal $J$ dari $\Z_3[x]$ sehingga $I\subset J$ dan $J\ne \Z_3[x]$. Jelaskan jawaban anda\\
        \jawab\\
        Polinomial $x^3+x+1$ dapat difaktorkan menjadi 
        \[x^3+x+1=(x+2)(x^2+x+2)\]
        Perhatikan bahwa $x^2+x+2$ dan $x+2$ masing masing sudah tidak dapat difaktorkan.
        Dari sini kita dapat memilih $J=\langle x+2 \rangle$ atau $J=\langle x^2+x+2 \rangle$, mengapa?
        \begin{itemize}
            \item Untuk $J=\langle x+2 \rangle$ jika ditulis secara definisi adalah sebagai berikut 
            \[J=\{f(x)(x+2)\,|\,f(x)\in\Z_3[x]\}\]
            Ketika kita memilih $g(x)=(x^2+x+2)h(x)$ untuk suatu $h(x)\in J$, maka
            berakibat $\langle g(x) \rangle\subset J$. Atau
            \[\langle g(x) \rangle=\{(x^2+x+2)(x+2)f(x)\,|\,f(x)\in\Z_3[x]\}=I\]
            Sehingga $I\subset J$.

            \item Dengan cara yang sama untuk $J=\langle x^2+x+2 \rangle$, 
            pilih $g(x)=(x+2)h(x)$ untuk suatu $h(x)\in J$, maka
            berakibat $\langle g(x) \rangle\subset J$. Atau
            \[\langle g(x) \rangle=\{(x+2)(x^2+x+2)f(x)\,|\,f(x)\in\Z_3[x]\}=I\]
            Pada akhirnya $I\subset J$ juga.
        \end{itemize}
    \end{enumerate}
\end{enumerate}
\end{document}