\documentclass{file/KP-ITS}
%Ridho Nur Rohman Wijaya

\makeatletter
\def\cleardoublepage{\clearpage%
	\if@twoside
	\ifodd\c@page\else
	\vspace*{\fill}
	\hfill
	\begin{center}
		\emph{ }
	\end{center}
	\vspace{\fill}
	\thispagestyle{empty}
	\newpage
	\if@twocolumn\hbox{}\newpage\fi
	\fi
	\fi
}
\makeatother
\newtheorem{defn}{Definisi}[section]
\newtheorem{teo}[defn]{Teorema}
\newtheorem{thm}{Teorema}[section]
\newtheorem{lemma}[defn]{Lemma}
\newtheorem{lemmas}[thm]{Lemma}
\newtheorem{cor}[defn]{Akibat}
\theoremstyle{definition}
\newtheorem{con}[defn]{Contoh}
\theoremstyle{definition}
\theoremstyle{plain}
\newtheorem{prop}[defn]{Proposisi}
\renewcommand{\proofname}{Bukti}
\renewcommand{\thethm}{\arabic{chapter}.\arabic{thm}}

\newcommand{\norm}[1]{\left\|#1\right\|} % Fungsi norm (||x||)

\newcommand\firstPar{0.75cm} % Indentasi 0.75cm pada tiap paragraf (manual untuk hspace)
\setlength{\parindent}{0.75cm} % Indentasi 0.75cm pada tiap paragraf

\usepackage{fancyhdr}
\pagestyle{fancy}
\renewcommand{\headrulewidth}{0pt}
\fancyhf{}
\usepackage{ifthen}
\fancyfoot[R]{\thepage}

\usepackage[labelsep=quad]{caption}
\captionsetup[table]{skip=5pt}

\usepackage{multirow}
\usepackage{longtable}

%%% Pewarnaan code
\usepackage{color}
\usepackage{listings}

\definecolor{codegreen}{rgb}{0,0.6,0}
\definecolor{codeblack}{rgb}{0,0,0}
\definecolor{codepurple}{rgb}{0.58,0,0.82}
\definecolor{backcolour}{rgb}{0.95,0.95,0.92}

\lstdefinestyle{mystyle}{
    commentstyle=\color{codegreen},
    keywordstyle=\color{magenta},
    numberstyle=\tiny\color{codeblack},
    stringstyle=\color{codepurple},
    basicstyle=\ttfamily\footnotesize,
    breakatwhitespace=false,         
    breaklines=true,                 
    captionpos=b,                    
    keepspaces=true,                 
    numbers=left,                    
    numbersep=5pt,                  
    showspaces=false,                
    showstringspaces=false,
    showtabs=false,                  
    tabsize=2
}
%%% Pewarnaan code

\hypersetup{ % Merubah warna link
    colorlinks,
    linkcolor={black},
    citecolor={black},
    urlcolor={black}
}

% \tolerance=1
% \emergencystretch = \maxdimen
% \hyphenpenalty=10000
% \hbadness=1000

\begin{document}

% input data
\NamaPerusahaan{Subdirektorat Koordinasi Perkuliahan Bersama}

\Judul{Perancangan Website Interaktif untuk Pembelajaran Kalkulus Menggunakan CortexJS}

\JudulEng{Design of an Interactive Website for Calculus Learning Using CortexJS}

\Nama{Teosofi Hidayah Agung}

\NamaKecil{Teosofi Hidayah Agung}

\NRP{5002221132}

\Departemen{Matematika}

\Department{Mathematics}

\BidangStudi{Aljabar dan Analisis}

\AlamatPenulis{Jalan Kendangsari Gang VII/22, RT.06 RW.03, Surabaya}

\Bulan{Mei} % Masuk lembar pengesahan

\Tahun{2025}

\TanggalDisetujui{31 Mei 2025} % Masuk lembar orisinilitas

\Fakultas{Sains dan Analitika Data}

\SingkatanFakultas{FSAD}

\Faculty{Scientics}

\SingkatanFakultasEng{SCIENTICS}

\Pembimbing{Dr. Didik Khusnul A, S.Si, M.Si}
          {} 	   

\NIPPembimbing{197309301997021001}
{} 
              
\Kadep{Dr. Didik Khusnul A, S.Si, M.Si}

\NIPKadep{197309301997021001}

\PembimbingMitra{Refais Akbar Zufira, S.Kom}

\NIPPembimbingMitra{1998202121057}

\KepalaInstansi{Dr. Bintoro Anang S, S.Si., M.Si.}

\JabatanKepalaInstansi{Kepala Subdirektorat Koordinasi \\Perkuliahan Bersama}

\NIPKepalaInstansi{197907192005011015}
\BagianAwal
\Cover
\LembarJudul
\TitlePage
\LembarPengesahanDepartemen
\LembarPengesahanInstansi
\LembarOrisinalitas

%%%%%%%%%%%%%%%%%%%%%%%%  Abstrak  %%%%%%%%%%%%%%%5%%%%%%%%%%
\restoregeometry
\KataPengantar

%%%%%%%%%%%%%%%%%%%%%%%%  Abstrak  %%%%%%%%%%%%%%%5%%%%%%%%%%

%%%%%%%%%%%%%%%%%%%%%%%%  Daftar  %%%%%%%%%%%%%%%5%%%%%%%%%%

\DaftarIsi\raggedbottom

\DaftarGambar

\DaftarTabel

\DaftarSimbol
\begin{flushleft}
\begin{tabular}{lrl}


$\oplus$ &:& Operasi \textit{max} dalam aljabar max-plus\\

\end{tabular}
\end{flushleft}
%%%%%%%%%%%%%%%%%%%%%%%%  Daftar  %%%%%%%%%%%%%%%5%%%%%%%%%%

\BagianInti

%%%%%%%%%%%%%%%%%%%%%%%%  Bab I  %%%%%%%%%%%%%%%5%%%%%%%%%%
\chapter{PENDAHULUAN}
\section{Latar Belakang}
Latar belakang meliputi :
\begin{itemize}
    \item Alasan mahasiswa matematika harus Kerja Praktik
    \item Alasan mahasiswa memilih Kerja Praktik di tempat tersebut
(sesuai dengan tempat Kerja Praktik masing- masing mahasiswa)
\end{itemize}

\section{Tujuan Kerja Praktik}
\subsection{Tujuan Umum}
Mengenali system kerja di tempat KP

\subsection{Tujuan Khusus}
Menyelesaikan tugas khusus yang diberikan di tempat Kerja Praktik (bila ada), dan atau mengidentifikasi serta memperkenalkan ilmu matematika yang dapat digunakan.

\section{Manfaat}
Berisikan manfaat yang diperoleh dari tujuan umum maupun
%%%%%%%%%%%%%%%%%%%%%%%%  Bab I  %%%%%%%%%%%%%%%5%%%%%%%%%%

%%%%%%%%%%%%%%%%%%%%%%%%  Bab II  %%%%%%%%%%%%%%%5%%%%%%%%%%

\pagebreak
\chapter{GAMBARAN UMUM}

\section{Sejarah Tempat Kerja Praktik}
\section{Struktur Organisasi}
\section{Kegiatan Perusahaan}


% (\ref*{pers1}). % Merujuk persamaan tinggal menuliskan labelnya


%%%%%%%%%%%%%%%%%%%%%%%%  Bab II  %%%%%%%%%%%%%%%5%%%%%%%%%%

%%%%%%%%%%%%%%%%%%%%%%%%  Bab III  %%%%%%%%%%%%%%%5%%%%%%%%%%

\pagebreak
\chapter{PELAKSANAAN KERJA PRAKTIK}

\section{Pelaksanaan Kerja Praktik}
...

\section{Metodologi Penyelesaian Tugas Khusus}
...


\section{Urutan pelaksanaan penelitian
}
...

\section{Jadwal Kegiatan}
Berikut ini disajikan tabel jadwal kegiatan yang akan dilakukan selama 3 bulan dan berkoresponden dengan metodologi.
	% Membuat tabel
%  \begin{table}[H]
%  \caption{Jadwal Kegiatan}
%  \centering
% 	\begin{tabular}{|C{0.6cm}|L{5.7cm}|C{0.25cm}|C{0.25cm}|C{0.25cm}|C{0.25cm}|C{0.25cm}|C{0.25cm}|C{0.25cm}|C{0.25cm}|C{0.25cm}|C{0.25cm}|C{0.25cm}|C{0.25cm}|}	\hline
		
% 		&&\multicolumn{12}{c|}{\textbf{BULAN}}\\\cline{3-14}
% 		\multicolumn{1}{|c|}{\textbf{NO}}&\multicolumn{1}{c|}{\textbf{NAMA KEGIATAN}}&\multicolumn{4}{c|}{1}&\multicolumn{4}{c|}{2}&\multicolumn{4}{c|}{3}\\\cline{3-14}
% 		&&1&2&3&4&1&2&3&4&1&2&3&4\\\cline{1-14}
		
% 1&Studi literatur&\cellcolor{black!}&\cellcolor{black!}&&&&&&&&&&\\\hline
% 2&Identifikasi \textit{cyclicity Latin square} dalam aljabar max-plus&&&\cellcolor{black!}&\cellcolor{black!}&\cellcolor{black!}&&&&&&&\\\hline
% 3&Identifikasi \textit{transient Latin square} dalam aljabar max-plus&&&&&\cellcolor{black!}&\cellcolor{black!}&\cellcolor{black!}&&&&&\\\hline
% 4&Penarikan kesimpulan&&&&&&&&\cellcolor{black!}&\cellcolor{black!}&\cellcolor{black!}&&\\\hline
% 5&Penulisan laporan tugas akhir&&&&&&&&\cellcolor{black!}&\cellcolor{black!}&\cellcolor{black!}&\cellcolor{black!}&\cellcolor{black!}\\\hline
% \end{tabular}
% \label{TabelJadwalKegiatan}
% \end{table}

%%%%%%%%%%%%%%%%%%%%%%%%  Bab III  %%%%%%%%%%%%%%%5%%%%%%%%%%

\chapter{HASIL KERJA PRAKTIK}

\section{Hasil Penelitian}

%%%%%%%%%%%%%%%%%%%%%%%%  Bab IV %%%%%%%%%%%%%%%5%%%%%%%%%%

\chapter{PENUTUP}

\section{Kesimpulan}
\section{Saran}

%%%%%%%%%%%%%%%%%%%%%%%%  Bab V %%%%%%%%%%%%%%%5%%%%%%%%%%

\pagebreak
\DaftarPustaka
%%%%%%%%%%%%%%%%%%%%%%%%  Dapus  %%%%%%%%%%%%%%%5%%%%%%%%%%

\pagebreak
\lampiran{\textit{Logbook} Kegiatan}
\lampiran{...}
%%%%%%%%%%%%%%%%%%%%%%%%  Lampiran %%%%%%%%%%%%%%%5%%%%%%%%%%

\pagebreak
\Biodata{foto/FotoPenulis.jpg}
Hai bg
%%%%%%%%%%%%%%%%%%%%%%%%  Bio Penulis  %%%%%%%%%%%%%%%5%%%%%%%%%%
\end{document}