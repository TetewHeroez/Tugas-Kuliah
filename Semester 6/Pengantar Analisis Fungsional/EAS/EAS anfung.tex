\documentclass{article}
\usepackage{amsmath,amssymb,amsfonts,amsthm}
\usepackage{multicol}
\usepackage{multirow}
\usepackage{mathtools}
\usepackage{soul}
\usepackage{hyperref}
\hypersetup{
    colorlinks=true,
    linkcolor=blue,
    filecolor=magenta,      
    urlcolor=cyan,
    pdftitle={Overleaf Example},
    pdfpagemode=FullScreen,
    }
\usepackage{color}
\usepackage[table]{xcolor}
\usepackage[T1]{fontenc}
\usepackage{etoolbox}
\usepackage{multicol}
\usepackage{multirow}
\usepackage{fancyhdr}
\usepackage{graphicx}
\usepackage{tcolorbox}
\usepackage{array}
\usepackage{amsthm}
\usepackage{titlesec}
\usepackage{tikz, tkz-euclide}
  \usetikzlibrary{arrows.meta,calc}
  \newcommand\rightAngle[4]{
  \pgfmathanglebetweenpoints{\pgfpointanchor{#2}{center}}{\pgfpointanchor{#1}{center}}
  \coordinate (tmpRA) at ($(#2)+(\pgfmathresult+45:#4)$);
  \draw[red!60!black,thick] ($(#2)!(tmpRA)!(#1)$) -- (tmpRA) -- ($(#2)!(tmpRA)!(#3)$);
}
\renewcommand{\baselinestretch}{1.2}

\titleformat*{\section}{\large\bfseries}
\titleformat*{\subsection}{\normalsize\bfseries}

\newtheorem{theorem}{Theorem}
\newtheorem*{teorema}{Teorema}
\newtheorem*{definisi}{Definisi}
\theoremstyle{definition}
\newtheorem*{bukti}{Bukti}

\newcommand{\Arg}{\text{Arg}}
\newcommand{\R}{\mathbb{R}}
\newcommand{\C}{\mathbb{C}}
\newcommand{\N}{\mathbb{N}}
\newcommand{\Z}{\mathbb{Z}}

\newtcolorbox{solution}[1][]{
    colback=blue!5!white, 
    colframe=blue!75!black,
    fonttitle=\bfseries, 
    colbacktitle=blue!85!black,
    title=Solusi,
    #1
}

\begin{document}
  \fancyhead[L]{\textit{Teosofi Hidayah Agung}}
  \fancyhead[R]{\textit{5002221132}}
  \pagestyle{fancy}

  \begin{enumerate}
    \item Diberikan transformasi linear terbatas \( T : C_R[0,1] \to \mathbb{R} \) dengan 
    \[
        T(f) = \int_0^1 (1 - x)f(x)\, dx.
    \]
    Tunjukkan bahwa \( \|T\| \leq 1 \). Jika \( g \in C_R[0,1] \) dengan \( g(x) = 1 \), \( \forall x \in [0,1] \), dapatkan \( |T(g)| \) dan \( \|T\| \).

    \item Misalkan \( X \) ruang Banach dan \( \{T_n\} \) barisan operator yang invertibel di \( B(X) \) yang konvergen ke \( T \in B(X) \). Jika \( \|T_n^{-1}\| < 1 \), maka tunjukkan bahwa \( T \) invertibel.

    \item Diberikan \( c = \{c_n\} \in \ell^\infty \) dan \( T_c \in B(\ell^2) \) dengan 
    \[
        T_c(\{x_n\}) = \{c_n x_n\}.
    \]
    \begin{enumerate}
        \item Dapatkan operator adjoint \( T^* \).
        \item Jika \( c_n \in \mathbb{R} \), \( \forall n \in \mathbb{N} \), tunjukkan \( T_c \) self-adjoint.
        \item Jika \( |c_n| = 1 \), \( \forall n \in \mathbb{N} \), tunjukkan \( T_c \) unitary.
    \end{enumerate}

    \item Misalkan \( \mathcal{H} \) ruang Hilbert kompleks dan \( U \in B(\mathcal{H}) \) unitary. Tunjukkan transformasi linear 
    \[
        f : B(\mathcal{H}) \to B(\mathcal{H}) \quad \text{dengan} \quad f(T) = U^*TU
    \]
    adalah isometri.
\end{enumerate}

\end{document}