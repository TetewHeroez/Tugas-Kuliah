\documentclass[a4paper]{article}
\usepackage{amsmath,amssymb,amsfonts,amsthm}
\usepackage{multicol}
\usepackage{multirow}
\usepackage{mathtools}
\usepackage{soul}
\usepackage{hyperref}
\hypersetup{
    colorlinks=true,
    linkcolor=blue,
    filecolor=magenta,      
    urlcolor=cyan,
    pdftitle={Overleaf Example},
    pdfpagemode=FullScreen,
    }
\usepackage{color}
\usepackage[table]{xcolor}
\usepackage[T1]{fontenc}
\usepackage{etoolbox}
\usepackage{multicol}
\usepackage{multirow}
\usepackage{fancyhdr}
\usepackage{graphicx}
\usepackage{array}
\usepackage{amsthm}
\usepackage{titlesec}
\usepackage{tikz}
\usetikzlibrary{arrows.meta,calc}
\renewcommand{\baselinestretch}{1.2}

\titleformat*{\section}{\large\bfseries}
\titleformat*{\subsection}{\normalsize\bfseries}

\graphicspath{{C:/Users/teoso/OneDrive/Documents/Tugas Kuliah/Template Math Depart/}}

\newtheorem{theorem}{Theorem}
\newtheorem*{teorema}{Teorema}
\newtheorem*{definisi}{Definisi}
\theoremstyle{definition}
\newtheorem*{bukti}{Bukti}

\newcommand{\Arg}{\text{Arg}}
\newcommand{\R}{\mathbb{R}}
\newcommand{\C}{\mathbb{C}}
\newcommand{\N}{\mathbb{N}}
\newcommand{\Z}{\mathbb{Z}}

\begin{document}
\fancyhead[L]{\textit{Teosofi Hidayah Agung}}
\fancyhead[R]{\textit{5002221132}}
\pagestyle{fancy}
\section*{Contoh Ruang Metrik Lengkap}
\noindent Buktikan bahwa Ruang Metrik dengan himpunan $[0,1]\times[0,1]$ dan fungsi 
\[d(\mathbf{x},\mathbf{y})=\min\{1,\sqrt{(x_1-y_1)^2+(x_2-y_2)^2}\},\quad \] 
adalah ruang metrik lengkap.
\begin{bukti}
  Sebelumnya kita tahu bahwa jika $(X,l)$ merupakan ruang metrik, maka $(X,\min\{1,l\})$ juga merupakan ruang metrik. Sehingga dapat disimpulkan bahwa $([0,1]\times[0,1],d)$ adalah ruang metrik. 
  
  Selanjutnya misalkan barisan $(\mathbf{x}_n)_{n\in\N}=(x_n,y_n)$ adalah barisan Cauchy sehingga untuk setiap $\varepsilon>0$ terdapat $n,m>N$ yang berakibat
  \[d(\mathbf{x}_n,\mathbf{x}_m)<\varepsilon\]
  Perhatikan Teorema berikut
  \begin{teorema}
    Misalkan $(M, d)$ adalah ruang metrik lengkap, dan misalkan $X$ adalah subhimpunan dari $M$. Subruang $(X, d)$ adalah ruang metrik lengkap jika dan hanya jika $X$ adalah subhimpunan tertutup dari $M$.
  \end{teorema}
  Dari teorema di atas, dapat diingat bahwa $(\R^2,d)$ adalah ruang metrik lengkap dan $[0,1]\times[0,1]$ adalah subhimpunan tertutup dari $\R^2$. Sehingga $([0,1]\times[0,1],d)$ adalah ruang metrik lengkap.$\blacksquare$
\end{bukti} 
\end{document}