\documentclass{article}
\usepackage{amsmath,amssymb,amsfonts,amsthm}
\usepackage{multicol}
\usepackage{multirow}
\usepackage{mathtools}
\usepackage{soul}
\usepackage{hyperref}
\hypersetup{
    colorlinks=true,
    linkcolor=blue,
    filecolor=magenta,      
    urlcolor=cyan,
    pdftitle={Overleaf Example},
    pdfpagemode=FullScreen,
    }
\usepackage{color}
\usepackage[table]{xcolor}
\usepackage[T1]{fontenc}
\usepackage{etoolbox}
\usepackage{multicol}
\usepackage{multirow}
\usepackage{fancyhdr}
\usepackage{graphicx}
\usepackage{tcolorbox}
\usepackage{array}
\usepackage{amsthm}
\usepackage{titlesec}
\usepackage{tikz, tkz-euclide}
  \usetikzlibrary{arrows.meta,calc}
  \newcommand\rightAngle[4]{
  \pgfmathanglebetweenpoints{\pgfpointanchor{#2}{center}}{\pgfpointanchor{#1}{center}}
  \coordinate (tmpRA) at ($(#2)+(\pgfmathresult+45:#4)$);
  \draw[red!60!black,thick] ($(#2)!(tmpRA)!(#1)$) -- (tmpRA) -- ($(#2)!(tmpRA)!(#3)$);
}
\renewcommand{\baselinestretch}{1.2}

\titleformat*{\section}{\large\bfseries}
\titleformat*{\subsection}{\normalsize\bfseries}

\newtheorem{theorem}{Theorem}
\newtheorem*{teorema}{Teorema}
\newtheorem*{definisi}{Definisi}
\theoremstyle{definition}
\newtheorem*{bukti}{Bukti}

\newcommand{\Arg}{\text{Arg}}
\newcommand{\R}{\mathbb{R}}
\newcommand{\C}{\mathbb{C}}
\newcommand{\N}{\mathbb{N}}
\newcommand{\Z}{\mathbb{Z}}

\newtcolorbox{solution}[1][]{
    colback=blue!5!white, 
    colframe=blue!75!black,
    fonttitle=\bfseries, 
    colbacktitle=blue!85!black,
    title=Solusi,
    #1
}

\begin{document}
\fancyhead[L]{\textit{Teosofi Hidayah Agung}}
\fancyhead[R]{\textit{5002221132}}
\pagestyle{fancy}

\begin{enumerate}
  \item Jika $T: C_{\mathbb{R}}[0,1] \to \mathbb{R}$ adalah transformasi linier yang didefinisikan oleh:
  \[
   T(f) = \int_0^1 f(x) \, dx.
  \]
  Tunjukkan bahwa $T$ kontinu.
  
  \bigskip
  
  \textbf{Penyelesaian:}
  
  Ruang $C_{\mathbb{R}}[0,1]$ adalah ruang fungsi kontinu dari $[0,1]$ ke $\mathbb{R}$, dan dilengkapi dengan norma maksimum:
  \[
  \|f\|_\infty = \max_{x \in [0,1]} |f(x)|.
  \]
  
  Kita ingin menunjukkan bahwa terdapat konstanta $C > 0$ sehingga:
  \[
  |T(f)| = \left| \int_0^1 f(x) \, dx \right| \leq C \|f\|_\infty.
  \]
  
  Untuk semua $f \in C_{\mathbb{R}}[0,1]$, diperoleh:
  \[
  \left| T(f) \right| = \left| \int_0^1 f(x) \, dx \right| 
  \leq \int_0^1 |f(x)| \, dx 
  \leq \int_0^1 \|f\|_\infty \, dx 
  = \|f\|_\infty.
  \]
  
  Sehingga:
  \[
  |T(f)| \leq \|f\|_\infty.
  \]
  
  Dengan kata lain, terdapat konstanta $C = 1$ sehingga:
  \[
  |T(f)| \leq C \|f\|_\infty.
  \]
  Hal tersebut menunjukkan bahwa $T$ kontinu di $f(x)=0$ sehingga dapat disimpulkan $T$ kontinu di seluruh ruang $C_{\mathbb{R}}[0,1]$.
\end{enumerate}
\end{document}