\documentclass{article}
\usepackage{amsmath,amssymb,amsfonts,amsthm}
\usepackage{multicol}
\usepackage{multirow}
\usepackage{mathtools}
\usepackage{soul}
\usepackage{hyperref}
\hypersetup{
    colorlinks=true,
    linkcolor=blue,
    filecolor=magenta,      
    urlcolor=cyan,
    pdftitle={Overleaf Example},
    pdfpagemode=FullScreen,
    }
\usepackage{color}
\usepackage[table]{xcolor}
\usepackage[T1]{fontenc}
\usepackage{etoolbox}
\usepackage{multicol}
\usepackage{multirow}
\usepackage{fancyhdr}
\usepackage{graphicx}
\usepackage{tcolorbox}
\usepackage{array}
\usepackage{amsthm}
\usepackage{titlesec}
\usepackage{tikz, tkz-euclide}
  \usetikzlibrary{arrows.meta,calc}
  \newcommand\rightAngle[4]{
  \pgfmathanglebetweenpoints{\pgfpointanchor{#2}{center}}{\pgfpointanchor{#1}{center}}
  \coordinate (tmpRA) at ($(#2)+(\pgfmathresult+45:#4)$);
  \draw[red!60!black,thick] ($(#2)!(tmpRA)!(#1)$) -- (tmpRA) -- ($(#2)!(tmpRA)!(#3)$);
}
\renewcommand{\baselinestretch}{1.2}

\titleformat*{\section}{\large\bfseries}
\titleformat*{\subsection}{\normalsize\bfseries}

\newtheorem{theorem}{Theorem}
\newtheorem*{teorema}{Teorema}
\newtheorem*{definisi}{Definisi}
\theoremstyle{definition}
\newtheorem*{bukti}{Bukti}

\newcommand{\Arg}{\text{Arg}}
\newcommand{\R}{\mathbb{R}}
\newcommand{\C}{\mathbb{C}}
\newcommand{\N}{\mathbb{N}}
\newcommand{\Z}{\mathbb{Z}}

\newtcolorbox{solution}[1][]{
    colback=blue!5!white, 
    colframe=blue!75!black,
    fonttitle=\bfseries, 
    colbacktitle=blue!85!black,
    title=Solusi,
    #1
}

\begin{document}
\fancyhead[L]{\textit{Teosofi Hidayah Agung}}
\fancyhead[R]{\textit{5002221132}}
\pagestyle{fancy}

\begin{enumerate}
  \setcounter{enumi}{2}
  \item Jika $X$ dalam soal nomor 2 adalah ruang real, buktikan bahwa, secara konvers, relasi yang diberikan menyiratkan bahwa $x \perp y$. Tunjukkan bahwa hal ini mungkin tidak berlaku jika $X$ adalah kompleks. Berikan contoh.
  \begin{solution}
    Misalkan $x,y\in X$ memenuhi $\|x+y\|^2=\|x\|^2+\|y\|^2$, maka berdasarkan definisi norm berlaku
    \begin{align*}
      \|x+y\|^2&=\langle x+y, x+y \rangle\\
      \|x\|^2+\|y\|^2&=\langle x,x \rangle+\langle x, y \rangle+\langle y,x \rangle+\langle y,y \rangle\\
      \|x\|^2+\|y\|^2&=\|x\|^2+2\langle x,y \rangle+\|y\|^2\\
      2\langle x,y \rangle&=0\implies \langle x,y \rangle=0
    \end{align*}
    Jadi terbukti $x\perp y$.\\

    % Untuk $X=\C$ artinya definisi untuk sifat simetri berubah menjadi
    % \[
    % \langle z,w \rangle=\overline{\langle w,z \rangle} \quad \text{untuk } z,w\in \C.
    % \]
    % Sehingga untuk $z,w\in \C$ berlaku
    % \begin{align*}
    %   \|z+w\|^2&=\langle z+w,z+w \rangle\\
    %   &=\langle z,z \rangle+\langle z,w \rangle+\langle w,z \rangle+\langle w,w \rangle\\
    %   &=\|z\|^2+\overline{\langle z,w \rangle}+\langle z,w \rangle+\|w\|^2\\
    %   &=\|z\|^2+2\operatorname{Re}(\langle z,w \rangle)+\|w\|^2.
    % \end{align*}
    Misalkan $z,w\in \C$, kemudian pilih $z=-i$ dan $w=-1$. Jelas bahwa
    \[
    \|z+w\|^2=\left[\sqrt{1^2+(-1)^2}\right]^2=2=\|-i\|^2+\|-1\|^2=\|z\|^2+\|w\|^2.
    \]
    Namun 
    \[
    \langle z,w \rangle=\langle -i,-1 \rangle=-i\cdot(\overline{-1})=i\neq 0.
    \]
    Berarti $z\not\perp w$ untuk $z,w\in \C$.
  \end{solution}

  \item Jika $X$ adalah ruang hasil dalam real, buktikan bahwa kondisi $\|x\| = \|y\|$ berimplikasi $\langle x + y, x - y \rangle = 0$. Apa arti kondisi ini secara geometris jika $X = \mathbb{R}^2$? Apa arti kondisi ini jika $X$ adalah kompleks?
  \begin{solution}
    Misalkan $x,y\in X$ adalah ruang hasil dalam real.\\
    Jika $\|x\|=\|y\|$, maka berlaku
    \begin{align*}
      \langle x+y,x-y \rangle&=\langle x,x \rangle+\langle x,y \rangle-\langle y,x \rangle-\langle y,y \rangle\\
      &=\langle x,x \rangle-\langle y,y \rangle\\
      &=\|x\|^2-\|y\|^2=0.
    \end{align*}
    Jadi $\langle x+y,x-y \rangle=0$.\\

    Secara geometris, jika $X=\mathbb{R}^2$, maka $\mathbf{x}$ dan $\mathbf{y}$ membentuk sudut siku-siku.
    \begin{center}
      \begin{tikzpicture}[scale=1.5]
        \draw[->] (-2,0) -- (2,0) node[right] {$x$};
        \draw[->] (0,-2) -- (0,2) node[above] {$y$};
        \coordinate (B) at (1,1)  {};
        \coordinate (C) at (1,-1) {};
        \coordinate (0) at (0, 0) {};
        \draw[red,->,thick] (0,0) -- (1,1) node[above right] {$\mathbf{x+y}$};
        \draw[red,->,thick] (0,0) -- (1,-1) node[below right] {$\mathbf{x-y}$};
        \draw[blue,->,thick] (0,0) -- (1,0) node[below right] {$\mathbf{x}$};
        \draw[blue,->,thick] (0,0) -- (0,1) node[above right] {$\mathbf{y}$};
        \rightAngle{C}{0}{B}{0.3}
      \end{tikzpicture}
    \end{center}
  \end{solution}
  
  \setcounter{enumi}{7}

  \item Tunjukkan bahwa dalam ruang hasil dalam (inner product space), $x \perp y$ jika dan hanya jika $\|x + \alpha y\| \geq \|x\|$ untuk semua skalar $\alpha$.
  \begin{solution}
    Misalkan $x,y\in X$ adalah ruang hasil dalam.\\
    Jika $x\perp y$, maka $\langle x,y \rangle=0$.\\
    Maka berlaku
    \begin{align*}
      \|x+\alpha y\|^2&=\langle x+\alpha y,x+\alpha y \rangle\\
      &=\langle x,x \rangle+\langle x,\alpha y \rangle+\langle \alpha y,x \rangle+\langle \alpha y,\alpha y \rangle\\
      &=\|x\|^2+\overline{\alpha}\langle x,y \rangle+\alpha\langle y,x \rangle+\|\alpha y\|^2\\
      &=\|x\|^2+0+\|\alpha y\|^2\\
      &\geq \|x\|^2.
    \end{align*}
    Sebaliknya, jika $\|x+\alpha y\|\geq \|x\|$, maka berlaku
    \[
    0=\|x+\alpha y\|^2-\|x\|^2=\|\alpha y\|^2+\overline{\alpha}\langle x,y\rangle+\alpha\langle y,x\rangle.
    \]
    Dengan memilih $\alpha=1$ dan $\overline{\alpha}=-1$ kita mendapatkan $\langle x,y\rangle=0$. Jadi $x\perp y$.
  \end{solution}

  \item Misalkan $V$ adalah ruang vektor dari semua fungsi kontinu bernilai kompleks pada $J = [a,b]$. Misalkan
  $\displaystyle
  X_1 = (V, \|\cdot\|_\infty), \quad \text{dengan} \quad \|x\|_\infty = \max_{\substack{t \in J}} |x(t)|,
  $
  dan
  $
  X_2 = (V, \|\cdot\|_2),
  $
  dengan
  \[
  \|x\|_2 = \left\langle x, x \right\rangle^{1/2}, \quad \quad \quad \langle x, y \rangle = \int_a^b x(t)\overline{y(t)} \, dt.
  \]

  Tunjukkan bahwa pemetaan identitas $x \mapsto x$ dari $X_1$ ke $X_2$ adalah kontinu. (Pemetaan ini bukan homeomorfisme. $X_2$ tidak lengkap.)
\end{enumerate}

\end{document}
