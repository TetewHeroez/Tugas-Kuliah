\documentclass[aspectratio=169]{beamer}
\usepackage{colortbl,tabularx,mathrsfs,calligra}
\usepackage{amsmath,amsfonts,amssymb,amsthm}
\usepackage{ragged2e}
\usepackage{tikz}
\usepackage{caption}
\usepackage{wrapfig}
\usepackage{multirow}
\usepackage{multicol}
\usepackage{array}
\usepackage{physics}
\usepackage{pgfplots, tkz-euclide,calc}
    \pgfplotsset{compat=1.18}
\usepackage{listings}

\graphicspath{{C:/Users/teoso/OneDrive/Documents/Tugas Kuliah/Template Math Depart/}{./foto/}}

\definecolor{HIMAmuda}{HTML}{01D1FD}
\definecolor{HIMAtua}{HTML}{02016A}
\definecolor{HIMAabu}{HTML}{CBCBCC}

\usetheme{Madrid}

\setbeamercolor{palette primary}{bg=HIMAtua,fg=white}
\setbeamercolor{palette secondary}{bg=HIMAmuda,fg=black}
\setbeamercolor{palette tertiary}{bg=HIMAabu,fg=black}
\setbeamercolor{palette quaternary}{bg=HIMAmuda,fg=white}
\setbeamercolor{structure}{fg=HIMAmuda} % itemize, enumerate, etc
\setbeamercolor{section in toc}{fg=HIMAtua} % TOC sections
\setbeamercolor{bibliography item}{parent=palette secondary}
\setbeamercolor*{bibliography entry author}{parent=section in toc}
\setbeamercolor{framesubtitle}{fg=HIMAmuda} % Hanya warna teks subtitle


\usetikzlibrary{shapes.geometric, arrows}

\tikzstyle{startstop} = [ellipse, minimum width=1cm, minimum height=1cm,text centered, draw=black, fill=red!30]
\tikzstyle{process} = [rectangle, minimum width=2cm, minimum height=1cm, text centered, draw=black, fill=blue!30]
\tikzstyle{decision} = [diamond, minimum width=1cm, minimum height=1cm, text centered, draw=black, fill=blue!50]
\tikzstyle{arrow} = [thick,->,>=stealth]

\newcolumntype{L}[1]{>{\raggedright\let\newline\\\arraybackslash\hspace{0pt}}m{#1}}
\newcolumntype{C}[1]{>{\centering\let\newline\\\arraybackslash\hspace{0pt}}m{#1}}
\newcolumntype{R}[1]{>{\raggedleft\let\newline\\\arraybackslash\hspace{0pt}}m{#1}}

\usefonttheme{professionalfonts}
\setbeamertemplate{theorems}[numbered]
% \setbeamercovered{transparent}


\theoremstyle{definition}
% \numberwithin{subsection}{section}
\newtheorem{definisi}{Definisi}
\newtheorem{teorema}{Teorema}
\newtheorem{lema}{Lema}
\newtheorem{soal}{Soal}
\newcommand{\R}{\mathbb{R}}
\newcommand{\N}{\mathbb{N}}
\newcommand{\Z}{\mathbb{Z}}
\newcommand{\C}{\mathbb{C}}
\newcommand{\Q}{\mathbb{Q}}

\AtBeginEnvironment{definisi}{
    \setbeamercolor{block title}{fg=white,bg=HIMAtua}
    \setbeamercolor{block body}{parent=normal text,bg=HIMAtua!30!white}
    \setbeamercolor{item}{fg=HIMAtua}
}
\AtBeginEnvironment{teorema}{
    \setbeamercolor{block title}{bg=darkgray,fg=white}
    \setbeamercolor{block body}{parent=pallette tertiary,bg=HIMAabu!30!white}
    \setbeamercolor{item}{fg=darkgray}
}
\AtBeginEnvironment{lema}{
    \setbeamercolor{block title}{bg=gray,fg=white}
    \setbeamercolor{block body}{parent=pallette tertiary,bg=HIMAabu!50!white}
    \setbeamercolor{item}{fg=gray}
}
\AtBeginEnvironment{soal}{%
  \setbeamercolor{block title}{fg=white,bg=teal} 
  \setbeamercolor{block body}{parent=normal text,bg=teal!30!white} 
  \setbeamercolor{item}{fg=teal}
}

\date{Senin, 2 Juni 2025}
\title[Pengantar Analisis Fungsional]{Pembahasan Kuis 2 Pengantar Analisis Fungsional T.A 2024/2025}
\author[Trio Anfung]{By Trio Anfung 6 People}
\institute[Matematika ITS]{Departemen Matematika\\ Institut Teknologi Sepuluh Nopember}
\titlegraphic{\includegraphics[scale=0.15]{logoITS}$\quad$\includegraphics[scale=0.024]{M.png}}

\begin{document}

\begin{frame}
  \titlepage
\end{frame}

\subsection{Nomor 1}
\section{Pembahasan Soal}
\begin{frame}
  \frametitle{\insertsection}
  \begin{soal}
    Diberikan $c = \{c_n\} \in \ell^\infty$ dan $T_c \in \mathcal{B}(\ell^\infty)$ dengan $T_c(\{x_n\}) = \{c_n x_n\}$. Jika $\inf\{|c_n| : n \in \mathbb{N}\} > 0$ dan $d_n = \frac{1}{c_n}$, maka tunjukkan bahwa $d = \{d_n\} \in \ell^\infty$ dan $T_c T_d = T_d T_c = I$.
  \end{soal}

\end{frame}


\begin{frame}
  \frametitle{\insertsection}
  \framesubtitle{\insertsubsection}
  \begin{definisi}
    Ruang \( \ell^\infty \) didefinisikan sebagai himpunan semua barisan bilangan real (atau kompleks) \( x = (x_n)_{n=1}^\infty \) yang \textbf{terbatas}, yaitu:

    \[
      \ell^\infty = \left\{ (x_n)_{n=1}^\infty \,\middle|\, \sup_{n \in \mathbb{N}} |x_n| < \infty \right\}.
    \]

  \end{definisi}
  \textbf{Jawaban:}\\
  Diberikan $c = \{c_n\} \in \ell ^\infty$, artinya $c$ terbatas maka  $\exists M>0$ sehingga sup$|c_n| \leq M , \ \forall n \in \mathbb{N}$. Karena inf$\{|c_n| : n \in \mathbb{N}\}>0$ maka $\exists  m>0$ sedemikian sehingga
  \[|c_n| \ge m ,\forall n \in \mathbb{N}\]
  Diberikan $d_n = \frac{1}{c_n}$, karena $|c_n| \ge m$ akibatnya $\frac{1}{|c_n|} \leq \frac{1}{m}$ dengan demikian
  \[|d|  = |d_n|=\frac{1}{|c_n|}\leq \frac{1}{m}\]
  artinya sup$|d_n|<\infty$ sehingga $d=\{d_n\}\in \ell^\infty$.\\

  $\therefore d=\{d_n\}\in \ell^\infty$
\end{frame}

\begin{frame}
  \frametitle{\insertsection}
  \framesubtitle{\insertsubsection}
  Dari hasil sebelumnya diperoleh $T_c,T_d \in B(\ell^\infty)$. Akan ditunjukkan $T_cT_d = T_dT_c = I$. Dengan definisi $T_c(\{x_n\}) = \{c_nx_n\}$.
  \begin{align*}
    T_cT_d = T_cT_d(\{{x_n}\})=T_c(\{d_nx_n\})=\{c_nd_nx_n\}=\left\{c_n\cdot\frac{1}{c_n}x_n\right\}=\{x_n\}=I
    \\
    T_dT_c = T_dT_c(\{{x_n}\})=T_d(\{c_nx_n\})=\{d_nc_nx_n\}=\left\{\frac{1}{c_n}\cdot c_nx_n\right\}=\{x_n\}=I
  \end{align*}
  Diperoleh $T_c$ invertibel dan inversnya adalah $T_d$.

  $\therefore T_cT_d=T_dT_c=I$
\end{frame}

\subsection{Nomor 2}
\begin{frame}
  \frametitle{\insertsection}
  \begin{soal}
    Diberikan $T \colon C_R[0,1] \to \mathbb{R}$ transformasi linear terbatas dengan $T(f) = \int_0^1 f(x) \, dx$
    \begin{enumerate}
      \item Tunjukkan bahwa $\|T\| \leq 1$.
      \item Jika $g \in C_R[0,1]$ dengan $g(x) = 1$ untuk semua $x \in [0,1]$, dapatkan $|T(g)|$ dan $\|T\|$.
    \end{enumerate}
  \end{soal}
  \begin{lema}
    Misalkan \(X\) dan \(Y\) adalah ruang bernorma. Jika \(\norm{\cdot}: B(X,Y) \to \mathbb{R}\) didefinisikan sebagai
    \begin{equation*}
      \norm{T} = \sup\{\norm{T(x)}:\norm{x} \leq 1\},
    \end{equation*}
    maka \(\norm{\cdot}\) adalah norma pada \(B(X,Y)\).
  \end{lema}
\end{frame}


\begin{frame}
  \frametitle{\insertsection}
  \framesubtitle{\insertsubsection}
  \textbf{Jawaban:}
  \begin{enumerate}
    \item Karena $T$ operator linear terbatas, norma dari $T$ didefinisikan oleh
          \begin{align*}
            ||T|| & =\sup\{||T(f)||_\R:||f||_{C_\R[0,1]}\leq1\}          \\
                  & =\sup\{|T(f)|:\sup_{x\in[0,1]}\{|f(x)|\}\leq1\}      \\
                  & =\sup\{|T(f)|:|f(x)|\leq1,\forall x\in[0,1]\}        \\
                  & =\sup\{|T(f)|:-1\leq f(x)\leq1,\forall x\in [0,1]\}.
          \end{align*}
          Misalkan $f\in C_\R[0,1]$ suatu fungsi sehingga $||f||_{C_\R[0,1]}\leq1$, yang artinya \[-1\leq f(x)\leq1,\forall x\in [0,1]\]
  \end{enumerate}
\end{frame}

\begin{frame}
  \frametitle{\insertsection}
  \framesubtitle{\insertsubsection}
  Berdasarkan Teorema, berlaku
  \begin{align*}
    \int_0^1-1\ dx & \leq\int_0^1f(x)\ dx\leq\int_0^11\ dx \\
    -1             & \leq\int_0^1f(x)\ dx\leq1             \\
                   & -1\leq T(f)\leq1                      \\
                   & \implies|T(f)|\leq1.
  \end{align*}
  Akibatnya, $1$ merupakan batas atas dari Himpunan $\{|T(f)|:-1\leq f(x)\leq1,\forall x\in [0,1]\}$. Berdasarkan definisi, berlaku
  $$||T||=\sup\{|T(f)|:-1\leq f(x)\leq1,\forall x\in [0,1]\}\leq1.$$
\end{frame}
\begin{frame}
  \frametitle{\insertsection}
  \framesubtitle{\insertsubsection}
  \begin{enumerate}
    \setcounter{enumi}{1}
    \item Kita ketahui bahwa \(g(x) = 1,\,\forall x \in [0,1]\). Dengan menggunakan definisi dari \(T\), kita dapatkan
          \begin{equation*}
            \abs{T(g)} = \int_0^1g(x)\,\dd x = \int_0^1 \dd x = 1.
          \end{equation*}
          Dengan begitu, kita dapatkan \(\abs{T(g)} = \abs{1} = 1\). \\
          Selanjutnya, perhatikan bahwa norma dari \(g\) adalah
          \begin{equation*}
            \norm{g} = \sup_{x\in [0,1]}\abs{g(x)} = \sup_{x\in [0,1]} \abs{1} = 1.
          \end{equation*}
          Ekspresi di atas dapat dinyatakan kembali sebagai \(\norm{g} \leq 1\). Berdasarkan sifat dari supremum, kita peroleh
          \begin{equation*}
            1 = \abs{T(g)} \leq \sup\qty{\abs{T(f)}:\norm{f}\leq 1} = \norm{T}.
          \end{equation*}
          Dari jawaban soal (a) dan pertidaksamaan di atas, kita dapatkan bahwa \(\norm{T} \leq 1\) dan \(\norm{T} \geq 1\). Dengan demikian, kita dapat simpulkan bahwa \(\norm{T} = 1\).
  \end{enumerate}
\end{frame}

\subsection{Nomor 3}
\begin{frame}
  \frametitle{\insertsection}
  \begin{soal}
    Misalkan $\mathcal{P}$ adalah subspace linear dari $C_c[0,1]$ yang memuat semua fungsi polinomial. Jika $T \colon \mathcal{P} \to \mathbb{C}$ adalah transformasi linear dengan $T(p) = p'(1)$, maka tunjukkan $T$ tidak kontinu.  \end{soal}
  \begin{lema}
    $T$ dikatakan tidak terbatas jika $\exists p\in \mathcal{P}$ sehingga $\forall k>0$ berlaku $||T(p)||>k||p||$.
  \end{lema}
\end{frame}
\begin{frame}
  \frametitle{\insertsection}
  \framesubtitle{\insertsubsection}
  \textbf{Jawaban:}\\
  Ambil sembarang $k > 0$. Pilih $p_n(x) = x^n$ untuk $n > k$, $x \in [0,1]$. Maka:
  \[
    p_n'(x) = nx^{n-1}, \quad \|p_n\| = \sup\{|p_n(x)| : x \in [0,1]\} = 1.
  \]
  Selanjutnya:
  \[
    \|T(p_n)\| = |p_n'(1)| = |n(1)^{n-1}| = n > k = k \cdot \|p_n\|.
  \]
  Jadi, $T$ tidak terbatas.\\

  Sehingga dengan menggunakan Lema 2 dapat disimpulkan $T$ tidak kontinu
\end{frame}

\subsection{Nomor 4}
\begin{frame}
  \frametitle{\insertsection}
  \begin{soal}
    Jika $a = \{a_n\} \in \ell^1$ dan $\{x_n\} \in c_0$, tunjukkan bahwa $\{a_n x_n\} \in \ell^1$ dan transformasi linear $f_a \colon c_0 \to \mathbb{C}$ dengan $f_a(\{x_n\}) = \sum_{n=1}^\infty a_n x_n$ kontinu dengan $\|f_a\| \leq \|\{a_n\}\|_{\ell^1}$.
  \end{soal}
\end{frame}

\begin{frame}
  \frametitle{\insertsection}
  \framesubtitle{\insertsubsection}
  \textbf{Jawaban:}\\
  Diketahui bahwa $\{a_n\} \in \ell^1$ dan $\{x_n\} \in c_0$ dimana
  \[c_0:=\left\{(x_n)\,|\,\lim_{n \to \infty} x_n = 0\right\}\]
  Jelas barisan $a_nx_n$ merupakan barisan kompleks juga, selanjutnya akan ditunjukkan bahwa deret mutlak suku-sukunya akan konvergen. Perhatikan
  $$
    \sum_{n=1}^\infty |a_n x_n| = \sum_{n=1}^\infty |a_n|\cdot|x_n| \leq \sup_{n\in\N} |x_n| \cdot \sum_{n=1}^\infty |a_n|=||x_n||\sum_{n=1}^\infty |a_n|
  $$

  Karena $\{x_n\} \in c_0 \subset \ell^\infty$ maka $\exists M > 0$ sehingga $|x_n| \leq M$, disisi lain ingat bahwa $\{a_n\} \in \ell^1$ yang mempunyai sifat $\sum |a_n| < \infty$ atau deret $\sum |a_n|$ konvergen.

  $\therefore$ Terbukti bahwa $a_nx_n\in\ell^1$
\end{frame}

\begin{frame}
  \frametitle{\insertsection}
  \framesubtitle{\insertsubsection}
  Dapat dilihat bahwa $f_a$ transformasi linear sebab untuk $\{x_n\}$, $\{y_n\} \in c_0$, dan $\alpha, \beta \in \mathbb{C}$ diperoleh
  $$
    f_a(\alpha \{x_n\} + \beta \{y_n\}) = \sum_{n=1}^\infty a_n (\alpha x_n + \beta y_n) = \alpha \sum a_n x_n + \beta \sum a_n y_n = \alpha f_a(\{x_n\}) + \beta f_a(\{y_n\})
  $$

  Terakhir untuk $||f_a||$ dapat ditulis
  $$
    \|f_a\| = \sup_{\|x\| \leq 1} |f_a(x)|
  $$
  sehingga dari informasi diatas, ambil $x = \{x_n\} \in c_0$ dengan $\|x\|_\infty = \sup_n |x_n| \leq 1$, maka
  $$
    |f_a(x)| = \left| \sum_{n=1}^\infty a_n x_n \right| \leq \sum |a_n x_n| = \sum |a_n| \cdot |x_n| \leq \sup |x_n| \cdot \sum |a_n| \leq \sum |a_n| = ||a_n||_1
  $$

  $\therefore$ Terbukti bahwa $\|f_a\| \leq \|\{a_n\}\|_1
  $
\end{frame}

\subsection{Nomor 5}
\begin{frame}
  \frametitle{\insertsection}
  \begin{soal}
    \begin{enumerate}
      \item Jika $(x_1, x_2, x_3, \ldots) \in \ell^2$, tunjukkan $(0, 4x_1, x_2, 4x_3, x_4, \ldots) \in \ell^2$.
      \item Jika $T \colon \ell^2 \to \ell^2$ transformasi linear dengan
            \[
              T(x_1, x_2, x_3, \ldots) = (0, 4x_1, x_2, 4x_3, x_4, \ldots),
            \]
            Tunjukkan bahwa $T$ kontinu.
    \end{enumerate}
  \end{soal}
\end{frame}

\begin{frame}
  \frametitle{\insertsection}
  \framesubtitle{\insertsubsection}
  \textbf{Jawaban:}
  \begin{enumerate}
    \item Perhatikan bahwa $(x_1,x_2,x_3,\cdots)\in l^2$, yang berakibat
          \begin{equation*}
            \sum_{i=1}^\infty|x_i|^2<\infty.
          \end{equation*}
          Perhatikan pula bahwa $|x_i|^2\geq0$ untuk setiap $i\in\N$. Akibatnya, berlaku
          \begin{equation*}
            \sum_{i=1}^\infty|x_{2i-1}|^2<\sum_{i=1}^\infty|x_i|^2<\infty\implies\sum_{i=1}^\infty15|x_{2i-1}|^2=15\sum_{i=1}^\infty|x_{2i-1}|^2<\infty.
          \end{equation*}
  \end{enumerate}
\end{frame}

\begin{frame}
  \frametitle{\insertsection}
  \framesubtitle{\insertsubsection}
  Berdasarkan Teorema, berlaku
  \begin{align*}
    \sum_{i=1}^\infty|x_i|^2+\sum_{i=1}^\infty15|x_{2i-1}|^2 & <\infty  \\
    \sum_{i=1}^\infty|x_i|^2+15|x_{2i-1}|^2                  & <\infty  \\
    16|x_1|^2+|x_2|^2+16|x_3|^2+|x_4|^2+\cdots               & <\infty  \\
    0^2+|4x_1|^2+|x_2|^2+|4x_3|^2+|x_4|^2+\cdots             & <\infty.
  \end{align*}
  Demikian dari definisi, terbukti bahwa $(0,4x_1,x_2,4x_3,x_4\cdots)\in l^2.$
\end{frame}

\begin{frame}
  \frametitle{\insertsection}
  \framesubtitle{\insertsubsection}
  \textbf{Jawaban:}
  \begin{enumerate}
    \setcounter{enumi}{1}
    \item Untuk membuktikan kontinu, dapat dibuktikan bahwa ada $k \in \mathbb{R}^+$ sehingga $||T(x)||\leq k||x||$ untuk semua $x\in X$. Ambil sebarang $(x_n)\in l^2$, sehingga
          \begin{align*}
            ||T(x)||_2^2
             & =\sqrt{0^2+|4x_1|^2+|x_2|^2+|4x_3|^2+\ldots}^2 \notag                         \\
             & =16|x_1|^2+|x_2|^2+16|x_3|^2+\ldots \notag                                    \\
             & \leq 16|x_1|^2+16|x_2|^2+16|x_3|^2+\ldots \notag                              \\
             & =16(|x_1|^2+|x_2|^2+|x_3|^2+\ldots) \notag                                    \\
             & =16\sum_{n=1}^\infty|x_n|^2 =16||x||_2^2 \iff ||T(x)||_2 \leq 4||x||^2 \notag
          \end{align*}
          Karena ada $k=4$ yang memenuhi $||T(x)||_2 \leq k||x||^2$ untuk semua $x \in l^2$. Dengan demikian $T$ kontinu pada $l^2.$
  \end{enumerate}
\end{frame}

\end{document}