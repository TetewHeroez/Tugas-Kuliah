\documentclass{article}
\usepackage{amsmath,amssymb,amsfonts,amsthm}
\usepackage{multicol}
\usepackage{multirow}
\usepackage{mathtools}
\usepackage{soul}
\usepackage{hyperref}
\hypersetup{
    colorlinks=true,
    linkcolor=blue,
    filecolor=magenta,      
    urlcolor=cyan,
    pdftitle={Overleaf Example},
    pdfpagemode=FullScreen,
    }
\usepackage{color}
\usepackage[table]{xcolor}
\usepackage[T1]{fontenc}
\usepackage{etoolbox}
\usepackage{multicol}
\usepackage{multirow}
\usepackage{fancyhdr}
\usepackage{graphicx}
\usepackage{tcolorbox}
\usepackage{array}
\usepackage{amsthm}
\usepackage{titlesec}
\usepackage{tikz, tkz-euclide}
  \usetikzlibrary{arrows.meta,calc}
  \newcommand\rightAngle[4]{
  \pgfmathanglebetweenpoints{\pgfpointanchor{#2}{center}}{\pgfpointanchor{#1}{center}}
  \coordinate (tmpRA) at ($(#2)+(\pgfmathresult+45:#4)$);
  \draw[red!60!black,thick] ($(#2)!(tmpRA)!(#1)$) -- (tmpRA) -- ($(#2)!(tmpRA)!(#3)$);
}
\renewcommand{\baselinestretch}{1.2}

\titleformat*{\section}{\large\bfseries}
\titleformat*{\subsection}{\normalsize\bfseries}

\newtheorem{theorem}{Theorem}
\newtheorem*{teorema}{Teorema}
\newtheorem*{definisi}{Definisi}
\theoremstyle{definition}
\newtheorem*{bukti}{Bukti}

\newcommand{\Arg}{\text{Arg}}
\newcommand{\R}{\mathbb{R}}
\newcommand{\C}{\mathbb{C}}
\newcommand{\N}{\mathbb{N}}
\newcommand{\Z}{\mathbb{Z}}

\newtcolorbox{solution}[1][]{
    colback=blue!5!white, 
    colframe=blue!75!black,
    fonttitle=\bfseries, 
    colbacktitle=blue!85!black,
    title=Solusi,
    #1
}

\begin{document}
  \fancyhead[L]{\textit{Teosofi Hidayah Agung}}
  \fancyhead[R]{\textit{5002221132}}
  \pagestyle{fancy}

  \begin{enumerate}
    \item Diberikan \( (\mathbb{R}, d_1) \) ruang metrik Euclid dan \( (C([a,b], d_2) \) ruang metrik dengan
    $
        d_2(x, y) = \max \{ |x(t) - y(t)| : t \in [a, b] \}.
    $
    Dapatkan bola buka satuan dari masing-masing ruang metrik tersebut dan gambarkan.

    \item Misalkan \( c_0 \) adalah himpunan semua barisan bilangan real yang konvergen ke nol. Tunjukkan bahwa \( c_0 \subset \ell^\infty \) adalah \textit{closed vector subspace} dari \( \ell^\infty \).

    \item Diberikan \( X \) ruang inner product. Jika \( x \perp y \) di \( X \), tunjukkan bahwa
    \[
        \|x + y\|^2 = \|x\|^2 + \|y\|^2.
    \]
    Tunjukkan bahwa kebalikannya tidak berlaku jika \( X \) kompleks.

    \item Tunjukkan bahwa \( \ell^2 \) adalah ruang Hilbert dengan
    \[
        \langle x, y \rangle = \sum_{i=1}^{\infty} x_i \overline{y_i}.
    \]
\end{enumerate}

\end{document}