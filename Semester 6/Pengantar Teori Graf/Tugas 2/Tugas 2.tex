\documentclass[a4paper]{article}
\usepackage{amsmath,amssymb,amsfonts,amsthm}
\usepackage{multicol}
\usepackage{multirow}
\usepackage{mathtools}
\usepackage{soul}
\usepackage{hyperref}
\hypersetup{
    colorlinks=true,
    linkcolor=blue,
    filecolor=magenta,      
    urlcolor=cyan,
    pdftitle={Overleaf Example},
    pdfpagemode=FullScreen,
    }
\usepackage{color}
\usepackage[table]{xcolor}
\usepackage[T1]{fontenc}
\usepackage{etoolbox}
\usepackage{multicol}
\usepackage{multirow}
\usepackage{fancyhdr}
\usepackage{graphicx}
\usepackage{array}
\usepackage{amsthm}
\usepackage{titlesec}
\usepackage{tikz}
\usepackage{bm}
\usepackage{enumitem}
\usetikzlibrary{arrows.meta,calc}
\usetikzlibrary{positioning,automata}
\renewcommand{\baselinestretch}{1.2}

\titleformat*{\section}{\large\bfseries}
\titleformat*{\subsection}{\normalsize\bfseries}

\graphicspath{{C:/Users/teoso/OneDrive/Documents/Tugas Kuliah/Template Math Depart/}}

\newtheorem{theorem}{Theorem}
\newtheorem*{teorema}{Teorema}
\newtheorem*{definisi}{Definisi}
\theoremstyle{definition}
\newtheorem*{bukti}{Bukti}

\newcommand{\Arg}{\text{Arg}}
\newcommand{\N}{\mathbb{N}}
\newcommand{\Z}{\mathbb{Z}}

\begin{document}
\fancyhead[L]{\textit{Teosofi Hidayah Agung\\5002221132}}
\fancyhead[R]{\textit{Junika Irdia Indi Astudin\\5002221015}}
\pagestyle{fancy}
\noindent Buktikan bahwa jika \(G\) adalah sebuah pohon dan setiap simpulnya memiliki derajat ganjil, maka jumlah sisi pada \(G\) adalah ganjil.
\begin{bukti}
Menurut \textit{Handshaking Lemma}, didapatkan fakta bahwa 
\[\sum_{v \in V(G)} \text{deg}(v) = 2|E(G)|\]
Karena setiap simpul pada \(G\) memiliki derajat ganjil, maka berdasarkan lemma diatas haruslah berakibat jumlah simpul pada \(G\) adalah genap, katakanlah \(|V(G)| = 2k\) untuk suatu $k\in\N$. Kemudian diketahui bahwa \(G\) adalah pohon yang dimana banyak sisinya adalah \(|E(G)| = |V(G)| - 1 = 2k - 1\). Oleh karena itu, jumlah sisi pada \(G\) adalah ganjil. $\square$
\end{bukti}
\end{document}