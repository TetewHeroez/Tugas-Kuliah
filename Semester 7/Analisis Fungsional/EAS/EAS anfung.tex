\documentclass[11pt,a4paper]{article}
\usepackage{graphicx} 
\usepackage{multirow}
\usepackage{enumitem}
\usepackage{amssymb}
\usepackage{amsmath}
\usepackage{setspace}
\setstretch{1.5}
\usepackage{amsthm}
\usepackage{xcolor}
\usepackage{geometry}
  \geometry{
    left = 25mm,
    right = 25mm,
    top = 20mm,
    bottom = 30mm,
  }
\usepackage{fancyhdr}
\renewcommand{\headrulewidth}{0pt}
\pagestyle{fancy}

\graphicspath{{C:/Users/teoso/OneDrive/Documents/Tugas Kuliah/Template Math Depart/}{D:/Hada Touya/Tugas-Kuliah/Template Math Depart/}}

\newcommand{\R}{\mathbb{R}}
\newcommand{\N}{\mathbb{N}}
\newcommand{\Z}{\mathbb{Z}}
\newcommand{\Q}{\mathbb{Q}}
\newcommand{\jawab}{
  \begin{center}
    \textbf{SOLUSI}
  \end{center}
}

\newtheorem*{teorema}{Teorema}
\newtheorem*{definisi}{Definisi}



\begin{document}
\pagenumbering{gobble}
\begin{table}[h!]
  \centering
  \begin{tabular}{r c}
    \includegraphics[width=2cm]{ITS.png}
     & \begin{tabular}{lcll}
         \multicolumn{4}{l}{\begin{tabular}{c}
                             \MakeUppercase{evaluasi akhir semester gasal 2024/2025} \\
                             \MakeUppercase{departemen matematika fsad its}          \\
                             \MakeUppercase{program sarjana}                         \\
                           \end{tabular}}                    \\
         \\
         Matakuliah    & : & \multicolumn{2}{l}{Analisis Fungsional}                                   \\
         Hari, Tanggal & : & \multicolumn{2}{l}{Selasa, 9 Desember 2024}                               \\
         Waktu / Sifat & : & \multicolumn{2}{l}{11:00 -- 12:40 (100 menit) / \textit{Closed Book}}     \\
         Dosen         & : & \multicolumn{2}{l}{Dr. Mahmud Yunus, M.Si. dan Dr. Sunarsini, S.Si, M.Si} \\
                       &   & \hspace{8cm} \multirow{-12}{0cm}{\includegraphics[width=2cm]{M.png}}
       \end{tabular}

    \\ \hline
    \multicolumn{2}{|l|}{\color{red}\MakeUppercase{harap diperhatikan !!!}}                                                \\
    \multicolumn{2}{|l|}{\color{red}Segala jenis pelanggaran (mencontek, kerjasama, dsb) yang dilakukan pada saat ETS/EAS} \\
    \multicolumn{2}{|l|}{\color{red}akan dikenakan sanksi pembatalan matakuliah pada semester yang sedang berjalan.}       \\
    \hline
  \end{tabular}
\end{table}
\renewcommand{\arraystretch}{1.5}
\noindent\vspace*{-1cm}
\begin{enumerate}
  \item
        Pandang ruang vektor $X = C[-1,1]$ dan pemetaan $\| \cdot \|_{\infty} : X \to \mathbb{R}$ yang didefinisikan dengan
        $\displaystyle
          \| f \|_{\infty} = \sup_{x \in [-1,1]} \{ |f(x)| \},
        $
        serta barisan fungsi yang didefinisikan dengan
        \[
          f_n(x) =
          \begin{cases}
            0,                      & -1 \le x < -\tfrac{1}{n},           \\[4pt]
            \frac{1}{2}\,(n x + 1), & -\tfrac{1}{n} \le x < \tfrac{1}{n}, \\[6pt]
            1,                      & \tfrac{1}{n} \le x \le 1 .
          \end{cases}
        \]

        \begin{enumerate}[label=(\alph*)]
          \item Tunjukkan bahwa $\| \cdot \|_{\infty}$ di atas merupakan norma di $X$.

          \item Tunjukkan bahwa $(f_n)$ barisan di $X$ yang konvergen dan dapatkan limitnya.

          \item Apakah $(X,\|\cdot\|_{\infty})$ tersebut merupakan ruang Banach? Berikan penjelasan untuk jawaban Anda.
        \end{enumerate}

  \item
        Diketahui $\{ f_k \}_{k=1}^{\infty}$ barisan di ruang Hilbert $L^2[a,b]$ untuk suatu $a,b\in\mathbb{R}$, dan $\{\alpha_k\}_{k=1}^{\infty}$ barisan di $\ell^2(\mathbb{N})$. Jika $\sum_{k=1}^{\infty} \alpha_k f_k$ konvergen, tunjukkan bahwa untuk semua $f \in L^2[a,b]$ berlaku
        \[
          \left\langle f, \sum_{k=1}^{\infty} \alpha_k f_k \right\rangle
          = \sum_{k=1}^{\infty} \alpha_k \langle f, f_k \rangle .
        \]

  \item
        Pandang $\mathcal{H}$ suatu ruang Hilbert dan $G$ ruang bagian tertutup dari $\mathcal{H}$. Jika $\{ e_k \}_{k=1}^{\infty}$ adalah basis ortonormal untuk $G$, tunjukkan bahwa pemetaan $P : \mathcal{H} \to G$ yang didefinisikan dengan
        \[
          Pv = \sum_{k=1}^{\infty} \langle v, e_k \rangle e_k ,\qquad v \in \mathcal{H},
        \]
        adalah proyeksi ortogonal dari $\mathcal{H}$ pada $G$.

  \item
        Jika $f \in L^2(\mathbb{R})$ mempunyai tumpuan kompak, tunjukkan bahwa $f \in L^1(\mathbb{R})$.
        \begin{center}
          -----o0o-----
        \end{center}
\end{enumerate}

\newpage
\fancyfoot[C]{}
\fancyfoot[R]{\color{blue}\textit{Teosofi Hidayah Agung - 5002221132}}
\jawab
\begin{enumerate}
  \item
        \begin{enumerate}
          \item Untuk setiap $f,g \in X$ dan $\alpha \in \R$, kita buktikan tiga sifat norma:
                \begin{itemize}
                  \item (Positivitas) Karena $|f(x)| \ge 0$ untuk semua $x \in [-1,1]$, maka $\| f \|_{\infty} = \sup_{x \in [-1,1]} |f(x)| \ge 0$.
                  \item (Homogenitas) Jika $\| f \|_{\infty} = 0$, maka $|f(x)| = 0$ untuk semua $x$, sehingga $f(x) = 0$ untuk semua $x$, yaitu $f$ adalah fungsi nol. Sebaliknya, jika $f$ adalah fungsi nol, maka jelas $\| f \|_{\infty} = 0$.

                  \item (Skalar) Untuk setiap $\alpha \in \R$, berlaku
                        \[
                          \| \alpha f \|_{\infty} = \sup_{x \in [-1,1]} | \alpha f(x) | = |\alpha| \sup_{x \in [-1,1]} |f(x)| = |\alpha| \| f \|_{\infty} .
                        \]

                  \item (Ketaksamaan Segitiga) Untuk setiap $x \in [-1,1]$, berlaku
                        \[
                          |f(x) + g(x)| \le |f(x)| + |g(x)| .
                        \]
                        Mengambil supremum di kedua sisi menghasilkan
                        \[
                          \| f + g \|_{\infty} = \sup_{x \in [-1,1]} |f(x) + g(x)| \le \sup_{x \in [-1,1]} |f(x)| + \sup_{x \in [-1,1]} |g(x)| = \| f \|_{\infty} + \| g \|_{\infty} .
                        \]
                \end{itemize}
                Dengan demikian, $\| \cdot \|_{\infty}$ memenuhi keempat sifat norma.

          \item Klaim bahwa barisan $(f_n)$ konvergen ke fungsi $f$ yang didefinisikan sebagai
                \[
                  f(x) =
                  \begin{cases}
                    0,           & -1 \le x < 0, \\
                    \frac{1}{2}, & x = 0,        \\
                    1,           & 0 < x \le 1 .
                  \end{cases}
                \]
                Selanjutnya bagi kasus untuk setiap $x \in [-1,1]$:
                \begin{itemize}
                  \item Jika $x < 0$, maka pilih $N=\lceil -\frac{1}{x} \rceil$. Maka untuk setiap $n \ge N$, $x < -\frac{1}{n}$. Oleh karena itu, untuk semua $n \ge N$, $f_n(x) = 0 = f(x)$.

                  \item Jika $x = 0$, maka untuk setiap $n \in \mathbb{N}$, $f_n(0) = \frac{1}{2} = f(0)$.

                  \item Jika $x > 0$, maka pilih $N=\lceil \frac{1}{x} \rceil$. Maka untuk setiap $n \ge N$, $x > \frac{1}{n}$. Oleh karena itu, untuk semua $n \ge N$, $f_n(x) = 1 = f(x)$.
                \end{itemize}
                Dengan demikian, terbukati bahwa untuk $n\to \infty$, $f_n(x) \to f(x)$ untuk setiap $x \in [-1,1]$.
          \item Ruang $(X, \| \cdot \|_{\infty})$ bukan merupakan ruang Banach karena terdapat barisan $(f_n)$ di $X$ yang konvergen ke fungsi $f$ yang tidak kontinu pada $x=0$, sehingga $f \notin X$. Oleh karena itu, $X$ tidak lengkap terhadap norma $\| \cdot \|_{\infty}$.
        \end{enumerate}

  \item Pertama-tama definsikan deret parsial
        \[
          S_N = \sum_{k=1}^{N} \alpha_k f_k .
        \]
        Karena $\sum_{k=1}^{\infty} \alpha_k f_k$ konvergen, maka barisan $(S_N)$ konvergen ke suatu elemen $S = \sum_{k=1}^{\infty} \alpha_k f_k \in L^2[a,b]$. Dapat dituliskan sebagai $\| S_N - S \|_{L^2} \to 0$ saat $N \to \infty$.

        Selanjutnya perhatikan bahwa untuk setiap $f \in L^2[a,b]$, berlaku
        \[
          \left\langle f, S_N \right\rangle = \left\langle f, \sum_{k=1}^{N} \alpha_k f_k \right\rangle = \sum_{k=1}^{N} \alpha_k \langle f, f_k \rangle
        \]
        dan
        \[
          \left\langle f, S \right\rangle = \left\langle f, \sum_{k=1}^{\infty} \alpha_k f_k \right\rangle.
        \]
        Kemudian akan dibuktikan bahwa $\langle f, S_N \rangle \to \langle f, S \rangle$ saat $N \to \infty$. Dengan menggunakan Ketaksamaan Cauchy-Schwarz, perhatikan bahwa
        \[
          | \langle f, S_N \rangle - \langle f, S \rangle | = | \langle f, S_N - S \rangle | \le \| f \|_{L^2} \| S_N - S \|_{L^2} .
        \]
        Karena $\| S_N - S \|_{L^2} \to 0$ saat $N \to \infty$, maka $| \langle f, S_N \rangle - \langle f, S \rangle | \to 0$ saat $N \to \infty$. Dengan demikian, diperoleh
        \[
          \left\langle f, \sum_{k=1}^{\infty} \alpha_k f_k \right\rangle = \lim_{N \to \infty} \left\langle f, S_N \right\rangle = \lim_{N \to \infty} \sum_{k=1}^{N} \alpha_k \langle f, f_k \rangle = \sum_{k=1}^{\infty} \alpha_k \langle f, f_k \rangle .
        \]

  \item Untuk setiap $v \in \mathcal{H}$, kita perlu menunjukkan bahwa $Pv$ adalah proyeksi ortogonal dari $v$ pada $G$. Pertama-tama, perhatikan bahwa $Pv \in G$ karena merupakan kombinasi linear dari elemen basis ortonormal $\{ e_k \}_{k=1}^{\infty}$.

        Selanjutnya, kita perlu menunjukkan bahwa $v - Pv$ ortogonal terhadap setiap elemen di $G$. Untuk setiap $g \in G$, dapat dituliskan sebagai
        \[
          g = \sum_{j=1}^{\infty} \beta_j e_j ,
        \]
        untuk beberapa koefisien $\beta_j \in \R$. Maka,
        \[
          \langle v - Pv, g \rangle = \left\langle v - \sum_{k=1}^{\infty} \langle v, e_k \rangle e_k, \sum_{j=1}^{\infty} \beta_j e_j \right\rangle .
        \]
        Menggunakan linearitas dan ortonormalitas basis, diperoleh
        \[
          \langle v - Pv, g \rangle = \langle v, g \rangle - \sum_{k=1}^{\infty} \langle v, e_k \rangle \beta_k = 0 .
        \]
        Dengan demikian, $v - Pv$ ortogonal terhadap setiap elemen di $G$, sehingga $Pv$ adalah proyeksi ortogonal dari $\mathcal{H}$ pada $G$.
\end{enumerate}

\end{document}