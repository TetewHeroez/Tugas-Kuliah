\documentclass[a4paper]{article}
\usepackage{amsmath,amssymb,amsfonts,amsthm}
\usepackage{multicol}
\usepackage{multirow}
\usepackage{mathtools}
\usepackage{soul}
\usepackage{enumitem}
\usepackage{hyperref}
\hypersetup{
    colorlinks=true,
    linkcolor=blue,
    filecolor=magenta,      
    urlcolor=cyan,
    pdftitle={Overleaf Example},
    pdfpagemode=FullScreen,
    }
\usepackage{color}
\usepackage[table]{xcolor}
\usepackage[T1]{fontenc}
\usepackage{etoolbox}
\usepackage{multicol}
\usepackage{multirow}
\usepackage{fancyhdr}
\usepackage{graphicx}
\usepackage{array}
\usepackage{amsthm}
\usepackage{titlesec}
\usepackage{tikz}
\usetikzlibrary{arrows.meta,calc}
\renewcommand{\baselinestretch}{1.2}

\titleformat*{\section}{\large\bfseries}
\titleformat*{\subsection}{\normalsize\bfseries}

\graphicspath{{C:/Users/teoso/OneDrive/Documents/Tugas Kuliah/Template Math Depart/}}

\newtheorem{theorem}{Theorem}
\newtheorem*{teorema}{Teorema}
\newtheorem*{definisi}{Definisi}
\theoremstyle{definition}
\newtheorem*{bukti}{Bukti}

\newcommand{\Arg}{\text{Arg}}

\begin{document}
\fancyhead[L]{\textit{Teosofi Hidayah Agung}}
\fancyhead[R]{\textit{5002221132}}
\pagestyle{fancy}
\noindent
\textbf{Soal 3.15}
Misalkan $p \in [1, \infty)$ dan tinjau pemetaan
\[
  T : \ell^p(\mathbb{N}) \to \ell^p(\mathbb{N}), \qquad
  T\{x_k\}_{k=1}^\infty := \{x_k + x_{k+1}\}_{k=1}^\infty.
\]
\begin{enumerate}
  \item[(i)] Tunjukkan bahwa $T$ benar-benar memetakan $\ell^p(\mathbb{N})$ ke $\ell^p(\mathbb{N})$.
  \item[(ii)] Tunjukkan bahwa $T$ bersifat linear dan terbatas.
\end{enumerate}

\bigskip

\textbf{Penyelesaian:}

\medskip

\begin{enumerate}[label=(\roman*)]
  \item Definisikan operator \textit{shift} $S : \ell^p(\mathbb{N}) \to \ell^p(\mathbb{N})$ sebagai
        \[
          S(x_1, x_2, x_3, \dots) = (x_2, x_3, x_4, \dots).
        \]
        Jelas bahwa jika $x = (x_k) \in \ell^p(\mathbb{N})$, maka
        \[
          \|Sx\|_p^p = \sum_{k=1}^\infty |x_{k+1}|^p = \sum_{k=2}^\infty |x_k|^p \le \sum_{k=1}^\infty |x_k|^p = \|x\|_p^p.
        \]
        Dengan demikian $Sx \in \ell^p(\mathbb{N})$.

        Selanjutnya, karena $T(x) = x + Sx$, dengan menggunakan \textbf{ketaksamaan Minkowski} diperoleh
        \[
          \|T(x)\|_p = \|x + Sx\|_p \le \|x\|_p + \|Sx\|_p \le 2\|x\|_p < \infty.
        \]
        Maka $T(x) \in \ell^p(\mathbb{N})$, sehingga $T$ memang memetakan $\ell^p(\mathbb{N})$ ke dalam $\ell^p(\mathbb{N})$.

  \item Ambil $x = (x_k)$ dan $y = (y_k)$ di $\ell^p(\mathbb{N})$, serta $\alpha, \beta \in \mathbb{R}$ (atau $\mathbb{C}$). Maka
        \begin{align*}
          T(\alpha x + \beta y) & = \{(\alpha x_k + \alpha x_{k+1} + \beta y_k + \beta y_{k+1})\}_{k=1}^\infty       \\
                                & = \{\alpha (x_k + x_{k+1}) + \beta (y_k + y_{k+1})\}_{k=1}^\infty                  \\
                                & = \alpha \{(x_k + x_{k+1})\}_{k=1}^\infty + \beta \{(y_k + y_{k+1})\}_{k=1}^\infty \\
                                & = \alpha T(x) + \beta T(y).
        \end{align*}
        Dengan demikian, $T$ adalah operator linear.

        \medskip

        Kemudian untuk setiap $x \in \ell^p(\mathbb{N})$, berlaku
        \[
          \|T(x)\|_p \le \|x\|_p + \|Sx\|_p.
        \]
        Selanjutnya,
        \[
          \|Sx\|_p^p = \sum_{k=1}^\infty |x_{k+1}|^p = \sum_{k=2}^\infty |x_k|^p \le \sum_{k=1}^\infty |x_k|^p = \|x\|_p^p.
        \]
        Dengan demikian, $\|Sx\|_p \le \|x\|_p$, dan memenuhi $\|S\| = 1$.

        Maka
        \[
          \|T(x)\|_p \le \|x\|_p + \|Sx\|_p \le 2\|x\|_p,
        \]
        sehingga $T$ adalah operator terbatas dengan
        \[
          \|T\| \le 2.
        \]
        Jadi, operator $T$ bersifat linear dan terbatas pada ruang $\ell^p(\mathbb{N})$.

        \hfill$\blacksquare$
\end{enumerate}



\end{document}