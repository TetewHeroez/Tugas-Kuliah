\documentclass{article}
\usepackage{graphicx} 
\usepackage{multirow}
\usepackage{enumitem}
\usepackage{amssymb}
\usepackage{amsmath}
\usepackage{xcolor}
\usepackage{cancel}
\usepackage{tcolorbox}
\usepackage{physics}
\usepackage{geometry}
\usepackage{tikz}
\usepackage{tikz-3dplot}
\usepackage{pgfplots, tkz-euclide,calc}

\usetikzlibrary{angles,quotes} % for pic (angle labels)
\usetikzlibrary{arrows.meta}
\usetikzlibrary{calc}
\usetikzlibrary{decorations.markings}
\usetikzlibrary{bending} % for arrow head angle
\tikzset{>=latex} % for LaTeX arrow head
\usepackage{xcolor}
\pgfplotsset{compat=newest}

\colorlet{xcol}{blue!60!black}
\colorlet{myred}{red!80!black}
\colorlet{myblue}{blue!80!black}
\colorlet{mygreen}{green!40!black}
\colorlet{mypurple}{red!50!blue!90!black!80}
\colorlet{mydarkred}{myred!80!black}
\colorlet{mydarkblue}{myblue!80!black}
\tikzstyle{xline}=[xcol,thick,smooth]
\tikzstyle{width}=[{Latex[length=5,width=3]}-{Latex[length=5,width=3]},thick]
\tikzstyle{mydashed}=[dash pattern=on 1.7pt off 1.7pt]
\tikzset{
  traj/.style 2 args={xline,postaction={decorate},decoration={markings,
    mark=at position #1 with {\arrow{<}},
    mark=at position #2 with {\arrow{<}}}
  }
}
\def\tick#1#2{\draw[thick] (#1)++(#2:0.12) --++ (#2-180:0.24)}
\def\N{100} % number of samples
    \usetikzlibrary{patterns,snakes,shapes.arrows,3d}
    \usepgfplotslibrary{fillbetween}
	\geometry{
		total = {160mm, 237mm},
		left = 25mm,
		right = 35mm,
		top = 30mm,
		bottom = 30mm,
	}

\newcommand{\jawab}{\textbf{\underline{Solusi}}:}
\newcommand{\del}{\partial}
\newcommand{\cis}{\text{cis}}
\begin{document}
\setlength{\parindent}{0pt}
    \pagenumbering{gobble}
    \noindent
    \begin{tabular}{|lcl|}
     \hline
     Nama&:&Teosofi Hidayah Agung\\
     NRP&:&5002221132\\
     \hline
    \end{tabular}\\~\\
    Diberikan sistem persamaan diferensial non-homogen sebagai berikut:
\[
\frac{dx}{dt} = 3x + 4y + \sin(t)
\]
\[
\frac{dy}{dt} = -2x + y + e^t
\]
\begin{enumerate}[label=(\arabic*)]
    \item Cari solusi untuk sistem homogen
    Abaikan terlebih dahulu suku non-homogen (\(\sin(t)\) dan \(e^t\)) untuk menyelesaikan sistem homogen:
    
    \[
    \frac{dx}{dt} = 3x + 4y
    \]
    \[
    \frac{dy}{dt} = -2x + y
    \]
    
    Ini adalah sistem persamaan diferensial homogen linear. Kita akan menggunakan metode \textbf{nilai eigen} untuk menyelesaikan sistem ini.

    \[
    \frac{d}{dt} \begin{pmatrix} x \\ y \end{pmatrix} = \begin{pmatrix} 3 & 4 \\ -2 & 1 \end{pmatrix} \begin{pmatrix} x \\ y \end{pmatrix}
    \]
    
    Cari nilai eigen dari matriks koefisien:
    \[
    A = \begin{pmatrix} 3 & 4 \\ -2 & 1 \end{pmatrix}
    \]
    Untuk mencari nilai eigen, kita perlu menghitung determinan \(\det(A - \lambda I)\), di mana \(\lambda\) adalah nilai eigen dan \(I\) adalah matriks identitas:
    
    \[
    \det \begin{pmatrix} 3 - \lambda & 4 \\ -2 & 1 - \lambda \end{pmatrix} = 0
    \]
    
    Hitung determinan:
    \[
    (3 - \lambda)(1 - \lambda) - (-2)(4) = 0
    \]
    \[
    (3 - \lambda)(1 - \lambda) + 8 = 0
    \]
    \[
    3 - \lambda - 3\lambda + \lambda^2 + 8 = 0
    \]
    \[
    \lambda^2 - 4\lambda + 11 = 0
    \]
    
    Gunakan rumus kuadrat untuk menyelesaikan:
    \[
    \lambda = \frac{-(-4) \pm \sqrt{(-4)^2 - 4(1)(11)}}{2(1)}
    \]
    \[
    \lambda = \frac{4 \pm \sqrt{16 - 44}}{2}
    \]
    \[
    \lambda = \frac{4 \pm \sqrt{-28}}{2}
    \]
    \[
    \lambda = 2 \pm i\sqrt{7}
    \]
    
    Jadi, nilai eigen adalah \(\lambda = 2 \pm i\sqrt{7}\), yang merupakan nilai kompleks. Karena ada bagian kompleks, solusi dari sistem homogen akan memiliki bentuk eksponensial dengan sinus dan kosinus.

    Solusi umum dari sistem homogen adalah:
    \[
    \begin{pmatrix} x_h(t) \\ y_h(t) \end{pmatrix} = e^{2t} \begin{pmatrix} C_1 \cos(\sqrt{7}t) + C_2 \sin(\sqrt{7}t) \\ C_3 \cos(\sqrt{7}t) + C_4 \sin(\sqrt{7}t) \end{pmatrix}
    \]
    di mana \(C_1, C_2, C_3,\) dan \(C_4\) adalah konstanta.
    
    \item Solusi khusus untuk bagian non-homogen
    Sekarang, kita cari solusi khusus untuk suku non-homogen. Untuk persamaan pertama yang memiliki suku \(\sin(t)\), kita coba solusi dalam bentuk:
    \[
    x_p(t) = A \sin(t) + B \cos(t)
    \]
    
    Untuk persamaan kedua yang memiliki suku \(e^t\), kita coba solusi dalam bentuk:
    \[
    y_p(t) = C e^t
    \]
    
    \item Substitusi solusi khusus
    Substitusi \(x_p(t)\) dan \(y_p(t)\) ke dalam sistem non-homogen:
    
    \begin{itemize}
        \item Persamaan pertama:
        \[
        \frac{dx_p}{dt} = 3x_p + 4y_p + \sin(t)
        \]
        Hitung turunan \(x_p(t)\):
        \[
        \frac{dx_p}{dt} = A \cos(t) - B \sin(t)
        \]
        Substitusi ke dalam persamaan:
        \[
        A \cos(t) - B \sin(t) = 3(A \sin(t) + B \cos(t)) + 4C e^t + \sin(t)
        \]
        Pisahkan bagian yang bergantung pada \(\sin(t)\), \(\cos(t)\), dan \(e^t\).
     
        \item Persamaan kedua:
        \[
        \frac{dy_p}{dt} = -2x_p + y_p + e^t
        \]
        Hitung turunan \(y_p(t)\):
        \[
        \frac{dy_p}{dt} = C e^t
        \]
        Substitusi ke dalam persamaan:
        \[
        C e^t = -2(A \sin(t) + B \cos(t)) + C e^t + e^t
        \]
    \end{itemize}
    Baik, kita lanjutkan dengan mencari solusi khusus.
\begin{itemize}
    \item Persamaan pertama:
    \[
    \frac{dx_p}{dt} = 3x_p + 4y_p + \sin(t)
    \]
    Dengan asumsi solusi khusus:
    \[
    x_p(t) = A \sin(t) + B \cos(t)
    \]
    \[
    y_p(t) = C e^t
    \]
    
    Hitung turunan dari \(x_p(t)\):
    \[
    \frac{dx_p}{dt} = A \cos(t) - B \sin(t)
    \]
    
    Substitusi \(x_p(t)\) dan \(y_p(t)\) ke dalam persamaan pertama:
    \[
    A \cos(t) - B \sin(t) = 3(A \sin(t) + B \cos(t)) + 4C e^t + \sin(t)
    \]
    
    Pisahkan bagian yang mengandung \(\sin(t)\), \(\cos(t)\), dan \(e^t\):
    \[
    A \cos(t) - B \sin(t) = 3A \sin(t) + 3B \cos(t) + 4C e^t + \sin(t)
    \]
    
    Sekarang, kita pisahkan menjadi tiga persamaan berdasarkan fungsi yang berbeda:
    
    1. Untuk suku \(\sin(t)\):
    \[
    -B = 3A + 1
    \]
    \[
    B = -3A - 1
    \]
    
    2. Untuk suku \(\cos(t)\):
    \[
    A = 3B
    \]
    
    3. Untuk suku \(e^t\):
    \[
    0 = 4C
    \]
    Dari sini, kita dapatkan \(C = 0\).
    
    \item Persamaan kedua:
    \[
    \frac{dy_p}{dt} = -2x_p + y_p + e^t
    \]
    Dengan \(y_p(t) = 0\) karena \(C = 0\), maka persamaan ini disederhanakan menjadi:
    \[
    0 = -2(A \sin(t) + B \cos(t)) + e^t
    \]
    Tidak ada suku \(e^t\) di sebelah kiri, sehingga tidak memberikan informasi tambahan.
    
    \item Temukan nilai \(A\) dan \(B\). Dari sistem persamaan yang kita dapatkan:
    
    1. \(B = -3A - 1\)
    2. \(A = 3B\)
    
    Substitusi \(B = -3A - 1\) ke dalam \(A = 3B\):
    \[
    A = 3(-3A - 1)
    \]
    \[
    A = -9A - 3
    \]
    \[
    10A = -3
    \]
    \[
    A = -\frac{3}{10}
    \]
    
    Substitusi nilai \(A\) ke \(B = -3A - 1\):
    \[
    B = -3\left(-\frac{3}{10}\right) - 1
    \]
    \[
    B = \frac{9}{10} - 1
    \]
    \[
    B = -\frac{1}{10}
    \]
\end{itemize}

Solusi khusus adalah:
\[
x_p(t) = -\frac{3}{10} \sin(t) - \frac{1}{10} \cos(t)
\]
\[
y_p(t) = 0
\]

\item Solusi Akhir dari sistem persamaan diferensial adalah gabungan dari solusi homogen dan solusi khusus:
\[
x(t) = e^{2t}(C_1 \cos(\sqrt{7}t) + C_2 \sin(\sqrt{7}t)) - \frac{3}{10} \sin(t) - \frac{1}{10} \cos(t)
\]
\[
y(t) = e^{2t}(C_3 \cos(\sqrt{7}t) + C_4 \sin(\sqrt{7}t))
\]  
    
\end{enumerate}

\end{document}