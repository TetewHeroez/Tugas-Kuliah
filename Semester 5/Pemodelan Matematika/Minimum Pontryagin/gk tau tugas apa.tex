\documentclass{article}
\usepackage{graphicx} 
\usepackage{multirow}
\usepackage{enumitem}
\usepackage{amssymb}
\usepackage{amsmath}
\usepackage{xcolor}
\usepackage{cancel}
\usepackage{tcolorbox}
\usepackage{physics}
\usepackage{geometry}
\usepackage{tikz}
\usepackage{tikz-3dplot}
\usepackage{pgfplots, tkz-euclide,calc}

\usetikzlibrary{angles,quotes} % for pic (angle labels)
\usetikzlibrary{arrows.meta}
\usetikzlibrary{calc}
\usetikzlibrary{decorations.markings}
\usetikzlibrary{bending} % for arrow head angle
\tikzset{>=latex} % for LaTeX arrow head
\usepackage{xcolor}
\pgfplotsset{compat=newest}

\colorlet{xcol}{blue!60!black}
\colorlet{myred}{red!80!black}
\colorlet{myblue}{blue!80!black}
\colorlet{mygreen}{green!40!black}
\colorlet{mypurple}{red!50!blue!90!black!80}
\colorlet{mydarkred}{myred!80!black}
\colorlet{mydarkblue}{myblue!80!black}
\tikzstyle{xline}=[xcol,thick,smooth]
\tikzstyle{width}=[{Latex[length=5,width=3]}-{Latex[length=5,width=3]},thick]
\tikzstyle{mydashed}=[dash pattern=on 1.7pt off 1.7pt]
\tikzset{
  traj/.style 2 args={xline,postaction={decorate},decoration={markings,
    mark=at position #1 with {\arrow{<}},
    mark=at position #2 with {\arrow{<}}}
  }
}
\def\tick#1#2{\draw[thick] (#1)++(#2:0.12) --++ (#2-180:0.24)}
\def\N{100} % number of samples
    \usetikzlibrary{patterns,snakes,shapes.arrows,3d}
    \usepgfplotslibrary{fillbetween}
	\geometry{
		total = {160mm, 237mm},
		left = 25mm,
		right = 35mm,
		top = 30mm,
		bottom = 30mm,
	}

\newcommand{\jawab}{\textbf{\underline{Solusi}}:}
\newcommand{\del}{\partial}
\newcommand{\cis}{\text{cis}}
\begin{document}
\setlength{\parindent}{0pt}
    \pagenumbering{gobble}
    \noindent
    \begin{tabular}{|lcl|}
     \hline
     Nama&:&Teosofi Hidayah Agung\\
     NRP&:&5002221132\\
     \hline
    \end{tabular}\\~\\
Seorang pelaut ingin memindahkan kapalnya dari posisi awal $x(0) = 0$ ke posisi akhir $x(T) = 10$ dalam waktu sesingkat mungkin. Dinamika kapal diberikan oleh:
\[
\dot{x}(t) = v(t), \quad \dot{v}(t) = u(t),
\]
dengan:
\[
|u(t)| \leq 1.
\]

Tujuannya adalah meminimalkan waktu perjalanan:
\[
J = \int_0^T 1 \, dt.
\]

Gunakan Prinsip Minimum Pontryagin untuk menentukan kontrol optimal $u(t)$ dan tentukan waktu terminal $T$.

\section*{Penyelesaian}

\subsection*{1. Fungsi Hamiltonian}
Fungsi Hamiltonian untuk sistem ini adalah:
\[
H = 1 + \lambda_1 \dot{x} + \lambda_2 \dot{v}.
\]
Dari persamaan dinamika:
\[
\dot{x} = v, \quad \dot{v} = u,
\]
maka:
\[
H = 1 + \lambda_1 v + \lambda_2 u.
\]

\subsection*{2. Persamaan Adjoin}
Persamaan adjoin diberikan oleh:
\[
\dot{\lambda}_i = -\frac{\partial H}{\partial x_i}.
\]
Karena $H$ tidak secara eksplisit bergantung pada $x$, maka:
\[
\dot{\lambda}_1 = -\frac{\partial H}{\partial x} = 0 \implies \lambda_1 \text{ adalah konstan.}
\]

Untuk $\lambda_2$:
\[
\dot{\lambda}_2 = -\frac{\partial H}{\partial v} = -\lambda_1 \implies \lambda_2(t) = -\lambda_1 t + C,
\]
dengan $C$ adalah konstanta integrasi.

\subsection*{3. Kondisi Optimalitas}
Prinsip Minimum Pontryagin mengharuskan kontrol $u(t)$ meminimalkan $H$:
\[
H = 1 + \lambda_1 v + \lambda_2 u.
\]
Karena $|u(t)| \leq 1$, nilai $u(t)$ yang meminimalkan $H$ adalah nilai ekstrem:
\[
u(t) = \begin{cases}
1 & \text{jika } \lambda_2 < 0, \\
-1 & \text{jika } \lambda_2 > 0.
\end{cases}
\]

Substitusi $\lambda_2(t) = -\lambda_1 t + C$:
\begin{itemize}
    \item Jika $-\lambda_1 t + C < 0$, maka $u(t) = 1$.
    \item Jika $-\lambda_1 t + C > 0$, maka $u(t) = -1$.
\end{itemize}

\subsection*{4. Dinamika Sistem}
Kontrol $u(t)$ akan terdiri dari dua fase:
\begin{enumerate}
    \item \textbf{Akselerasi Maksimum ($u = 1$):}
    \[
    \dot{v} = 1 \implies v(t) = t + v(0),
    \]
    \[
    \dot{x} = v \implies x(t) = \frac{1}{2}t^2 + v(0)t + x(0).
    \]
    \item \textbf{Deselerasi Maksimum ($u = -1$):}
    \[
    \dot{v} = -1 \implies v(t) = -t + v(T_1),
    \]
    \[
    \dot{x} = v \implies x(t) = -\frac{1}{2}t^2 + v(T_1)t + x(T_1).
    \]
\end{enumerate}

\subsection*{5. Kondisi Batas}
Kondisi batas adalah:
\[
\begin{aligned}
&x(0) = 0, \quad v(0) = 0, \\
&x(T) = 10, \quad v(T) = 0.
\end{aligned}
\]

Dari solusi fase akselerasi dan deselerasi:
\begin{itemize}
    \item Akselerasi terjadi hingga waktu $T_1$, dengan $x(T_1)$ dan $v(T_1)$ sebagai posisi dan kecepatan saat transisi.
    \item Deselerasi berlangsung dari $T_1$ hingga $T$, dengan kondisi $v(T) = 0$.
\end{itemize}

Dari persamaan dinamika, solusi waktu optimal diperoleh dengan menyelesaikan:
\[
\begin{aligned}
\text{Fase akselerasi: } & x(T_1) = \frac{1}{2}T_1^2, \quad v(T_1) = T_1, \\
\text{Fase deselerasi: } & x(T) = \frac{1}{2}T_1^2 + T_1(T - T_1) - \frac{1}{2}(T - T_1)^2 = 10.
\end{aligned}
\]

Selesaikan sistem untuk mendapatkan $T_1$ dan $T$. Jawabannya adalah:
\[
T_1 = 2\sqrt{5}, \quad T = 4\sqrt{5}.
\]

\subsection*{6. Kontrol Optimal}
Kontrol optimal $u(t)$ diberikan oleh:
\[
u(t) = \begin{cases}
1 & 0 \leq t < T_1, \\
-1 & T_1 \leq t \leq T.
\end{cases}
\]

\end{document}
