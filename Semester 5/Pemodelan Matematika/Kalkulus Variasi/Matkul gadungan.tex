\documentclass{article}
\usepackage{graphicx} 
\usepackage{multirow}
\usepackage{enumitem}
\usepackage{amssymb}
\usepackage{amsmath}
\usepackage{xcolor}
\usepackage{cancel}
\usepackage{tcolorbox}
\usepackage{physics}
\usepackage{geometry}
\usepackage{tikz}
\usepackage{tikz-3dplot}
\usepackage{pgfplots, tkz-euclide,calc}

\usetikzlibrary{angles,quotes} % for pic (angle labels)
\usetikzlibrary{arrows.meta}
\usetikzlibrary{calc}
\usetikzlibrary{decorations.markings}
\usetikzlibrary{bending} % for arrow head angle
\tikzset{>=latex} % for LaTeX arrow head
\usepackage{xcolor}
\pgfplotsset{compat=newest}

\colorlet{xcol}{blue!60!black}
\colorlet{myred}{red!80!black}
\colorlet{myblue}{blue!80!black}
\colorlet{mygreen}{green!40!black}
\colorlet{mypurple}{red!50!blue!90!black!80}
\colorlet{mydarkred}{myred!80!black}
\colorlet{mydarkblue}{myblue!80!black}
\tikzstyle{xline}=[xcol,thick,smooth]
\tikzstyle{width}=[{Latex[length=5,width=3]}-{Latex[length=5,width=3]},thick]
\tikzstyle{mydashed}=[dash pattern=on 1.7pt off 1.7pt]
\tikzset{
  traj/.style 2 args={xline,postaction={decorate},decoration={markings,
    mark=at position #1 with {\arrow{<}},
    mark=at position #2 with {\arrow{<}}}
  }
}
\def\tick#1#2{\draw[thick] (#1)++(#2:0.12) --++ (#2-180:0.24)}
\def\N{100} % number of samples
    \usetikzlibrary{patterns,snakes,shapes.arrows,3d}
    \usepgfplotslibrary{fillbetween}
	\geometry{
		total = {160mm, 237mm},
		left = 25mm,
		right = 35mm,
		top = 30mm,
		bottom = 30mm,
	}

\newcommand{\jawab}{\textbf{\underline{Solusi}}:}
\newcommand{\del}{\partial}
\newcommand{\cis}{\text{cis}}
\begin{document}
\setlength{\parindent}{0pt}
    \pagenumbering{gobble}
    \noindent
    \begin{tabular}{|lcl|}
     \hline
     Nama&:&Teosofi Hidayah Agung\\
     NRP&:&5002221132\\
     \hline
    \end{tabular}\\~\\
Carilah fungsi \( y(x) \) yang meminimalkan integral berikut:

\[
J[y] = \int_{0}^{1} \left( y'^2 - xy \right) \, dx
\]

dengan syarat batas:
\[
y(0) = 0 \quad \text{dan} \quad y(1) = 1.
\]
\jawab

Rumus Euler-Lagrange adalah:
\[
\frac{\partial F}{\partial y} - \frac{d}{dx} \left( \frac{\partial F}{\partial y'} \right) = 0,
\]
di mana \( F = y'^2 - xy \).

  \[
  \frac{\partial F}{\partial y} = \frac{\partial}{\partial y} \left( y'^2 - xy \right) = -x.
  \]
  \[
  \frac{\partial F}{\partial y'} = \frac{\partial}{\partial y'} \left( y'^2 - xy \right) = 2y'.
  \]
  \[
  \frac{d}{dx} \left( \frac{\partial F}{\partial y'} \right) = \frac{d}{dx} \left( 2y' \right) = 2y''.
  \]

Masukkan hasil turunan ke persamaan Euler-Lagrange:
\[
\frac{\partial F}{\partial y} - \frac{d}{dx} \left( \frac{\partial F}{\partial y'} \right) = 0,
\]
sehingga:
\[
-x - 2y'' = 0.
\]

Atau dapat ditulis sebagai:
\[
y'' + \frac{x}{2} = 0.
\]

Persamaan diferensial tersebut adalah:
\[
y'' = -\frac{x}{2}.
\]

Integrasikan kedua ruas terhadap \( x \) untuk mendapatkan \( y' \):
\[
y' = -\frac{x^2}{4} + C_1,
\]
dengan \( C_1 \) adalah konstanta integrasi.

Integrasikan lagi untuk mendapatkan \( y \):
\[
y = -\frac{x^3}{12} + C_1x + C_2,
\]
dengan \( C_2 \) adalah konstanta integrasi kedua.


Dari syarat batas \( y(0) = 0 \) dan \( y(1) = 1 \):

- Ketika \( x = 0 \), \( y(0) = 0 \):
  \[
  0 = -\frac{0^3}{12} + C_1(0) + C_2 \implies C_2 = 0.
  \]

- Ketika \( x = 1 \), \( y(1) = 1 \):
  \[
  1 = -\frac{1^3}{12} + C_1(1) + 0 \implies C_1 = 1 + \frac{1}{12} = \frac{13}{12}.
  \]

Sehingga, fungsi \( y(x) \) adalah:
\[
y(x) = -\frac{x^3}{12} + \frac{13}{12}x.
\]

Kita memverifikasi bahwa fungsi $y(x)$ memenuhi syarat batas:
\begin{itemize}
    \item Ketika \( x = 0 \), \( y(0) = 0 \).
    \item Ketika \( x = 1 \), \( y(1) = -\frac{1^3}{12} + \frac{13}{12}(1) = 1 \).
\end{itemize}

Fungsi ini juga meminimalkan integral \( J[y] \) sesuai dengan persamaan Euler-Lagrange.

Fungsi yang meminimalkan integral adalah:
\[
y(x) = -\frac{x^3}{12} + \frac{13}{12}x.
\]
\end{document}