\documentclass{article}
\usepackage{graphicx} 
\usepackage{multirow}
\usepackage{enumitem}
\usepackage{amssymb}
\usepackage{amsmath}
\usepackage{xcolor}
\usepackage{cancel}
\usepackage{tcolorbox}
\usepackage{physics}
\usepackage{geometry}
\usepackage{tikz}
\usepackage{tikz-3dplot}
\usepackage{pgfplots, tkz-euclide,calc}

\usetikzlibrary{angles,quotes} % for pic (angle labels)
\usetikzlibrary{arrows.meta}
\usetikzlibrary{calc}
\usetikzlibrary{decorations.markings}
\usetikzlibrary{bending} % for arrow head angle
\tikzset{>=latex} % for LaTeX arrow head
\usepackage{xcolor}
\pgfplotsset{compat=newest}

\colorlet{xcol}{blue!60!black}
\colorlet{myred}{red!80!black}
\colorlet{myblue}{blue!80!black}
\colorlet{mygreen}{green!40!black}
\colorlet{mypurple}{red!50!blue!90!black!80}
\colorlet{mydarkred}{myred!80!black}
\colorlet{mydarkblue}{myblue!80!black}
\tikzstyle{xline}=[xcol,thick,smooth]
\tikzstyle{width}=[{Latex[length=5,width=3]}-{Latex[length=5,width=3]},thick]
\tikzstyle{mydashed}=[dash pattern=on 1.7pt off 1.7pt]
\tikzset{
  traj/.style 2 args={xline,postaction={decorate},decoration={markings,
    mark=at position #1 with {\arrow{<}},
    mark=at position #2 with {\arrow{<}}}
  }
}
\def\tick#1#2{\draw[thick] (#1)++(#2:0.12) --++ (#2-180:0.24)}
\def\N{100} % number of samples
    \usetikzlibrary{patterns,snakes,shapes.arrows,3d}
    \usepgfplotslibrary{fillbetween}
	\geometry{
		total = {160mm, 237mm},
		left = 25mm,
		right = 35mm,
		top = 30mm,
		bottom = 30mm,
	}

\newcommand{\jawab}{\textbf{\underline{Solusi}}:}
\newcommand{\del}{\partial}
\newcommand{\cis}{\text{cis}}
\begin{document}
\setlength{\parindent}{0pt}
    \pagenumbering{gobble}
    \noindent
    \begin{tabular}{|lcl|}
     \hline
     Nama&:&Teosofi Hidayah Agung\\
     NRP&:&5002221132\\
     \hline
    \end{tabular}\\~\\
    Diberikan sistem persamaan diferensial non-linear non-homogen:
    \begin{align*}
    \frac{dx}{dt} &= x^2 + y + 2t, \\
    \frac{dy}{dt} &= y^2 - x + t^2.
    \end{align*}
    
    Matriks Jacobian dari sistem tersebut adalah:
    \[
    J = \begin{bmatrix} \frac{\partial f}{\partial x} & \frac{\partial f}{\partial y} \\ \frac{\partial g}{\partial x} & \frac{\partial g}{\partial y} \end{bmatrix} = \begin{bmatrix} 2x & 1 \\ -1 & 2y \end{bmatrix}.
    \]
    Pada titik keseimbangan \((0, 0)\), diperoleh:
    \[
    J = \begin{bmatrix} 0 & 1 \\ -1 & 0 \end{bmatrix}.
    \]

    Mencari nilai eigen dengan menyelesaikan:
    \[
    \det(J - \lambda I) = \det \begin{bmatrix} -\lambda & 1 \\ -1 & -\lambda \end{bmatrix} = \lambda^2 + 1 = 0.
    \]
    Solusi:
    \[
    \lambda = \pm i.
    \]
    
    Vektor eigen untuk \(\lambda = i\):
    \[
    \mathbf{v}_1 = \begin{bmatrix} 1 \\ i \end{bmatrix}.
    \]
    Vektor eigen untuk \(\lambda = -i\):
    \[
    \mathbf{v}_2 = \begin{bmatrix} 1 \\ -i \end{bmatrix}.
    \]

    Solusi umum homogen:
    \[
    \mathbf{x}_h(t) = c_1 e^{it} \begin{bmatrix} 1 \\ i \end{bmatrix} + c_2 e^{-it} \begin{bmatrix} 1 \\ -i \end{bmatrix}.
    \]
    Dengan identitas Euler \( e^{it} = \cos(t) + i\sin(t) \), solusi homogen dalam bentuk real:
    \[
    \mathbf{x}_h(t) = \begin{bmatrix} c_1 \cos(t) + c_2 \sin(t) \\ c_1 \sin(t) - c_2 \cos(t) \end{bmatrix}.
    \]

    Rumus solusi partikulir:
    \[
    \mathbf{x}_p(t) = \Phi(t) \int \Phi^{-1}(t) \mathbf{g}(t) \, dt,
    \]
    dengan:
    \[
    \Phi(t) = \begin{bmatrix} \cos(t) & \sin(t) \\ -\sin(t) & \cos(t) \end{bmatrix}, \quad \Phi^{-1}(t) = \begin{bmatrix} \cos(t) & -\sin(t) \\ \sin(t) & \cos(t) \end{bmatrix}.
    \]
    
    Bagian non-homogen:
    \[
    \mathbf{g}(t) = \begin{bmatrix} 2t \\ t^2 \end{bmatrix}.
    \]
    
    Perkalian:
    \[
    \Phi^{-1}(t) \mathbf{g}(t) = \begin{bmatrix} \cos(t) & -\sin(t) \\ \sin(t) & \cos(t) \end{bmatrix} \begin{bmatrix} 2t \\ t^2 \end{bmatrix} = \begin{bmatrix} 2t \cos(t) - t^2 \sin(t) \\ 2t \sin(t) + t^2 \cos(t) \end{bmatrix}.
    \]
    
    Integrasi:
    \[
    \int (2t \cos(t) - t^2 \sin(t)) \, dt = t^2 \cos(t), \quad \int (2t \sin(t) + t^2 \cos(t)) \, dt = t^2 \sin(t).
    \]
    
    Solusi partikulir:
    \[
    \mathbf{x}_p(t) = \Phi(t) \begin{bmatrix} t^2 \cos(t) \\ t^2 \sin(t) \end{bmatrix} = \begin{bmatrix} t^2 \\ 0 \end{bmatrix}.
    \]

    Solusi lengkap dari sistem adalah:
    \[
    \mathbf{x}(t) = \mathbf{x}_h(t) + \mathbf{x}_p(t) = \begin{bmatrix} c_1 \cos(t) + c_2 \sin(t) + t^2 \\ c_1 \sin(t) - c_2 \cos(t) \end{bmatrix}.
    \]
\end{document}