\documentclass{article}
\usepackage{graphicx} 
\usepackage{multirow}
\usepackage{enumitem}
\usepackage{amssymb}
\usepackage{amsmath}
\usepackage{xcolor}
\usepackage{cancel}
\usepackage{tcolorbox}
\usepackage{physics}
\usepackage{geometry}
\usepackage{tikz}
\usepackage{tikz-3dplot}
\usepackage{pgfplots, tkz-euclide,calc}

\usetikzlibrary{angles,quotes} % for pic (angle labels)
\usetikzlibrary{arrows.meta}
\usetikzlibrary{calc}
\usetikzlibrary{decorations.markings}
\usetikzlibrary{bending} % for arrow head angle
\tikzset{>=latex} % for LaTeX arrow head
\usepackage{xcolor}
\pgfplotsset{compat=newest}

\colorlet{xcol}{blue!60!black}
\colorlet{myred}{red!80!black}
\colorlet{myblue}{blue!80!black}
\colorlet{mygreen}{green!40!black}
\colorlet{mypurple}{red!50!blue!90!black!80}
\colorlet{mydarkred}{myred!80!black}
\colorlet{mydarkblue}{myblue!80!black}
\tikzstyle{xline}=[xcol,thick,smooth]
\tikzstyle{width}=[{Latex[length=5,width=3]}-{Latex[length=5,width=3]},thick]
\tikzstyle{mydashed}=[dash pattern=on 1.7pt off 1.7pt]
\tikzset{
  traj/.style 2 args={xline,postaction={decorate},decoration={markings,
    mark=at position #1 with {\arrow{<}},
    mark=at position #2 with {\arrow{<}}}
  }
}
\def\tick#1#2{\draw[thick] (#1)++(#2:0.12) --++ (#2-180:0.24)}
\def\N{100} % number of samples
    \usetikzlibrary{patterns,snakes,shapes.arrows,3d}
    \usepgfplotslibrary{fillbetween}
	\geometry{
		total = {160mm, 237mm},
		left = 25mm,
		right = 35mm,
		top = 30mm,
		bottom = 30mm,
	}

\newcommand{\jawab}{\textbf{\underline{Solusi}}:}
\newcommand{\del}{\partial}
\newcommand{\cis}{\text{cis}}
\begin{document}
\setlength{\parindent}{0pt}
    \pagenumbering{gobble}
    \noindent
    \begin{tabular}{|lcl|}
     \hline
     Nama&:&Teosofi Hidayah Agung\\
     NRP&:&5002221132\\
     \hline
    \end{tabular}\\~\\

Diberikan sistem persamaan diferensial non-linier homogen berikut:

\[
\begin{aligned}
\frac{dx}{dt} &= e^x + y \\
\frac{dy}{dt} &= \ln(x+1) - y
\end{aligned}
\]

Untuk menemukan titik kesetimbangan, kita cari nilai \( x \) dan \( y \) yang memenuhi:

\[
\frac{dx}{dt} = 0 \quad \text{dan} \quad \frac{dy}{dt} = 0.
\]

Dari persamaan pertama:

\[
e^x + y = 0 \quad \Rightarrow \quad y = -e^x.
\]

Dari persamaan kedua:

\[
\ln(x+1) - y = 0 \quad \Rightarrow \quad y = \ln(x+1).
\]

Kemudian kita substitusi \( y = -e^x \) ke dalam persamaan kedua, sehingga

\[
-e^x = \ln(x+1).
\]

Persamaan ini non-linier dan membutuhkan solusi numerik untuk \( x \), namun kita akan fokus pada titik sederhana \( x = 0 \) dan \( y = 0 \) untuk analisis lokal.

Selanjutnya di sekitar titik kesetimbangan \( (x_0, y_0) = (0, 0) \), kita linearkan sistem dengan menghitung \textbf{matriks Jacobian} dari sistem non-linier tersebut. Didefinisikan

\begin{flalign*}
f(x, y) &= e^x + y, \\
g(x, y) &= \ln(x+1) - y.
\end{flalign*}

Maka matriks Jacobian dari sistem persamaan diferensial non-linier adalah

\[
J(0, 0) = \begin{pmatrix}
    \partial f/\partial x & \partial f/\partial y \\
    \partial g/\partial x & \partial g/\partial y
\end{pmatrix}
= \begin{pmatrix}
    e^x & 1 \\
    \frac{1}{x+1} & -1
\end{pmatrix}=
\begin{pmatrix}
1 & 1 \\
1 & -1
\end{pmatrix}.
\]

Setelah sistem dilinearkan, kita perkenalkan notasi \( x^* \) dan \( y^* \) sebagai variabel yang menggambarkan penyimpangan kecil dari titik kesetimbangan \( (x_0, y_0) \). Dengan demikian, sistem persamaan diferensial linier yang dilinearkan menjadi:

\[
\frac{d}{dt} \begin{pmatrix} x^* \\ y^* \end{pmatrix} = J(0, 0) \begin{pmatrix} x^* \\ y^* \end{pmatrix}.
\]

Secara eksplisit, sistem linier yang dilinearkan adalah:

\[
\begin{aligned}
\frac{dx^*}{dt} &= x^* + y^* \\
\frac{dy^*}{dt} &= x^* - y^*
\end{aligned}
\]

Untuk menyelesaikan sistem linier ini, kita bisa mencari solusi eigen dari matriks Jacobian \( J(0, 0) \). Eigenvalue yang telah kita hitung adalah:

\[
\lambda_1 = \sqrt{2}, \quad \lambda_2 = -\sqrt{2}.
\]

Dengan demikian, solusi umum untuk sistem linier yang dilinearkan adalah:

\[
\begin{pmatrix} x^*(t) \\ y^*(t) \end{pmatrix} = c_1 \begin{pmatrix} 1 \\ 1 \end{pmatrix} e^{\sqrt{2} t} + c_2 \begin{pmatrix} 1 \\ -1 \end{pmatrix} e^{-\sqrt{2} t}.
\]

Penjelasan lebih lanjut:
- Vektor eigen yang sesuai dengan \( \lambda_1 = \sqrt{2} \) adalah \( \begin{pmatrix} 1 \\ 1 \end{pmatrix} \), sehingga solusinya adalah \( c_1 e^{\sqrt{2} t} \begin{pmatrix} 1 \\ 1 \end{pmatrix} \).
- Vektor eigen yang sesuai dengan \( \lambda_2 = -\sqrt{2} \) adalah \( \begin{pmatrix} 1 \\ -1 \end{pmatrix} \), sehingga solusinya adalah \( c_2 e^{-\sqrt{2} t} \begin{pmatrix} 1 \\ -1 \end{pmatrix} \).

Solusi umum untuk \( x^*(t) \) dan \( y^*(t) \) adalah:

\[
x^*(t) = c_1 e^{\sqrt{2} t} + c_2 e^{-\sqrt{2} t}
\]
\[
y^*(t) = c_1 e^{\sqrt{2} t} - c_2 e^{-\sqrt{2} t}.
\]

Sekarang kita substitusi kondisi awal \( x(0) = 0 \) dan \( y(0) = 1 \), yang diterjemahkan menjadi \( x^*(0) = 0 \) dan \( y^*(0) = 1 \), untuk menentukan konstanta \( c_1 \) dan \( c_2 \):

- Dari \( x^*(0) = 0 \):
  \[
  c_1 + c_2 = 0 \quad \Rightarrow \quad c_1 = -c_2.
  \]

- Dari \( y^*(0) = 1 \):
  \[
  c_1 - c_2 = 1.
  \]

Substitusi \( c_1 = -c_2 \) ke dalam persamaan kedua:

\[
-c_2 - c_2 = 1 \quad \Rightarrow \quad -2c_2 = 1 \quad \Rightarrow \quad c_2 = -\frac{1}{2}.
\]

Sehingga \( c_1 = \frac{1}{2} \).

Dengan \( c_1 = \frac{1}{2} \) dan \( c_2 = -\frac{1}{2} \), solusi akhirnya untuk \( x^*(t) \) dan \( y^*(t) \) adalah:

\[
x^*(t) = \frac{1}{2} e^{\sqrt{2} t} - \frac{1}{2} e^{-\sqrt{2} t}
\]
\[
y^*(t) = \frac{1}{2} e^{\sqrt{2} t} + \frac{1}{2} e^{-\sqrt{2} t}.
\]

Solusi ini dapat disederhanakan dengan menggunakan fungsi hiperbolik:

\[
x^*(t) = \sinh(\sqrt{2} t)
\]
\[
y^*(t) = \cosh(\sqrt{2} t).
\]

\end{document}