\documentclass[aspectratio=169]{beamer}
\usepackage{pgf}
\usepackage{colortbl,tabularx,mathrsfs,calligra,xcolor}
\usepackage{amsmath,amsfonts,amssymb,amsthm}
\usepackage{ragged2e}
\usepackage{setspace}
\usepackage{filecontents}
\usepackage{caption}
\usepackage{subcaption}
\usepackage{contour}
\usepackage{fancybox}
\usepackage{wrapfig}
\usepackage{multirow}
\usepackage{multicol}
\usepackage{tikz, pgfplots, tkz-euclide,calc}
    \usetikzlibrary{patterns,snakes,shapes.arrows}
\usepackage{listings}
\usepackage{enumitem}
\usepackage{pifont}
\usepackage[scaled]{berasans}
    \renewcommand*\familydefault{\sfdefault}  %% Only if the base font of the document is to be sans serif
\usepackage[T1]{fontenc}
\usepackage{hyperref}
\hypersetup{
    filecolor=magenta,      
    urlcolor=cyan,
    pdftitle={Overleaf Example},
    pdfpagemode=FullScreen,
    }
\renewcommand*\familydefault{\sfdefault} %% Only if the base font of the document is to be sans serif

\graphicspath{{C:/Users/teoso/OneDrive/Documents/Asisten Dosen & Lab/Asisten Laboratorium/Alpro 1/PPT/Graphicx/}}

\definecolor{HIMAmuda}{HTML}{01D1FD}
\definecolor{HIMAtua}{HTML}{02016A}
\definecolor{HIMAabu}{HTML}{CBCBCC}
\definecolor{PastelGreen}{HTML}{77DD77}
\definecolor{pgray}{rgb}{0.5,0.5,0.5}
\definecolor{pblue}{rgb}{0.13,0.13,1}
\definecolor{pgreen}{rgb}{0,0.5,0}
\definecolor{pred}{rgb}{0.9,0,0}
\definecolor{pgrey}{rgb}{0.46,0.45,0.48}
\definecolor{pcyan}{HTML}{D4EFFC}
\definecolor{lblue}{HTML}{00AEEF}
\definecolor{input}{HTML}{AAE1FA}
\definecolor{bg}{rgb}{0.95, 0.95, 0.92}

\usepackage{listings}

\lstdefinestyle{standard}{
    language            =Java,
    showspaces          =false,
    showtabs            =false,
    breaklines          =true,
    showstringspaces    =false,
    breakatwhitespace   =true,
    commentstyle        =\color{pgray},
    keywordstyle        =\color{pblue},
    stringstyle         =\color{pgreen},
    basicstyle          =\small\ttfamily,
    frame               =single,
    backgroundcolor     =\color{bg},
    escapeinside        ={(*}{*)},
    numbers             = left, % {none, left, right}
    numberstyle         = \scriptsize\color{black},
    numbersep           = -8pt,
    }

\lstset{style=standard}

\usetheme{Madrid}

\setbeamercolor{palette primary}{bg=HIMAtua,fg=white}
\setbeamercolor{palette secondary}{bg=HIMAmuda,fg=black}
\setbeamercolor{palette tertiary}{bg=HIMAabu,fg=black}
\setbeamercolor{palette quaternary}{bg=HIMAmuda,fg=white}
\setbeamercolor{structure}{fg=HIMAmuda} % itemize, enumerate, etc
\setbeamercolor{section in toc}{fg=HIMAtua} % TOC sections
\setbeamercolor{block title alerted}{fg=white,bg=magenta}
\setbeamercolor{block body alerted}{fg=black!90,bg=pink}

\usefonttheme{professionalfonts}
\setbeamertemplate{theorems}[numbered]
\setbeamertemplate{itemize items}[circle]

\usebackgroundtemplate{%
\tikz[overlay,remember picture] \node[opacity=0.1, at=(current page.center)]{\includegraphics[width=\paperwidth]{plana class}};
}

\renewcommand\thesubfigure{\arabic{subfigure}}
\theoremstyle{definition}
\newcommand{\R}{\mathbb{R}}

\AtBeginEnvironment{theorem}{
    \setbeamercolor{block title}{bg=darkgray,fg=white}
    \setbeamercolor{block body}{parent=pallette tertiary,bg=HIMAabu!30!white}
}

\renewenvironment<>{proof}[1][\proofname]{%
    \par
    \def\insertproofname{Proof #1}%
    \pushQED{\qed}
    \usebeamertemplate{proof begin}#2}
  {\popQED\usebeamertemplate{proof end}}

\AtBeginEnvironment{proof}{
    \setbeamercolor{block title}{bg=HIMAtua,fg=white}
    \setbeamercolor{block body}{parent=pallette tertiary,bg=HIMAtua!10!white}
}

\author[Teo - 5002221132]{Teosofi Hidayah Agung}
\title[Linear Algebra]{Pengenalan Bahasa Pemrograman Java}
\institute[Matematika ITS]{Departemen Matematika\\ Institut Teknologi Sepuluh Nopember}
\titlegraphic{{\includegraphics[scale=0.02]{M.png}$\quad$\includegraphics[scale=0.2]{Provicom.png}}}

\begin{document}
    \section{Orthogonal}
    \begin{frame}
        \begin{theorem}
            \frametitle{\insertsection}
            Suppose $W$ is a subspace of an inner product space $V$. Then:
            \begin{enumerate}[label=(\alph*)]
              \item $W^\perp$ is a subspace of $V$.
              \item Only the zero vector, $\mathbf{0}$, is common to both $W$ and $W^\perp$.
              \item $(W^\perp)^\perp = W$. In other words, the orthogonal complement of $W^\perp$ is $W$.
            \end{enumerate}
          \end{theorem}
          \onslide<2->{
            In this case, assume (a) has been proven, thus use (a) to help for proofing (b) and (c).
          }
    \end{frame}
    
    \begin{frame}
        \begin{proof}[]
            Hence $W$ and $W^\perp$ is a subspace of $V$, its obvious that $\mathbf{0}$ is common to both $W$ and $W^\perp$.\\~\\
            Assume there is a vector $\mathbf{v}\neq \mathbf{0}$ that is common to both $W$ and $W^\perp$. Then $\mathbf{v}\in W$ and $\mathbf{v}\in W^\perp$. Since $\mathbf{v}\in W$, then $\langle\mathbf{v},\,\mathbf{v}\rangle = 0$. This implies that $\mathbf{v} = \mathbf{0}$, which is a contradiction. Therefore, only the zero vector, $\mathbf{0}$, is common to both $W$ and $W^\perp$.
        \end{proof}
    \end{frame}

    \begin{frame}
        \begin{proof}[]
            Let any $\mathbf{u}\in W$, then $\langle\mathbf{u},\,\mathbf{v}\rangle = 0$ for all $\mathbf{v}\in W^\perp$. Thus by definition, any $\mathbf{u}$ is orthogonal to any $\mathbf{v}\in W^\perp$ or can be written as $\mathbf{u}\in(W^\perp)^\perp$.
            \begin{align}
                W \subseteq (W^\perp)^\perp
            \end{align}
            Next consider $\mathbf{w}\in (W^\perp)^\perp$ but $\mathbf{w}\notin W$. By definition, exist $\mathbf{v}_0\in W^\perp$ such that is not orthogonal to $\mathbf{w}$. Other words, $\langle\mathbf{v}_0,\,\mathbf{w}\rangle = 0$ because $\mathbf{w}\in (W^\perp)^\perp$. So its contradict with the assumption that $\mathbf{w}\notin W$. Therefore, $\mathbf{w}\in W$.
            \begin{align}
                (W^\perp)^\perp \subseteq W 
            \end{align}
            Therefore, $(W^\perp)^\perp = W$.
        \end{proof}
    \end{frame}
\end{document}