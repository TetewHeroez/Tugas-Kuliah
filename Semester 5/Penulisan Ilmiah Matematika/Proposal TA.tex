\documentclass[12pt]{article}
\usepackage[a4paper, portrait, top=3cm, bottom=2.5cm, left=3cm, right=2.5cm]{geometry}
\usepackage{graphicx}
\usepackage{tabularx}
\usepackage{setspace}
\usepackage{fancyhdr}
\usepackage{array}
\usepackage{courier}
\usepackage{lscape}
\usepackage{titlesec}
\usepackage{amssymb}
\usepackage{amsmath}
%\usepackage{times}
\usepackage{courier}
\usepackage{hyperref}
\usepackage[table]{xcolor}
\usepackage{tikz}

\graphicspath{{C:/Users/teoso/OneDrive/Documents/Tugas Kuliah/Template Math Depart/}}

\newcolumntype{L}[1]{>{\raggedright\let\newline\\\arraybackslash\hspace{0pt}}m{#1}}
\newcolumntype{C}[1]{>{\centering\let\newline\\\arraybackslash\hspace{0pt}}m{#1}}
\newcolumntype{R}[1]{>{\raggedleft\let\newline\\\arraybackslash\hspace{0pt}}m{#1}}

\titleformat*{\section}{\normalsize\bfseries}
\titleformat*{\subsection}{\normalsize\bfseries}

\renewcommand{\thesection}{\Roman{section}.}
\renewcommand{\thesubsection}{\arabic{section}.\arabic{subsection}}
\renewcommand{\contentsname}{Daftar Isi}
\renewcommand{\arraystretch}{1.2}
\renewcommand{\refname}{}

\usepackage{amsthm}
\renewcommand*{\proofname}{\textbf{Bukti.}}
\newtheorem{theorem}{Teorema}
\renewcommand*{\thetheorem}{\Alph{theorem}}
\newtheorem{corollary}{Corollary}[theorem]
\newtheorem{lemma}{Lemma}
\numberwithin{lemma}{subsection}
\newtheorem{dfn}[lemma]{Definisi}
\newtheorem{proposition}[lemma]{Proposisi}
\newtheorem{example}[lemma]{Contoh}
\newtheorem{thm}[lemma]{Teorema}
\newtheorem{remark}[lemma]{Catatan}
\usetikzlibrary{shapes.geometric, arrows}
\tikzstyle{startstop} = [rectangle, rounded corners=0.5cm, minimum width=3cm, minimum height=1cm,text centered, draw=black, fill=red!40]
\tikzstyle{process} = [rectangle, minimum width=10cm, minimum height=1cm, text width=12cm, text centered, draw=black]
\tikzstyle{processRR} = [rectangle, minimum width=3cm, minimum height=1cm, text width=4cm, text centered, draw=black]
\tikzstyle{decision} = [diamond, aspect=2, minimum width=3cm, minimum height=1cm, text centered, draw=black, fill=green!20]
\tikzstyle{pj} = [rectangle, minimum width=2cm, minimum height=2cm, text width=2cm, text centered, draw=orange!50, fill=orange!40]
\tikzstyle{pb} = [rectangle, minimum width=2cm, minimum height=2cm, text width=2cm, text centered, draw=blue!30, fill=blue!30]
\tikzstyle{mon} = [rectangle, minimum width=2cm, minimum height=1.25cm, text width=5cm, text centered, draw=black!20, fill=black!20]
\tikzstyle{invol} = [rectangle, minimum width=2cm, minimum height=1.25cm, text width=3.5cm, text centered, draw=white]
\tikzstyle{op} = [rectangle, minimum width=2cm, minimum height=1.75cm, text width=4cm, text centered, draw=yellow!50, fill = yellow!30]
\tikzstyle{res} = [rectangle, minimum width=2cm, minimum height=1.8cm, text width=4cm, text centered, draw=green!50, fill = green!30]
\tikzstyle{arrow} = [thick,->,>=stealth]
\tikzstyle{arrowX} = [thick,->,>=stealth]
\tikzstyle{arrowLeft} = [thick,<-,>=stealth]
\thispagestyle{empty}
\begin{document}
	\begin{center}
		\textbf{PROPOSAL TUGAS AKHIR}\\
		\vspace{1cm}
		\textbf{JUDUL BAHASA INDONESIA}\\
		\vspace{1cm}
		\textbf{\textit{ENGLISH TITLE}}\\
		\vspace{2cm}
		\includegraphics[width=8cm]{logoITSBiru}\\
		\vspace{3cm}
		Oleh:\\
		Teosofi Hidayah Agung\\
		5002221132\\
		\vspace{3.5cm}
		\textbf{
		LABORATORIUM ANALISIS, ALJABAR, DAN PEMBELAJARAN MATEMATIKA\\
		DEPARTEMEN MATEMATIKA\\
		FAKULTAS SAINS DAN ANALITIKA DATA\\
		INSTITUT TEKNOLOGI SEPULUH NOPEMBER\\
		SURABAYA\\
		2019}
		

	\end{center}
\pagebreak
\pagenumbering{roman}
% \begin{center}
% 	\textbf{LEMBAR PENGESAHAN\\PROPOSAL TUGAS AKHIR
% 	}\\
% \vspace{1cm}
% 	\textbf{JUDUL BAHASA INDONESIA}\\
% \vspace{1cm}
% \textbf{\textit{ENGLISH TITLE}}\\
% \vspace{1cm}
% 	Dipersiapkan dan diusulkan oleh,\\
% 	Mohamad Ilham Dwi Firmansyah\\
% 	NRP. 06111640000003\vspace{0.25cm}\\
% 	Telah dipertahankan dan diterima pada Seminar Proposal pada tanggal 20 September 2019\\
% 	Surabaya, 30 September 2019\vspace{0.5cm}\\
% 	Menyetujui,\\
% 	\begin{tabular}{C{6cm}C{6cm}}
% 		&\\
% 		Dosen Pembimbing II,&Dosen Pembimbing I,\\
% 		&\\	
% 		&\\
% 		&\\
% 		\underline{Dr. Chairul Imron, M.I.Komp}&\underline{Dr. Mahmud Yunus, M.Si}\\
% 		NIP. 19611115 198703 1 003&NIP. 19620407 198703 1 005 001\\
% 	\end{tabular}\\
% 	\vspace{1cm}
% 	Mengetahui\\
% 	Kaprodi S1 Departemen Matematika\\
% 	FMKSD ITS\\
% 	\vspace{2.5cm}
% 	\underline{Dr. Didik Khusnul Arif, S.Si, M.Si}\\
% 	NIP. 19730930 199702 1 001\\
% \end{center}

%\pagebreak
\setstretch{1}

\begin{center}
	\textbf{ABSTRAK}
	
\end{center}
$\,$\hskip 1cm Dalam matematika, analisis fungsional merupakan bidang yang sering dikaji oleh para ilmuan, salah satu konsep dari analisis fungsional yang sering dikaji adalah ruang metrik. Konsep ruang metrik dapat diperluas, beberapa perluasan ruang metrik antara lain ruang metrik parsial dan ruang metrik rectangular. Ruang metrik yang akan dibahas pada penelitian ini adalah ruang metrik parsial rectangular,  ruang ini merupakan gabungan konsep ruang metrik parsial dan ruang metrik rectangular. Ruang metrik parsial memiliki perbedaan dengan ruang metrik. Pada ruang metrik, jarak suatu titik ke dirinya sendiri selalu bernilai nol, sedangkan untuk ruang metrik parsial jarak suatu titik ke dirinya sendiri tidak selalu nol. Sedangakan perbedaan antara ruang metrik dengan ruang metrik parsial recatngular terletak pada sifat pertidaksamannya, pada ruang metrik pertidaksamaan yang berlaku adalah pertidaksamaan segitiga, sedangkan untuk ruang metrik rectangular menggunakan pertidaksamaan rectangular. Kemudian akan dibahas pada penelitian ini tentang ruang metrik parsial rectangular dengan permasalahan eksistensi dan ketunggalan titik tetap pemetaan pada ruang tersebut denga prinsip pemetaan quasi-kontraksi.  \\

\textbf{Kata kunci :} \textit{ruang metrik, ruang metrik parsial, ruang metrik rectangular, teorema titik tetap, pemetaan quasi-kontraksi.}

\pagebreak
\setstretch{1}
\pagenumbering{arabic}
\section{Pendahuluan}
$\,$\hskip 1cm Pada bab ini akan dijelaskan tentang latar belakang, rumusan masalah, batasan masalah, tujuan penelitian, dan manfaat penelitian. 
\subsection{Latar Belakang}
$\,$\hskip 1cm Dalam perkembangan analisis fungsional salah satu bahasan yang sering dikaji oleh ilmuwan adalah konsep ruang metrik\cite{aufa}. %cara sitasi
\subsection{Rumusan Masalah}

\subsection{Batasan Masalah}


\subsection{Tujuan Penelitian}

\subsection{Manfaat Penelitian}


\pagebreak
\section{Tinjauan Pustaka}
\hskip 1cm 

\subsection{Penelitian Terdahulu}
\hskip 0.8cm

\subsection{Ruang Metrik}

\subsection{Ruang Metrik Parsial}

\subsection{Ruang Metrik Rectangular}

\hskip 1cm
\subsection{Ruang Metrik Parsial Rectangular}

\hskip 0.7cm
\subsection{Teorema Titik Tetap Pemetaan Kontraksi-Quasi}
\hskip 0.7cm 



\pagebreak
\section{Metode Penelitian}
%\hskip 1cm Pada bab ini akan dijelaskan langkah – langkah dalam mengerjakan penelitian tentang eksintensi dan ketunggalan titik tetap pada ruang metrik parsial rectangular. Berikut ini langkah – langkah yang akan dilaksanakan pada penilitian ini terdiri atas:
%\begin{enumerate}
%	\item \textbf{Studi literatur} \\Pada tahap ini akan dicari referensi yang berkaitan dengan ruang metrik parsial rectangular. Referensi yang dicari meliputi pengertian definisi ruang metrik, ruang metrik rectangular, ruang parsial, ,teorema titik tetap dan lain sebagainya yang berhubungan dengan penelitian ini. Referensi – referensi yang dicari dapat diperoleh melalui jurnal – jurnal yang sesuai dengan topik tugas akhir ini. Selain melalui jurnal – jurnal yang terkait, studi literatur dapat diperoleh melalui buku teks yang berkaitan dengan ruang metrik parsila rectangular.
%	\item \textbf{Mengkaji ruang metrik parsial rectangular kemudian memberikan contoh himpunan dan fungsi metriknya serta pembuktiannya}\\
%	Setelah mempelajari dan memahami referensi yang ada, pada tahap ini akan dikaji jenis – jenis ruang metrik yang berkaitan dengan ruang metrik parsial rectangular. Kemudian akan diberikan contoh dari ruang – ruang tersebut yang dilengkapi dengan pembuktiannya.  
%	\item \textbf{Mengkaji karakterisasi teorema titik tetap dan prinsip pemetaan quasi-kontraksi pada ruang metrik parsial rectangular}\\  Selanjutnya adalah mengkaji prinsip pemetaan kontraksi quasi pada ruang metrik parsial rectangular untuk mengetahui eksistensi dan ketunggalan titik tetap pada ruang tersebut. Tahap ini merupakan langkah untuk menjawab rumusan masalah.
%	\item \textbf{Pembuktian teorema titik tetap pada ruang metrik parsial rectangular dan mengkonstruksi contoh ruang dan pemetaan lainnya}\\
%	Setelah melakukan kajian terhadap konsep ruang metrik parsial rectangular dan teorema titik tetap pemetaan kontrasi quasi, dilakukan pembuktian teorema titik tetap pada ruang metrik parsial rectangular, serta mengembangkan dan mengkonstruksi contoh-contoh sehingga dapat menhasilkan jawaban atas rumusan masalah.
%	\hskip 1cm 
%	\item \textbf{Penarikan kesimpulan dan pembukuan tugas akhir}\\
%	Pada tahap yang terakhir, akan di tarik kesimpulan dari penelitian yang dilakukan sebelumnya dan akan dilakukan pembukuan tugas akhir. 
%\end{enumerate}
\pagebreak
\section{Jadwal Kegiatan}
%	Berikut ini disajikan tabel jadwal kegiatan yang akan dilakukan selama 3 bulan yang berkoresponden dengan metode penelitian.\vspace{0.5cm}\\
%\begin{tabular}{|C{0.6cm}|L{6.5cm}|C{0.25cm}|C{0.25cm}|C{0.25cm}|C{0.25cm}|C{0.25cm}|C{0.25cm}|C{0.25cm}|C{0.25cm}|C{0.25cm}|C{0.25cm}|C{0.25cm}|C{0.25cm}|}	\hline
%	&&\multicolumn{12}{c|}{\textbf{BULAN}}\\\cline{3-14}
%	\multicolumn{1}{|c|}{\textbf{NO}}&\multicolumn{1}{c|}{\textbf{NAMA KEGIATAN}}&\multicolumn{4}{c|}{1}&\multicolumn{4}{c|}{2}&\multicolumn{4}{c|}{3}\\\cline{3-14}
%	&&1&2&3&4&1&2&3&4&1&2&3&4\\\cline{1-14}
%	1&Studi literatur&\cellcolor{black!}&\cellcolor{black!}&\cellcolor{black!}&&&&&&&&&\\\hline
%	2&Mengkaji ruang metrik parsial rectangular kemudian memberikan contoh himpunan dan fungsi metriknya serta pembuktiannya&&\cellcolor{black!}&\cellcolor{black!}&\cellcolor{black!}&&&&&&&&\\\hline
%	3&Mengkaji karakterisasi teorema titik tetap dan prinsip pemetaan quasi-kontraksi pada ruang metrik parsial rectangular&&&\cellcolor{black!}&\cellcolor{black!}&\cellcolor{black!}&&&&&&&\\\hline
%	4&Pembuktian teorema titik tetap pada ruang metrik parsial rectangular dan mengkonstruksi contoh ruang dan pemetaan lainnya&&&&&\cellcolor{black!}&\cellcolor{black!}&\cellcolor{black!}&\cellcolor{black!}&&&&\\\hline
%	
%	5&Penarikan kesimpulan dan pembukuan tugas akhir&&&\cellcolor{black!}&\cellcolor{black!}&\cellcolor{black!}&\cellcolor{black!}&\cellcolor{black!}&\cellcolor{black!}&\cellcolor{black!}&\cellcolor{black!}&\cellcolor{black!}&\cellcolor{black!}\\\hline
%\end{tabular}
\pagebreak

\addcontentsline{toc}{section}{BIBLIOGRAPHY}
\setstretch{0.95}
\pagestyle{empty}
\section{DAFTAR PUSTAKA}
\vspace{-1cm}
\begin{thebibliography}{10}
	
	\bibitem{aufa} Biahdillah, A. (2019). Kajin Prinsip Kontraksi Banach Pada Ruang b-Metrik Rectangular Bernilai Kompleks dan Penerapannya pada Sistem Persamaan Linear. Tugas Akhir. Fakultas Matematika Komputasi dan Sains Data. Institut Teknologi Sepuluh Nopember:Surabaya.
	
	\bibitem{parsial} Matthews, SG (1994). Partial Metric Topology. Annals of the New York Academy of Sciences, 183-197
	
	\bibitem{recta}	Hassen A, Erdal K dan Hossein L. (2012). “Fixed point results on a class of generalized
	metric spaces
   ”. Mathematical Sciences a Spinger Open Jurnal. Hal 1-6
   
   
	
	\bibitem{rmp}S. Shukla. (2014). Partial rectangular metric spaces and fixed point theorems. Sci. World J. 1 –12, in press.
	
	\bibitem{rpm1} N.V.Dung \& V.T.Le Hang. (2014). A Note on Partial Rectangular Metric Spaces. Mathematica Moravica Vol. 18-1 , 1–8
	
	\bibitem{metrik1}Kreyzig, E., (1978). “Introductory Functional Analysis with Applications”. New York: John Wiley.
	

	
	\bibitem{m.yunus}Singh, R. Aggarwal, J. (2016). Introduction to Metric Space. ResearChgate. 
	
	\bibitem{exParsial} Alghamdi, Maryam A, Naseer S, Oscar V. (2012). On Fixed Theory in Partial Matrix Spaces. Fixed Point Theory and Applications 2012, 2012:175, 1-25
	
	\bibitem{barPar}Soemarsono, R, S. Teorema Titik Tetap Pemetaan Kontraktif dan Kannan Lemah Pada Ruang Metrik Parsial. Tugas Akhir. Fakultas Matematika Komputasi dan Sains Data. Institut Teknologi Sepuluh Nopember. Surabaya.

 
	\bibitem{complex}Ege, O. (2015). “Complex Valued Rectangular b-metric Spaces and an Application to Linear Equations”. J. Nonlinear Sci. Appl. Vol 8. Hal 1014 – 1021.
	
	
	\end{thebibliography}

\end{document}