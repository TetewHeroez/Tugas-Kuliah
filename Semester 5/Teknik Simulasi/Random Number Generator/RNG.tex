\documentclass{article}
\usepackage{amsmath,amssymb,amsfonts,amsthm}
\usepackage{multicol}
\usepackage{multirow}
\usepackage{mathtools}
\usepackage{pgfplots}
\usepackage{soul}
\usepackage{hyperref}
\hypersetup{
    colorlinks=true,
    linkcolor=blue,
    filecolor=magenta,      
    urlcolor=cyan,
    pdftitle={Overleaf Example},
    pdfpagemode=FullScreen,
    }
\usepackage{color}
\usepackage[table]{xcolor}
\usepackage[T1]{fontenc}
\usepackage{etoolbox}
\usepackage{multicol}
\usepackage{multirow}
\usepackage{fancyhdr}
\usepackage{graphicx}
\usepackage{array}
\usepackage{animate}
\usepackage{amsthm}
\usepackage{caption}
\usepackage{listings}
\usepackage{titlesec}

\titleformat*{\section}{\large\bfseries}

\graphicspath{{C:/Users/teoso/OneDrive/Documents/Tugas Kuliah/Template Math Depart/}}

\definecolor{HIMAmuda}{HTML}{01D1FD}
\definecolor{HIMAtua}{HTML}{02016A}
\definecolor{HIMAabu}{HTML}{CBCBCC}
\definecolor{pgray}{rgb}{0.5,0.5,0.5}
\definecolor{pblue}{rgb}{0.13,0.13,1}
\definecolor{pgreen}{rgb}{0,0.5,0}
\definecolor{pred}{rgb}{0.9,0,0}
\definecolor{pgrey}{rgb}{0.46,0.45,0.48}
\definecolor{pcyan}{HTML}{D4EFFC}
\definecolor{lblue}{HTML}{00AEEF}
\definecolor{input}{HTML}{AAE1FA}
\definecolor{bg}{rgb}{0.95, 0.95, 0.92}
\definecolor{vscode}{HTML}{282A36}
\definecolor{PastelGreen}{HTML}{77DD77}

\lstdefinestyle{standard}{
    language            = Python,
    showspaces          = false,
    showtabs            = false,
    breaklines          = true,
    showstringspaces    = false,
    breakatwhitespace   = true,
    commentstyle        = \color{pgray},
    keywordstyle        = \color{pblue},
    stringstyle         = \color{pgreen},
    basicstyle          = \footnotesize\ttfamily,
    frame               = single,
    backgroundcolor     = \color{brown!10!white},
    escapeinside        = {(*}{*)},
    numbers             = left, % {none, left, right}
    numberstyle         = \scriptsize\color{gray},
    numbersep           = -8pt,
    }

\lstdefinestyle{output}{
    language            =Python,
    backgroundcolor     =\color{vscode},
    basicstyle          =\footnotesize\ttfamily\color{white},
    frame               =shadowbox,
    escapeinside        ={(*}{*)},
    showspaces          =false,
    showtabs            =false,
    breaklines          =true,
    showstringspaces    =false,
    breakatwhitespace   =true,
    rulesepcolor        =\color{HIMAtua!50!white},
    rulecolor           =\color{HIMAtua!50!white},
    numbers             =none,
    }

\newtheorem{theorem}{Theorem}
\newtheorem*{teorema}{Teorema}
\newtheorem*{definisi}{Definisi}

\renewcommand{\lstlistingname}{Code}

\begin{document}
\fancyhead[L]{\textit{Teosofi Hidayah Agung}}
\fancyhead[R]{\textit{5002221132}}
\pagestyle{fancy}
\section{Midsquare Method}
The midsquare method algorithm is as follows:
\begin{enumerate}
    \item Choose an integer seed $X_0$ that has four digits.
    \item Calculate $X_0^2$.
    \item Extract the middle four digits of $X_0^2$ as the next random number.
    \item Repeat the process from step 2.
\end{enumerate}
The following is the implementation of the midsquare method algorithm in Python:
\begin{lstlisting}[style=standard,caption={Midsquare Method Algorithm in Python}]
    def midsquare(seed, n):
        # Initialize an empty list
        random_numbers = []

        # Loop n times to generate n random numbers
        for i in range(n):
            # Square the current seed
            seed = str(seed ** 2)

            # Add leading zero's if the result has odd length
            if len(seed) % 2 == 1:
                seed = '0' + seed

            # Extract the middle 4 digits of the squared result
            seed = int(seed[2:6])

            # Append the new seed (random number) to the list
            random_numbers.append(seed)

        # Return the list of generated random numbers
        return random_numbers
\end{lstlisting}
\section{Linear Congruential Method}
The linear congruential method algorithm is as follows:
\begin{enumerate}
    \item Choose four integers $a$, $c$, $m$, and $X_0$. The integer must satisfy the following conditions: $m > 0$ and $a,\,c,\,X_0 < m$.
    \item Calculate $X_{i+1} \equiv (aX_i + c) \mod m$.
    \item Convert $X_{i+1}$ to a random number by dividing it by $m$.
    \item Repeat the process from step 2.
\end{enumerate}
The following is the implementation of the linear congruential method algorithm in Python:
\begin{lstlisting}[style=standard,caption={Linear Congruential Method Algorithm in Python}]
    def linear_congruential(a, c, m, seed, n):
        # Initialize an empty list
        random_numbers = []

        # Loop n times to generate n random numbers
        for i in range(n):
            # Calculate the next random number
            seed = (a * seed + c) % m

            # Append the new seed (random number) to the list
            random_numbers.append(seed / m)

        # Return the list of generated random numbers
        return random_numbers
\end{lstlisting}
\begin{theorem}
    The LCM has a full period if and only if:
    \begin{enumerate}
        \item $c$ and $m$ are relatively prime,
        \item if $q$ is a prime number that divides $m$, then $q$ divides $a - 1$,
        \item $a - 1$ is divisible by 4 if $m$ is divisible by 4.
    \end{enumerate}
\end{theorem}
\begin{proof}
    The full period means that the sequence must not be repeated until the $m$-th number. Consider the recurrence relation:
    \begin{equation}
        X_{i+1} \equiv (aX_i + c) \mod m \label{eq1}
    \end{equation}
    The sequence will be repeated if $X_{i+1} = X_i$. This means that:
    \begin{equation}
        (aX_i + c) \mod m = X_i \label{eq2}
    \end{equation}
    Rearranging equation \eqref{eq2} gives:
    \begin{equation}
        aX_i \equiv X_i - c \mod m \label{eq3}
    \end{equation}
    Since $X_i < m$, then $X_i - c < m$. This means that $X_i - c$ is a non-negative number. Therefore, $X_i - c \mod m = X_i - c$. Substituting this into equation \eqref{eq3} gives:
    \begin{equation}
        aX_i \equiv X_i - c \mod m \label{eq4}
    \end{equation}
    Rearranging equation \eqref{eq4} gives:
    \begin{equation}
        (a - 1)X_i \equiv -c \mod m \label{eq5}
    \end{equation}
    Since $c < m$, then $-c < m$. This means that $-c$ is a non-negative number. Therefore, $-c \mod m = -c$. Substituting this into equation \eqref{eq5} gives:
    \begin{equation}
        (a - 1)X_i \equiv -c \mod m \label{eq6}
    \end{equation}
    Since $X_i < m$, then $X_i$ is relatively prime to $m$. This means that $X_i$ has a multiplicative inverse modulo $m$. Multiplying both sides of equation \eqref{eq6} by $X_i^{-1}$ gives:
    \begin{equation}
        a - 1 \equiv -cX_i^{-1} \mod m \label{eq7}
    \end{equation}
    Since $X_i$ is relatively prime to $m$, then $X_i^{-1}$ exists. This means that equation \eqref{eq7} has a solution if and only if $c$ and $m$ are relatively prime. This proves the first condition. The second condition can be proven similarly. The third condition can be proven by considering the case when $m$ is divisible by 4. This completes the proof.
\end{proof}

\section{Combined Linear Congruential Generators}
The combined linear congruential generator (CLCG) is a method to generate random numbers by combining multiple LCGs. The algorithm is as follows:
\begin{enumerate}
    \item Choose $k$ LCGs with parameters $a_i$, $c_i$, $m_i$, and $X_{0i}$ for $i = 1, 2, \ldots, k$.
    \item Calculate $X_{i+1} \equiv \left(\sum_{j=1}^{k} a_jX_{ij} + c_j\right) \mod m_j$ for $i = 1, 2, \ldots$.
    \item Convert $X_{i+1}$ to a random number by dividing it by $m_j$.
    \item Repeat the process from step 2.
\end{enumerate}
The following is the implementation of the CLCG algorithm in Python:
\begin{lstlisting}[style=standard,caption={Combined Linear Congruential Generator Algorithm in Python}]
    def clcg(a, c, m, seed, n):
        # Initialize an empty list
        random_numbers = []

        # Loop n times to generate n random numbers
        for i in range(n):
            # Calculate the next random number
            seed = sum([ai * xi for ai, xi in zip(a, seed)]) + c
            seed = [si % mi for si, mi in zip(seed, m)]

            # Append the new seed (random number) to the list
            random_numbers.append(seed)

        # Return the list of generated random numbers
        return random_numbers
\end{lstlisting}
    
\end{document}