\documentclass{exam}
\usepackage{graphicx} 
\usepackage{multirow}
\usepackage{multicol}
\usepackage{enumitem}
\usepackage{amssymb}
\usepackage{amsmath}
\usepackage{amsfonts}
\usepackage{amsthm}
\usepackage{xcolor}
\usepackage{cancel}
\usepackage{tcolorbox}
\usepackage{geometry}
\usepackage{tikz}
\usepackage{tikz-3dplot}
\usepackage{accents}
\usepackage{pgfplots, tkz-euclide,calc}
    \usetikzlibrary{patterns,snakes,shapes.arrows,3d}
    \usepgfplotslibrary{fillbetween}
	\geometry{
		total = {160mm, 237mm},
		left = 25mm,
		right = 35mm,
		top = 30mm,
		bottom = 30mm,
	}

\newcommand{\jawab}{\textbf{Solution}:}
\newcommand{\del}{\partial}
\newcommand{\R}{\mathbb{R}}
\newcommand{\N}{\mathbb{N}}

\newtheorem{teorema}{Teorema}

\footer{}{\thepage}{}

\lhead{\textit{Teosofi Hidayah Agung}}
\rhead{\textit{5002221132}}
\headrule

\renewcommand{\solutiontitle}{\noindent\textbf{\underline{Solusi:}}\par\noindent}
\printanswers

\begin{document}
\pagenumbering{gobble}

\begin{enumerate}
    \setcounter{enumi}{1}
    \item Data berikut terdiri atas pengukuran berat (dalam ons) untuk 60 bola \textit{Major League Baseball}::
    \[
    \begin{array}{llllllllll}
    5.09 & 5.08 & 5.21 & 5.17 & 5.07 & 5.24 & 5.12 & 5.16 & 5.18 & 5.19 \\
    5.26 & 5.10 & 5.28 & 5.29 & 5.27 & 5.09 & 5.24 & 5.26 & 5.17 & 5.13 \\
    5.27 & 5.26 & 5.17 & 5.19 & 5.28 & 5.28 & 5.18 & 5.27 & 5.25 & 5.26 \\
    5.26 & 5.18 & 5.13 & 5.08 & 5.25 & 5.17 & 5.09 & 5.16 & 5.24 & 5.23 \\
    5.28 & 5.24 & 5.23 & 5.23 & 5.27 & 5.22 & 5.26 & 5.27 & 5.24 & 5.27 \\
    5.25 & 5.28 & 5.24 & 5.26 & 5.24 & 5.24 & 5.27 & 5.26 & 5.22 & 5.09
    \end{array}
    \]
    Asumsikan bahwa data tersebut merupakan nilai-nilai yang diamati dari sebuah sampel acak yang berasal dari distribusi normal.
    \begin{enumerate}
        \item Tentukan interval kepercayaan 99\% untuk rata-rata berat bola \textit{Major League Baseball}.
        \item Tentukan interval kepercayaan 99\% untuk simpangan baku.
    \end{enumerate}
    \setcounter{enumi}{6}
    \item Misalkan \( X_1, X_2, \ldots, X_n \) adalah sampel acak dari distribusi Weibull, \( X \sim \text{WEI}(\theta, 2) \).
    \begin{enumerate}
        \item Tunjukkan bahwa $\displaystyle Q = 2 \sum_{i=1}^n X_i^2/\theta^2 \sim \chi^2(2n)$.
        \item Gunakan \( Q \) untuk menentukan batas interval kepercayaan 100\( \gamma\% \) untuk \( \theta \).
        \item Tentukan batas bawah interval kepercayaan 100\( \gamma\% \) untuk $P(X > t) = \exp\left[-(t/\theta)^2\right]$.
        \item Tentukan batas atas interval kepercayaan 100\( \gamma\% \) untuk persentil ke-\( p \) dari distribusi tersebut.
    \end{enumerate}
    \begin{solution}
        \begin{enumerate}
            \item Diketahui bahwa \( X \sim \text{WEI}(\theta, 2) \) memiliki CDF sebagai berikut:
            \[F(x) = 1 - \exp\left[-\left(\frac{x}{\theta}\right)^2\right], \quad x > 0\]
            Disini akan digunakan metode CDF untuk mencari distribusi dari $Z=\frac{X^2}{\theta^2}$
            \begin{align*}
                P(Z \leq z) &= P\left(\frac{X^2}{\theta^2} \leq z\right) \\
                &= P\left(X^2 \leq z\theta^2\right) 
            \end{align*}
            Perhatikan karena $X>0$, maka $X^2 \leq z\theta^2$ setara dengan $X \leq \theta \sqrt{z}$. Sehingga
            \begin{align*}
                P(Z \leq z)&= P\left(X \leq \theta \sqrt{z}\right) \\
                &= F(\theta \sqrt{z}) \\
                &= 1 - \exp\left[-\left(\frac{\theta \sqrt{z}}{\theta}\right)^2\right] \\
                &= 1 - \exp\left(-z\right)
            \end{align*}
            Sehingga, CDF dari $Z$ adalah $1 - \exp(-z)$, yang merupakan CDF dari distribusi eksponensial. Dengan demikian, \( Z \sim \text{EXP}(1) \).\\
            Selanjutnya untuk $\sum_{i=1}^n X_i^2/\theta^2= \sum_{i=1}^n Z_i$ menggunakan sifat MGF yang independen, didapat bahwa $\sum_{i=1}^n Z_i \sim \text{GAM}(1, n)$. Yang terakhir dengan menggunakan hubungan antara distribusi gamma dan chi-square, maka $2\sum_{i=1}^n Z_i \sim \chi^2(2n)$.\\

            $\therefore Q \sim \chi^2(2n)$.

            \item Interval kepercayaan 100\( \gamma\% \) untuk \( \theta \) adalah
            \begin{align*}
                P\left(\chi^2_{\frac{1-\gamma}{2}}(2n) \leq Q \leq \chi^2_{\frac{1+\gamma}{2}}(2n)\right) &=\gamma \\
                P\left(\chi^2_{\frac{1-\gamma}{2}}(2n) \leq 2\sum_{i=1}^n X_i^2/\theta^2 \leq \chi^2_{\frac{1+\gamma}{2}}(2n)\right) &= \gamma \\
                P\left(\frac{2\sum_{i=1}^n X_i^2}{\chi^2_{\frac{1+\gamma}{2}}(2n)} \leq \theta^2 \leq \frac{2\sum_{i=1}^n X_i^2}{\chi^2_{\frac{1-\gamma}{2}}(2n)}\right) &= \gamma \\
                P\left(\sqrt{\frac{2\sum_{i=1}^n X_i^2}{\chi^2_{\frac{1+\gamma}{2}}(2n)}} \leq \theta \leq \sqrt{\frac{2\sum_{i=1}^n X_i^2}{\chi^2_{\frac{1-\gamma}{2}}(2n)}}\right) &= \gamma
            \end{align*}
            Sehingga, interval kepercayaan 100\( \gamma\% \) untuk \(\theta\) adalah
            \[\left[\sqrt{\frac{2\sum_{i=1}^n X_i^2}{\chi^2_{\frac{1+\gamma}{2}}(2n)}}, \sqrt{\frac{2\sum_{i=1}^n X_i^2}{\chi^2_{\frac{1-\gamma}{2}}(2n)}}\right]\]

            \item Dari poin sebelumnya, diketahui bahwa 
            \begin{align*}
                \frac{\chi^2_{\frac{1-\gamma}{2}}(2n)}{2\sum_{i=1}^n X_i^2} \leq\frac{1}{\theta^2}& \leq \frac{\chi^2_{\frac{1+\gamma}{2}}(2n)}{2\sum_{i=1}^n X_i^2}\\
                \frac{t^2\chi^2_{\frac{1-\gamma}{2}}(2n)}{2\sum_{i=1}^n X_i^2} \leq\left(\frac{t}{\theta}\right)^2& \leq \frac{t^2\chi^2_{\frac{1+\gamma}{2}}(2n)}{2\sum_{i=1}^n X_i^2}\\
                -\frac{t^2\chi^2_{\frac{1+\gamma}{2}}(2n)}{2\sum_{i=1}^n X_i^2} \leq-\left(\frac{t}{\theta}\right)^2& \leq -\frac{t^2\chi^2_{\frac{1-\gamma}{2}}(2n)}{2\sum_{i=1}^n X_i^2}\\
                \exp\left(-\frac{t^2\chi^2_{\frac{1+\gamma}{2}}(2n)}{2\sum_{i=1}^n X_i^2}\right) \leq\exp\left[-\left(\frac{t}{\theta}\right)^2\right]& \leq \exp\left(-\frac{t^2\chi^2_{\frac{1-\gamma}{2}}(2n)}{2\sum_{i=1}^n X_i^2}\right)\\
            \end{align*}
            Sehingga, batas bawah interval kepercayaan 100\( \gamma\% \) untuk \( P(X > t) = \exp\left[-(t/\theta)^2\right] \) adalah
            \[P(X > t)\geq\exp\left(-\frac{t^2\chi^2_{\frac{1+\gamma}{2}}(2n)}{2\sum_{i=1}^n X_i^2}\right)\]
            \item Persentil ke-\( p \) adalah nilai \( x_p\) sehingga \( P(X \leq x_p) = p \). Dengan kata lain
            \begin{align*}
                P(X \leq x_p) &= p \\
                1 - \exp\left[-\left(\frac{x_p}{\theta}\right)^2\right] &= p \\
                \exp\left[-\left(\frac{x_p}{\theta}\right)^2\right] &= 1 - p \\
                -\left(\frac{x_p}{\theta}\right)^2 &= \ln(1-p) \\
                \left(\frac{x_p}{\theta}\right)^2 &= -\ln(1-p)
            \end{align*}
            Karena $x_p\geq 0$, maka $x_p = \theta\sqrt{-\ln(1-p)}$. Kemudian dengan menggunakan hasil dari poin (b), didapat bahwa
            \[x_p\leq\sqrt{\frac{-2\ln(1-p)\sum_{i=1}^n X_i^2}{\chi^2_{\frac{1-\gamma}{2}}(2n)}}\]
        \end{enumerate}
    \end{solution}
    
    \item Misalkan sampel acak ukuran \( n \) berasal dari distribusi uniform \( X_i \sim \text{UNIF}(0, \theta) \), \( \theta > 0 \), dan \( X_{n:n} \) adalah order statistik terbesar.
    \begin{enumerate}
        \item Tentukan probabilitas bahwa interval acak \( (X_{n:n}, 2X_{n:n}) \) memuat \( \theta \).
        \item Tentukan konstanta \( c \) sehingga \( (X_{n:n}, cX_{n:n}) \) adalah interval kepercayaan 100\( (1-\alpha)\% \) untuk \( \theta \).
    \end{enumerate}
\end{enumerate}
\end{document}