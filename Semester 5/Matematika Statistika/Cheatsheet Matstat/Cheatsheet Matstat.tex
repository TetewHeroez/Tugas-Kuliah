\documentclass[a4paper,extrafontsizes, 9pt]{memoir}

\usepackage{amsmath, amssymb, amsfonts, amsthm}
\usepackage{multicol}
\usepackage{multirow}
\usepackage{lipsum}
\usepackage{mathtools}
\usepackage{pgfplots}
\usepackage{soul}
\usepackage{hyperref}
\usepackage{enumitem}
\usepackage{fancyhdr}

\pagestyle{empty}

\setlrmarginsandblock{1cm}{1cm}{*}
\setulmarginsandblock{1cm}{1cm}{*}
\checkandfixthelayout

\DeclareMathOperator{\Cov}{Cov}
\DeclareMathOperator{\Var}{Var}

\let\bf\textbf{}

\setlength{\parindent}{0pt}
\renewcommand{\arraystretch}{1.5}

\newcommand{\textbetweenrules}[2][.4pt]{%
  \par\vspace{\topsep}
  \noindent\makebox[\textwidth]{%
    \sbox0{#2}%
    \dimen0=.5\dimexpr\ht0+#1\relax
    \dimen2=-.5\dimexpr\ht0-#1\relax
    \leaders\hrule height \dimen0 depth \dimen2\hfill
    \quad #2\quad
    \leaders\hrule height \dimen0 depth \dimen2\hfill
  }\par\nopagebreak\vspace{\topsep}
}

\begin{document}
\footnotesize \textbf{{Cheatsheet -- Matematika Statistika}} \hfill \textit{Teosofi Hidayah Agung/5002221132}
	\begin{multicols}{3}			
        \subsection*{\small Metode Transformasi}
            \subsubsection*{\small Multidimensi}
                    Untuk kasus diskrit sama seperti sebelumnya, hanya perlu menambah peubah acak pada fungsinya. Namun untuk kasus kontinu kita perlu meninjau Jacobian dari fungsi transformasi tersebut.

                    Misal $X=(X_1,\dots,X_n)$ variabel acak kontinu dengan pdf bersama $f_X(x_1, x_2,\dots, x_n) > 0$ atas $A$, dan $Y = (Y_1, Y_2,\dots, Y_n)$ didefinisikan
                    oleh transformasi satu-satu $Y_i = u_i(X_1,X_2,\dots ,X_n)$, maka pdf bersama dari $Y$ adalah
                    \[f_Y(y_1,y_2,\dots,y_n) = f_X(x_1,x_2,\dots,x_n)\left|J\right|\]
                    dengan $J=\begin{vmatrix}
                        \frac{\partial x_1}{\partial y_1} & \frac{\partial x_1}{\partial y_2} & \dots & \frac{\partial x_1}{\partial y_n}\\
                        \vdots & \vdots & \ddots & \vdots\\
                        \frac{\partial x_n}{\partial y_1} & \frac{\partial x_n}{\partial y_2} & \dots & \frac{\partial x_n}{\partial y_n}
                    \end{vmatrix}$ 
                
                Jika transformasi tidak satu-satu, dengan cara yang sama yaitu kita partisi $A$ sehingga $u(x)$ satu-satu pada $A_i$. Kemudian jumlahkan semua pdf nya.
            \subsection*{\small Order Statistik}
                Misal $X_1, X_2, \dots, X_n$ adalah sampel acak. Kemudian didefinisikan
                \begin{itemize}
                    \item $X_{(1)}$ adalah sampel terkecil dari $X_1, X_2, \dots, X_n$.
                    \item $X_{(i)}$ adalah sampel terkecil ke-$i$ dari $X_1, X_2, \dots, X_n$ untuk $i=2,3,\dots,n-1$.
                    \item $X_{(n)}$ adalah sampel terbesar dari $X_1, X_2, \dots, X_n$.
                \end{itemize}
                Misalkan $Y_k=X_{(k)}$, maka pdf bersama untuk order statistik $Y_1,Y_2,\dots,Y_n$ adalah
                \[g(y_1,y_2,\dots,y_n) = n!f(y_1)f(y_2)\dots f(y_n)\]
                \[y_1<y_2<\dots<y_n\]
                Kemudian untuk pdf dari $Y_k$ adalah
                \[g_{k}(y) = \dfrac{n!F(y)^{k-1}[1-F(y)]^{n-k}f(y)}{(k-1)!(n-k)!},\,a<y_k<b\]
                \begin{itemize}
                    \item PDF $Y_1$: $g_1(y_1) = nf(y_1)[1-F(y_1)]^{n-1}$
                    \item PDF $Y_n$: $g_n(y_n) = nf(y_n)[F(y_n)]^{n-1}$
                    \item CDF $Y_k$: $G_k(y_k) = \sum_{i=k}^{n}\binom{n}{i}F(y_k)^{i}[1-F(y_k)]^{n-i}$
                    \item CDF $Y_1$: $G_1(y_1) = 1-[1-F(y_1)]^{n}$
                    \item CDF $Y_n$: $G_n(y_n) = [F(y_n)]^{n}$
                \end{itemize}
                \subsection*{\small Teorema Limit Pusat}
                Misal $Y_1,Y_2,\dots$ adalah barisan variabel acak dengan CDF $G_1(y),G_2(y),\dots$ dan MGF $M_1(t),M_2(t),\dots$. Jika $\displaystyle\lim_{n\to\infty}M_n(t)=M(t)$ untuk setiap $t$ dalam suatu interval $-h<t<h$, maka $\displaystyle\lim_{n\to\infty}G_n(y)=G(y)$.\\

                Jika $X_1,X_2,\dots,X_n$ adalah sampel acak dari distribusi dengan mean $\mu$ dan varian $\sigma^2$, maka limiting distribution dari
                \[Z_n=\dfrac{\bar{X}-\mu}{\sigma/\sqrt{n}}\]
                adalah distribusi normal standar, $Z_n\xrightarrow{d} N(0,1)$.

                \subsection*{\small Asimtotik Distribusi Normal}
                Jika $Y_1,Y_2,...$ adalah barisan variabel acak dan $m$ dan $c$ adalah konstan sedemikian sehingga
                \[Z_n=\dfrac{Y_n-m}{c}\xrightarrow{d}Z\sim N(0,1)\]
                untuk $n\to\infty$, maka $Y_n$ dikatakan berdistribusi normal asimtotik dengan mean $m$ dan varians $c^2/n$.\\

                Misalkan $X_1,X_2,\dots,X_n$ adalah sampel acak dari distribusi kontinu dengan PDF $f(x)$ dan tidak nol pada persentil ke-$p$. Jika $k/n\to p$, maka barisan order statistik ke-$k$, $X_{k:n}$ adalah normal asimtotik dengan mean $x_p$ dan varians $c^2/n$, dimana 
                \[c^2 = \dfrac{p(1-p)}{f^2(x_p)}\]

                \subsection*{\small Konvergen Stokastik}
                Suatu barisan variabel acak $Y_1,Y_2,\dots$ dikatakan konvergen stokastik menuju sebuah konstanta $c$ jika barisan tersebut terdistribusi terbatas pada satu nilai $y=c$, dapat ditulis
                \[\lim_{n\to\infty}G_n(y)=G(y)=\begin{cases}
                    0 &, y<c\\
                    1 &, y\geq c
                \end{cases}\]
                note: $\displaystyle\lim_{n\to\infty}\left(1+\dfrac{c}{n}\right)^{nb}=e^{bc}$

                Barisan $Y_1,Y_2,\dots$ konvergen stokastik ke $c$ jika dan hanya jika untuk setiap $\epsilon>0$, 
                \[\lim_{n\to\infty}P(|Y_n-c|<\epsilon)=1\]
                Barisan tersebut dapat dikatakan konvergen dalam peluang menuju suatu konstanta $c$, yang dinotasikan dengan $Y_n\xrightarrow{P}c$.\\

                Ketaksamaan Chebyshev: $P(|X-\mu|\geq k\sigma)\leq\dfrac{1}{k^2}$ atau dapat ditulis $P(|X-\mu|< k\sigma)\geq 1-\dfrac{1}{k^2}$,
                dengan $X$ adalah variabel acak, $\mu$ adalah mean, dan $\sigma$ adalah standar deviasi.\\

                Jika $X_1,X_2,\dots,X_n$ adalah sampel acak dari distribusi dengan mean $\mu$ dan varian $\sigma^2$, maka barisan sampel mean konvergen dalam peluang menuju $\mu$, dinotasikan dengan $\bar{X}_n\xrightarrow{P}\mu$.
                \vspace*{-0.3cm}
                \subsection*{\small Sifat-sifat}
                Jika $X_n\xrightarrow{P}c$ dan $Y_n\xrightarrow{P}d$, maka
                \begin{itemize}
                    \setlength\itemsep{0.5mm}
                    \item $X_n+Y_n\xrightarrow{P}c+d$
                    \item $X_n-Y_n\xrightarrow{P}c-d$
                    \item $X_nY_n\xrightarrow{P}cd$
                    \item $X_n/c\xrightarrow{P}1$ jika $c\neq 0$
                    \item $\sqrt{X_n}\xrightarrow{P}\sqrt{c}$
                \end{itemize}
            \section*{\small Distribusi Sampel}
            \subsection*{\small Kombinasi Linear dari Variabel Normal}
                Jika $X_i\sim N(\mu_i,\sigma_i^2)$ dengan $i=1,2,\dots,n$ menotasikan variabel normal independen, maka
                \[Y=\sum_{i=1}^{n}a_iX_i\sim N\left(\sum_{i=1}^{n}a_i\mu_i,\sum_{i=1}^{n}a_i^2\sigma_i^2\right)\]
            \subsection*{\small Distribusi Chi-Square}
                Jika $Y\sim\chi^2(\nu)$, maka
                \begin{itemize}
                    \item MGF: $M_Y(t)=(1-2t)^{-\nu/2}$
                    \item $E(Y)=\nu$
                    \item $\Var(Y)=2\nu$
                \end{itemize}
                Jika $X\sim GAM(\theta,\kappa)$, maka $Y=2X/\theta \sim\chi^2(2\kappa)$.

                Persentil ke-$p$ dari distribusi gamma dapat diperoleh dari $x_p=\theta\chi^2_p(2\kappa)$.

                Jika $Y_i\sim\chi^2(\nu_i);i=1,\dots,n$ adalah variabel chi-square independen, maka
                \[V=\sum_{i=1}^{n}Y_i\sim\chi^2\left(\sum_{i=1}^{n}\nu_i\right)\]

                Jika $Z\sim N(0,1)$, maka $Z^2\sim\chi^2(1)$.

                Jika $X_1,\dots,X_n$ menotasikan suatu sampel acak dari $N(\mu,\sigma^2)$, maka
                \[\sum_{i=1}^{n}\left(\dfrac{X_i-\mu}{\sigma}\right)^2\sim\chi^2(n)\]
                \[\frac{n(\bar{X}-\mu)^2}{\sigma^2}\sim\chi^2(1)\]
                \[\dfrac{(n-1)S^2}{\sigma^2}\sim\chi^2(n-1)\]
            \subsection*{\small Distribusi Student's t}
                Jika $Z\sim N(0,1)$ dan $V\sim\chi^2(\nu)$, untuk $Z$ dan $V$ independen berakibat distribusi $\displaystyle T=\dfrac{Z}{\sqrt{V/\nu}}$ adalah distribusi t dengan $\nu$ derajat kebebasan. Dinotasikan dengan $T\sim t(\nu)$, dimana pdf dari $T$ adalah
                \[f(t;\nu)=\dfrac{\Gamma\left(\dfrac{\nu+1}{2}\right)}{\sqrt{\nu\pi}\Gamma\left(\dfrac{\nu}{2}\right)}\left(1+\dfrac{t^2}{\nu}\right)^{-\dfrac{(\nu+1)}{2}}\]

                Jika $T\sim t(\nu)$, maka untuk $\nu>2r$
                \[E(T^{2r})=\dfrac{\Gamma\left(\dfrac{2r+1}{2}\right)\Gamma\left(\dfrac{\nu-2r}{2}\right)}{\sqrt{\pi}\Gamma(\nu/2)}v^r\]
                \[E(T^{2r-1})=0 \text{ untuk } r=1,2,\dots\]
                \[\Var(T)=\dfrac{\nu}{\nu-2},\,\nu>2\]

                Jika $X_1,\dots,X_n$ adalah sampel acak dari $N(\mu,\sigma^2)$, maka
                \[\dfrac{\bar{X}-\mu}{S/\sqrt{n}}\sim t(n-1)\]

            \subsection*{\small Distribusi F}
                Jika $V_1\sim\chi^2(\nu_1)$ dan $V_2\sim\chi^2(\nu_2)$ adalah independen, maka distribusi $X=\dfrac{V_1/\nu_1}{V_2/\nu_2}$ adalah distribusi F dengan derajat kebebasan $\nu_1$ dan $\nu_2$. Dinotasikan dengan $X\sim F(\nu_1,\nu_2)$.\\

                JIka $X\sim F(\nu_1,\nu_2)$, maka
                \[E(X)=\dfrac{\nu_2}{\nu_2-2}\]
                \[\Var(X)=\dfrac{2\nu_2^2(\nu_1+\nu_2-2)}{\nu_1(\nu_2-2)^2(\nu_2-4)}\]

                Fakta bahwa $X\sim F(\nu_1,\nu_2)$ dan $1/X\sim F(\nu_2,\nu_1)$. berakibat 
                \[f_{1-\gamma}(\nu_1,\nu_2)=\dfrac{1}{f_{\gamma}(\nu_1,\nu_2)}\]

            \subsection*{\small Distribusi Beta}
                Suatu variabel F dapat dtransformasikan untuk distribusi beta. Jika $X\sim F(\nu_1,\nu_2)$, maka
                \[Y=\dfrac{(\nu_1/\nu_2)X}{1+(\nu_1/\nu_2)X}\sim BETA\left(a,b\right)\]
                dengan $a=\nu_1/2$ dan $b=\nu_2/2$ yang memiliki pdf 
                \[f(y;a,b)=\dfrac{\Gamma(a+b)}{\Gamma(a)\Gamma(b)}y^{a-1}(1-y)^{b-1}\]
                Rataan dan varians dari distribusi beta adalah
                \[E(Y)=\dfrac{a}{a+b}\]
                \[\Var(Y)=\dfrac{ab}{(a+b)^2(a+b+1)}\]
                Sedangkan, persentil dri suatu distribusi beta dapat diekspresikan dalam bentuk persentil dari distribusi $F$ sebagai hasil dari persamaan sebelumnya
                \[y_\gamma(a,b)=\frac{af_{\gamma}(2a,2b)}{bf_{\gamma}(2a,2b)}\]
            
            \subsection*{\small Pendekatan Sampel Ukuran Besar}
                Jika $Y_\nu\sim \chi^2(\nu)$, maka
                \[Z_\nu=\dfrac{Y_\nu-\nu}{\sqrt{2\nu}}\xrightarrow{d}N(0,1)\]
                ketika $\nu\to\infty$.

                \textbf{Pendekatan Wilson-Hilferty} diberikan oleh
                \[\chi^2_{\gamma}(\nu)=\nu\left(1-\dfrac{2}{9\nu}+z_{\gamma}\sqrt{\dfrac{2}{9\nu}}\right)^3\]
                ketika $\nu\to\infty$.
	\end{multicols}

    \newpage
    \textbetweenrules[2pt]{\textbf{Distribusi Bersama}}
    \vspace*{-10pt}
	\section*{\small Tabel Distribusi Diskrit}
	\begin{tabular}{|c|c|c|c|c|c|}
        \hline
        Nama & Notasi dan & \textbf{PDF} Diskrit & \textbf{Ekspektasi} & \textbf{Varian} & \textbf{MGF}\\
        Distribusi& Parameter & $f(x)$ & $E(X)$ & $\Var(X)$ & $M_{X}(t)$\\
        \hline
        \hline
        Bernoulli & $X\sim B(1,p)$ & $p^{x}(1-p)^{1-x}$ & $p$ & $p(1-p)$ & $1-p+pe^{t}$\\
        \hline
        Binomial & $X\sim B(n,p)$ & $\displaystyle\binom{n}{x}p^{x}(1-p)^{n-x}$ & $np$ & $np(1-p)$ & $(1-p+pe^{t})^{n}$\\
        \hline
        Negatif Binomial & $X\sim NB(r,p)$ & $\displaystyle\binom{x-1}{r-1}p^{r}(1-p)^{x}$ & $\dfrac{r}{p}$ & $\dfrac{r(1-p)}{p^{2}}$ & $\left(\dfrac{p}{1-(1-p)e^{t}}\right)^{r}$\\
        \hline
        Geometrik & $X\sim G(p)$ & $p(1-p)^{x-1}$ & $\dfrac{1}{p}$ & $\dfrac{1-p}{p^{2}}$ & $\dfrac{p}{1-(1-p)e^{t}}$\\
        \hline
        Hypergeometrik & $X\sim H(n,M,N)$ & $\dfrac{\displaystyle\binom{M}{x}\binom{N-M}{n-x}}{\displaystyle\binom{N}{n}}$ & $n\dfrac{M}{N}$ & $n\dfrac{M}{N}\left(1-\dfrac{M}{N}\right)\dfrac{N-n}{N-1}$ & -\\
        \hline
		Multinomial & $X\sim M(n,p_{1},\ldots,p_{k})$ & $\dfrac{n!}{x_{1}!\cdots x_{k}!}p_{1}^{x_{1}}\cdots p_{k}^{x_{k}}$ & $np_{i}$ & $np_{i}(1-p_{i})$ & $\left(\sum_{i=1}^{k}p_{i}e^{t_{i}}\right)^{n}$\\
		\hline
        Poisson & $X\sim P(\mu)$ & $\dfrac{e^{-\mu}\mu^{x}}{x!}$ & $\mu$ & $\mu$ & $e^{\mu(e^{t}-1)}$\\
        \hline
        Uniform Diskrit & $X\sim U(a,b)$ & $\dfrac{1}{b-a}$ & $\dfrac{a+b}{2}$ & $\dfrac{(b-a)^{2}}{12}$ & $\dfrac{e^{tb}-e^{ta}}{t(b-a)}$\\
        \hline
    \end{tabular}
    \section*{\small Tabel Distribusi Kontinu}
    \begin{tabular}{|c|c|c|c|c|c|}
        \hline
        Nama & Notasi dan & \textbf{PDF} Kontinu & \textbf{Ekspektasi} & \textbf{Varian} & \textbf{MGF}\\
        Distribusi& Parameter & $f(x)$ & $E(X)$ & $\Var(X)$ & $M_{X}(t)$\\
        \hline
        \hline
        Uniform & $X\sim UNIF(a,b)$ & $\dfrac{1}{b-a}$ & $\dfrac{a+b}{2}$ & $\dfrac{(b-a)^{2}}{12}$ & $\dfrac{e^{bt}-e^{at}}{(b-a)t}$\\
        \hline
        Normal & $X\sim N(\mu,\sigma^{2})$ & $\dfrac{1}{\sqrt{2\pi}\sigma}e^{-\frac{(x-\mu)^{2}}{2\sigma^{2}}}$ & $\mu$ & $\sigma^{2}$ & $e^{\mu t+\dfrac{\sigma^{2}t^{2}}{2}}$\\
        \hline
        Gamma & $X\sim GAM(\theta,\kappa)$ & $\dfrac{1}{\theta^\kappa\Gamma(\kappa)}x^{\kappa-1}e^{-x/\theta}$ & $\kappa\theta$ & $\kappa\theta^2$ & $\left(\dfrac{1}{1-\theta t}\right)^{\kappa}$\\
        \hline
        Exponential & $X\sim EXP(\theta)$ & $\dfrac{1}{\theta} e^{-x/\theta}$ & $\theta$ & $\theta^2$ & $\dfrac{1}{1-\theta t}$\\
        \hline
        Weibull & $X\sim WEI(\theta,\beta)$ & $\dfrac{\beta}{\theta^\beta}x^{\beta-1}e^{-(x/\theta)^{\beta}}$ & $\theta\Gamma\left(1+\dfrac{1}{\beta}\right)$ & $\theta^{2}\left[\Gamma\left(1+\dfrac{2}{\beta}\right)-\Gamma^2\left(1+\dfrac{1}{\beta}\right)\right]$ & -\\
        \hline
        Pareto & $X\sim PAR(\theta,\kappa)$ & $\dfrac{\kappa}{\theta(1+x/\theta)^{\kappa+1}}$ & $\dfrac{\theta}{\kappa-1}$ & $\dfrac{\theta^{2}\kappa}{(\kappa-1)^{2}(\kappa-2)}$ & -\\
        \hline
    \end{tabular}
\end{document}