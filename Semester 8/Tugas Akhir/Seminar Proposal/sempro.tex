\documentclass[aspectratio=169]{beamer}
\usepackage[style=ieee,backend=biber]{biblatex}
    \addbibresource{reference.bib}
\usepackage{csquotes}
\usepackage{colortbl,tabularx,mathrsfs,calligra}
\usepackage{amsmath,amsfonts,amssymb,amsthm}
\usepackage{ragged2e}
\usepackage[bahasa]{babel}
\usepackage{tikz}
\usepackage{caption}
\usepackage{wrapfig}
\usepackage{multirow}
\usepackage{multicol}
\usepackage{array}
\usepackage{pgfplots, tkz-euclide,calc}
    \pgfplotsset{compat=1.18}
\usepackage{listings}

\graphicspath{{C:/Users/teoso/OneDrive/Documents/Tugas Kuliah/Template Math Depart/}{./foto/}}

\definecolor{HIMAmuda}{HTML}{01D1FD}
\definecolor{HIMAtua}{HTML}{02016A}
\definecolor{HIMAabu}{HTML}{CBCBCC}

\usetheme{Madrid}

\setbeamercolor{palette primary}{bg=HIMAtua,fg=white}
\setbeamercolor{palette secondary}{bg=HIMAmuda,fg=black}
\setbeamercolor{palette tertiary}{bg=HIMAabu,fg=black}
\setbeamercolor{palette quaternary}{bg=HIMAmuda,fg=white}
\setbeamercolor{structure}{fg=HIMAmuda} % itemize, enumerate, etc
\setbeamercolor{section in toc}{fg=HIMAtua} % TOC sections
\setbeamercolor{bibliography item}{parent=palette secondary}
\setbeamercolor*{bibliography entry author}{parent=section in toc}

\usetikzlibrary{shapes.geometric, arrows}

\tikzstyle{startstop} = [ellipse, minimum width=1cm, minimum height=1cm,text centered, draw=black, fill=red!30]
\tikzstyle{process} = [rectangle, minimum width=2cm, minimum height=1cm, text centered, draw=black, fill=blue!30]
\tikzstyle{decision} = [diamond, minimum width=1cm, minimum height=1cm, text centered, draw=black, fill=blue!50]
\tikzstyle{arrow} = [thick,->,>=stealth]

\colorlet{cOne}{white}     % 1 putih
\colorlet{cTwo}{yellow}   % 2 kuning
\colorlet{cThree}{cyan}   % 3 merah muda
\colorlet{cFour}{green}   % 4 hijau

\newcommand{\numcolor}[1]{%
  \ifnum#1=1 cOne\fi
  \ifnum#1=2 cTwo\fi
  \ifnum#1=3 cThree\fi
  \ifnum#1=4 cFour\fi
}


\newcommand{\permcell}[3]{%
  % #1 = x-shift
  % #2 = nilai sigma(i)
  % #3 = indeks i

  % kotak
  \draw[thick, fill=\numcolor{#3}]
    (#1,0) rectangle ++(1,1);

  % lingkaran
  \draw[thick, fill=\numcolor{#2}]
    (#1+0.5,0.5) circle (0.3);

  % angka
  \node at (#1+0.5,0.5) {\small #2};

  % fixed point
  \ifnum#2=#3
    \draw[red!60, very thick] (#1,0) -- (#1+1,1);
    \draw[red!60, very thick] (#1,1) -- (#1+1,0);
  \fi
}


\newcommand{\permrow}[4]{%
  \begin{tikzpicture}[baseline,scale=0.8]
    \permcell{0}{#1}{1}
    \permcell{1}{#2}{2}
    \permcell{2}{#3}{3}
    \permcell{3}{#4}{4}
  \end{tikzpicture}
}

\newcolumntype{L}[1]{>{\raggedright\let\newline\\\arraybackslash\hspace{0pt}}m{#1}}
\newcolumntype{C}[1]{>{\centering\let\newline\\\arraybackslash\hspace{0pt}}m{#1}}
\newcolumntype{R}[1]{>{\raggedleft\let\newline\\\arraybackslash\hspace{0pt}}m{#1}}

\usefonttheme{professionalfonts}
\setbeamertemplate{theorems}[numbered]
\setbeamertemplate{bibliography item}{\insertbiblabel}
% \setbeamercovered{transparent}


\theoremstyle{definition}
% \numberwithin{subsection}{section}
\newtheorem{definisi}{Definisi}
\numberwithin{definisi}{subsection}
\newtheorem{teorema}[definisi]{Teorema}
\newtheorem{lema}[definisi]{Lema}
\newtheorem{proposisi}[definisi]{Proposisi}
\newcommand{\R}{\mathbb{R}}
\newcommand{\N}{\mathbb{N}}
\newcommand{\Z}{\mathbb{Z}}
\newcommand{\C}{\mathbb{C}}


\AtBeginEnvironment{contoh}{%
  \setbeamercolor{block title}{use=example text,fg=white,bg=example text.fg!75!black}
  \setbeamercolor{block body}{parent=normal text,use=block title example,bg=block title example.bg!10!bg}
}
\AtBeginEnvironment{definisi}{
    \setbeamercolor{block title}{fg=white,bg=HIMAtua}
    \setbeamercolor{block body}{parent=normal text,bg=HIMAtua!30!white}
}

\date{Desember 2025}
\title[Proposal TA]{Latin Square Komutatif atas Aljabar Max-Plus}
\author[Teosofi]{Teosofi Hidayah Agung}
\institute[Matematika ITS]{Departemen Matematika\\ Institut Teknologi Sepuluh Nopember}
\titlegraphic{\includegraphics[width=2cm]{logoITS}$\quad$\includegraphics[width=2cm]{M.png}}

\begin{document}
\begin{frame}
  \titlepage
\end{frame}

\AtBeginSection[]
{
  \begin{frame}{Daftar isi}
    \tableofcontents[currentsection,hidesubsections]
  \end{frame}
}

\section{Pendahuluan}
\subsection{Latar Belakang}
\begin{frame}
  \frametitle{\insertsection}
  \framesubtitle{\insertsubsection}
  \begin{itemize}
    \item Aljabar max-plus adalah struktur aljabar yang terdiri dari himpunan bilangan real yang diperluas dengan elemen $-\infty$, dilengkapi dengan dua operasi biner yaitu:
          \begin{align*}
            a \oplus b  & = \max\{a,b\} \\
            a \otimes b & = a + b
          \end{align*}
    \item Merupakan cabang dari aljabar tropikal yang dikembangkan oleh Imre Simon pada tahun 1988.
    \item Memiliki sifat idempoten dan non-linier yang sangat berguna untuk berbagai aplikasi.
  \end{itemize}
\end{frame}

\begin{frame}
  \frametitle{\insertsection}
  \framesubtitle{\insertsubsection}
  \textit{Latin square} adalah susunan $n\times n$ yang diisi dengan $n$ simbol berbeda, sehingga setiap simbol muncul tepat satu kali pada setiap baris dan setiap kolom.
  \begin{itemize}
    \item Pertama kali diperkenalkan oleh Leonhard Euler pada abad ke-18.
    \item Memiliki hubungan erat dengan representasi matriks permutasi.
    \item Setiap \textit{Latin square} dapat direpresentasikan sebagai matriks berukuran $n \times n$.
  \end{itemize}
\end{frame}

\begin{frame}
  \frametitle{\insertsection}
  \framesubtitle{\insertsubsection}
  Hubungan antara aljabar max-plus dan kriptografi:
  \begin{itemize}
    \item Sifat idempoten dan non-linier aljabar max-plus dapat meningkatkan keamanan sistem kriptografi.
    \item Banyak protokol kriptografi berbasis aljabar tropikal yang telah diusulkan.
    \item Contohnya adalah Protokol Stickel yang diajukan Eberhard Stickel pada tahun 2005.
    \item Protokol Stickel awalnya menggunakan grup matriks biasa, namun pada perkembangan selanjutnya diadaptasi menggunakan semiring tropikal.
  \end{itemize}
\end{frame}

\subsection{Rumusan Masalah}
\begin{frame}
  \frametitle{\insertsection}
  \framesubtitle{\insertsubsection}
  \begin{enumerate}
    \item \onslide<2->{Apa saja syarat perlu untuk dua buah \textit{Latin square} $A,B\in\R^{n\times n}_{\max}$ agar komutatif terhadap operasi $\otimes$?}
    \item \onslide<3->{Apa saja syarat cukup untuk dua buah \textit{Latin square} $A,B\in\R^{n\times n}_{\max}$ agar komutatif terhadap operasi $\otimes$?}
    \item \onslide<4->{Bagaimana cara membangun pasangan \textit{Latin square} komutatif dari $L_S(n)$?}
  \end{enumerate}
\end{frame}

\subsection{Batasan Masalah}
\begin{frame}
  \frametitle{\insertsection}
  \framesubtitle{\insertsubsection}
  \begin{itemize}
    \item \onslide<2->{Penelitian ini hanya membahas \textit{Latin square} berukuran $n\times n$ dengan setiap baris dan kolomnya merupakan permutasi dari himpunan $\underline{n}=\{1,2,\ldots,n\}$.}
    \item \onslide<3->{Penelitian ini membatasi ordo \textit{Latin square} hanya pada $n \leq 5$.}
  \end{itemize}
\end{frame}

\subsection{Tujuan}
\begin{frame}
  \frametitle{\insertsection}
  \framesubtitle{\insertsubsection}
  \begin{enumerate}
    \item \onslide<2->{Menentukan syarat perlu bagi dua buah \textit{Latin square} $A,B\in\R^{n\times n}_{\max}$ agar komutatif terhadap operasi $\otimes$.}
    \item \onslide<3->{Menentukan syarat cukup bagi dua buah \textit{Latin square} $A,B\in\R^{n\times n}_{\max}$ agar komutatif terhadap operasi $\otimes$.}
    \item \onslide<4->{Membangun pasangan \textit{Latin square} komutatif dari $L_S(n)$.}
  \end{enumerate}
\end{frame}

\subsection{Manfaat}
\begin{frame}
  \frametitle{\insertsection}
  \framesubtitle{\insertsubsection}
  \begin{itemize}
    \item \onslide<2->{\textbf{Manfaat Teoritis}: Menambah khasanah ilmu pengetahuan khususnya dalam bidang aljabar max-plus dan teori \textit{Latin square}.}
    \item \onslide<3->{\textbf{Manfaat Praktis}: Memberikan kontribusi dalam pengembangan protokol sistem keamanan kunci privat berbasis aljabar tropikal.}
    \item \onslide<4->{\textbf{Referensi Penelitian}: Menjadi bahan referensi untuk penelitian selanjutnya yang berkaitan dengan aljabar max-plus dan \textit{Latin square}.}
  \end{itemize}
\end{frame}

\section{Tinjauan Pustaka}
\subsection{Hasil Penelitian Terdahulu}
\begin{frame}
  \frametitle{\insertsection}
  \framesubtitle{\insertsubsection}
  \begin{table}[H]
    \centering
    \begin{tabular}{|c|m{3cm}|c|m{8cm}|}
      \hline
      \small
      \textbf{No} & \textbf{Peneliti}                                  & \textbf{Tahun}                                   & \textbf{Hasil Penelitian}                                                                                             \\ \hline
      1           & \citeauthor{mufid2014eigenvalues}                  & \citeyear{mufid2014eigenvalues}                  & Menyelesaikan permasalahan nilai eigen dan vektor eigen dari Latin square pada aljabar max-plus.                      \\ \hline
      2           & \citeauthor{linde2014matricescommutinggivennormal} & \citeyear{linde2014matricescommutinggivennormal} & Menyatakan syarat perlu dan cukup bagi dua matriks tropikal agar komutatif berdasarkan nilai diagonal utama.          \\ \hline
      3           & \citeauthor{trivialeigenvectorsym14061101}         & \citeyear{trivialeigenvectorsym14061101}         & Mengkaji sifat-sifat vektor eigen trivial pada aljabar max-plus.                                                      \\ \hline
      4           & \citeauthor{zufar2023konstruksigrup}               & \citeyear{zufar2023konstruksigrup}               & Mengkonstruksi grup Latin square pada aljabar max-plus dengan operasi perkalian matriks tropikal.                     \\ \hline
      5           & \citeauthor{alhussaini2024securityinitialtropical} & \citeyear{alhussaini2024securityinitialtropical} & Memperkenalkan protokol kriptografi berbasis aljabar tropikal yang terinspirasi dari pertukaran kunci Diffie-Hellman. \\ \hline
    \end{tabular}
  \end{table}
\end{frame}

\subsection{Aljabar Max-Plus}
\begin{frame}
  \frametitle{\insertsection}
  \framesubtitle{\insertsubsection}
  \begin{definisi}[\cite{subiono2015minmaxplus}]
    Aljabar max-plus adalah himpunan $\mathbb{R}_{\max} = \mathbb{R} \cup \{\varepsilon\}$ yang dilengkapi dengan dua operasi biner yaitu:
    \begin{align*}
      a \oplus b  & = \max\{a,b\}, \quad \forall a, b \in \mathbb{R}_{\max} \\
      a \otimes b & = a + b, \quad \forall a, b \in \mathbb{R}_{\max}
    \end{align*}
    dengan elemen identitas $\varepsilon=-\infty$ untuk $\oplus$ dan $0$ untuk $\otimes$.
  \end{definisi}
  $(\mathbb{R}_{\max}, \oplus, \otimes)$ disebut semiring max-plus.
\end{frame}

\begin{frame}
  \frametitle{\insertsection}
  \framesubtitle{Matriks Aljabar Max-Plus}
  \begin{definisi}[\cite{Baccelli1994maxplus}]
    Misalkan $A = (a_{ij})$ matriks $m \times n$ dan $B = (b_{ij})$ matriks $n \times p$ dengan elemen dari $\mathbb{R}_{\max}$. Operasi matriks:
    \begin{align*}
      (A \oplus B)_{ij}  & = a_{ij} \oplus b_{ij} = \max\{a_{ij}, b_{ij}\}                                          \\
      (A \otimes B)_{ij} & = \bigoplus_{k=1}^{n} a_{ik} \otimes b_{kj} = \max_{1 \leq k \leq n} \{a_{ik} + b_{kj}\}
    \end{align*}
  \end{definisi}
  Matriks identitas $I_n$ memiliki diagonal $0$ dan elemen lain $\varepsilon$.
\end{frame}

\begin{frame}
  \frametitle{\insertsection}
  \framesubtitle{Contoh Operasi Matriks Max-Plus}
  Misalkan $A = \begin{pmatrix} 2 & 3 \\ 5 & 1 \end{pmatrix}$ dan $B = \begin{pmatrix} 4 & 0 \\ 2 & 6 \end{pmatrix}$
  \begin{align*}
    A \oplus B  & = \begin{pmatrix} \max\{2,4\} & \max\{3,0\} \\ \max\{5,2\} & \max\{1,6\} \end{pmatrix} = \begin{pmatrix} 4 & 3 \\ 5 & 6 \end{pmatrix} \\
    A \otimes B & = \begin{pmatrix} \max\{6,5\} & \max\{2,9\} \\ \max\{9,3\} & \max\{5,7\} \end{pmatrix} = \begin{pmatrix} 6 & 9 \\ 9 & 7 \end{pmatrix}
  \end{align*}
\end{frame}

\begin{frame}
  \frametitle{\insertsection}
  \framesubtitle{Polinomial dalam Aljabar Max-Plus}
  \begin{definisi}[\cite{muanalifah2020modifyingtropical}]
    Polinomial max-plus atau tropikal adalah fungsi $x\mapsto p(x)$:
    \[
      p(x) = \bigoplus_{k=0}^{n} a_{k} \otimes x^{\otimes k} = \max_{0 \leq k \leq n} \{a_{k} + kx\}
    \]
    dengan $a_k \in \mathbb{R}_{\max}$ untuk setiap $k = 0, 1, \ldots, n$.
  \end{definisi}
\end{frame}

\begin{frame}
  \frametitle{\insertsection}
  \framesubtitle{Matriks Linde-de la Puente}
  \begin{definisi}[\cite{alhussaini2024implementationstickel}]
    Untuk $r \leq 0$ dan $k \geq 0$, $[2r, r]_{n}^{k}$ adalah himpunan matriks $A_{n\times n}$ dengan $a_{ii} = k$ untuk semua $i$ dan $a_{ij} \in [2r, r]$ untuk $i \neq j$.
  \end{definisi}
  \begin{teorema}[\cite{alhussaini2024securityinitialtropical}]
    Jika $A \in [2r, r]_{n}^{k_{1}}, \, B \in [2s, s]_{n}^{k_{2}}$ untuk $r, s \leq 0$, $k_1, k_2 \geq 0$, maka
    \[
      A \otimes B = B \otimes A = k_{2} \otimes A \oplus k_{1} \otimes B
    \]
  \end{teorema}
\end{frame}

\begin{frame}
  \frametitle{\insertsection}
  \framesubtitle{Contoh Matriks Linde-de la Puente}
  Misalkan $A\in [-4, -2]_{3}^{3}$ dan $B\in [-3, -1]_{3}^{2}$:
  \[
    A = \begin{pmatrix} 3 & -2 & -4 \\ -3 & 3 & -2 \\ -2 & -4 & 3 \end{pmatrix}, \quad
    B = \begin{pmatrix} 2 & -1 & -3 \\ -3 & 2 & -1 \\ -1 & -3 & 2 \end{pmatrix}
  \]
  Maka $A \otimes B = B \otimes A = \begin{pmatrix} 5 & 2 & 0 \\ 0 & 5 & 2 \\ 2 & 0 & 5 \end{pmatrix}$ (komutatif)
\end{frame}

\subsection{Grup}
\begin{frame}
  \frametitle{\insertsection}
  \framesubtitle{\insertsubsection}
  Grup adalah himpunan $G$ dengan operasi $*$ yang memenuhi:
  \begin{itemize}
    \item \textbf{Tertutup}: $a * b \in G$ untuk setiap $a, b \in G$
    \item \textbf{Asosiatif}: $(a * b) * c = a * (b * c)$
    \item \textbf{Identitas}: $\exists e \in G$ sehingga $e * a = a * e = a$
    \item \textbf{Invers}: Untuk setiap $a \in G$, $\exists a^{-1}$ sehingga $a * a^{-1} = e$
  \end{itemize}
  \textbf{Contoh}: Grup permutasi $S_n$ dengan operasi komposisi fungsi.
\end{frame}

\begin{frame}
  \frametitle{\insertsection}
  \framesubtitle{Homomorfisma dan Isomorfisma Grup}
  \begin{itemize}
    \item \textbf{Homomorfisma}: Pemetaan $f: G \to H$ yang mempertahankan struktur grup, yaitu $f(a * b) = f(a) \cdot f(b)$
    \item \textbf{Isomorfisma}: Homomorfisma yang bijektif (satu-satu dan pada)
    \item Dua grup $G$ dan $H$ disebut \textbf{isomorfik} ($G\cong H$) jika terdapat isomorfisma antara keduanya
    \item Isomorfik menunjukkan kedua grup memiliki struktur yang sama
  \end{itemize}
\end{frame}

\subsection{Permutasi}
\begin{frame}
  \frametitle{\insertsection}
  \framesubtitle{\insertsubsection}
  Permutasi adalah fungsi bijektif dari himpunan ke dirinya sendiri:
  \begin{itemize}
    \item $\sigma: X \to X$ dengan $X = \{1, 2, \ldots, n\}$, bersifat satu-satu dan pada
    \item Himpunan semua permutasi dari $X$ membentuk grup simetris $S_n$ terhadap komposisi fungsi
  \end{itemize}
  \textbf{Contoh}: Untuk $X = \{1, 2, 3\}$, permutasi $\sigma(1) = 2, \sigma(2) = 3, \sigma(3) = 1$ dapat dituliskan:
  \[
    \sigma = \begin{pmatrix} 1 & 2 & 3 \\ 2 & 3 & 1 \end{pmatrix}
  \]
\end{frame}

\begin{frame}
  \frametitle{\insertsection}
  \framesubtitle{Sikel}
  \begin{definisi}
    Orbit $\mathcal{O}_{\tau}(i_1) = \{\, i_1,\ \tau(i_1),\ \tau^{2}(i_1),\ \ldots,\ \tau^{k-1}(i_1)\,\}$ yang memuat lebih dari satu elemen disebut \textbf{sikel}, dituliskan $(i_1\ i_2\ \ldots\ i_k)$.
  \end{definisi}
  \begin{teorema}[\cite{subiono2022aljabar}]
    Setiap permutasi $\tau\in S_n$ dapat dituliskan sebagai hasil komposisi dari sikel-sikel yang saling asing.
  \end{teorema}
\end{frame}

\begin{frame}
  \frametitle{\insertsection}
  \framesubtitle{Contoh Notasi Sikel}
  Untuk $\tau = \begin{pmatrix} 1 & 2 & 3 & 4 & 5 & 6 & 7 & 8 & 9 \\ 3 & 1 & 2 & 5 & 4 & 6 & 8 & 7 & 9 \end{pmatrix} \in S_9$

  Dapat dituliskan: $\tau = (1\ 3\ 2)(4\ 5)(6)(7\ 8)(9)$
  \begin{columns}
    \begin{column}{0.5\textwidth}
      \begin{itemize}
        \item Sikel $(1\ 3\ 2)$: $1 \to 3 \to 2 \to 1$
        \item Sikel $(4\ 5)$: $4 \to 5 \to 4$
        \item Titik tetap: $(6)$ dan $(9)$
      \end{itemize}
    \end{column}
    \begin{column}{0.5\textwidth}
      \centering
      \begin{tikzpicture}[scale=0.5]
        \node[circle, draw, fill=blue!20] (1) at (0, 0) {1};
        \node[circle, draw, fill=blue!20] (2) at (2.5, 0) {2};
        \node[circle, draw, fill=blue!20] (3) at (1.25, 2.165) {3};
        \node[circle, draw, fill=green!20] (4) at (4.5, 0) {4};
        \node[circle, draw, fill=green!20] (5) at (6.5, 0) {5};
        \node[circle, draw, fill=red!20] (6) at (8.5, 0) {6};
        \node[circle, draw, fill=orange!20] (7) at (4.5, 2.165) {7};
        \node[circle, draw, fill=orange!20] (8) at (6.5, 2.165) {8};
        \node[circle, draw, fill=red!20] (9) at (8.5, 2.165) {9};

        \draw[->, thick, blue] (1) to[bend left=20] (3);
        \draw[->, thick, blue] (3) to[bend left=20] (2);
        \draw[->, thick, blue] (2) to[bend left=20] (1);

        \draw[->, thick, green] (4) to[bend left=30] (5);
        \draw[->, thick, green] (5) to[bend left=30] (4);

        \path[red] (6) edge[->, loop right, thick] ();

        \draw[->, thick, orange] (7) to[bend left=30] (8);
        \draw[->, thick, orange] (8) to[bend left=30] (7);

        \path[red] (9) edge[->, loop right, thick] ();
      \end{tikzpicture}
    \end{column}
  \end{columns}
\end{frame}

\begin{frame}
  \frametitle{\insertsection}
  \framesubtitle{Matriks Permutasi}
  \begin{definisi}
    Untuk permutasi $\sigma : \{1,2,\dots,n\} \to \{1,2,\dots,n\}$, matriks permutasi max-plus $P_{\sigma} = [p_{ij}]$ didefinisikan:
    \[
      p_{ij} = \begin{cases} 0, & \text{jika } i = \sigma(j) \\ \varepsilon, & \text{lainnya} \end{cases}
    \]
  \end{definisi}
  \textbf{Contoh}: Untuk $\sigma=(1\ 3\ 2)\in S_3$, maka
  \[
    P_{\sigma} = \begin{pmatrix} \varepsilon & \varepsilon & 0 \\ 0 & \varepsilon & \varepsilon \\ \varepsilon & 0 & \varepsilon \end{pmatrix}
  \]
\end{frame}

\begin{frame}
  \frametitle{\insertsection}
  \framesubtitle{Teorema Matriks Permutasi}
  \begin{teorema}[\cite{Farlow2009}]
    Matriks $A \in \mathbb{R}_{\max}^{n \times n}$ memiliki invers kanan jika dan hanya jika terdapat suatu permutasi $\sigma$ serta nilai-nilai $\lambda_i > \varepsilon$, $i \in \{1,2,\dots,n\}$, sehingga
    \[
      A = P_{\sigma} \otimes D(\lambda)
    \]
    dengan $D(\lambda)$ adalah matriks diagonal max-plus dengan entri diagonal $\lambda_1, \lambda_2, \ldots, \lambda_n$.
  \end{teorema}
\end{frame}

\begin{frame}
  \frametitle{\insertsection}
  \framesubtitle{Matriks Sirkulan}
  Matriks sirkulan adalah matriks persegi di mana setiap baris adalah pergeseran siklik dari baris sebelumnya.

  Representasi polinomial:
  \[
    C = \bigoplus_{i=0}^{n-1} c_i P_{\sigma}^{i}
  \]
  dengan $\sigma = (1\ 2\ \ldots\ n)$ siklik.
  \begin{teorema}[\cite{huang2024newsemiring}]
    Dua matriks sirkulan $A, B$ komutatif: $A \otimes B = B \otimes A$.
  \end{teorema}


\end{frame}

\begin{frame}
  \frametitle{\insertsection}
  \framesubtitle{Matriks Sirkulan-$t$}
  \begin{definisi}[\cite{buchinskiy2024analysisfour}]
    Matriks sirkulan-$t$ adalah variasi matriks sirkulan dengan parameter $t$, yang dapat direpresentasikan:
    \begin{align*}
      C^{(t)} & = c_0 I_n \oplus c_1 (P^{(t)}) \oplus \cdots \oplus c_{n-1} (P^{(t)})^{n-1} \\
      C_{(t)} & = c_0 I_n \oplus c_1 (P_{(t)}) \oplus \cdots \oplus c_{n-1} (P_{(t)})^{n-1}
    \end{align*}
  \end{definisi}
  \begin{teorema}[\cite{buchinskiy2024analysisfour}]
    Dua matriks sirkulan-$t$ (bawah atau atas keduanya) komutatif terhadap $\otimes$.
  \end{teorema}
\end{frame}

\subsection{Derangement}
\begin{frame}
  \frametitle{\insertsection}
  \framesubtitle{\insertsubsection}
  \begin{itemize}
    \item \textbf{Titik tetap}: elemen $i$ yang memenuhi $\sigma(i)=i$
    \item \textbf{Derangement}: permutasi tanpa titik tetap (setiap objek tidak pada posisi aslinya)
    \item Dinotasikan $D_n$ untuk banyaknya derangement dari $n$ objek berbeda
  \end{itemize}
  \textbf{Contoh}: Dalam $S_3$, permutasi $(1), (1\ 2)(3), (1)(2\ 3), (1\ 3)(2)$ memiliki titik tetap, sedangkan $(1\ 2\ 3), (1\ 3\ 2)$ adalah derangement.
\end{frame}

\begin{frame}
  \frametitle{\insertsection}
  \framesubtitle{Contoh Derangement $S_4$}
  \begin{figure}[H]
    \centering
    \permrow{1}{2}{3}{4}\quad\permrow{2}{1}{3}{4}\quad\permrow{3}{1}{2}{4}\quad\permrow{4}{1}{2}{3}\\
    \permrow{1}{2}{4}{3}\quad\permrow{2}{1}{4}{3}\quad\permrow{3}{1}{4}{2}\quad\permrow{4}{1}{3}{2}\\
    \permrow{1}{3}{2}{4}\quad\permrow{2}{3}{1}{4}\quad\permrow{3}{2}{1}{4}\quad\permrow{4}{2}{1}{3}\\
    \permrow{1}{3}{4}{2}\quad\permrow{2}{3}{4}{1}\quad\permrow{3}{2}{4}{1}\quad\permrow{4}{2}{3}{1}\\
    \permrow{1}{4}{2}{3}\quad\permrow{2}{4}{1}{3}\quad\permrow{3}{4}{1}{2}\quad\permrow{4}{3}{1}{2}\\
    \permrow{1}{2}{4}{3}\quad\permrow{2}{4}{3}{1}\quad\permrow{3}{4}{2}{1}\quad\permrow{4}{3}{2}{1}\\
    \caption{\emph{derangement} dari himpunan $S_4$}
    \label{fig:derangement_s4}
  \end{figure}
\end{frame}

\begin{frame}
  \frametitle{\insertsection}
  \framesubtitle{Representasi Latin Square}
  \begin{teorema}
    Misalkan $A \in L_S(n)$ dengan simbol $\{a_1, a_2, \ldots, a_n\}$. Untuk setiap $i$, definisikan $\tau_i \in S_n$ dengan $\tau_i(r) = c \iff [A]_{r c} = a_i$. Maka
    \[
      A = \bigoplus_{i=1}^{n} a_i P_{\tau_i}
    \]
    dengan $P_{\tau_i}$ adalah matriks permutasi max-plus yang bersesuaian dengan $\tau_i$.
  \end{teorema}
\end{frame}

\section{Metodologi}
\subsection{Alur Penelitian}
\begin{frame}
  \frametitle{\insertsection}
  \framesubtitle{\insertsubsection}
  \begin{center}
    \begin{tikzpicture}[node distance=1.5cm and 0.8cm, font=\small, scale=0.95, transform shape]
      % Nodes
      \node[startstop] (start) {Mulai};
      \node[process, right=of start] (studi) {Studi literatur};
      \node[process, right=of studi] (identifikasi) {\parbox{2.2cm}{\centering Identifikasi \emph{Latin square} komutatif}};
      \node[process, below=of identifikasi] (investigasi) {\parbox{2.2cm}{\centering Investigasi sifat \emph{Latin square} komutatif}};
      \node[process, left=of investigasi] (kesimpulan) {\parbox{2cm}{\centering Penarikan kesimpulan}};
      \node[process, left=of kesimpulan] (penulisan) {\parbox{2cm}{\centering Penulisan tugas akhir}};
      \node[startstop, below=of penulisan] (end) {Selesai};

      % Arrows
      \draw[arrow] (start) -- (studi);
      \draw[arrow] (studi) -- (identifikasi);
      \draw[arrow] (identifikasi) -- (investigasi);
      \draw[arrow] (investigasi) -- (kesimpulan);
      \draw[arrow] (kesimpulan) -- (penulisan);
      \draw[arrow] (penulisan) -- (end);
    \end{tikzpicture}
  \end{center}
\end{frame}

\subsection{Tahapan Penelitian}
\begin{frame}
  \frametitle{\insertsection}
  \framesubtitle{\insertsubsection}
  \begin{enumerate}
    \item \textbf{Studi Literatur dan Perumusan Masalah}: Mempelajari literatur yang berkaitan dengan aljabar max-plus dan \textit{Latin square}, memahami konsep dasar dan teori-teori yang relevan.
    \item \textbf{Perancangan Skema Normalisasi dan Model Operasi}: Merancang skema normalisasi dan model operasi untuk \textit{Latin square} komutatif.
    \item \textbf{Pengembangan Algoritme Konstruksi dan Verifikasi}: Mengembangkan algoritme untuk konstruksi dan verifikasi \textit{Latin square} komutatif.
    \item \textbf{Eksperimen dan Enumerasi Orde Kecil-Menengah}: Melakukan eksperimen dan enumerasi terhadap \textit{Latin square} berordo kecil hingga menengah.
    \item \textbf{Analisis Hasil dan Karakterisasi Aljabar}: Menganalisis hasil investigasi dan karakterisasi aljabar dari \textit{Latin square} komutatif.
    \item \textbf{Penyusunan Kesimpulan dan Saran}: Menyimpulkan hasil investigasi syarat perlu, syarat cukup, dan konstruksi \textit{Latin square} komutatif.
    \item \textbf{Penulisan Laporan Tugas Akhir}: Menyusun laporan tugas akhir dan dokumentasi lengkap dari seluruh proses penelitian.
  \end{enumerate}
\end{frame}
% \begin{frame}
%     \begin{tikzpicture}[node distance=3cm, font=\scriptsize]

%         % Nodes
%         \node (start) [startstop] {Mulai};
%         \node (data) [process, right of=start] {\parbox{1.5cm}{Identifikasi Data Awal}};
%         \node (formulasi) [process, right of=data] {\parbox{2.5cm}{Formulasi Sistem dalam Aljabar Max-Plus}};
%         \node (dinamika) [process, right of=formulasi] {\parbox{1.5cm}{Modeling Dinamika Jadwal}};
%         \node (validasi) [process, right of=dinamika] {\parbox{1.5cm}{Analisis Validitas Model}};
%         \node (optimasi) [process, below of=validasi] {Optimasi Jadwal};
%         \node (evaluasi) [decision, left of=optimasi] {\parbox{1.5cm}{Evaluasi dan Validasi}};
%         \node (implementasi) [process, left of=evaluasi] {\parbox{2cm}{Implementasi dan Simulasi}};
%         \node (kesimpulan) [process, left of=implementasi] {\parbox{2cm}{Kesimpulan dan Dokumentasi}};
%         \node (end) [startstop, left of=kesimpulan] {Selesai};

%         % Arrows
%         \draw [arrow] (start) -- (data);
%         \draw [arrow] (data) -- (formulasi);
%         \draw [arrow] (formulasi) -- (dinamika);
%         \draw [arrow] (dinamika) -- (validasi);
%         \draw [arrow] (validasi) -- (optimasi);
%         \draw [arrow] (optimasi) -- (evaluasi);
%         \draw [arrow,red] (evaluasi) -- node[midway,above left]{no} ++(0,1.5) -| (formulasi);
%         \draw [arrow,green] (evaluasi) --node[midway,above]{yes} (implementasi);
%         \draw [arrow] (implementasi) -- (kesimpulan);
%         \draw [arrow] (kesimpulan) -- (end);
%     \end{tikzpicture}
% \end{frame}

% \begin{frame}
%     \begin{table}
%         \caption{Jadwal Kegiatan}
%         \centering
%         \begin{tabular}{|C{0.6cm}|L{5.7cm}|C{0.25cm}|C{0.25cm}|C{0.25cm}|C{0.25cm}|C{0.25cm}|C{0.25cm}|C{0.25cm}|C{0.25cm}|C{0.25cm}|C{0.25cm}|C{0.25cm}|C{0.25cm}|}
%         \hline
%         &&\multicolumn{12}{c|}{\textbf{BULAN}}\\\cline{3-14}
%         \multicolumn{1}{|c|}{\textbf{NO}}&\multicolumn{1}{c|}{\textbf{NAMA KEGIATAN}}&\multicolumn{4}{c|}{1}&\multicolumn{4}{c|}{2}&\multicolumn{4}{c|}{3}\\\cline{3-14}
%         &&1&2&3&4&1&2&3&4&1&2&3&4\\\cline{1-14}

%         1&Identifikasi Data Awal&\cellcolor{HIMAtua}&\cellcolor{HIMAtua}&&&&&&&&&&\\\hline
%         2&Formulasi Sistem dalam Aljabar Max-Plus&&&\cellcolor{HIMAtua}&\cellcolor{HIMAtua}&\cellcolor{HIMAtua}&&&&&&&\\\hline
%         3&Modeling Dinamika Jadwal&&&&&\cellcolor{HIMAtua}&\cellcolor{HIMAtua}&\cellcolor{HIMAtua}&&&&&\\\hline
%         4&Analisis Validitas Model&&&&&&&&\cellcolor{HIMAtua}&\cellcolor{HIMAtua}&\cellcolor{HIMAtua}&&\\\hline
%         5&Optimasi Jadwal&&&&&&&&\cellcolor{HIMAtua}&\cellcolor{HIMAtua}&\cellcolor{HIMAtua}&&\\\hline
%         6&Implementasi dan Simulasi&&&&&&&&&\cellcolor{HIMAtua}&\cellcolor{HIMAtua}&\cellcolor{HIMAtua}&\\\hline
%         7&Kesimpulan dan Dokumentasi&&&&&&&&&\cellcolor{HIMAtua}&\cellcolor{HIMAtua}&\cellcolor{HIMAtua}&\cellcolor{HIMAtua}\\\hline

%         \end{tabular}
%         \label{TabelJadwalKegiatan}
%     \end{table}
% \end{frame}

\section{Referensi}
\begin{frame}[allowframebreaks]
  \printbibliography
\end{frame}
\end{document}