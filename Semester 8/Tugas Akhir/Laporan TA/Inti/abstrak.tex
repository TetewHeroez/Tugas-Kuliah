Aljabar max-plus, yang dinotasikan dengan $(\mathbb{R}_{\max}, \oplus, \otimes)$ dengan $a \oplus b = \max\{a,b\}$ dan $a \otimes b = a + b$, merupakan salah satu struktur aljabar idempoten yang banyak digunakan dalam pemodelan sistem diskrit serta mulai
dikaji dalam pengembangan skema kriptografi non-konvensional. Dalam konteks ini,
keberadaan pasangan matriks $A,B \in \mathbb{R}_{\max}^{n \times n}$ yang memenuhi
$A \otimes B = B \otimes A$ menjadi aspek penting. Salah satu kelas matriks yang
menarik untuk dikaji adalah matriks \textit{Latin square}, karena memiliki struktur
kombinatorial yang teratur dan keterkaitan erat dengan teori permutasi.

Penelitian ini bertujuan untuk mengkaji sifat kekomutatifan matriks Latin square pada
aljabar max-plus, khususnya kondisi yang menyebabkan dua matriks $A,B \in L_S(n)$
bersifat komutatif melalui representasi kombinasi matriks permutasi $\{P_{\sigma}\}_{\sigma
  \in S_n}$. Dengan memanfaatkan konsep matriks sirkulan dan karakteristik pola diagonal,
penelitian ini diharapkan menghasilkan kriteria komutatif yang jelas serta prosedur
konstruktif untuk membangun pasangan Latin square komutatif. Hasil kajian ini
diharapkan berkontribusi pada pengembangan teori Latin square dalam aljabar tropikal
dan menjadi landasan bagi skema kriptografi berbasis aljabar max-plus.

\katakunci{Aljabar Max-Plus, Latin Square, Matriks Komutatif, Matriks Sirkulan, Permutasi, Semiring Tropikal}
