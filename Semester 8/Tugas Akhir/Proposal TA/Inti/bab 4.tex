\pagebreak
\chapter{Hasil dan Pembahasan}

\section{Analisis Model Matematika Rudal}
Pada penelitian tugas akhir ini, jenis rudal yang digunakan adalah rudal balistik RX-200 TC (BRIN). Model matematika yang digunakan terdiri dari model kinematika rudal dengan 3 derajat kebebasan dan model dinamika rudal dengan 3 derajat kebebasan. Nilai parameter rudal balistik RX-200 TC (BRIN) terdapat pada \ref{tab: parameter rudal}.

% \begin{table}[htbp]
%     \setlength{\tabcolsep}{10pt}
%     \renewcommand{\arraystretch}{1.8}
%     \centering
%     \caption{Nilai Parameter pada Rudal Balistik RX-200 TC (BRIN)}
%     \begin{tabular}{|c| c| c| c| c| c|c| c| c|}
%     \hline
%         \textbf{No} & \textbf{Notasi} & \textbf{Nilai} & \textbf{No} & \textbf{Notasi} & \textbf{Nilai} & \textbf{No} & \textbf{Notasi} & \textbf{Nilai} \\ \hline 
%         1 & $\alpha$ &  & 17 & $C_X$ &  & 33 & $M$ &  \\ \hline
%         2 & $\beta$ &  & 18 & $C_{X_{\dot{\alpha}}}$ &  & 34 & $\dot{M}$ &  \\ \hline
%         3 & $C_{l_{\beta}}$ &  & 19 & $C_{X_{q}}$ &  & 35 & $g$ &  \\ \hline
%         4 & $C_{l_{p}}$ &  & 20 & $C_{X_{\delta_a}}$ &  & 36 & $\hat{q}$ &  \\ \hline
%         5 & $C_{l_{r}}$ &  & 21 & $C_{X_{\delta_e}}$ &  & 37 & $c$ &  \\ \hline
%         6 & $C_{l_{\delta_a}}$ &  & 22 & $C_{X_{\delta_r}}$ &  & 38 & $S$ &  \\ \hline
%         7 & $C_{l_{\delta_r}}$ &  & 23 & $C_{Y_{\beta}}$ &  & 39 & $d$ &  \\ \hline
%         8 & $C_{m}$ &  & 24 & $C_{Y_{p}}$ &  & 40 & $I_x$ &  \\ \hline
%         9 & $C_{m_{{\dot{\alpha}}}}$ &  & 25 & $C_{Y_{r}}$ &  & 41 & $I_y$ &  \\ \hline
%         10 & $C_{m_{q}}$ &  & 26 & $C_{Y_{\delta_r}}$ &  & 42 & $I_z$ &  \\ \hline
%         11 & $C_{m_{\delta_e}}$ &  & 27 & $C_{Y_{\delta_a}}$ &  & 43 & $\dot{I_x}$ &  \\ \hline
%         12 & $C_{n_{\beta}}$ &  & 28 & $C_{Z}$ &  & 44 & $\dot{I_y}$ &  \\ \hline
%         13 & $C_{n_{p}}$ &  & 29 & $C_{Z_{\dot{\alpha}}}$ &  & 45 & $\dot{I_z}$ &  \\ \hline 
%         14 & $C_{n_r}$ &  & 30 & $C_{Z_{q}}$ &  & 46 & $m$ &  \\ \hline 
%         15 & $C_{n_{\delta_a}}$ &  & 31 & $C_{Z_{\delta_e}}$ &  & 47 & $x_e$ & \\ \hline 
%         16 & $C_{n_{\delta_r}}$ &  & 32 & $T$ &  &  &  &  \\ \hline
%     \end{tabular}
%     \label{tab: parameter rudal}
% \end{table}

\begin{table}[htbp]
 \setlength{\tabcolsep}{10pt}
 %\renewcommand{\arraystretch}{1.5}
    \centering
    \caption{Nilai Parameter Rudal Balistik RX-200 TC (BRIN)}
    \begin{tabular}{|c|l|c|}
    \hline 
        \textbf{Notasi} & \multicolumn{1}{c|}{\textbf{Parameter}}  & \textbf{Nilai} \\
        \hline
        $m$ & Massa roket & 197.03 $kg$\\
        \hline
        $\dot{m}$ & Perubahan massa roket & 8.05 $kg/s$ \\
        \hline
        $T$ & Gaya dorong roket & \dots  \\
        \hline
        $g$ & gravitasi bumi & 9.806 $m/s^2$ \\
        \hline
        $\hat{q}$ & Tekanan dinamik & 63460 $N/m^2$ \\
        \hline
        $c$ & Diameter roket & 0.203 $m$ \\
        \hline
        $S$ & Luas penampang badan roket & 0.0324 $m^2$ \\
        \hline
        $\dot{\alpha}$ & \textit{Rate of change of Angle of attack} & -0.00407 \\
        \hline
        % $\beta$ & \textit{Side slip angle}  &  \\
        % \hline
        $x_e$ & pusat aliran massa & 1247 \\
        \hline 
        $d$ & Perubahan titik pusat massa & \dots \\
        \hline
        $I_x$ & Momen Inersia $x$ & 1.252 $kg\,m^2$ \\
        \hline
        $I_y$ & Momen Inersia $y$ & 188.470 $kg\,m^2$ \\
        \hline
        $I_z$ & Momen Inersia $z$ & 188.472  $kg\,m^2$ \\
        \hline
        $\dot{I_x}$ & Perubahan Momen Inersia $x$ & 0.036  $kg\,m^2/s$ \\
        \hline
        $\dot{I_y}$ & Perubahan Momen Inersia $y$ & 4.612  $kg\,m^2/s$ \\
        \hline
        $\dot{I_z}$ & Perubahan Momen Inersia $z$ & 4.612 $kg\,m^2/s$ \\
        \hline
    \end{tabular}
    \label{tab: parameter rudal}
\end{table}


\subsection{Model Matematika Kinematika Rudal}

Model matematika kinematika rudal 3 derajat kebabasan terdiri dari kinematika translasi dan kinematika rotasi. Pada model kinematika translasi terdapat tiga macam gerakan, yaitu gerakan pada sumbu-$x$, sumbu-$y$, dan sumbu-$z$. Persamaan kinematika translasi dapat dituliskan sebagai berikut
\begin{align}
    \begin{bmatrix}
        \dot{x} \\
        \dot{y} \\
        \dot{z}
    \end{bmatrix} = \vb{K}^{-1}  
    \begin{bmatrix}
        u \\ v \\ w
    \end{bmatrix}
\end{align}
dengan
\begin{align*}
    \vb{K}^{-1} = \begin{bmatrix}
        \cos\theta\cos\psi & \sin\phi\sin\theta\cos\psi-\cos\phi\sin\psi & \cos\phi\sin\theta\cos\psi+\sin\phi\sin\psi \\
        \cos\theta\sin\psi & \sin\phi\sin\theta\sin\psi+\cos\phi\cos\psi & \cos\phi\sin\theta\sin\psi-\sin\phi\cos\psi \\
        -\sin\theta & \sin\phi\cos\theta & \cos\phi\cos\theta 
    \end{bmatrix}
\end{align*}
kemudian untuk kinematika rotasi diperoleh dari kecepatan \textit{roll}, \textit{pitch}, dan \textit{yaw}, sehingga didapat persamaan kinematika rotasi 
\begin{align}
    \begin{bmatrix}
        \dot{\phi} \\
        \dot{\theta} \\
        \dot{\psi}
    \end{bmatrix} = \begin{bmatrix}
        1 & \sin\phi\tan\theta & \cos\phi\tan\theta \\
        0 & \cos\phi & -\sin\phi \\
        0 & \sin\phi\sec\theta & \cos\phi\sec\theta
    \end{bmatrix} \begin{bmatrix}
        p \\ q \\ r
    \end{bmatrix}
\end{align}

\subsection{Model Matematika Dinamika Rudal}

Persamaan dinamika translasi
\begin{align}
    \dot{u}&=vr-wq+\frac{1}{m}\Bigl(C_X+\frac{c}{2U}C_{X_{\dot{\alpha}}}\dot{\alpha}+\frac{c}{2U}C_{X_{q}}q+C_{X_{\delta_a}}\delta_a+C_{X_{\delta_e}}\delta_e+C_{X_{\delta_r}}\delta_r\Bigr)\hat{q}S \notag\\ 
    & \;\;\;\; -\frac{\dot{m}u}{m}-g\sin\theta+\frac{T}{m} \\  
    \dot{v}&=wp-ur+\frac{1}{m}\Bigl(C_{Y_\beta}+\frac{c}{2U}C_{Y_p}p+\frac{c}{2U}C_{Y_r}r+C_{Y_{\delta_r}}\delta_r+C_{Y_{\delta_a}}\delta_a\Bigr)\hat{q}S \notag\\
    & \;\;\;\; -\frac{\dot{m}v}{m}+g\cos\theta \sin\phi \\
    \dot{w}&=uq-vp+\frac{1}{m}\Bigl(C_Z+\frac{c}{2U}C_{Z_{\dot{\alpha}}}\dot{\alpha}+\frac{c}{2U}C_{Z_{q}}q+C_{Z_{\delta_e}}\delta_e\Bigr)\hat{q}S \notag \\
    &\;\;\;\; -\frac{\dot{m}w}{m}+g\cos\theta \cos\phi \label{dinamika translasi}
\end{align}

Persamaan dinamika rotasi
\begin{align}
    \Dot{p} &= \frac{1}{I_x}\Bigl[\bigl(C_{l_{\beta}}\beta+\frac{c}{2U}C_{l_p}p+\frac{c}{2U}C_{l_r}r+C_{l_{\delta_a}}\delta_a+C_{l_{\delta_r}}\delta_r\bigr)\hat{q}cS \notag \\
    &\quad -\Dot{I}_xp+qr(I_y-I_z)\Bigr]  \\
    \Dot{q} &= \frac{1}{I_y}\Bigl[\bigl(C_m+\frac{c}{2U}C_{m_{\Dot{\alpha}}}\Dot{\alpha}+\frac{c}{2U}C_{m_q}q+C_{m_{\delta_e}}\delta_e\bigr)\hat{q}cS \notag \\ 
    &\quad -\Dot{I}_yq-pr(I_x-I_z)-\Dot{m}qx_e^2 \notag \\
    &\quad + \bigl(C_Z+\frac{c}{2U}C_{Z_{\alpha}}\alpha+\frac{c}{2U}C_{Z_q}q+C_{Z_{\delta_e}}\delta_e\bigr)\hat{q}S d \Bigr]  \\
    \Dot{r} &= \frac{1}{I_z}\Bigl[\bigl(C_{n_{\beta}}\beta+\frac{c}{2U}C_{n_p}p+\frac{c}{2U}C_{n_r}r+C_{n_{\delta_a}}\delta_a+C_{n_{\delta_r}}\delta_r\bigr)\hat{q}cS \notag \\ 
    &\quad -\Dot{I}_zr+pq(I_x-I_y)-\Dot{m}rx_e^2 \notag \\ 
    &\quad -\bigl(C_{Y_{\beta}}\beta + \frac{c}{2U}C_{Y_p}p+\frac{c}{2U}C_{Y_r}r+C_{Y_{\delta_r}}\delta_r+C_{Y_{\delta_a}}\delta_a\bigr)\hat{q}Sd\Bigr]
\end{align}

Sehingga berdasarkan model kinematika rudal (\ref{model kinematika}) dan model dinamika rudal (\ref{model dinamika}), didapatkan model matematika rudal dalam bentuk kontinu  
\begin{align}
    \dot{\vb{x}}(t) = \vb{f}(\vb{x}(t),\vb{u}(t)) \label{model kontinu}
\end{align}
dengan 12 variabel \textit{state}, $\vb{x}=[x\;\;y\;\;z\;\;\phi\;\;\theta\;\;\psi\;\;u\;\;v\;\;w\;\;p\;\;q\;\;r]^T$
dan 3 variabel \textit{input}, $\vb{u}=[\delta_a\;\;\delta_e\;\;\delta_r]^T$.

Nilai koefisien pada persamaan dinamika translasi dan dinamika rotasi dapat dilihat pada \ref{tab: nilai koefisien}. 
\begin{table}[h]
 \setlength{\tabcolsep}{10pt}
 \renewcommand{\arraystretch}{1.2}
    \centering
    \caption{Nilai Koefisien Rudal}
    \begin{tabular}{|c|c|c|c|}
    \hline 
        \textbf{Koefisien} & \textbf{Nilai} & \textbf{Koefisien} & \textbf{Nilai} \\
        \hline
        $C_X$ & $-0.5694$ & $C_{l_{\beta}}$ & 0.03056 \\
        \hline
        $C_{X_{\dot{\alpha}}}$ & $\hdots$ & $C_{l_{p}}$ & $-0.53$ \\
        \hline
        $C_{X_{q}}$ & 0 & $C_{l_{r}}$ & 0.002545\\
        \hline
        $C_{X_{\delta_a}}$ & $\hdots$ & $C_{l_{\delta_a}}$ & $\hdots$ \\
        \hline
        $C_{X_{\delta_e}}$ & 0.5694 & $C_{l_{\delta_r}}$ & $\hdots$ \\
        \hline
        $C_{X_{\delta_r}}$ & $\hdots$ & $C_{m}$ & 0.0004975\\
        \hline 
        $C_{Y_{\beta}}$ & $-0.3248$ & $C_{m_{{\dot{\alpha}}}}$ & $-0.8239$\\
        \hline
        $C_{Y_{p}}$ & 0.0001727 & $C_{m_{q}}$ & $-49.21$\\
        \hline
        $C_{Y_{r}}$ & 5.12 & $C_{m_{\delta_e}}$ & $-0.005589$\\
        \hline
        $C_{Y_{\delta_r}}$ & $\hdots$ & $C_{n_{\beta}}$ & 0.8444\\
        \hline
        $C_{Y_{\delta_a}}$ & $\hdots$ & $C_{n_{p}}$ & $-0.001315$\\
        \hline
        $C_{Z}$ & $-0.001538$ & $C_{n_r}$ & $-48.88$\\
        \hline
        $C_{Z_{\dot{\alpha}}}$ & 1.908 & $C_{n_{\delta_a}}$ & $\hdots$ \\
        \hline
        $C_{Z_{q}}$ & 5.164 & $C_{n_{\delta_r}}$ & $\hdots$\\
        \hline
        $C_{Z_{\delta_e}}$ & 0.002179 & & \\
        \hline
    \end{tabular}
    \label{tab: nilai koefisien}
\end{table}

\section{Pendiskritan Model Matematika}
Model matematika rudal yang diperoleh dari model kinematika pada persamaan (\ref{model kinematika}) dan model dinamika pada persamaan (\ref{model dinamika}) merupakan model kontinu. Oleh karena itu, perlu dilakukan diskritisasi pada model tersebut untuk mendapat model matematika rudal dalam bentuk diskrit. Proses diskritisasi dilakukan menggunakan metode beda maju sebagai berikut
\begin{align}
    \dot{\pmb{\mathcal{X}}} = \frac{\pmb{\mathcal{X}}(k+1)-\pmb{\mathcal{X}}(k)}{\Delta t} \label{beda maju}
\end{align}
dengan $\pmb{\mathcal{X}}=[x\;\;y\;\;z\;\;\phi\;\;\theta\;\;\psi\;\;u\;\;v\;\;w\;\;p\;\;q\;\;r]^T$ dan $\Delta t$ merupakan \textit{time step} . Selanjutnya dari persamaan (\ref{model kontinu}) dan (\ref{beda maju}) dapat diperoleh model matematika dalam bentuk diskrit yang ditulis
\begin{align}
    \pmb{\mathcal{X}}(k+1) = \pmb{\mathcal{X}}(k)+\pmb{f}\bigl(\pmb{\mathcal{X}}(k),\pmb{u}(k)\bigr)\Delta t
\end{align}
% atau dapat juga ditulis sebagai
% \begin{align}
%     \pmb{\mathcal{X}}(k+1) = \pmb{f_d}\bigl(\pmb{\mathcal{X}}(k),\pmb{u}(k)\bigr)
% \end{align}
% dengan $\pmb{f_d}(\pmb{\mathcal{X}}(k),\pmb{u}(k))=\pmb{\mathcal{X}}(k)+\pmb{f}(\pmb{\mathcal{X}}(k),\pmb{u}(k))\Delta t$.

Hasil diskritisasi menggunakan metode beda maju pada model kinematika translasi sebagai berikut

\begin{align}
    \begin{bmatrix}
        x(k+1) \\
        y(k+1) \\
        z(k+1)
    \end{bmatrix} = \begin{bmatrix}
        x(k) \\
        y(k) \\
        z(k)
    \end{bmatrix} + \Delta t \cdot \vb{K}^{-1}  
    \begin{bmatrix}
        u(k) \\ v(k) \\ w(k)
    \end{bmatrix}
\end{align}
\begin{align*}
\vb{K}^{-1} = \begin{bmatrix}
        \cos\theta\cos\psi & \sin\phi\sin\theta\cos\psi-\cos\phi\sin\psi & \cos\phi\sin\theta\cos\psi+\sin\phi\sin\psi \\
        \cos\theta\sin\psi & \sin\phi\sin\theta\sin\psi+\cos\phi\cos\psi & \cos\phi\sin\theta\sin\psi-\sin\phi\cos\psi \\
        -\sin\theta & \sin\phi\cos\theta & \cos\phi\cos\theta 
    \end{bmatrix}
\end{align*}

sedangkan untuk kinematika rotasi diperoleh

\begin{align}
    \begin{bmatrix}
        \phi(k+1) \\
        \theta(k+1) \\
        \psi(k+1)
    \end{bmatrix} = \begin{bmatrix}
        \phi(k) \\
        \theta(k) \\
        \psi(k)
    \end{bmatrix} + \Delta t \begin{bmatrix}
        1 & \sin\phi\tan\theta & \cos\phi\tan\theta \\
        0 & \cos\phi & -\sin\phi \\
        0 & \sin\phi\sec\theta & \cos\phi\sec\theta
    \end{bmatrix} \begin{bmatrix}
        p(k) \\ q(k) \\ r(k)
    \end{bmatrix}
\end{align}

selanjutnya model dinamika translasi didapat

\begin{align}
    u(k+1) &= u(k) + \Delta t \Bigl[v(k)r(k)-w(k)q(k) \notag \\ 
    &\quad +\frac{1}{m}\bigl(C_X+\frac{c}{2U}C_{X_{\alpha}}\alpha(k)+\frac{c}{2U}C_{X_{q}}q(k)+C_{X_{\delta_a}}\delta_a(k) \notag \\
    &\quad +C_{X_{\delta_e}}\delta_e(k)+C_{X_{\delta_r}}\delta_r(k) \bigr)\hat{q}S -\frac{\Dot{m}u(k)}{m}-g\sin\theta(k)+\frac{T}{m} \Bigr] \\ 
    \notag \\
    v(k+1) &= v(k) + \Delta t \Bigl[ w(k)p(k)-u(k)r(k) \notag \\ 
    &\quad +\frac{1}{m}\bigl(C_{Y_{\beta}}\beta(k) + \frac{c}{2U}C_{Y_p}p(k)+\frac{c}{2U}C_{Y_r}r(k)+C_{Y_{\delta_r}}\delta_r(k) \notag \\ 
    &\quad +C_{Y_{\delta_a}}\delta_a(k)\bigr)\hat{q}S-\frac{\Dot{m}v(k)}{m}+g\cos\theta(k)\sin\phi(k) \Bigr] \\ 
    \notag \\
    w(k+1) &= w(k) + \Delta t \Bigl[] u(k)q(k)-v(k)p(k) \notag \\
    &\quad +\frac{1}{m}\bigl(C_Z+\frac{c}{2U}C_{Z_{\alpha}}\alpha(k)+\frac{c}{2U}C_{Z_q}q(k)+C_{Z_{\delta_e}}\delta_e(k)\bigr)\hat{q}S \notag \\
    &\quad -\frac{\Dot{m}w(k)}{m}+g\cos\theta(k)\cos\phi(k) \Bigr]
\end{align}

% \begin{align}
%     v(k+1) &= v(k) + \Delta t \Bigl[ w(k)p(k)-u(k)r(k) \notag \\ 
%     &\quad +\frac{1}{m}\bigl(C_{Y_{\beta}}\beta(k) + \frac{c}{2U}C_{Y_p}p(k)+\frac{c}{2U}C_{Y_r}r(k)+C_{Y_{\delta_r}}\delta_r(k) \notag \\ 
%     &\quad +C_{Y_{\delta_a}}\delta_a(k)\bigr)\hat{q}S-\frac{\Dot{m}v(k)}{m}+g\cos\theta(k)\sin\phi(k) \Bigr] 
% \end{align}

% \begin{align}
%     w(k+1) &= w(k) + \Delta t \Bigl[] u(k)q(k)-v(k)p(k) \notag \\
%     &\quad +\frac{1}{m}\bigl(C_Z+\frac{c}{2U}C_{Z_{\alpha}}\alpha(k)+\frac{c}{2U}C_{Z_q}q(k)+C_{Z_{\delta_e}}\delta_e(k)\bigr)\hat{q}S \notag \\
%     &\quad -\frac{\Dot{m}w(k)}{m}+g\cos\theta(k)\cos\phi(k) \Bigr]
% \end{align}

dan dinamika rotasi
\begin{align}
    p(k+1) &= p(k) + \Delta t \cdot \frac{1}{I_x}\Bigl[\bigl(C_{l_{\beta}}\beta(k)+\frac{c}{2U}C_{l_p}p(k)+\frac{c}{2U}C_{l_r}r(k)+C_{l_{\delta_a}}\delta_a(k) \notag \\
    &\quad +C_{l_{\delta_r}}\delta_r(k)\bigr)\hat{q}cS-\Dot{I}_xp(k)+q(k)r(k)(I_y-I_z)\Bigr] \\
    \notag \\
    q(k+1) &= q(k) + \Delta t \cdot \frac{1}{I_y}\Bigl[\bigl(C_m+\frac{c}{2U}C_{m_{\alpha}}\alpha(k)+\frac{c}{2U}C_{m_q}q(k) \notag \\ 
    &\quad +C_{m_{\delta_e}}\delta_e(k)\bigr)\hat{q}cS-\Dot{I}_yq(k)-p(k)r(k)(I_x-I_z)-\Dot{m}q(k)x_e^2 \notag \\
    &\quad + \bigl(C_Z+\frac{c}{2U}C_{Z_{\alpha}}\alpha(k)+\frac{c}{2U}C_{Z_q}q(k)+C_{Z_{\delta_e}}\delta_e(k)\bigr)\hat{q}S d \Bigr]  \\
    \notag \\
    r(k+1) &= r(k) + \Delta t \cdot \frac{1}{I_z}\Bigl[\bigl(C_{n_{\beta}}\beta(k)+\frac{c}{2U}C_{n_p}p(k)+\frac{c}{2U}C_{n_r}r(k)+C_{n_{\delta_a}}\delta_a(k) \notag \\ 
    &\quad +C_{n_{\delta_r}}\delta_r(k)\bigr)\hat{q}cS-\Dot{I}_zr(k)+p(k)q(k)(I_x-I_y)-\Dot{m}r(k)x_e^2 \notag \\ 
    &\quad -\bigl(C_{Y_{\beta}}\beta(k) + \frac{c}{2U}C_{Y_p}p(k)+\frac{c}{2U}C_{Y_r}r(k)+C_{Y_{\delta_r}}\delta_r(k) \notag \\
    &\quad +C_{Y_{\delta_a}}\delta_a(k)\bigr)\hat{q}Sd\Bigr]
\end{align}
 
\section{Kendali Rudal dengan Laguerre-NMPC}
Pada tahap ini, akan dilakukan pembentukan kendali rudal menggunakan metode Laguerre-NMPC.

\subsection{Fungsi Objektif}

\subsection{Prediksi Variabel \textit{State} dan \textit{Output}}

\subsection{Pembentukan Kendala}
Kendala pada penelitian ini
\begin{enumerate}
    \item \textbf{Kendala \textit{Input}} \\
    Kendala input diberikan untuk membatasi nilai masukan kontrol yang masuk pada sistem. Batas input diberikan sebagai berikut
    \begin{align*}
        \pmb{u}_{\min} &\leq \pmb{u}(k+i-1) \leq \pmb{u}_{\max} 
    \end{align*}
    jika menggunakan parameter Laguerre didapat
    \begin{align*}
        \pmb{u}_{\min} &\leq \sum_{j=0}^{i-1}\pmb{L}(j)^T\pmb{\eta}(k)+\pmb{u}(k-1) \leq \pmb{u}_{\max} \\
        \pmb{u}_{\min}-\pmb{u}(k-1) &\leq \sum_{j=0}^{i-1}\pmb{L}(j)^T\pmb{\eta}(k) \leq \pmb{u}_{\max}-\pmb{u}(k-1)
    \end{align*}
    atau dapat ditulis sebagai 
    \begin{align*}
        -\sum_{j=0}^{i-1}\pmb{L}(j)^T\pmb{\eta}(k) &\leq \pmb{u}(k-1)-\pmb{u}_{\min} \\
        \sum_{j=0}^{i-1}\pmb{L}(j)^T\pmb{\eta}(k) &\leq \pmb{u}_{\max}-\pmb{u}(k-1)
    \end{align*}
    pertidaksamaan diatas dalam bentuk matriks menjadi
    \begin{align*}
        \begin{bmatrix}
            -1 \\ 1
        \end{bmatrix} \sum_{j=0}^{i-1}\pmb{L}(j)^T\pmb{\eta}(k) \leq \begin{bmatrix}
            \pmb{u}(k-1)-\pmb{u}_{\min} \\ \pmb{u}_{\max}-\pmb{u}(k-1)
        \end{bmatrix}
    \end{align*}
    untuk $i=1,\hdots,N_p$ diperoleh
    \begin{align*}
        \begin{bmatrix}
            -1 & 0 & \hdots & 0 \\ 
            1 & 0 & \hdots & 0 \\
            0 & -1 & \hdots & 0 \\
            0 & 1 & \hdots & 0 \\
            \vdots & \vdots & \ddots & 0\\
            0 & 0 & \hdots & -1 \\
            0 & 0 & \hdots & 1 
        \end{bmatrix} \begin{bmatrix}
            \pmb{L}(0)^T\pmb{\eta}(k) \\
            \sum\limits_{j=0}^{1}\pmb{L}(j)^T\pmb{\eta}(k) \\
            \sum\limits_{j=0}^{2}\pmb{L}(j)^T\pmb{\eta}(k) \\
            \vdots \\
            \sum\limits_{j=0}^{N_p-1}\pmb{L}(j)^T\pmb{\eta}(k)
        \end{bmatrix} \leq \begin{bmatrix}
            \pmb{u}(k-1)-\pmb{u}_{\min} \\ \pmb{u}_{\max}-\pmb{u}(k-1) \\ \pmb{u}(k-1)-\pmb{u}_{\min} \\ \pmb{u}_{\max}-\pmb{u}(k-1) \\ \vdots \\ \pmb{u}(k-1)-\pmb{u}_{\min} \\ \pmb{u}_{\max}-\pmb{u}(k-1)
        \end{bmatrix} 
    \end{align*}
    Formulasi kendala \textit{input} diatas dapat dibentuk menjadi pertidaksaman matriks
    \begin{align*}
        S \pmb{\eta} \leq \vb{P} 
    \end{align*}

    \item \textbf{Kendala \textit{Increment Input}} \\
    Kendala \textit{Increment Input} diberikan sebagai berikut
    \begin{align*}
     \Delta\pmb{u}_{\min} &\leq \Delta\pmb{u}(k+i-1) \leq \Delta \pmb{u}_{\max} 
    \end{align*}
    dengan parameter Laguerre didapat
    \begin{align*}
        \Delta\pmb{u}_{\min} &\leq \pmb{L}(i-1)^T\pmb{\eta}(k) \leq \Delta \pmb{u}_{\max}
    \end{align*}
    atau dapat ditulis sebagai pertidaksamaan sebagai berikut
    \begin{align*}
        -\pmb{L}(i-1)^T\pmb{\eta}(k) &\leq -\Delta\pmb{u}_{\min} \\
        \pmb{L}(i-1)^T\pmb{\eta}(k) &\leq \Delta \pmb{u}_{\max}
    \end{align*}
    untuk $i=1,\hdots,N_p$ diperoleh pertidaksamaan diatas dalam bentuk matriks 
    \begin{align*}
        \begin{bmatrix}
            -1 & 0 & \hdots & 0 \\ 
            1 & 0 & \hdots & 0 \\
            0 & -1 & \hdots & 0 \\
            0 & 1 & \hdots & 0 \\
            \vdots & \vdots & \ddots & 0\\
            0 & 0 & \hdots & -1 \\
            0 & 0 & \hdots & 1 
        \end{bmatrix} \begin{bmatrix}
            \pmb{L}(0)^T\pmb{\eta}(k) \\
            \pmb{L}(1)^T\pmb{\eta}(k) \\
            \pmb{L}(2)^T\pmb{\eta}(k) \\
            \vdots \\
            \pmb{L}(N_p-1)^T\pmb{\eta}(k)
        \end{bmatrix} \leq \begin{bmatrix}
            -\Delta\pmb{u}_{\min} \\ \Delta\pmb{u}_{\max} \\ -\Delta\pmb{u}_{\min} \\ \Delta\pmb{u}_{\max} \\ \vdots \\ -\Delta\pmb{u}_{\min} \\ \Delta\pmb{u}_{\max}
        \end{bmatrix}
    \end{align*}
    Formulasi kendala \textit{increment input} dalam dibentuk pertidaksaman matriks menjadi
    \begin{align*}
        E \pmb{\eta} \leq \vb{F}
    \end{align*}
    
    \item \textbf{Kendala Variabel \textit{State}} \\
    Bentuk variabel \textit{state} pada sistem adalah sebagai berikut
    \begin{align*}
        \pmb{\mathcal{X}}(k+i|k) &= \pmb{f_d}\bigl(\pmb{\mathcal{X}}(k+i-1|k),\pmb{L}(i-1)^T \pmb{\eta}(k)\bigr) \\
    \pmb{\mathcal{X}}_{\min} &\leq \pmb{\mathcal{X}}(k+i|k) \leq \pmb{\mathcal{X}}_{\max}
    \end{align*}
    formulasi bentuk pertidaksamaan
    \begin{align*}
        \begin{bmatrix}
            -1 & 0 & \hdots & 0 \\ 
            1 & 0 & \hdots & 0 \\
            0 & -1 & \hdots & 0 \\
            0 & 1 & \hdots & 0 \\
            \vdots & \vdots & \ddots & 0\\
            0 & 0 & \hdots & -1 \\
            0 & 0 & \hdots & 1 
        \end{bmatrix} \begin{bmatrix}
            \pmb{f_d}\bigl(\pmb{\mathcal{X}}(k|k),\pmb{L}(0)^T \pmb{\eta}(k)\bigr) \\
            \pmb{f_d}\bigl(\pmb{\mathcal{X}}(k+1|k),\pmb{L}(1)^T \pmb{\eta}(k)\bigr) \\
            \pmb{f_d}\bigl(\pmb{\mathcal{X}}(k+2|k),\pmb{L}(2)^T \pmb{\eta}(k)\bigr) \\
            \vdots \\
            \pmb{f_d}\bigl(\pmb{\mathcal{X}}(k+N_p-1|k),\pmb{L}(N_p-1)^T \pmb{\eta}(k)\bigr)
        \end{bmatrix} \leq \begin{bmatrix}
            -\pmb{\mathcal{X}}_{\min} \\ \pmb{\mathcal{X}}_{\max} \\ -\pmb{\mathcal{X}}_{\min} \\ \pmb{\mathcal{X}}_{\max} \\ \vdots \\ -\pmb{\mathcal{X}}_{\min} \\ \pmb{\mathcal{X}}_{\max}
        \end{bmatrix}
    \end{align*}
    jika pertidaksamaan diatas ditulis dalam bentuk matriks menjadi
    \begin{align*}
        M \pmb{\eta} \leq \vb{N}
    \end{align*}
    pada penelitian ini diberikan batasan pada $\phi,\theta,\psi$, dengan nilai batasan 
    \begin{align*}
        |\phi| \leq \frac{\pi}{2}, \quad |\theta| \leq \frac{\pi}{2}, \quad |\psi| \leq \frac{\pi}{2}
    \end{align*}
\end{enumerate}

\section{Proses Optimasi Menggunakan MATLAB}
Optimasi akan dilakukan menggunakan \textit{Quadratic Programming} pada MATLAB. Fungsi objektif harus diubah ke fungsi objektif \textit{Quadratic Programming}, yang dapat dirumuskan sebagai berikut
\begin{align*}
    J=\frac{1}{2}\pmb{\eta}^TH\pmb{\eta}+g^T\pmb{\eta}
\end{align*}
bentuk kendala pada \textit{Quadratic Programming} adalah
\begin{align*}
    V\pmb{\eta} \leq \vb{W}
\end{align*}
dengan
\begin{align*}
    V=\begin{bmatrix}
      S \\ E \\ M  
    \end{bmatrix}, \quad \vb{W}=\begin{bmatrix}
        \vb{P} \\ \vb{F} \\ \vb{N}
    \end{bmatrix}
\end{align*}

\section{Simulasi}

\begin{table}[htbp]
\setlength{\tabcolsep}{5pt}
% \renewcommand{\arraystretch}{1.5}
    \centering
    \caption{Inisialisasi Nilai Awal Variabel \textit{State} dan Variabel \textit{Input}}
    \begin{tabular}{|c|c|c|c|c|c|}
    \hline 
        \textbf{Variabel} & \textbf{Nilai} & \textbf{Variabel} & \textbf{Nilai} & \textbf{Variabel} & \textbf{Nilai} \\
        \hline
        $x$ & 0 & $\psi$ & 0 & $q$ & 0\\
        \hline
        $y$ & 0 & $u$ & 21.5 & $r$ & 0\\
        \hline
        $z$ & 0 & $v$ & 0 & $\delta_a$ & 0\\
        \hline
        $\phi$ & 0 & $w$ & 0 & $\delta_e$ & 0 \\
        \hline
        $\theta$ & $1.2211$ & $p$ & 0 & $\delta_r$ & 0\\
        \hline
    \end{tabular}
    \label{tab: inisialisasi variabel}
\end{table}


