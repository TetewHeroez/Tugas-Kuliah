Aljabar max-plus menyediakan struktur idempoten yang relevan untuk membangun skema kriptografi berbasis masalah komputasi sulit. Salah satu komponen kuncinya adalah pasangan matriks yang komutatif terhadap operasi $\otimes$, khususnya matriks yang berasal dari \textit{Latin square}. Penelitian ini bertujuan (1) menurunkan syarat perlu agar dua \textit{Latin square} $A,B\in\R^{n\times n}_{\max}$ komutatif, dan (2) menyusun prosedur konstruksi pasangan \textit{Latin square} komutatif dari $L_S(n)$ yang dapat dipakai sebagai blok dasar protokol kunci privat tropikal. Metodologi yang ditempuh meliputi studi literatur aljabar max-plus, matriks sirkulan-$t$, serta karakterisasi diagonal utama (Linde--de la Puente); identifikasi bentuk \textit{Latin square} melalui representasi matriks permutasi dan orbit siklus (termasuk derangement); analisis teoretis terhadap relasi komutasi berbasis struktur permutasi dan batas entri diagonal; serta verifikasi eksperimental pada \textit{Latin square} berordo kecil–menengah untuk menguji kekomutatifan dan pola konstruksi yang diusulkan. Hasil yang diharapkan adalah kriteria komutasi yang eksplisit beserta algoritme konstruktif yang memperkaya kajian \textit{Latin square} pada aljabar max-plus dan mendukung perancangan protokol kriptografi tropikal yang aman.
\katakunci{aljabar max-plus, Latin square, matriks komutatif, matriks sirkulan, permutasi, kriptografi tropikal}
