Aljabar max-plus, yang dinotasikan dengan $(\mathbb{R}_{\max}, \oplus, \otimes)$ dengan $a \oplus b = \max\{a,b\}$ dan $a \otimes b = a + b$, merupakan salah satu struktur aljabar idempoten yang banyak digunakan dalam pemodelan sistem diskrit serta mulai
dikaji dalam pengembangan skema kriptografi non-konvensional. Dalam konteks ini,
keberadaan pasangan matriks $A,B \in \mathbb{R}_{\max}^{n \times n}$ yang memenuhi
$A \otimes B = B \otimes A$ menjadi aspek penting. Salah satu kelas matriks yang
menarik untuk dikaji adalah matriks \textit{Latin square}, karena memiliki struktur
kombinatorial yang teratur dan keterkaitan erat dengan teori permutasi.

Penelitian ini membahas sifat kekomutatifan matriks Latin square pada aljabar
max-plus. Fokus utama penelitian adalah mengkaji kondisi-kondisi yang menyebabkan dua
Latin square $A,B \in LS(n)$ bersifat komutatif, serta menganalisis struktur internal
Latin square melalui representasinya sebagai kombinasi matriks permutasi
$\{P_{\pi}\}_{\pi \in S_n}$. Kajian dilakukan dengan memanfaatkan konsep-konsep dasar
aljabar max-plus, matriks sirkulan dan sirkulan-tergeser, serta karakterisasi
kekomutatifan matriks berdasarkan pola diagonal dan relasi komutatif pada grup
permutasi $S_n$.

Luaran yang diharapkan dari penelitian ini adalah diperolehnya kriteria komutatif yang
jelas untuk matriks Latin square dalam aljabar max-plus, beserta prosedur konstruktif
untuk membangun pasangan Latin square komutatif dengan memanfaatkan struktur matriks
permutasi sirkulan. Hasil kajian ini diharapkan dapat memberikan kontribusi pada
pengembangan teori Latin square dalam aljabar tropikal serta menjadi dasar bagi kajian
lanjutan pada skema kriptografi berbasis aljabar max-plus.

\katakunci{Aljabar Max-Plus, Latin Square, Matriks Komutatif, Matriks Sirkulan, Permutasi, Semiring Tropikal}
