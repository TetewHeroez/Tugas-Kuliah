\pagebreak
\chapter{TINJAUAN PUSTAKA}
\section{Hasil Penelitian Terdahulu}


\section{Latin Squares}


\section{Aljabar Max-Plus}
Aljabar max-plus didefinisikan pada $\mathbb{R}\cup\{-\infty\}$ dengan $a\oplus b=\max\{a,b\}$ dan $a\otimes b=a+b$. Struktur ini merupakan semiring idempoten dan banyak digunakan pada penjadwalan, optimisasi, dan sistem kejadian diskret. Sejumlah kajian dalam kerangka tropis juga muncul di bidang kriptografi dan keamanan protokol, yang memperlihatkan relevansi struktur idempoten dalam konteks komputasi dan implementasi \cite{stickelImpl,tropicalStickelSecurity}.

\section{Celah Riset dan Arah Penelitian}
Keterhubungan langsung antara persegi Latin (terutama yang komutatif) dengan konstruksi berbasis aljabar max-plus relatif belum dieksplorasi luas. Tantangan utama ialah memastikan hasil operasi biner kembali ke himpunan simbol hingga. Untuk itu, diperlukan skema normalisasi $\varphi: S\times S\to S$ yang menjaga sifat Latin (keunikan tiap baris/kolom) sekaligus komutativitas. Pendekatan ini, disertai verifikasi berbantuan komputer dan perbandingan dengan praktik di ranah tropis lainnya \cite{tropicalStickelSecurity}, menjadi dasar arah penelitian ini, dengan acuan penulisan dari karya-karya terkait \cite{zulfikarTA}.

\section{Ringkasan}
Tinjauan menunjukkan peluang kontribusi pada definisi, konstruksi, dan karakterisasi persegi Latin komutatif di lingkungan max-plus, serta potensi aplikasi pada penjadwalan tropis dan desain kombinatorial.


