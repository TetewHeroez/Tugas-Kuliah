\pagebreak
\chapter{TINJAUAN PUSTAKA}
\section{Hasil Penelitian Terdahulu}

\begin{table}[h]
  \centering
  \begin{tabular}{|p{3cm}|p{1.2cm}|p{3cm}|p{6cm}|}
    \hline
    \textbf{Penulis}                                                                   & \textbf{Tahun} & \textbf{Judul} & \textbf{Rangkuman}                             \\
    \hline
    Muhammad Zulfikar Zufar                                                            & 2023           & Kontruksi Grup
    Latin Square pada Aljabar Max-Plus                                                 &                                                                                  \\
    \hline

    Muhammad Syifa'ul Mufid and Subiono                                                &
    2014                                                                               &
    Eigenvalues and eigenvectors of Latin Squares In Max-plus Algebra                  &
    Diselesaikan Permasalahan Eigen dari \textit{latin square} pada Aljabar Max-plus dengan Memperhatikan Permutasi dari Angka-angka pada \textit{latin square} Tersebut. \\
    \hline

    Kasie G Farlow                                                                     &
    2009                                                                               &
    Max-Plus Algebra                                                                   &
    Dibahas Mengenai Beberapa Hal Terkait Aljabar Max-plus, dan Salahsatu Hal Menarik yang Dibahas pada Paper ini adalah Aljabar Linear pada Aljabar Max-plus.            \\
    \hline

    Fazal Abbas and Mubasher Umer and Umar Hayat and Ikram Ullah                       &
    2022                                                                               &
    Trivial and Nontrivial eigenvectors for \textit{latin squares} in Max-Plus Algebra &
    Mengkaji Permasalahan Eigen pada \textit{Non-Symmetric} Latin Square pada Aljabar Max-plus.                                                                           \\
    \hline
  \end{tabular}
  \caption{Ringkasan Literatur terkait Max-Plus Algebra dan Latin Squares}
\end{table}


\section{Persegi Latin, Kuasigrup, dan Komutativitas}
Persegi Latin berordo $n$ adalah matriks $n\times n$ yang memuat $n$ simbol sehingga setiap simbol muncul tepat satu kali pada setiap baris dan setiap kolom. Struktur ini berkorespondensi dengan kuasigrup pada himpunan simbol melalui operasi biner yang tabel Cayley-nya adalah persegi Latin. Komutativitas pada persegi Latin dicirikan oleh kesimetrian terhadap diagonal utama, yaitu $L(i,j)=L(j,i)$ untuk seluruh indeks $i,j$.

\section{Aljabar Max-Plus}
Aljabar max-plus didefinisikan pada $\mathbb{R}\cup\{-\infty\}$ dengan $a\oplus b=\max\{a,b\}$ dan $a\otimes b=a+b$. Struktur ini merupakan semiring idempoten dan banyak digunakan pada penjadwalan, optimisasi, dan sistem kejadian diskret. Sejumlah kajian dalam kerangka tropis juga muncul di bidang kriptografi dan keamanan protokol, yang memperlihatkan relevansi struktur idempoten dalam konteks komputasi dan implementasi \cite{stickelImpl,tropicalStickelSecurity}.

\section{Celah Riset dan Arah Penelitian}
Keterhubungan langsung antara persegi Latin (terutama yang komutatif) dengan konstruksi berbasis aljabar max-plus relatif belum dieksplorasi luas. Tantangan utama ialah memastikan hasil operasi biner kembali ke himpunan simbol hingga. Untuk itu, diperlukan skema normalisasi $\varphi: S\times S\to S$ yang menjaga sifat Latin (keunikan tiap baris/kolom) sekaligus komutativitas. Pendekatan ini, disertai verifikasi berbantuan komputer dan perbandingan dengan praktik di ranah tropis lainnya \cite{tropicalStickelSecurity}, menjadi dasar arah penelitian ini, dengan acuan penulisan dari karya-karya terkait \cite{zulfikarTA}.

\section{Ringkasan}
Tinjauan menunjukkan peluang kontribusi pada definisi, konstruksi, dan karakterisasi persegi Latin komutatif di lingkungan max-plus, serta potensi aplikasi pada penjadwalan tropis dan desain kombinatorial.


