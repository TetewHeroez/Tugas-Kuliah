\chapter{METODOLOGI}

Pada bab ini dijelaskan langkah-langkah pelaksanaan penelitian tugas akhir beserta alur proses dan jadwal kegiatan.
\begin{figure}[H]
  \centering
  \begin{tikzpicture}[node distance=1cm and 1cm,font=\scriptsize,scale=1,transform shape]

    % Nodes
    \node[startstop] (start) {Mulai};
    \node[process, right=of start] (studi) {Studi literatur};
    \node[process, right=of studi] (identifikasi) {Identifikasi ...};
    \node[process, right=of identifikasi] (investigasi) {Investigasi ...};
    \node[process, below=of investigasi] (kesimpulan) {Penarikan kesimpulan};
    \node[process, left=of kesimpulan] (penulisan) {Penulisan tugas akhir};
    \node[startstop, left=of penulisan] (end) {Selesai};

    % Arrows
    \draw[arrow] (start) -- (studi);
    \draw[arrow] (studi) -- (identifikasi);
    \draw[arrow] (identifikasi) -- (investigasi);
    \draw[arrow] (investigasi) -- (kesimpulan);
    \draw[arrow] (kesimpulan) -- (penulisan);
    \draw[arrow] (penulisan) -- (end);

  \end{tikzpicture}
  \caption{Alur Proses Penelitian Tugas Akhir}
  \label{fig:metodologi}
\end{figure}

\section{Studi Literatur}
Tahap ini dilakukan dengan mempelajari berbagai literatur yang berkaitan dengan aljabar \textit{max-plus} dan \textit{latin square}, baik berupa buku, jurnal, artikel, maupun sumber-sumber terpercaya lainnya.
Tujuannya adalah untuk memahami konsep dasar, teori-teori yang relevan, serta penelitian-

\section{Identifikasi...}
Pada tinjauan pustaka telah dijelaskan mengenai definisi dan teorema yang berkaitan dengan matriks aljabar max-plus. Selanjutnya,

\section{Investigasi...}

\section{Penarikan Kesimpulan}
Bagian 3.2 dan 3.3 telah menjawab rumusan masalah yang telah diajukan pada bab 1, sehingga pada tahap ini dilakukan penarikan kesimpulan dari hasil investigasi yang telah dilakukan.

\section{Penulisan Tugas Akhir}
Tahap akhir dari penelitian ini adalah menyusun laporan tugas akhir berdasarkan hasil penelitian yang telah dilakukan pada tahap-tahap sebelumnya.
% \chapter*{JADWAL PENELITIAN}
% Berikut jadwal pelaksanaan tahap-tahap penelitian tugas akhir yang akan dilakukan selama 3 bulan sesuai dengan metode penelitian. \vspace{0.5cm}
% \begin{table}[htbp]
%   \centering
%   \caption{Jadwal Pelaksanaan Penelitian Tugas Akhir}
%   \renewcommand{\arraystretch}{1.5}
%   \begin{tabular}{|C{0.6cm}|L{6.5cm}|C{0.25cm}|C{0.25cm}|C{0.25cm}|C{0.25cm}|C{0.25cm}|C{0.25cm}|C{0.25cm}|C{0.25cm}|C{0.25cm}|C{0.25cm}|C{0.25cm}|C{0.25cm}|}	\hline
%                                       &                                                  & \multicolumn{12}{c|}{\textbf{BULAN}}                                                                                                                                                                                                                                       \\\cline{3-14}
%     \multicolumn{1}{|c|}{\textbf{NO}} & \multicolumn{1}{c|}{\textbf{NAMA KEGIATAN}}      & \multicolumn{4}{c|}{1}               & \multicolumn{4}{c|}{2} & \multicolumn{4}{c|}{3}                                                                                                                                                                                     \\\cline{3-14}
%                                       &                                                  & 1                                    & 2                      & 3                      & 4                 & 1                 & 2                 & 3                 & 4                 & 1                 & 2                 & 3                 & 4                 \\\cline{1-14}
%     1                                 & Studi literatur dan perumusan masalah            & \cellcolor{blue!}                    & \cellcolor{blue!}      &                        &                   &                   &                   &                   &                   &                   &                   &                   &                   \\\hline
%     2                                 & Perancangan skema normalisasi dan model operasi  &                                      & \cellcolor{blue!}      & \cellcolor{blue!}      &                   &                   &                   &                   &                   &                   &                   &                   &                   \\\hline
%     3                                 & Pengembangan algoritme konstruksi dan verifikasi &                                      &                        & \cellcolor{blue!}      & \cellcolor{blue!} &                   & \cellcolor{blue!} & \cellcolor{blue!} &                   &                   &                   &                   &                   \\\hline
%     4                                 & Eksperimen dan enumerasi orde kecil--menengah    &                                      &                        &                        & \cellcolor{blue!} & \cellcolor{blue!} & \cellcolor{blue!} & \cellcolor{blue!} &                   &                   &                   &                   &                   \\\hline
%     5                                 & Analisis hasil dan karakterisasi aljabar         &                                      &                        &                        &                   &                   &                   & \cellcolor{blue!} & \cellcolor{blue!} & \cellcolor{blue!} &                   &                   &                   \\\hline
%     6                                 & Penyusunan kesimpulan dan saran                  &                                      &                        &                        &                   &                   &                   &                   &                   & \cellcolor{blue!} & \cellcolor{blue!} &                   &                   \\\hline
%     7                                 & Penulisan laporan proposal/TA                    &                                      &                        &                        &                   &                   &                   &                   &                   &                   & \cellcolor{blue!} & \cellcolor{blue!} & \cellcolor{blue!} \\\hline
%   \end{tabular}
% \end{table}
