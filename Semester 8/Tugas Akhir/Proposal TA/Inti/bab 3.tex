\pagebreak
\chapter{METODOLOGI}

Pada bab ini dijelaskan langkah-langkah pelaksanaan penelitian tugas akhir beserta alur proses dan jadwal kegiatan.

\pagebreak
\chapter*{JADWAL PENELITIAN}
Berikut jadwal pelaksanaan tahap-tahap penelitian tugas akhir yang akan dilakukan selama 3 bulan sesuai dengan metode penelitian. \vspace{0.5cm}
\begin{table}[htbp]
  \centering
  \caption{Jadwal Pelaksanaan Penelitian Tugas Akhir}
  \renewcommand{\arraystretch}{1.5}
  \begin{tabular}{|C{0.6cm}|L{6.5cm}|C{0.25cm}|C{0.25cm}|C{0.25cm}|C{0.25cm}|C{0.25cm}|C{0.25cm}|C{0.25cm}|C{0.25cm}|C{0.25cm}|C{0.25cm}|C{0.25cm}|C{0.25cm}|}	\hline
                                      &                                                  & \multicolumn{12}{c|}{\textbf{BULAN}}                                                                                                                                                                                                                                       \\\cline{3-14}
    \multicolumn{1}{|c|}{\textbf{NO}} & \multicolumn{1}{c|}{\textbf{NAMA KEGIATAN}}      & \multicolumn{4}{c|}{1}               & \multicolumn{4}{c|}{2} & \multicolumn{4}{c|}{3}                                                                                                                                                                                     \\\cline{3-14}
                                      &                                                  & 1                                    & 2                      & 3                      & 4                 & 1                 & 2                 & 3                 & 4                 & 1                 & 2                 & 3                 & 4                 \\\cline{1-14}
    1                                 & Studi literatur dan perumusan masalah            & \cellcolor{blue!}                    & \cellcolor{blue!}      &                        &                   &                   &                   &                   &                   &                   &                   &                   &                   \\\hline
    2                                 & Perancangan skema normalisasi dan model operasi  &                                      & \cellcolor{blue!}      & \cellcolor{blue!}      &                   &                   &                   &                   &                   &                   &                   &                   &                   \\\hline
    3                                 & Pengembangan algoritme konstruksi dan verifikasi &                                      &                        & \cellcolor{blue!}      & \cellcolor{blue!} &                   & \cellcolor{blue!} & \cellcolor{blue!} &                   &                   &                   &                   &                   \\\hline
    4                                 & Eksperimen dan enumerasi orde kecil--menengah    &                                      &                        &                        & \cellcolor{blue!} & \cellcolor{blue!} & \cellcolor{blue!} & \cellcolor{blue!} &                   &                   &                   &                   &                   \\\hline
    5                                 & Analisis hasil dan karakterisasi aljabar         &                                      &                        &                        &                   &                   &                   & \cellcolor{blue!} & \cellcolor{blue!} & \cellcolor{blue!} &                   &                   &                   \\\hline
    6                                 & Penyusunan kesimpulan dan saran                  &                                      &                        &                        &                   &                   &                   &                   &                   & \cellcolor{blue!} & \cellcolor{blue!} &                   &                   \\\hline
    7                                 & Penulisan laporan proposal/TA                    &                                      &                        &                        &                   &                   &                   &                   &                   &                   & \cellcolor{blue!} & \cellcolor{blue!} & \cellcolor{blue!} \\\hline
  \end{tabular}
\end{table}
