Persegi Latin berordo $n$ adalah matriks $n\times n$ yang memuat $n$ simbol sehingga setiap simbol muncul tepat satu kali pada setiap baris dan kolom. Dalam penelitian ini, kami mengkaji persegi Latin komutatif yang dikonstruksi dalam kerangka aljabar max-plus, yaitu semiring idempoten pada $\mathbb{R}\cup\{-\infty\}$ dengan operasi $\oplus$ sebagai maksimum dan $\otimes$ sebagai penjumlahan. Fokus kajian meliputi: (i) perumusan definisi formal ``persegi Latin komutatif atas aljabar max-plus'' pada himpunan hingga melalui skema normalisasi agar keluaran operasi kembali ke himpunan simbol; (ii) kondisi perlu/cukup bagi eksistensi; (iii) metode konstruksi eksplisit beserta prosedur verifikasinya; dan (iv) studi kasus, enumerasi awal, serta sifat-sifat aljabar yang menyertainya. Kontribusi diharapkan pada jembatan antara desain kombinatorial (persegi Latin/kuasigrup) dan aljabar tropis, sekaligus membuka potensi aplikasi pada penjadwalan tropis dan pemodelan sistem kejadian diskret.

\katakunci{persegi Latin; komutatif; kuasigrup; aljabar max-plus; aljabar tropis; normalisasi}
