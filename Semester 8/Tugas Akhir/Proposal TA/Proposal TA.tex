\documentclass{file/TA-ITS}
%Ridho Nur Rohman Wijaya

\makeatletter
\def\cleardoublepage{\clearpage%
	\if@twoside
	\ifodd\c@page\else
	\vspace*{\fill}
	\hfill
	\begin{center}
		\emph{ }
	\end{center}
	\vspace{\fill}
	\thispagestyle{empty}
	\newpage
	\if@twocolumn\hbox{}\newpage\fi
	\fi
	\fi
}
\makeatother
\theoremstyle{definition}
\newtheorem{definisi}{Definisi}[section]
\newtheorem{teorema}{Teorema}[section]
\newtheorem{lemma}{Lemma}
\newtheorem{akibat}{Akibat}
\newtheorem{contoh}{Contoh}
\theoremstyle{plain}
\newtheorem{prop}{Proposisi}
\renewcommand{\proofname}{Bukti}

\newcommand{\norm}[1]{\left\|#1\right\|} % Fungsi norm (||x||)

\newcommand\firstPar{0.75cm} % Indentasi 0.75cm pada tiap paragraf (manual untuk hspace)
\setlength{\parindent}{0.75cm} % Indentasi 0.75cm pada tiap paragraf

\usepackage{fancyhdr}
\pagestyle{fancy}
\renewcommand{\headrulewidth}{0pt}
\fancyhf{}
\usepackage{ifthen}
\fancyfoot[R]{\ifthenelse{\isodd{\value{page}}}{\thepage}{}}
\fancyfoot[L]{\ifthenelse{\isodd{\value{page}}}{}{\thepage}}

\usepackage[labelsep=quad]{caption}
\captionsetup[table]{skip=5pt}

\usepackage{multirow}
\usepackage{longtable}
\usepackage{physics}
\usepackage{ragged2e} 
\usepackage[mathscr]{eucal}

% Pengaturan tabel
\usepackage{multicol,multirow}
\newcommand{\blue}{\cellcolor{blue!100}}
\newcommand{\red}{\cellcolor{red!75}}
\newcommand{\yellow}{\cellcolor{yellow!75}}
\newcommand{\gray}{\cellcolor{gray!75}}
\newcommand{\cyan}{\cellcolor{cyan!100}}
\usepackage{longtable} % Pembuatan tabel
\usepackage[table]{xcolor} % Pewarnaan tabel

%%% Pewarnaan code
\usepackage{xcolor}
\usepackage{color}
\usepackage{listings}

\definecolor{codegreen}{rgb}{0,0.6,0}
\definecolor{codeblack}{rgb}{0,0,0}
\definecolor{codepurple}{rgb}{0.58,0,0.82}
\definecolor{backcolour}{rgb}{0.95,0.95,0.92}

\lstdefinestyle{mystyle}{
    commentstyle=\color{codegreen},
    keywordstyle=\color{magenta},
    numberstyle=\tiny\color{codeblack},
    stringstyle=\color{codepurple},
    basicstyle=\ttfamily\footnotesize,
    breakatwhitespace=false,         
    breaklines=true,                 
    captionpos=b,                    
    keepspaces=true,                 
    numbers=left,                    
    numbersep=5pt,                  
    showspaces=false,                
    showstringspaces=false,
    showtabs=false,                  
    tabsize=2
}
%%% Pewarnaan code

\hypersetup{ % Merubah warna link
    colorlinks,
    linkcolor={black},
    citecolor={black},
    urlcolor={black}
}

% \tolerance=1
% \emergencystretch = \maxdimen
% \hyphenpenalty=10000
% \hbadness=1000

\begin{document}

% input data
\Judul{Optimisasi Jadwal Jaringan Bus di Terminal Purabaya Menggunakan Aljabar Max-Plus}

\JudulEng{Optimization of Bus Network Scheduling at Purabaya Terminal Using Max-Plus Algebra}

\Nama{Teosofi Hidayah Agung}

\NamaKecil{Teosofi Hidayah Agung}

\NRP{5002221132}

\Departemen{Matematika}

\Department{Mathematics}

\BidangStudi{Aljabar dan Analisis}

\Bulan{November} % Masuk lembar pengesahan

\Tahun{2024}

\TanggalDisetujui{22 November 2024} % Masuk lembar orisinilitas

\Fakultas{Sains dan Analitika Data}

\SingkatanFakultas{FSAD}

\Faculty{Scientics}

\SingkatanFakultasEng{SCIENTICS}

\Pembimbing{Pembimbing 1}
          {} % Isi {} untuk pembimbing 2
		   

\NIPPembimbing{NIP}
{} % Isi {} untuk pembimbing 2
              
              
\Penguji{Penguji 1}
        {Penguji 2}
        {Penguji 3}

\NIPPenguji{NIP Penguji 1}
           {NIP Penguji 2}
           {NIP Penguji 3}
\Kadep{Nama Kadept}

\NIPKadep{NIP Pak Kadept}
%\Cover
\BagianAwal
\LembarJudul
\TitlePage
%\LembarPengesahan
%\LembarOrisinalitas
\LembarPengesahanProposal
\LembarAbstrak
\LembarAbstract

%%%%%%%%%%%%%%%%%%%%%%%%  Abstrak  %%%%%%%%%%%%%%%5%%%%%%%%%%
%Aljabar max-plus menyediakan struktur idempoten yang relevan untuk membangun skema kriptografi berbasis masalah komputasi sulit. Salah satu komponen kuncinya adalah pasangan matriks yang komutatif terhadap operasi $\otimes$, khususnya matriks yang berasal dari \textit{Latin square}. Penelitian ini bertujuan (1) menurunkan syarat perlu agar dua \textit{Latin square} $A,B\in\R^{n\times n}_{\max}$ komutatif, dan (2) menyusun prosedur konstruksi pasangan \textit{Latin square} komutatif dari $L_S(n)$ yang dapat dipakai sebagai blok dasar protokol kunci privat tropikal. Metodologi yang ditempuh meliputi studi literatur aljabar max-plus, matriks sirkulan-$t$, serta karakterisasi diagonal utama (Linde--de la Puente); identifikasi bentuk \textit{Latin square} melalui representasi matriks permutasi dan orbit siklus (termasuk derangement); analisis teoretis terhadap relasi komutasi berbasis struktur permutasi dan batas entri diagonal; serta verifikasi eksperimental pada \textit{Latin square} berordo kecil–menengah untuk menguji kekomutatifan dan pola konstruksi yang diusulkan. Hasil yang diharapkan adalah kriteria komutasi yang eksplisit beserta algoritme konstruktif yang memperkaya kajian \textit{Latin square} pada aljabar max-plus dan mendukung perancangan protokol kriptografi tropikal yang aman.
\katakunci{aljabar max-plus, Latin square, matriks komutatif, matriks sirkulan, permutasi, kriptografi tropikal}


%%%%%%%%%%%%%%%%%%%%%%%%  Abstrak  %%%%%%%%%%%%%%%5%%%%%%%%%%


%%%%%%%%%%%%%%%%%%%%%%%%  Daftar  %%%%%%%%%%%%%%%5%%%%%%%%%%

\DaftarIsi\raggedbottom

\DaftarGambar
%%%%%%%%%%%%%%%%%%%%%%%%  Daftar  %%%%%%%%%%%%%%%5%%%%%%%%%%

\DaftarTabel
%%%%%%%%%%%%%%%%%%%%%%%%  Daftar  %%%%%%%%%%%%%%%5%%%%%%%%%%

\DaftarSimbol
%%%%%%%%%%%%%%%%%%%%%%%%  Daftar  %%%%%%%%%%%%%%%5%%%%%%%%%%

\BagianInti

%%%%%%%%%%%%%%%%%%%%%%%%  Bab I-III  %%%%%%%%%%%%%%%5%%%%%%%%%%

\pagebreak
\chapter{PENDAHULUAN}
\section{Latar Belakang}
Persegi Latin (Latin square) berordo $n$ adalah sebuah matriks berukuran $n\times n$ yang diisi oleh $n$ simbol sedemikian sehingga setiap simbol muncul tepat satu kali pada setiap baris dan setiap kolom. Struktur kombinatorial ini memainkan peranan penting dalam berbagai bidang, seperti desain eksperimen, teori kode, kriptografi, penjadwalan, dan teori graf. Dari sudut pandang aljabar, setiap persegi Latin berkaitan erat dengan konsep kuasigrup, yakni himpunan beroperasi biner yang tabel Cayley-nya merupakan persegi Latin. Sifat komutatif pada persegi Latin merefleksikan komutativitas operasi yang melandasinya, dan membawa konsekuensi struktural yang khas terhadap simetri dan isotopi.

Di sisi lain, aljabar tropis—khususnya aljabar max-plus—telah menjadi medan kajian yang berkembang pesat. Aljabar max-plus adalah semiring idempoten pada $\mathbb{R}\cup\{-\infty\}$ dengan operasi $a\oplus b = \max\{a,b\}$ dan $a\otimes b = a+b$. Kerangka ini telah terbukti efektif untuk memodelkan sistem kejadian diskret, penjadwalan, optimisasi, dan analisis jaringan karena sifat idempotensi dan keteraturan parsial yang kuat. Selain itu, studi pada kerangka tropis juga muncul dalam konteks kriptografi dan protokol kunci publik, yang menyoroti aspek keamanan dan implementasi dalam semiring idempoten \cite{tropicalStickelSecurity,stickelImpl}. Meskipun demikian, interaksi antara objek kombinatorial klasik seperti persegi Latin dengan struktur aljabar max-plus relatif belum banyak dieksplorasi, khususnya ketika kita membatasi perhatian pada persegi Latin komutatif.

Penelitian ini memfokuskan diri pada
``Latin Square Komutatif atas Aljabar Max-Plus'': merumuskan dan mengkaji cara-cara membangun, mengarakterisasi, serta memverifikasi persegi Latin komutatif yang diturunkan dari operasi max-plus pada himpunan hingga. Secara khusus, kami meninjau skema pelabelan/normalisasi agar hasil operasi max-plus pada pasangan elemen kembali ke himpunan simbol yang terbatas sehingga memenuhi sifat Latin. Kajian ini diharapkan membuka jembatan antara aljabar tropis dan desain kombinatorial, sekaligus menyediakan metode konstruksi baru dengan potensi aplikasi pada penjadwalan berbasis tropis dan pemodelan sistem diskret.

\section{Rumusan Masalah}
Rumusan masalah yang diangkat pada penelitian ini adalah sebagai berikut:
\begin{enumerate}
  \item Bagaimana merumuskan definisi yang ketat tentang ``persegi Latin (komutatif) atas aljabar max-plus'' pada himpunan simbol hingga, termasuk mekanisme pelabelan/normalisasi hasil operasi agar tetap berada dalam himpunan tersebut?
  \item Kondisi apa yang diperlukan dan/atau cukup bagi eksistensi persegi Latin komutatif berordo $n$ yang dibangun dari struktur max-plus tertentu (misalnya pada $S=\{a_0,\dots,a_{n-1}\}\subset \mathbb{Z}$ atau $\mathbb{R}$)?
  \item Bagaimana prosedur konstruksi eksplisit (berbasis parameter) untuk menghasilkan keluarga persegi Latin komutatif berordo $n$ dalam kerangka max-plus, dan bagaimana kompleksitas verifikasinya?
  \item Sifat-sifat aljabar apa (misalnya identitas netral, idempoten, isotopi/autotopi) yang tetap, berubah, atau muncul khas ketika persegi Latin dibangun melalui operasi max-plus dan normalisasi terkait?
  \item Untuk orde kecil hingga menengah, seperti apa hasil enumerasi dan klasifikasi (hingga isotopi/isosimetri) persegi Latin komutatif di bawah skema konstruksi yang diusulkan?
\end{enumerate}

\section{Batasan Masalah}
Batasan masalah dalam penelitian ini adalah sebagai berikut :
\begin{enumerate}
  \item Himpunan simbol $S$ yang dipertimbangkan adalah himpunan hingga $S=\{a_0,\dots,a_{n-1}\}\subset \mathbb{Z}$ atau $\mathbb{R}$, dengan operasi max-plus $\oplus$ dan $\otimes$ dibatasi melalui suatu pemetaan normalisasi $\varphi: S\times S\to S$ yang dipilih agar sifat Latin terpenuhi.
  \item Fokus pada persegi Latin komutatif, yaitu tabel yang simetris terhadap diagonal utama ($L(i,j)=L(j,i)$) dan merefleksikan komutativitas operasi yang melandasinya.
  \item Analisis teoretis dilengkapi studi kasus dan verifikasi komputasional untuk orde kecil hingga menengah; penetapan batas orde mengikuti kebutuhan bukti/komputasi dan ketersediaan waktu.
  \item Pembahasan diarahkan pada aspek struktur dan konstruksi; aplikasi praktis (misal implementasi industri skala besar) tidak menjadi fokus utama, kecuali disinggung sebagai motivasi.
  \item Tidak semua varian aljabar tropis ditinjau; perhatian utama pada semiring max-plus idempoten.
\end{enumerate}

\section{Tujuan}
\begin{enumerate}
  \item Merumuskan definisi formal persegi Latin komutatif atas aljabar max-plus pada himpunan hingga beserta skema normalisasi yang konsisten.
  \item Menurunkan kondisi perlu/cukup (ketika memungkinkan) bagi eksistensi dan komutativitas persegi Latin yang dibangun dari struktur max-plus.
  \item Mengembangkan metode konstruksi eksplisit dan prosedur verifikasi komputasional untuk menghasilkan dan menguji persegi Latin komutatif berordo $n$.
  \item Melakukan studi kasus, contoh, serta (bila dimungkinkan) klasifikasi untuk orde kecil hingga menengah, berikut analisis sifat-sifat aljabarnya.
  \item Mendiskusikan implikasi dan potensi aplikasi hasil kajian pada penjadwalan tropis dan desain kombinatorial.
\end{enumerate}

\section{Manfaat}
\begin{enumerate}
  \item Kontribusi teoretis pada persilangan antara desain kombinatorial (persegi Latin/kuasigrup) dan aljabar tropis (max-plus), berupa definisi, konstruksi, dan karakterisasi baru.
  \item Menyediakan pendekatan konstruktif yang dapat direplikasi untuk membangun persegi Latin komutatif di domain yang tidak konvensional (tropis/idempoten).
  \item Bahan awal untuk pengembangan alat verifikasi komputasional persegi Latin dalam kerangka max-plus, berguna untuk eksplorasi orde yang lebih besar.
  \item Menambah alternatif model dan inspirasi aplikasi pada penjadwalan, optimisasi diskret, dan desain eksperimen dalam perspektif tropis.
  \item Menjadi referensi dan bahan ajar untuk topik terkait aljabar max-plus, kombinatorika, dan struktur aljabar non-klasik.
\end{enumerate}

\pagebreak
\chapter{TINJAUAN PUSTAKA}
\section{Hasil Penelitian Terdahulu}


\section{Latin Squares}


\section{Aljabar Max-Plus}
Aljabar max-plus didefinisikan pada $\mathbb{R}\cup\{-\infty\}$ dengan $a\oplus b=\max\{a,b\}$ dan $a\otimes b=a+b$. Struktur ini merupakan semiring idempoten dan banyak digunakan pada penjadwalan, optimisasi, dan sistem kejadian diskret. Sejumlah kajian dalam kerangka tropis juga muncul di bidang kriptografi dan keamanan protokol, yang memperlihatkan relevansi struktur idempoten dalam konteks komputasi dan implementasi \cite{stickelImpl,tropicalStickelSecurity}.

\section{Celah Riset dan Arah Penelitian}
Keterhubungan langsung antara persegi Latin (terutama yang komutatif) dengan konstruksi berbasis aljabar max-plus relatif belum dieksplorasi luas. Tantangan utama ialah memastikan hasil operasi biner kembali ke himpunan simbol hingga. Untuk itu, diperlukan skema normalisasi $\varphi: S\times S\to S$ yang menjaga sifat Latin (keunikan tiap baris/kolom) sekaligus komutativitas. Pendekatan ini, disertai verifikasi berbantuan komputer dan perbandingan dengan praktik di ranah tropis lainnya \cite{tropicalStickelSecurity}, menjadi dasar arah penelitian ini, dengan acuan penulisan dari karya-karya terkait \cite{zulfikarTA}.

\section{Ringkasan}
Tinjauan menunjukkan peluang kontribusi pada definisi, konstruksi, dan karakterisasi persegi Latin komutatif di lingkungan max-plus, serta potensi aplikasi pada penjadwalan tropis dan desain kombinatorial.



\pagebreak
\chapter{METODOLOGI}

Pada bab ini akan dijelaskan langkah-langkah pelaksanaan penelitian tugas akhir yang dilengkapi dengan alur proses serta jadwal kegiatan yang disusun selama penelitian.
% \section{Studi Literatur}
% Pada tahap ini dilakukan pengumpulan referensi yang relevan dengan penelitian tugas akhir ini. Sumber referensi yang digunakan adalah buku, tugas akhir, tesis, paper, serta artikel yang mendukung topik penelitian. Referensi yang terkait dengan penelitian ini adalah mengenai kendali rudal menggunakan metode NMPC dengan pendekatan fungsi laguerre.

\section{Pengkajian Model Matematika Rudal}

\section{Pendiskritan Model}

\section{Pembentukan Kendali Rudal Menggunakan Laguerre-NMPC}

\section{Simulasi dan Analisis Hasil Simulasi}

\section{Penarikan Kesimpulan dan Saran}

\section{Penulisan Laporan Tugas Akhir}

\begin{figure}[htbp]
    \centering
    \includegraphics[width=\linewidth]{foto/Diagram Alir Penelitian Blue.png}
    \caption{Diagram Alir Penelitian}
    \label{fig:diagram penelitian}
\end{figure}

\pagebreak
\chapter*{JADWAL PENELITIAN}
Berikut jadwal pelaksanaan tahap-tahap penelitian tugas akhir yang akan dilakukan selama 3 bulan sesuai dengan metode penelitian. \vspace{0.5cm}
\begin{table}[htbp]
\centering
\caption{Jadwal Pelaksanaan Penelitian Tugas Akhir}
\renewcommand{\arraystretch}{1.5}
\begin{tabular}{|C{0.6cm}|L{6.5cm}|C{0.25cm}|C{0.25cm}|C{0.25cm}|C{0.25cm}|C{0.25cm}|C{0.25cm}|C{0.25cm}|C{0.25cm}|C{0.25cm}|C{0.25cm}|C{0.25cm}|C{0.25cm}|}	\hline
	&&\multicolumn{12}{c|}{\textbf{BULAN}}\\\cline{3-14}
	\multicolumn{1}{|c|}{\textbf{NO}}&\multicolumn{1}{c|}{\textbf{NAMA KEGIATAN}}&\multicolumn{4}{c|}{1}&\multicolumn{4}{c|}{2}&\multicolumn{4}{c|}{3}\\\cline{3-14}
	&&1&2&3&4&1&2&3&4&1&2&3&4\\\cline{1-14}
	1&Pengkajian model matematika rudal &\cellcolor{blue!}&\cellcolor{blue!}&&&&&&&&&&\\\hline
	2&Pendiskritan model&&\cellcolor{blue!}&\cellcolor{blue!}&&&&&&&&&\\\hline
	3&Pembentukan kendali rudal menggunakan Laguerre-NMPC &&&\cellcolor{blue!}&\cellcolor{blue!}&&&&&&&&\\\hline
	4&Pembuatan program menggunakan software MATLAB R2024b &&&&\cellcolor{blue!}&\cellcolor{blue!}&\cellcolor{blue!}&\cellcolor{blue!}&&&&&\\\hline
	5&Simulasi dan analisis hasil simulasi&&&&&&&\cellcolor{blue!}&\cellcolor{blue!}&\cellcolor{blue!}&&&\\\hline
	6&Penarikan kesimpulan dan saran&&&&&&&&&\cellcolor{blue!}&\cellcolor{blue!}&&\\\hline
	7&Penulisan laporan tugas akhir&&&&&&&&&&\cellcolor{blue!}&\cellcolor{blue!}&\cellcolor{blue!}\\\hline
\end{tabular}
\end{table}

%%%%%%%%%%%%%%%%%%%%%%%%  Dapus  %%%%%%%%%%%%%%%5%%%%%%%%%%akan

\pagebreak
\DaftarPustaka
%%%%%%%%%%%%%%%%%%%%%%%%  Dapus  %%%%%%%%%%%%%%%5%%%%%%%%%%

%%%%%%%%%%%%%%%%%%%%%%%%  Lampiran  %%%%%%%%%%%%%%%5%%%%%%%%%%

%\pagebreak
%\DaftarLampiran
%%%%%%%%%%%%%%%%%%%%%%%%  Lampiran  %%%%%%%%%%%%%%%5%%%%%%%%%%



\end{document}