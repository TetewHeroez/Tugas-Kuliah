\documentclass{article}
\usepackage{amsmath,amssymb,amsfonts,amsthm}
\usepackage{multicol}
\usepackage{multirow}
\usepackage{mathtools}
\usepackage{soul}
\usepackage{hyperref}
\hypersetup{
    colorlinks=true,
    linkcolor=blue,
    filecolor=magenta,      
    urlcolor=cyan,
    pdftitle={Overleaf Example},
    pdfpagemode=FullScreen,
    }
\usepackage{color}
\usepackage[table]{xcolor}
\usepackage{etoolbox}
\usepackage{multicol}
\usepackage{multirow}
\usepackage{fancyhdr}
\usepackage{graphicx}
\usepackage{tcolorbox}
\usepackage{array}
\usepackage{amsthm}
\usepackage{titlesec}
\usepackage{tikz, tkz-euclide}
  \usetikzlibrary{arrows.meta,calc}
  \newcommand\rightAngle[4]{
  \pgfmathanglebetweenpoints{\pgfpointanchor{#2}{center}}{\pgfpointanchor{#1}{center}}
  \coordinate (tmpRA) at ($(#2)+(\pgfmathresult+45:#4)$);
  \draw[red!60!black,thick] ($(#2)!(tmpRA)!(#1)$) -- (tmpRA) -- ($(#2)!(tmpRA)!(#3)$);
}
\renewcommand{\baselinestretch}{1.2}

\titleformat*{\section}{\large\bfseries}
\titleformat*{\subsection}{\normalsize\bfseries}

\newtheorem{theorem}{Theorem}
\newtheorem*{teorema}{Teorema}
\newtheorem*{definisi}{Definisi}
\theoremstyle{definition}
\newtheorem*{bukti}{Bukti}

\newcommand{\Arg}{\text{Arg}}
\newcommand{\R}{\mathbb{R}}
\newcommand{\C}{\mathbb{C}}
\newcommand{\N}{\mathbb{N}}
\newcommand{\Z}{\mathbb{Z}}

\newtcolorbox{solution}[1][]{
    colback=blue!5!white, 
    colframe=blue!75!black,
    fonttitle=\bfseries, 
    colbacktitle=blue!85!black,
    title=Solusi,
    #1
}

\begin{document}
\fancyhead[L]{\textit{Teosofi Hidayah Agung}}
\fancyhead[R]{\textit{5002221132}}
\pagestyle{fancy}
\begin{enumerate}
  \item Dapatkan determinan dari matriks
  \[
  A = \begin{bmatrix}
  2 & 1 & 0 & 0 \\
  0 & 1 & 1 & 2 \\
  0 & 1 & -2 & 2 \\
  2 & 3 & 0 & 9
  \end{bmatrix}
  \]
  

  \item Dengan metode Cramer, selesaikan untuk $y$ untuk sistem persamaan berikut:
  \[
  \begin{aligned}
    2x& - 2y& + z &= 3 \\
    -x& + 2y& + 3z &= 4 \\
    3x& - 2y& &= 5
  \end{aligned}
  \]

  \item Diberikan $v$ adalah vektor eigen dari matriks persegi $A$ yang bersesuaian dengan nilai eigen $\lambda$.
  \begin{enumerate}
    \item Tuliskan persamaan yang menghubungkan $A$, $v$ dan $\lambda$.
    \item Selidiki apakah $v$ juga vektor eigen dari $A^k$ untuk $k \geq 2$? Jelaskan jawaban Anda.
  \end{enumerate}

  \item Diberikan matriks
  \[
  A = \begin{bmatrix}
  2 & 3 \\
  4 & 1
  \end{bmatrix}
  \]
  \begin{enumerate}
    \item Dapatkan nilai eigen dan vektor eigen dari matriks $A$.
    \item Dapatkan matriks $P$ sedemikian hingga $D = P^{-1}AP$ dengan $D$ matriks diagonal.
  \end{enumerate}
\end{enumerate}
\end{document}