\documentclass[10pt]{article}
\usepackage{amsmath, amssymb}
\usepackage[margin=1in]{geometry}
\usepackage{fancyhdr}

\begin{document}

\begin{center}
    \Large\textbf{Quiz 1 Aljabar Linier Elementer}
\end{center}
\pagenumbering{gobble}
\vspace{0.5cm}

\begin{enumerate}
    \item Diberikan suatu SPL
    \begin{align*}
        x_1 + 3x_2 + x_3 + x_4 &= a \\
        2x_1 + 5x_2 + 2x_3 + x_4 &= b \\
        x_1 + 3x_2 + 8x_3 + 9x_4 &= c \\
        x_1 + 3x_2 + 2x_3 + 2x_4 &= d
    \end{align*}
    dengan $abcd$ merupakan 4 digit terakhir NRP anda. Tentukan solusi SPL tersebut dengan menggunakan \textbf{Metode Eliminasi Gauss-Jordan}.
    
    \item Diberikan suatu barisan Fibonacci $F_{n+2} = F_{n+1} + F_n$ untuk $n \geq 1$ dengan dua suku awal $F_1 = F_2 = 1$. Sebagai contoh, 5 suku awal dari barisan Fibonacci adalah
    \[
    1, 1, 2, 3, 5.
    \]
    Selanjutnya didefinisikan matriks
    \[
    A = \begin{bmatrix}
    F_2 & F_1 \\
    F_1 & 0
    \end{bmatrix}.
    \]
    Buktikan bahwa
    \[
    A^n = \begin{bmatrix}
    F_{n+1} & F_n \\
    F_n & 0
    \end{bmatrix}
    \quad \text{untuk setiap } n \geq 1.
    \]

    \item Tentukan semua matriks diagonal $A$ dengan ukuran $3 \times 3$ yang memenuhi
    \[
    A^2 - 3A - 4I = 0.
    \]

    \item Diberikan suatu matriks
    \[
    A = \begin{bmatrix}
    a_1 & a_2 & a_3 & a_4 \\
    1 & a_5 & a_6 & a_7 \\
    2 & 3 & a_8 & a_9 \\
    4 & 5 & 6 & a_{10}
    \end{bmatrix}
    \]
    dengan $a_1, a_2, \ldots, a_{10}$ merupakan digit-digit NRP anda dari depan.
    \begin{itemize}
        \item Hitung $\det(A)$.
        \item Apakah $A$ mempunyai invers? Jelaskan jawaban anda.
    \end{itemize}

    \item Selesaikan SPL pada soal no 1 dengan menggunakan \textbf{Metode Cramer}.

    \item Buktikan bahwa jika $A$ mempunyai invers, maka $\operatorname{adj}(A)$ mempunyai invers dan
    \[
    (\operatorname{adj}(A))^{-1} = \frac{A}{\det(A)} = \operatorname{adj}(A^{-1}).
    \]
\end{enumerate}

\vspace{1cm}

\noindent\textbf{Catatan:}
\begin{itemize}
    \item Kerjakan \textbf{hanya 5 soal}
    \item Setiap soal berbobot 20
    \item Kerjakan dengan penuh kejujuran
\end{itemize}
\newpage


\end{document}
