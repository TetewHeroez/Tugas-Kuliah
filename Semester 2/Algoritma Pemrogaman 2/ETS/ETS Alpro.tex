\documentclass{article}
\usepackage{amsmath}
\usepackage{amssymb}
\usepackage{enumitem}
\usepackage{algorithm}
\usepackage{listings}
\usepackage{color}
\usepackage[T1]{fontenc}
\usepackage{etoolbox}
\usepackage{multicol}
\usepackage{geometry}
\usepackage{tikz, pgfplots, tkz-euclide,calc}
    \usetikzlibrary{patterns,snakes,shapes.arrows,3d}
\geometry{
    total = {160mm, 237mm},
    left = 25mm,
    right = 35mm,
    top = 30mm,
    bottom = 30mm,
  }

\definecolor{pblue}{rgb}{0.13,0.13,1}
\definecolor{pgreen}{rgb}{0,0.5,0}
\definecolor{pred}{rgb}{0.9,0,0}
\definecolor{pgrey}{rgb}{0.46,0.45,0.48}

\usepackage{listings}
\lstset{language=Java,
  showspaces=false,
  showtabs=false,
  breaklines=true,
  showstringspaces=false,
  breakatwhitespace=true,
  commentstyle=\color{pgreen},
  keywordstyle=\color{pblue},
  stringstyle=\color{pred},
  basicstyle=\small\ttfamily,
  frame=none,
}
\newcommand{\jawab}{\textbf{Jawab}:}

\begin{document}
\pagenumbering{gobble}
    \setcounter{section}{3}
    \setcounter{subsection}{1}
    \begin{tabular}{|lcl|}
     \hline
     Nama&:&Teosofi Hidayah Agung\\
     NRP&:&5002221132\\
     \hline
    \end{tabular}
  \begin{enumerate}
    \item (Kompleksitas)\\
    Nilai apa yang di-return oleh fungsi berikut? Nyatakan jawabanmu sebagai fungsi n.\\~\\
    Tentukan kompleksitas dari algoritma tersebut dalam notasi Big Oh.
    \begin{lstlisting}
function mystery(n)
r:=0
for i:= 1 to n - 1 do
    for j := i + 1 to n do
        for k := 1 to j do
            r := r + 1
return(r)
    \end{lstlisting}
    \jawab\\
    Nilai yang di-return fungsi tersebut adalah perhitungan berapa banyaknya loop/perulanagan 
    pada fungsi tersebut. Misalkan $T(n)$ adalah fungsi yang mempresentasikan algoritma 
    di atas.
    \begin{flalign*}
      T(n)&=\sum_{i=1}^{n-1}\sum_{j=i+1}^{n}\sum_{k=1}^{j}1=\sum_{i=1}^{n-1}\sum_{j=i+1}^{n}j=\sum_{i=1}^{n-1}\left.\Delta^{-1}j^{(1)}\right|^{n+1}_{i+1}=\sum_{i=1}^{n-1}\left.\frac{j^{(2)}}{2}\right|^{n+1}_{i+1}=\sum_{i=1}^{n-1}\left.\frac{j(j-1)}{2}\right|^{n+1}_{i+1}&\\
      &=\sum_{i=1}^{n-1}\left(\frac{n(n+1)}{2}-\frac{i(i+1)}{2}\right)=\sum_{i=1}^{n-1}\frac{n(n+1)}{2}-\sum_{i=1}^{n-1}\frac{i(i+1)}{2}&\\
      &=\frac{1}{2}\left[n(n+1)(n-1)\right]-\frac{1}{2}\left[\sum_{i=1}^{n-1}i^2+\sum_{i=1}^{n-1}i\right]&\\
      &=\frac{1}{2}\left[n^3-n\right]-\frac{1}{2}\left[\frac{n(n-1)(2n-1)}{6}+\frac{n(n-1)}{2}\right]&\\\
      &=\frac{1}{2}\left[n^3-n-\frac{2n^3-3n^2+n}{6}-\frac{n^2-n}{2}\right]&\\
      &=\frac{1}{12}\left[6n^3-6n-2n^3+3n^2-n-3n^2+3n\right]&\\
      &=\frac{1}{12}\left[4n^3-4n\right]=\boxed{\frac{1}{3}[n^3-n]\in O(n^3)}
    \end{flalign*}

    \item (Algoritma)\\
    Kontruksi algoritma untuk mencari jarak terdekat antara dua elemen dalam suatu array bilangan.
    \begin{lstlisting}
MinDistance(A[0..n - 1])
      //input: Array A[0.. n - 1] of numbers
      //Output: Jarak minimum antara elemen-elemen di array
    \end{lstlisting}
    \jawab\\
    Algoritma yang akan dibuat adalah mengecek satu persatu selisih dari dua elemen kemudian 
    memilih jarak terkecil antar salah satu pasangan elemen tersebut.
    \begin{lstlisting}
public static void MinDistance(int[] array){
  int min = Math.abs(array[0]-array[1]);
  for (int i = 0; i < array.length; i++) {
      for (int j = 0; j < i; j++) {
          int distance = Math.abs(array[j]-array[i]);
          if (distance<min) {
              min=distance;
          }
      }
  }
  System.out.println("Jarak terdekat antara dua elemen adalah "+min);
}//jika lupa syntax nilai mutlak, dapat juga menggunakan definisi nilai mutlak
    \end{lstlisting}
    \item (\textit{Searching})\\
    Buatlah coding dengan menggunakan algoritma binary search untuk mendapatkan nilai 
    pembulatan dari akar bilangan bulat antara 0 dan 9.\\
    petunjuk: gunakan method \texttt{int findsqrt(int x)} untuk menampilkan output:
    \begin{lstlisting}
sqrt(0) = 0
sqrt(1) = 1
sqrt(2) = 1
sqrt(3) = 1
sqrt(4) = 2
sqrt(5) = 2
sqrt(6) = 2
sqrt(7) = 2
sqrt(8) = 2
sqrt(9) = 3
    \end{lstlisting}
    \jawab\\
    Idenya adalah dengan menggunakan teori bilangan dimana ketika kita membagi bilangan x dengan 
    2 lalu dikuadratkan hasilnya, maka akan ada 3 kasus yaitu hasil kuadrat sama dengan, kurang 
    dari, atau lebih dari x.
    \begin{lstlisting}
public static void main(String[] args) {
    for (int i = 0; i <= 9; i++) {
        System.out.println("sqrt("+i+") = "+findsqrt(i));
    }
}
public static int findsqrt(int x){
    int mid=0;
    int start=0;
    int end=x;
    int ans=0;
    while (start<=end) {
        mid = (start+end)/2;
        int square = mid*mid;
        if(square==x){
           return mid; 
        }
        else if (square<x) {
            start=mid+1;
            ans=mid;
        }
        else{
            end=mid-1;
        }
    }
return ans;
}
    \end{lstlisting}

    \item (Sorting)\\
    Lengkapilah potongan program dibawah ini untuk mendapatkan tiga bilangan terbesar dari 
    sebuah array $[6,4,3,4,8,2,5]$ yaitu $[5,6,8]$ berdasarkan algoritma \textit{sorting} 
    berikut:
    \begin{enumerate}
      \item \textit{Insertion sort}
      \begin{lstlisting}[numbers=left,frame=single,numbersep= -8pt]
    public class FindThreeLargest{
      public static void findThreeLargest(int[] arr)
        {
        int n = arr.length;
        int largest1 = Integer.MIN_VALUE;
        int largest2 = Integer.MIN_VALUE;
        int largest3 = Integer.MIN_VALUE;

        for (int i = 0; i < n; i++) {
            int current = ...;

            if(...){
                largest3 = largest2;
                largest2 = largest1;
                largest1 = current;
            } else if (...&&...) {
                largest3 = largest2;
                largest2 = current;
            } else if (...&&...) {
                largest3 = current;
            }
        }
        System.out.println("Three largest elements;");
        System.out.println(largest1);
        System.out.println(largest2);
        System.out.println(largest3);
      }
      public static void main(String[] args) {
        int[] arr = {6,4,3,1,8,2,5};
        findThreeLargest(arr);
      }
    }
      \end{lstlisting}
      \jawab
      \begin{lstlisting}[numbers=left,frame=single,numbersep= -8pt,firstnumber = 10]
      int current = arr[i];
      \end{lstlisting}
      \begin{lstlisting}[numbers=left,frame=single,numbersep= -8pt,firstnumber = 12]
      if(current>largest1)
      \end{lstlisting}
      \begin{lstlisting}[numbers=left,frame=single,numbersep= -8pt,firstnumber = 16]
      else if (current<largest1 && current>largest2)
      \end{lstlisting}
      \begin{lstlisting}[numbers=left,frame=single,numbersep= -8pt,firstnumber = 19]
        else if (current<largest2 && current>largest3)
      \end{lstlisting}
      \item \textit{Selection sort}
      \begin{lstlisting}[numbers=left,frame=single,numbersep= -8pt]
    public class FindThreeLargest{

      public static void findThreeLargest(int[] arr)
      {
          int n = arr.length;
          int largest1 = Integer.MIN_VALUE;
          int largest2 = Integer.MIN_VALUE;
          int largest3 = Integer.MIN_VALUE;
  
          for (int i = 0; i < n; i++) {
              int largestIndex = i;
              for (int j = i+1; j < n; j++) {
                  if (...) {
                   largestIndex = ...;
                  }
              }
              if (...) {
                  largest3 = largest2;
                  largest2 = largest1;
                  largest1 = arr[largestIndex];
              } else if (...&&...) {
                  largest3 = largest2;
                  largest2 = arr[largestIndex];
              } else if (...&&...) {
                  largest3 = arr[largestIndex];
              }
  
              int temp = arr[i];
              arr[i] = arr[largestIndex];
              arr[largestIndex] = temp;
          }
          System.out.println("Three largest elements;");
          System.out.println(largest1);
          System.out.println(largest2);
          System.out.println(largest3);
      }
      public static void main(String[] args) {
          int[] arr = {6,4,3,1,8,2,5};
          findThreeLargest(arr);
      }
    }
      \end{lstlisting}
      \jawab
      \begin{lstlisting}[numbers=left,frame=single,numbersep= -8pt,firstnumber = 13]
      if (j>largestIndex)
        \end{lstlisting}
        \begin{lstlisting}[numbers=left,frame=single,numbersep= -8pt,firstnumber = 14]
      largestIndex = j;
        \end{lstlisting}
        \begin{lstlisting}[numbers=left,frame=single,numbersep= -8pt,firstnumber = 18]
      if (arr[largestIndex]>largest1)
        \end{lstlisting}
        \begin{lstlisting}[numbers=left,frame=single,numbersep= -8pt,firstnumber = 22]
      else if (arr[largestIndex]<largest1 && arr[largestIndex]>largest2)
        \end{lstlisting}
        \begin{lstlisting}[numbers=left,frame=single,numbersep= -8pt,firstnumber = 25]
      else if (arr[largestIndex]<largest2 && arr[largestIndex]>largest3)
        \end{lstlisting}
    \end{enumerate}
    \item (\textit{Exception Handling})\\
    Perhatikan potongan program berikut:
    \begin{lstlisting}[numbers=left,frame=single,numbersep= -8pt]
    String a;
    String[] nama = new String[1];
    try {
        statment1;
        nama[2] = "Linda";
        statment2;
    } 
    catch (ArithmeticException ex1) {
        System.out.println(ex1);
    } 
    catch (Exception ex2){
        System.out.println(ex2);
    }
    finally {
        System.out.println("bakso");
    }
    System.out.println("rawon");
    \end{lstlisting}
    Apa output dari program tersebut apabila terjadi kondisi seperti ini, jelaskan:
    \begin{enumerate}[label=\alph*.]
      \item Jika statment1 dan statment2 tidak terjadi exception (pernyatannya benar)\\
      \jawab
      \begin{lstlisting}[frame=single]
java.lang.ArrayIndexOutOfBoundsException: Index 2 out of bounds for length 1
bakso
rawon
      \end{lstlisting}
      \item Jika statment1 diganti dengan \texttt{a=3;} dan statment2 tidak terjadi exception\\
      \jawab\\
      Program error saat di-running
      \item Jika statment1 tidak terjadi exception dan statment2 diganti dengan\\
      \texttt{System.out.println(1/0);}\\
      \jawab
      \begin{lstlisting}[frame=single]
java.lang.ArrayIndexOutOfBoundsException: Index 2 out of bounds for length 1
bakso
rawon
      \end{lstlisting}
      \item Jika statment1 diganti dengan \texttt{System.out.println(1/0);} dan statment2 tidak 
      terjadi exception\\
      \jawab
      \begin{lstlisting}[frame=single]
java.lang.ArithmeticException: / by zero
bakso
rawon
      \end{lstlisting}
    \end{enumerate}
  \end{enumerate}
\end{document}