\documentclass{article}
\usepackage{graphicx}
\usepackage{amsmath, amsfonts, amssymb, amsthm}
\usepackage{tikz, pgfplots, tkz-euclide,calc}
    \usetikzlibrary{patterns,snakes,shapes.arrows}
\usepackage{fancyhdr}
\usepackage{enumerate,enumitem}
\usepackage{cancel}
\usepackage{varwidth}

% TAMBAHKAN PACKAGE SENDIRI KALAU KURANG

\usepackage{geometry}
\geometry{
	total = {160mm, 237mm},
	left = 25mm,
	right = 20mm,
	top = 30mm,
	bottom = 30mm,
}
\title{Script Video Asdos}
\author{Teosofi Agung}
\date{November 2023}

\begin{document}

\maketitle

\begin{enumerate}
    \item Nyatakan bilangan kompleks berikut ke dalam bentuk kutub dan gambarlah dalam bidang kompleks:
    \begin{enumerate}
        \item[(e)] $z=-\sqrt{6}-\sqrt{2}i$\\
        \textbf{Penyelesaian. } Modulus dari $z$ adalah
        \[r=|-\sqrt{6}-\sqrt{2}i|=\sqrt{(-\sqrt{6})^2+(-\sqrt{2})^2}=\sqrt{6+2}=2\sqrt{2}\]
        $\theta$ atau Argumen dari $z$ adalah
        \[\tan^{-1}\left(\frac{b}{a}\right)=\tan^{-1}\left(\frac{-\sqrt{2}}{-\sqrt{6}}\right)=\tan^{-1}\left(\frac{1}{\sqrt{3}}\right)=\left\{\frac{\pi}{6},\frac{7\pi}{6}\right\}\]
        Karena $a$ dan $b$ keduanya negatif maka $Arg(z)$ atau $\theta$ adalah $\frac{7\pi}{6}$.\\
        $\therefore$ Bentuk kutubnya adalah $z=2\sqrt{2}\left(\cos{\frac{7\pi}{6}}+i\sin{\frac{7\pi}{6}}\right)$\\

        \begin{tikzpicture}[scale=1.5]
            \draw[->] (-3.2,0) -- (1.2,0) node[right] {$Re(z)$};
            \draw[->] (0,-2.2) -- (0,1.2) node[above] {$Im(z)$};

            \coordinate (o) at (0,0);
            \coordinate (z) at ({-sqrt(6)},{-sqrt(2)});
            \coordinate (a) at ({-sqrt(6)},0);
            \coordinate (b) at (0,{-sqrt(2)});
            \coordinate (a') at ({sqrt(6)},0);

            \draw (o)--(z) node[below left] {$z$};
            \node[circle,fill=black,inner sep=0pt,minimum size=7pt] at (z) {};

            \tkzMarkAngle (a',o,z);
            \tkzLabelAngle[pos=0.6](a',o,z) {$210^\circ$};

            \tkzLabelSegment[above left](o,z) {$2\sqrt{2}$};
            \draw[dashed] (z)--(a) node[above] {$-\sqrt{6}$};
            \draw[dashed] (z)--(b) node[right] {$-\sqrt{2}$};
        \end{tikzpicture}
    \end{enumerate}

    \item Nyatakan bilangan kompleks berikut ke dalam bentuk $z=a+bi$ dan gambarlah dalam bidang kompleks:
    \begin{enumerate}
        \item[(f)] $2e^{-\frac{\pi}{4}i}$\\
        \textbf{Penyelesaian. } Ingat bahwa \framebox{$re^{\theta i}=r\textrm{cis}(\theta)$}
        \[2e^{-\frac{\pi}{4}i}=2\textrm{cis}\left(-\frac{\pi}{4}\right)=2\left(\cos\left(-\frac{\pi}{4}\right)+i\sin\left(-\frac{\pi}{4}\right)\right)=2\left(\cos\left(\frac{\pi}{4}\right)-i\sin\left(\frac{\pi}{4}\right)\right)=2\left(\frac{1}{2}\sqrt{2}-\frac{1}{2}\sqrt{2}i\right)\]
        $\therefore 2e^{-\frac{\pi}{4}i}=\sqrt{2}-\sqrt{2}i$\\
        \begin{tikzpicture}[scale=1.5]
            \draw[->] (-1.2,0) -- (2.2,0) node[right] {$Re(z)$};
            \draw[->] (0,-2.2) -- (0,1.2) node[above] {$Im(z)$};

            \coordinate (o) at (0,0);
            \coordinate (z) at ({sqrt(2)},{-sqrt(2)});
            \coordinate (a) at ({sqrt(2)},0);
            \coordinate (b) at (0,{-sqrt(2)});
            \coordinate (a') at ({sqrt(2)},0);

            \draw (o)--(z) node[below right] {$z$};
            \node[circle,fill=black,inner sep=0pt,minimum size=7pt] at (z) {};

            \tkzMarkAngle (z,o,a');
            \tkzLabelAngle[pos=0.6](z,o,a') {$-45^\circ$};

            \tkzLabelSegment[below left](o,z) {$2$};
            \draw[dashed] (z)--(a) node[above] {$\sqrt{2}$};
            \draw[dashed] (z)--(b) node[left] {$-\sqrt{2}$};
        \end{tikzpicture}
    \end{enumerate}

    \item Diberikan $z_1=2\left(\cos{\frac{\pi}{4}}+i\sin{\frac{\pi}{4}}\right)$ dan $z_2=3\left(\cos{\frac{\pi}{6}}+i\sin{\frac{\pi}{6}}\right)$.\\
    \begin{enumerate}
        \item $z_1z_2$\\
        \textbf{Penyelesaian. } \textit{Teorema De Moivre} \framebox{$z_1z_2=r_1r_2(\cos(\theta_1+\theta_2)+i\sin(\theta_1+\theta_2))$}
        \[z_1z_2=2\cdot3\left(\cos{\left(\frac{\pi}{4}+\frac{\pi}{6}\right)}+i\sin{\left(\frac{\pi}{4}+\frac{\pi}{6}\right)}\right)=6\left(\cos{\left(\frac{5\pi}{12}\right)}+i\sin{\left(\frac{5\pi}{12}\right)}\right)\]

        \item $\frac{z_1}{z_2}$\\
        \textbf{Penyelesaian. } \textit{Teorema De Moivre} \framebox{$\frac{z_1}{z_2}=\frac{r_1}{r_2}(\cos(\theta_1-\theta_2)+i\sin(\theta_1-\theta_2))$}
        \[\frac{z_1}{z_2}=\frac{2}{3}\left(\cos{\left(\frac{\pi}{4}-\frac{\pi}{6}\right)}+i\sin{\left(\frac{\pi}{4}-\frac{\pi}{6}\right)}\right)=\frac{2}{3}\left(\cos{\left(\frac{\pi}{12}\right)}+i\sin{\left(\frac{\pi}{12}\right)}\right)\]
    \end{enumerate}

    \item Hitunglah
    \begin{enumerate}
        \item[(e)]$\left(\frac{1+\sqrt{3}i}{1-\sqrt{3}i}\right)^{10}$\\
        \textbf{Penyelesaian. } Ubah kedalam bentuk kutub  \framebox{$1+\sqrt{3}i=2\textrm{cis}\left(\frac{\pi}{3}\right)$} dan
        \framebox{$1-\sqrt{3}i=2\textrm{cis}\left(-\frac{\pi}{3}\right)$}. Sehinggan didapatkan
        \[\left(\frac{1+\sqrt{3}i}{1-\sqrt{3}i}\right)^{10}=\left(\frac{2\textrm{cis}\left(\frac{\pi}{3}\right)}{2\textrm{cis}\left(-\frac{\pi}{3}\right)}\right)^{10}=\left(\textrm{cis}\left(\frac{\pi}{3}-\left(-\frac{\pi}{3}\right)\right)\right)^{10}=\left(\textrm{cis}\left(\frac{2\pi}{3}\right)\right)^{10}=\textrm{cis}\left(\frac{20\pi}{3}\right)=\textrm{cis}\left(\frac{2\pi}{3}+6\pi\right)\]
        \[=\textrm{cis}\left(\frac{2\pi}{3}\right)=\cos\left(\frac{2\pi}{3}\right)+i\sin\left(\frac{2\pi}{3}\right)=\frac{1}{2}+\frac{1}{2}\sqrt{3}i\]
        \item[(f)] $\left(\frac{\sqrt{3}-i}{\sqrt{3}+i}\right)^4\left(\frac{1-i}{1+i}\right)^5$\\
        \textbf{Penyelesaian. } Dengan cara yang sama seperti 4e, Didapatkan
        \[\left(\frac{\sqrt{3}-i}{\sqrt{3}+i}\right)^4\left(\frac{1-i}{1+i}\right)^5=\left(\frac{2\textrm{cis}\left(\frac{11\pi}{6}\right)}{2\textrm{cis}\left(\frac{\pi}{6}\right)}\right)^4\left(\frac{2\textrm{cis}\left(\frac{7\pi}{4}\right)}{2\textrm{cis}\left(\frac{\pi}{4}\right)}\right)^5=\left(\textrm{cis}\left(\frac{5\pi}{3}\right)\right)^4\left(\textrm{cis}\left(\frac{3\pi}{2}\right)\right)^5=\textrm{cis}\left(\frac{20\pi}{3}\right)\textrm{cis}\left(\frac{15\pi}{2}\right)\]
        \[=\textrm{cis}\left(\frac{2\pi}{3}\right)\textrm{cis}\left(\frac{3\pi}{2}\right)=\textrm{cis}\left(\frac{2\pi}{3}+\frac{3\pi}{2}\right)=\textrm{cis}\left(\frac{5\pi}{6}\right)=\cos\left(\frac{5\pi}{6}\right)+i\sin\left(\frac{5\pi}{6}\right)=-\frac{1}{2}\sqrt{3}+\frac{1}{2}i\]
    \end{enumerate}

    \item[7.] Carilah semua bilangan kompleks $z$ yang memenuhi persamaan berikut:
    \begin{enumerate}
        \item[(c)]$z^2+(-2+i)z+3-i=0$\\
        \textbf{Penyelesaian. } Dengan menggunakan rumus abc didapatkan:
        \begin{flalign*}
            z&=\frac{2-i\pm\sqrt{(2-i)^2-4(1)(3-i)}}{2(1)}=\frac{2-i\pm\sqrt{3-4i-12+4i}}{2}=\frac{2-i\pm\sqrt{-9}}{2}=\frac{2-i\pm3i}{2}\\
            \therefore&\: z_1=1+i\quad\vee\quad z_2=1-2i
        \end{flalign*}
    \end{enumerate}
    \item[8.]  Dapatkan semua nilai dari
    \[z=(2-2i)^{\frac{3}{5}}\]
    \textbf{Penyelesaian. } Ubah kedalam bentuk kutub
    \[(2-2i)^{\frac{3}{5}}=\left(2\sqrt{2}\textrm{cis}\left(-\frac{\pi}{4}\right)\right)^{\frac{3}{5}}=\left(2\sqrt{2}\textrm{cis}\left(2k\pi-\frac{\pi}{4}\right)\right)^{\frac{3}{5}}=(2^\frac{3}{2})^{\frac{3}{5}}\textrm{cis}\left(\frac{8k\pi-\pi}{4}\cdot\frac{3}{5}\right)=2^\frac{9}{10}\textrm{cis}\left(\frac{(8k-1)3\pi}{20}\right)\]
    Sehingga didapatkan
    \begin{flalign*}
        k=1\Longrightarrow&\:z_1=2^\frac{9}{10}\textrm{cis}\left(\frac{(8(1)-1)3\pi}{20}\right)=2^\frac{9}{10}\textrm{cis}\left(\frac{21\pi}{20}\right)&\\
        k=2\Longrightarrow&\:z_2=2^\frac{9}{10}\textrm{cis}\left(\frac{(8(2)-1)3\pi}{20}\right)=2^\frac{9}{10}\textrm{cis}\left(\frac{9\pi}{4}\right)&\\
        k=3\Longrightarrow&\:z_3=2^\frac{9}{10}\textrm{cis}\left(\frac{(8(3)-1)3\pi}{20}\right)=2^\frac{9}{10}\textrm{cis}\left(\frac{69\pi}{20}\right)=2^\frac{9}{10}\textrm{cis}\left(\frac{29\pi}{20}\right)&\\
        k=4\Longrightarrow&\:z_4=2^\frac{9}{10}\textrm{cis}\left(\frac{(8(4)-1)3\pi}{20}\right)=2^\frac{9}{10}\textrm{cis}\left(\frac{93\pi}{20}\right)=2^\frac{9}{10}\textrm{cis}\left(\frac{13\pi}{20}\right)&\\
        k=5\Longrightarrow&\:z_5=2^\frac{9}{10}\textrm{cis}\left(\frac{(8(5)-1)3\pi}{20}\right)=2^\frac{9}{10}\textrm{cis}\left(\frac{117\pi}{20}\right)=2^\frac{9}{10}\textrm{cis}\left(\frac{37\pi}{20}\right)&\\
    \end{flalign*}
\end{enumerate}

\end{document}
